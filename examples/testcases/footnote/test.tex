% Test the 'hook' code.
\input "paratext2.tex"
\def\doLines{\doGridLines}%
\stylesheet{usfm.sty}
\stylesheet{ptx2pdf.sty}
\stylesheet{default-custom.sty}
\stylesheet{test.sty}
\input "usfmTex-settings.tex"
%\input "usfmTex-ext.tex"
%\input "hooks.tex"
%\tracingmacros=1
%\tracingassigns=1
\tracing{f}
\tracing{sP}
\tracing{n}
\tracing{nS}
\tracing{F}
\tracing{o}
\tracing{e}
\tracing{i}
%\ParagraphedNotes{fe}
\AutoCallers{fe}{A,B,C,D,E,F,G,H,I,J,K}
%\scrollmode
%\nonstopmode
% Set added material in desired form
\def\addmtlquteht{2pt}
\def\addmtlqutewd{1.5pt}
\def\addmtlqutelw{0.2pt}
\csname ColorFontstrue\endcsname
\def\notebreak{\endgraf\kern 1sp}
\tracing{s}
\tracing{j}
\ptxfile{test.usfm}
\ParagraphedNotes{fe}
\newparnotesfalse
\ptxfile{test.usfm}
\unbalancedtrue
\tracing{t}
%\tracingassigns=1
\newcount\tp
\def\logoutput{1}
\def\testpoint{%\tracinggroups=1
  \edef\thetp{test point \the\tp}\expandafter\write\expandafter\logoutput\expandafter{\thetp}\message{\thetp}\global\advance\tp by 1\relax}
%\tracingassigns=1
\tracing{n}
\tracing{f}
\tracing{p}
\global\lastptxfiletrue
%\expandafter\def\csname bookend-all\endcsname{\message{ENDOFBOOK}==END+OF+BOOK}%
\ptxfile{table.usfm}
\bye
