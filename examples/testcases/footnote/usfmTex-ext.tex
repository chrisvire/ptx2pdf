% usfmTex-ext.tex
%
% Custom TeX setup file for the usfmTex macro package. These
% additional TeX commands will be loaded in after the stylesheet
% and style overrides are loaded

%%%%%%%%%%%%%%%%%%%%%%%  Extra Tweaks %%%%%%%%%%%%%%%%%%%%%%

% In a perfect world none of these would be needed but when
% your publication throws you a curve, perhaps one of these
% work-arounds might help you do what you want to do. Below
% you will find a number of extra Tweaks that may not be included
% in the normal functions of this macro package. Hopefully the
% comments and documentation included with each one will be
% helpful and enable you to use them to your satisfaction. You
% can comment, uncomment and modify them to work with this
% specific project. You can also add more as needed. Have fun!


%%%%% Depricated and special USFM Markers
% Process \b 
% This is often frowned upon but if you want to add extra
% spaces around poetry, uncomment this next line.
%\def\b{\vskip 0.5\baselineskip}


%%%%% Baselineskip Adjustment Hooks
% This hook provides a means to adjust the baselineskip on a
% specific style. It provides a place to put the initial 
% setting so the hook can make the change and then go back
% to the initial setting when done.
\newdimen\remblskip \remblskip=\baselineskip

% Baselineskip Adjustment Hook Example
%\sethook{start}{s1}{\remblskip=\baselineskip \baselineskip=10pt}
%\sethook{after}{s1}{\baselineskip=\remblskip}


%%%%% Header output
% To adjust the size of the page number in the header or footer
% use the following code. Adjust font name and size as necessary.
%\font\mysmallfont="[../Fonts/CharisSIL/CharisSILB.ttf]" at 10pt
%\def\pagenumber{{\mysmallfont \folio}}

% This will output only the book name in the header that is
% found in \h. (This should be added to ptx2pdf.)
\catcode`\@=11
\def\bookname{\x@\extr@ctfirst\p@gefirstmark\relax\@book}
\catcode`\@=12


%%%%% Footnote tweaks
% Footnote caller kerning - To adjust space around the
% footnote caller use the following code Adjust the kern

% Inter Note Skip - Adjust the horizontal space between footnotes,
% both paragraphed and non-paragraphed
\catcode`\@=11
  \internoteskip=1em plus 0em
  \InterNoteSpace=0pt
\catcode`\@=12

% Inter-note Penalty - Control the amount of "tension" between
% parts of a footnote to help control line breaking. If you
% use the highest setting, 10000, it will never break. A lower
% setting, like 9999, will lossen it up. Default is 9999.
\def\internotepenalty{9999}


%%%%% Substituting Characters
% Some times, when a character does not exist in a font
% you can substitute from another if no special rendering
% is needed. This code will do that. Modify as needed.
%% Example 1 - None Rapuma font
%\font\cwi="[../Fonts/Padauk/Padauk.ttf]" at 10pt
%\catcode"A92E=\active                          % Make U+A92E an active character
%\def^^^^a92e{\leavevmode{\cwi\char"A92E}}      % Define it to print itself

%% Example 2 - Rapuma (secondary) font
%\crossmaltese=\wingdingfontregular\char"2720
%\catcode"2720=\active                          % Make U+2720 an active character
%\def^^^^2720{\leavevmode{\crossmaltese}}       % Define it to replace itself
                                                % with the character found in
                                                % the specified font


%%%%% Non-standard Spaces
% Some publications may use non-standard (U+0020) between words.
% But TeX (and XeTeX) will treat spaces other than U+0020 as
% non-breaking which messes up your justification. This is a
% work around to force TeX to break and stretch words with
% another space character in a controled way.
%\catcode"2009=13
%\def^^^^2009{\hskip .2em plus.1em minus.1em\relax}


%%%%% Heading space
% There always seems to be problems with extra space between 
% the section heading and the top of the column when the 
% section head is at the top of the column. To take up the
% slack this code will usually help. Any adjustments needed
% should be done in the .sty and \VerticalSpaceFactor.
% Trying to adjust this code doesn't seem to make any dif.
% This is normally on by default.
\catcode`\@=11
\catcode`\@=12


%%%%% Chapter/Verse Number Settings
% Some additional macros and commands to gain more control over verse numbers

% Superscript tweaks (uncomment \def settings to override default values)
% Any style that has \Superscript in it will be affected by these modified settings
%\def\SuperscriptRaise{0.85ex}   % Raise or lower a verse number or caller
%\def\SuperscriptFactor{0.75}    % Scale up or down the size of the verse number or caller


%%%%% Special commands
% These are extra commands that can be inserted in the text


