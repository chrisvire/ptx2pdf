% Test the 'hook' code.
%\tracingassigns=1
%\tracinggroups=1
\input "paratext2.tex"
\stylesheet{usfm.sty}
\stylesheet{default-custom.sty}
%\stylesheet{test.sty}
\input "usfmTex-settings.tex"
%\input "usfmTex-ext.tex"
%\input "hooks.tex"
%\tracingmacros=1
%\tracinggroups=1
%\tracingifs=1
%\tracing{g}
%\tracing{G}
%\tracing{s}
\tracing{Ds}
\tracing{sP}
%\tracing{F}
\CropMarkstrue
%\def\doLines{\doGraphPaper}
%\def\doLines{\doNoteLines}
%\tracing{P}
\tracingoutput=1
% Set added material in desired form
\def\addmtlquteht{2pt}
\def\addmtlqutewd{1.5pt}
\def\addmtlqutelw{0.2pt}
\BindingGutter=3cm
\BindingGuttertrue
\def\NoteLineY{0.6cm}
\csname ColorFontstrue\endcsname
\csname lastptxfiletrue\endcsname
\csname unbalancedtrue\endcsname
\def\NoteLineRuleType{rule}
\def\NoteLineColMinor{0 0 0}
\def\NoteLineLineMinor{0.3pt}
\def\NoteLineXoffset{1cm}
\showboxbreadth=30
\tracing{C}
\showboxdepth=30
\catcode`\@=11
\def\p@gebotmark{}
\catcode`\@=12
\IndentAtChapterfalse
\IndentAfterHeadingfalse
\def\lmz{\expandafter\gdef\csname p:leftmargin\endcsname{0}}
\def\vfour{\gdef\ptxversion{4}}
\def\vnormal{\gdef\ptxversion{0}}
\ptxfile{test.usfm}
\bye
