% Test the 'hook' code.
%\tracingassigns=1
%\tracinggroups=1
\input "paratext2.tex"
\input "ptx-ptxprint.tex"
\stylesheet{usfm_sb.sty}
\stylesheet{ptx2pdf.sty}
\stylesheet{default-custom.sty}
\stylesheet{test.sty}
\input "usfmTex-settings.tex"
%\input "usfmTex-ext.tex"
%\input "hooks.tex"
%\tracingmacros=1
%\tracinggroups=1
%\tracingifs=1
%\tracing{g}
%\tracing{s}
\tracing{C}
%\tracing{F}
\tracing{T}
\tracing{A}
%\tracing{m}
\SetTriggerParagraphSeparator{=@=}
\tracing{P}
\tracing{o}
\tracing{i}
\tracingpages=1
%\MarkTriggerPointstrue
%\tracingifs=1
% Set added material in desired form
\def\addmtlquteht{2pt}
\def\addmtlqutewd{1.5pt}
\def\addmtlqutelw{0.2pt}
\csname ColorFontstrue\endcsname
%\ptxfile{frt.usfm}
\setCutoutSlop{droppic3}{0}{0}
\KeepFigure{TST}{Borogroves}{1}
\KeepFigure{TST}{k.Brillig}{1}
\KeepFigure{TST}{k.(Slithy)Toves}{2}
\tracingpages=1
\showboxdepth=99
\showboxbreadth=990
\StudyNotes{f}
\ptxfile{test.usfm}
\bye
