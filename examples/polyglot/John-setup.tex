
%
% This file defines some basic parameters that control the format of the output

% Dimensions of A7 paper
%\PaperWidth=74.25mm
%\PaperHeight=105mm
% Dimensions of A6 paper
%\PaperWidth=105mm
%\PaperHeight=148.5mm
% Dimensions of A5 paper
\PaperWidth=148.5mm
\PaperHeight=210mm
% Dimensions of A4 paper
%\PaperHeight=296.9mm
%\PaperWidth=210mm
% A4 landscape
%\PaperHeight=210mm
%\PaperWidth=296.9mm

\diglottrue %Diglot edition (2 languages/translations)
\diglotSepNotestrue % Keep footnotes separate between translations? (Probably best set to false if only one side has them)
\diglotBalNotesfalse% Balance column lengths from footnotes?

%\CropMarkstrue
\CropMarksfalse
\ColorFontsfalse
% Basic unit for margins; changing this will alter them all
\MarginUnit=.5in

% Relative sizes of margins, based on the unit above
\def\TopMarginFactor{1.0}
\def\BottomMarginFactor{1.0}
\def\SideMarginFactor{0.75}

% Fonts to use for "plain", "bold", "italic", and "bold italic" from the Paratext stylesheet
% (they need not really be italic, etc, of course)
\def\regular{"Gentium Plus"}
\def\bold{"TeX Gyre Termes/B"}
\def\italic{"Gentium Basic/I"}
\def\bolditalic{"TeX Gyre Schola/BI"}

% Can override the defaults for Left or Right:  (It doesn't make sense to do both)

% Use right-to-left layout mode
%\RTLtrue

% Unit for font sizes in the stylesheet; changing this will scale all text proportionately
\FontSizeUnit=0.9pt
\FontSizeUnitR=1.1pt
\FontSizeUnitC=1.1pt


% Scaling factor used to adjust line spacing, relative to font size
\def\LineSpacingFactor{1.04}
\def\VerticalSpaceFactor{1.0}

% Information to include in the running header (at top of pages, except first)
% We set the items to print at left/center/right of odd and even pages separately
% Possible contents:
%   \rangeref = Scripture reference of the range of text on the page;
%   \firstref = reference of the first verse on the page)
%   \lastref = reference of the last verse on the page)
%   \pagenumber = the page number
%   \empty = print nothing in this position
\def\RHevenleft{\rangerefL}
\def\RHevencenter{\pagenumber}
\def\RHevenright{\rangerefB}

\def\RHoddleft{\rangerefR}
\def\RHoddcenter{\pagenumber}
\def\RHoddright{\rangerefC}

\def\RHtitleleft{\empty}
\def\RHtitlecenter{\empty}
\def\RHtitleright{\empty}

\def\RFoddcenter{\empty}
\def\RFevencenter{\empty}
\def\RFtitlecenter{\pagenumber}

\VerseRefstrue % whether to include verse number in running head, or only chapter

\OmitVerseNumberOnetrue % whether to skip printing verse number 1 at start of chapter
%\IndentAtChaptertrue % whether to use paragraph indent at drop-cap chapter numbers

\AutoCallers{f}{*,†,‡,¶,§}
\PageResetCallers{f}
\PageResetCallers{fR}
%\NumericCallers{f}
%\OmitCallerInNote{f}

\ParagraphedNotes{x} % reformat \x notes as a single paragraph
\ParagraphedNotes{f} % reformat \f notes as a single paragraph

\TitleColumns=1
\IntroColumns=1
\BodyColumns=1

\def\ColumnGutterFactor{15} % gutter between double cols, relative to font size

%\BindingGuttertrue % add extra margin of \BindingGutter on binding side
%\BindingGutter=10pt
%\DoubleSidedfalse

%\GenerateTOC[Table of Contents]{toc-auto.tex}


\def\addmtlquteht{2pt}
\def\addmtlqutewd{1.5pt}
\def\addmtlqutelw{0.2pt}

%Only Left column gets NIV-style corner-brackets for added material
\sethook{before}{addL}{\lower 1pt \hbox{\vrule width \addmtlqutelw height \addmtlquteht\vrule width \addmtlqutewd height \addmtlqutelw}\kern -1pt\ignorespaces}
\sethook{after}{addL}{\kern -1pt \lower 1pt \hbox{\vrule width \addmtlqutewd height \addmtlqutelw\vrule width \addmtlqutelw height \addmtlquteht}}
\sethook{end}{xo}{\dimen0=\lastskip\unskip\penalty 10000\hskip \dimen0  minus 0.5\dimen0\penalty 10000}
\sethook{start}{xt}{\hyphenpenalty=-10\linepenalty=900}

%\sethook{before}{add}{\lower 1pt \hbox{\vrule width \addmtlqutelw height \addmtlquteht\vrule width \addmtlqutewd height \addmtlqutelw}\kern -1pt}
%\sethook{after}{add}{\kern -1pt \lower 1pt \hbox{\vrule width \addmtlqutewd height \addmtlqutelw\vrule width \addmtlqutelw height \addmtlquteht}}

% Define the Paratext stylesheet to be used as a basis for formatting
\stylesheet{usfm.sty}
\stylesheet{usfmTex.sty}
%\stylesheet{default-custom.sty}
%\stylesheet{Galatians.sty}
