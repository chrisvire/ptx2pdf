
%%%%%%%%%%%%%%%%%%%%%%%%%%%%%%%%%%%%%%%%%%%%%%%%%%%%%%%%%%%%%%%%%%%%%%%
% Part of the ptx2pdf macro package for formatting USFM text
% copyright (c) 2007 by SIL International
% written by Jonathan Kew
%
% Permission is hereby granted, free of charge, to any person obtaining  
% a copy of this software and associated documentation files (the  
% "Software"), to deal in the Software without restriction, including  
% without limitation the rights to use, copy, modify, merge, publish,  
% distribute, sublicense, and/or sell copies of the Software, and to  
% permit persons to whom the Software is furnished to do so, subject to  
% the following conditions:
%
% The above copyright notice and this permission notice shall be  
% included in all copies or substantial portions of the Software.
%
% THE SOFTWARE IS PROVIDED "AS IS", WITHOUT WARRANTY OF ANY KIND,  
% EXPRESS OR IMPLIED, INCLUDING BUT NOT LIMITED TO THE WARRANTIES OF  
% MERCHANTABILITY, FITNESS FOR A PARTICULAR PURPOSE AND  
% NONINFRINGEMENT. IN NO EVENT SHALL SIL INTERNATIONAL BE LIABLE FOR  
% ANY CLAIM, DAMAGES OR OTHER LIABILITY, WHETHER IN AN ACTION OF  
% CONTRACT, TORT OR OTHERWISE, ARISING FROM, OUT OF OR IN CONNECTION  
% WITH THE SOFTWARE OR THE USE OR OTHER DEALINGS IN THE SOFTWARE.
%
% Except as contained in this notice, the name of SIL International  
% shall not be used in advertising or otherwise to promote the sale,  
% use or other dealings in this Software without prior written  
% authorization from SIL International.
%%%%%%%%%%%%%%%%%%%%%%%%%%%%%%%%%%%%%%%%%%%%%%%%%%%%%%%%%%%%%%%%%%%%%%%

% Paratext formatting macros, spanning footnotes version

%+c_pt_intro
% Here we declare generic useful stuff. Including log output routines.
\TeXXeTstate=1 % enable the eTeX bidi extensions, in case we need RTL support
\catcode`\@=11
\count11=20 \count14=20 % need more temporary dimens and boxes
\let\x@=\expandafter
\newif\iftemp \tempfalse
\def\stripqu@tes#1"#2"#3\relax{#2}% Strip quotes if present
\edef\t@mp{\jobname}%
\xdef\j@bname{\x@\x@\x@\stripqu@tes\x@\t@mp\x@"\t@mp"\relax}

%There is no equivalent definition to maxdimen for counts
\def\m@xcount{2147483647}
\def\m@ncount{-2147483647} % Technically ...8, but TeX moans about athat.

\def\MSG{\immediate\write16 } % shorthand to write a message to the terminal
\def\TRACE#1{}%\let\TRACE=\MSG % default - consume messages

% This lot has to be early
\newif\ifinn@te\inn@tefalse
\newif\ifm@rksonpage % Try to keep track of marks, so we can kill them in end-sections. 
\m@rksonpagefalse
%-c_pt_intro

%+c_define-hooks
% `\addtoendhooks` collects macros to be executed at the end of the job
\def\addtoendhooks#1{\x@\global\x@\@ndhooks\x@{\the\@ndhooks #1}}
\def\addtoendptxhooks#1{\x@\global\x@\@ndptxhooks\x@{\the\@ndptxhooks #1}}
\newtoks\@ndhooks
\newtoks\@ndptxhooks
\let\s@ve@nd=\end
\def\end{\ifsk@pping\egroup\fi\par\vfill\supereject  \the\@ndhooks \s@ve@nd}

%: `\addtoinithooks` is for stuff we do during one-time-setup before the first PTX file
\def\addtoinithooks#1{\x@\global\x@\@nithooks\x@{\the\@nithooks #1}}
\newtoks\@nithooks

%: `\addtoidhooks` is for stuff we do at an id or a periph
\def\addtoidhooks#1{\x@\global\x@\@idhooks\x@{\the\@idhooks #1}}
\newtoks\@idhooks

%: `\addtoeveryparhooks` is for the start of every paragraph
\def\addtoeveryparhooks#1{\x@\global\x@\@veryparhooks\x@{\the\@veryparhooks #1}}
\newtoks\@veryparhooks

\def\addtoeveryparstarthooks#1{\x@\global\x@\@veryparstarthooks\x@{\the\@veryparstarthooks #1}}
\newtoks\@veryparstarthooks

%: `\addtoparstylehooks` is for stuff to do at each new parstyle marker
\def\addtoparstylehooks#1{\x@\global\x@\p@rstylehooks\x@{\the\p@rstylehooks #1}}
\newtoks\p@rstylehooks
%-c_define-hooks

%+c_timestamp
% initialize a \timestamp macro for the cropmarks etc to use
\edef\timestamp{\number\year.% print the date and time of the run
  \ifnum\month<10 0\fi \number\month.%
  \ifnum\day<10 0\fi \number\day\space :: }%
\count255=\time \divide\count255 by 60
\edef\hrsmins{\ifnum\count255<10 0\fi \number\count255:}%
\multiply\count255 by 60 \advance\count255 by -\time
\count255=-\count255
\edef\hrsmins{\hrsmins
  \ifnum\count255<10 0\fi \number\count255}%
\edef\timestamp{\timestamp \hrsmins}

%-c_timestamp

%+c_evenpage
% Some things need to start on an odd-numbered page, others on an even numbered page
\def\TPILB{}
\def\EmptyPage{\message{Blank page for \the\pageno}\bgroup\m@rksonpagefalse\pagecontents\def\pagecontents{\vfill\TPILB}\plainoutput\egroup}
\def\need@evenpage{\ifodd\pageno\EmptyPage\fi}
\def\need@oddpage{\ifodd\pageno\else\EmptyPage\fi}
\def\need@quadpage{\ifodd\pageno\EmptyPage\need@quadpage\fi \ifodd\numexpr \pageno/2\relax \EmptyPage\EmptyPage\fi}

%-c_evenpage

%+c_imports
\input ptx-constants.tex
\input ptx-diglot.tex
\input ptx-tracing.tex
\input ptx-para-style.tex
\input ptx-char-style.tex
\input ptx-milestone-style.tex
\input ptx-note-style.tex
\input ptx-stylesheet.tex % must come after the ptx-*-style.tex macros
\input ptx-attribute.tex %Must come after ptx-stylesheet
\input ptx-references.tex
\input ptx-cropmarks.tex
\input ptx-toc.tex
\input ptx-tables.tex
\input ptx-triggers.tex % must come before adj-list and pic-list
\input ptx-adj-list.tex % must come after ptx-stylesheet.tex
\input ptx-pic-list.tex % must come after ptx-stylesheet.tex
\input ptx-cutouts.tex
\input ptx-callers.tex % must come after ptx-note-style.tex
\input ptx-figure.tex % figure-handling
% Additional modules that are not part of the normal ptx2pdf module
\input ptxplus-character-kerning.tex
\input ptx-unicode.tex
% We want to import this module here, but it has to come after the stylesheet is loaded.
%\input ptxplus-marginalverses.tex
\input ptx-extended.tex % \esb and friends.
\input ptx-borders.tex % Borders for esb etc
\input ptx-labels.tex % \zlabel and friends
\input ptx-plugins.tex % Load optional plugins based on contents of \pluginlist, if defined
%-c_imports

%+c_fonts-basic
% default font names (override in setup file)
\ifx\regular\undefined   \def\regular{"Times New Roman"}      \fi
\ifx\bold\undefined      \def\bold{"Times New Roman/B"}       \fi
\ifx\italic\undefined    \def\italic{"Times New Roman/I"}     \fi
\ifx\bolditalic\undefined\def\bolditalic{"Times New Roman/BI"}\fi
%-c_fonts-basic

%+c_makenote
% footnote macros based on plain.tex \footnote, \vfootnote
\newbox\he@dingnotes
\newbox\t@blenotes
%:
% #1 ~ class to store the note in. E.g. f, x. The results of `\NoteBlendInto`.
% #2 ~ class to use for styling (what the source had in it)
% #3 ~ styled text for main text caller
% #4 ~ styled text for in note caller
\def\@wrap{}
\def\m@kenote#1#2#3#4{\let\@sf\empty
  % text is read later
  \ifhmode\edef\@sf{\spacefactor\the\spacefactor}\/\fi%
  % if footnote is on a chapter number ...
  \ifinextended #3\edef\@wrap{\global\setbox\sid@barnotes=\vbox\bgroup\unvbox\sid@barnotes}\else
    \ifdoingt@ble #3\edef\@wrap{\global\setbox\t@blenotes=\vbox\bgroup\unvbox\t@blenotes}\else%
      \ifhe@dings #3\edef\@wrap{\global\setbox\he@dingnotes=\vbox\bgroup\unvbox\he@dingnotes}%
      \else\edef\@wrap{}%
        \x@\ifx\csname ch@pter\g@tdstat waiting\endcsname\relax #3
        \else
          \everypar={}\x@\global\x@\let\csname ch@pter\g@tdstat waiting\endcsname\relax
          \global\setbox\ch@pternote=\hbox{\box\ch@pternote #3}%
        \fi % output caller
      \fi
    \fi
  \fi%
  % @sf preserves the space factor (e.g. extra space after a period), restore it now
  \@sf \@wrap\vm@kenote{#1}{#2}{\getp@ram{notecallerstyle}{#2}{#2}\ifx\p@ram\relax #3\else #4\fi}}
%-c_makenote

%+c_vmakenote
\newtoks\d@note
\def\vm@kenote#1#2#3{%
  \inn@tetrue%
  \let\next\relax%
  \def\n@tetype{#1\g@tndstat}%
  \TRACE{vm@kenote \n@tetype}%
  \x@\let\x@\th@sins\csname note-no-insert-\n@tetype\endcsname
  \checkp@ranotes{#1}% check whether this note class is to be paragraphed 
  \ifp@ranotes\n@tewidth=\maxdimen \else\setn@tewidth{#1}\fi
  \ifx\th@sins\relax \d@note{\x@\insert\csname note-\n@tetype\endcsname}%
  \else \d@note{\th@sins}\fi
  \trace{f}{Footnote (\n@tetype) [\reference] will be \the\n@tewidth \space wide (page is \the\hsize)}%
  \the\d@note\bgroup\setbox9=\vbox\bgroup% insert note-f (or note-x)
    % \XeTeXuseglyphmetrics=3 % so that notes sit on the baseline based on content and not font descender
    \hsize=\n@tewidth
%%% single-column notes:
    \interlinepenalty\interfootnotelinepenalty % set penalty to break lines
    \floatingpenalty\@MM % make sure note does not float away from caller to another page
    \let\styst@k\empty
    \parfillskip=0pt plus 1fil
    \leftskip=0pt \rightskip=0pt
    \getp@ram{justification}{#1}{#1}%
    \ifx\p@ram\c@nter
     \leftskip=\noteRag \rightskip=\leftskip \parfillskip=0pt
    \else\ifx\p@ram\l@ft
     \rightskip=\noteRag \ifRTL\parfillskip=0pt \fi
    \else\ifx\p@ram\r@ght
     \leftskip=\noteRag \ifRTL\else\parfillskip=0pt \fi
    \fi\fi\fi
    \trace{f}{note(#1) justification=\p@ram, leftskip=\the\leftskip, rightskip=\the\rightskip, parfillskip=\the\parfillskip}%
    \ifx\th@sins\relax
      \s@tbaseline{#1}{#1}%
    \else
      \ifp@ranotes
        \baselineskip=0pt
      \else
        \s@tbaseline{#1}{#1}%
      \fi
    \fi
    \ifp@ranotes\else
      \getp@ram{spacebefore}{#1}{#1}\trace{f}{note(#1) spacebefore: \p@ram}\ifx\p@ram\relax\else\kern\p@ram\verticalsp@ceunit\fi
      \setbox0=\hbox{\XeTeXuseglyphmetrics=0 \char32}\dimen0=\ht0
      \ifdim\dimen0<\baselineskip
        \dimen1=\baselineskip\advance\dimen1 by -\dimen0
        \vskip\dimen1
		\trace{f}{\reference : Footnote vskip=\the\dimen1 baselineskip=\the\baselineskip strut height=\the\dimen0}%
	\fi\fi
    \leavevmode % begin paragraph
    \ifdiglot \x@\the\csname diglot\c@rrdstat ho@ks\endcsname \fi%
    \ifp@ranotes\else\getp@ram{firstindent}{#1}{#1}\trace{f}{note(#1) firstindent: \p@ram}%
      \ifx\p@ram\relax\else\kern\p@ram\IndentUnit\fi\fi
    \ifRTL\setbox2=\lastbox\beginR\box2\fi % if RTL text this paragraph needs to be RTL
    % if note omitting caller from note (i.e. all callers are *'s)
    \let\stylet@pe\ss@Note\mcpush{\stylet@pe}{#2}%
    \testomitc@ller{#2}\ifomitc@ller\trace{f}{Omitting caller}\else
    % save copy of caller in temporary box, if non-empty add a little space
    \setbox0=\hbox{#3}%
    \ifdim\wd0>0pt \ifdim\wd0 <\NoteCallerWidth% Allow for wide callers, but standardise normal callers to \NoteCallerWidth, for better alignment
      \setbox0=\hbox{\hbox to \NoteCallerWidth{\hfil\box0\hfil}}%
    \fi\fi
    \setbox1=\copy0\unhbox1\ifdim\wd0>0pt \kern\NoteCallerSpace\fi
  \fi
  % currently we do not allow footnotes to break to the next page, so this may not be necessary
%  \splittopskip\ht\f@@tstrut % top baseline for broken footnotes
%  \splitmaxdepth\dp\f@@tstrut
  %\getp@ram{fontsize}{#2}\edef\c@rrfontsize{\ifx\p@ram\relax12\else\p@ram\fi}%
  \s@tfont{#2}{#2}%
  \futurelet\next\fo@t}% use plain.tex footnote processor
%-c_vmakenote

%+c_endnotes

\def\NoteAtEnd#1{%Treat the specified note as an end-note.
  \x@\newb@x\csname endn@te-#1\endcsname
  \x@\gdef\csname addt@endnote-#1\endcsname{%thanks to afterassignment, this gets put immediately after the start of the vbox
    \checkp@ranotes{#1}%
    \ifp@ranotes
      \hsize=\maxdimen
      \x@\ifvoid \csname endn@te-#1\endcsname\else
        %\x@\showbox\csname endn@te-#1\endcsname
        \vbox{}%Sacrificial box for makehboxofhboxes to consume
        \x@\unvbox\csname endn@te-#1\endcsname\makehboxofhboxes%
        %\showbox0
        \setbox0=\hbox{\unhbox0 \removehboxes}%
        \unhbox0 %
      \fi
    \else
      \x@\ifvoid \csname endn@te-#1\endcsname
        \hbox{}%Empty hbox to force some baselineskip at the top of the box
      \else
        %\x@\showbox\csname endn@te-#1\endcsname
        \unvbox\csname endn@te-#1\endcsname
      \fi
    \fi
    }%
  \x@\gdef\csname note-no-insert-#1\endcsname{\x@\afterassignment\csname addt@endnote-#1\endcsname\global\setbox\csname endn@te-#1\endcsname\vbox}%
  \x@\gdef\csname ztestnotes-#1\endcsname{% Set the @ndnotesfound flag based on one type of note. 
      \ifvoid\csname endn@te-#1\endcsname\@ndnotesfoundfalse\else\@ndnotesfoundtrue\fi}%
  \x@\gdef\csname zplacenotes-#1\endcsname{\ifvmode\else\par\fi \dimen0=\prevdepth %
    %\showbox\csname endn@te-#1\endcsname
    \checkp@ranotes{#1}%
    \ifp@ranotes
      \ifdim \dimen0>0pt \kern-\dimen0\fi
      \ifvoid\csname endn@te-#1\endcsname\else
        \maken@tepara{\csname endn@te-#1\endcsname}{#1}%
      \fi
    \else 
      \s@tbaseline{#1}{#1}%
      {%\x@\showbox\csname endn@te-#1\endcsname
      \setbox0=\vbox{\unvbox\csname endn@te-#1\endcsname}%
      %\showbox0
      \unvbox0}%
    \fi
    }% Individual note placement
  \x@\endn@teclasses\x@{\the\endn@teclasses \\{#1}}%
  \x@\plac@ndnotes\x@{\the\plac@ndnotes\csname zplacenotes-#1\endcsname}%
}
\let\newb@x=\newbox
\let\newc@unt=\newcount
\let\newt@ks=\newtoks
\let\newdim@n=\newdimen
\let\newins@rt=\newinsert
\let\n@wif=\newif
\newtoks\endn@teclasses
\newtoks\plac@ndnotes
\newif\if@ndnotesfound
\newif\ifpartialfr@med \partialfr@medfalse %Is there a textborder on partial
\def\pretextb@rderskip{0pt}
\def\posttextb@rderskip{0pt}
\plac@ndnotes{\ifhmode\endgraf\fi\endn@terule} % Pre-fill the toklist
\def\ch@ckendnotes#1{\x@\ifvoid\csname endn@te-#1\endcsname\else\global\@ndnotesfoundtrue\fi}
\def\zplaceallnotes{\bgroup\setbox0\vbox{\the\plac@ndnotes}\unvbox0\egroup}

\newif\ifnotesEachBook
\notesEachBooktrue
\NoteAtEnd{fe}

%-c_endnotes

% use \OmitCallerInNote{f} to omit callers from the note at foot of page
% but leave them in the body text (e.g. all the callers are *'s)
\def\OmitCallerInNote#1{%
  \expandafter\let\csname omit-in-note #1\endcsname=1}
%
\def\testomitc@ller#1{\expandafter\ifx\csname omit-in-note #1\endcsname\relax
  \omitc@llerfalse \else \omitc@llertrue \fi}
\newif\ifomitc@ller

% create a "strut" (see TeXbook) of suitable size for the note style
\def\footstrut{\s@tfont{\newn@testyle}{\newn@testyle}%
  \s@tbaseline{\newn@testyle}{\newn@testyle}%
  \setbox\f@@tstrut=\hbox to 0pt{\XeTeXuseglyphmetrics=0 \char32 \hss}%
  \dimen0=\ht\f@@tstrut \dimen2=\dp\f@@tstrut
  \dimen4=\dimen0 \advance\dimen4 by \dimen2
%  \ifdim\dimen4<\baselineskip
    \dimen6=100\baselineskip \divide\dimen6 by \dimen4
    \multiply\dimen0 by \dimen6 \divide\dimen0 by 100
    \multiply\dimen2 by \dimen6 \divide\dimen2 by 100
%  \fi
  \setbox\f@@tstrut=\hbox{}\ht\f@@tstrut=\dimen0 \dp\f@@tstrut=\dimen2
  \copy\f@@tstrut}

%+c_foot
% called at end of note
\def\@foot{\ifp@ranotes \parfillskip=0pt \else\parfillskip=0pt plus 1fil\fi%\else\strut\fi
  \par
  \ifp@ranotes\else\getp@ram{spaceafter}{\n@tetype}{\n@tetype}\trace{f}{note(\n@tetype) spaceafter: \p@ram}%
    \ifx\p@ram\relax\penalty0\else\vskip\p@ram\verticalsp@ceunit\fi
    \setbox0=\hbox to 0pt{\XeTeXuseglyphmetrics=0 \char32 \hss}%
    \dimen0=\baselineskip \advance\dimen0 by -\ht0 \advance\dimen0 by -\prevdepth
    \vskip \dimen0
  \fi
  \egroup\end@llpoppedstyles{N*}%
  \def\d@##1+##2\E{\if ##1N\else\MSG{Bad marker ##2\space but expected a closing note marker}\fi}\mctopnoms
  \mcpop\trace{i}{note \n@tetype, height=\the\ht9, \ifn@npublishablec@t non\fi publishable}\ifn@npublishablec@t\global\n@npublishablec@tfalse\else\unvbox9\fi\egroup
  }
\newbox\f@@tstrut
\def\n@teglue{2em plus 1em minus .5em\relax} % glue to be used between paragraphed notes
%-c_foot

% set baseline appropriately for the given style (may be using much smaller font than body)
% baselineskip = leading
\def\s@tbaseline#1#2{% Setbaseline-normal
  \trace{F}{s@tbaseline #1\c@rrdstat /\noexpand#2=#2}%
  \def\s@urce{no data}%
  \def\f@ntstyle{#1\c@rrdstat}%
  \@s@tbaseline{#1}{#2}%
}

\def\s@tbaseline@#1#2{% Setting-Side specific already
  \trace{F}{s@tbaseline@ #1#2}%
  \def\s@urce{no data}%
  \def\f@ntstyle{#1#2}%
  \@s@tbaseline{#1#2}{#1#2}%
}

\def\@s@tbaseline#1#2{%
  \getp@ram{baseline}{#1}{#2}\ifx\p@ram\relax
    \getp@ram{fontsize}{#1}{#2}\ifx\p@ram\relax\else
      \def\s@urce{calcn:\f@ntstyle}%
      \dimen0=\p@ram\le@dingunit
      \multiply\dimen0 by \LineSpaceBase \divide\dimen0 by 12 % default .75 shift of .85ex (ex=.5 fontsize) against 14/12
      \trace{F}{baseline [\f@ntstyle] = \the\dimen0 = \p@ram * \the\le@dingunit  * \LineSpaceBase / 12}%
      \setp@ram{baseline}{\f@ntstyle}{\the\dimen0}\baselineskip=\dimen0
      \should@xist{\f@ntstyle}%
    \fi
  \else
    \def\s@urce{store:\f@ntstyle}%
    \baselineskip=\p@ram\fi
  \trace{j}{set baselineskip=\the\baselineskip (\s@urce)}}

\newif\ifkeepn@tes
\newif\ifst@dynotes
\newif\ifnotst@dynotes
\newif\ifonlyst@dynotes
\def\n@ffin#1{}
\def\ins@rtn@tecl@ss#1{% insert the given note class, either paragraphed or separately
  % set appropriate hsize width for diglot or not.
  \checkp@ranotes{#1}\setn@tewidth{#1}\ifst@dynotes\ifnotst@dynotes\let\n@xt=\n@ffin\else\let\n@xt=\studyins@rtn@tecl@ss\fi\else
    \ifonlyst@dynotes\let\n@xt=\n@ffin\else
      \ifp@ranotes\let\n@xt=\parains@rtn@tecl@ss\else\let\n@xt=\separateins@rtn@tecl@ss\fi
  \fi\fi
  \n@xt{#1}}

% insert a note class in which each note is on its own line
\def\separateins@rtn@tecl@ss#1{%
  \def\cl@ss{#1\g@tndstat}%
  \trace{n}{seperateins@rtn@tecl@ss \cl@ss}%
  \x@\let\x@\th@cl@ss\csname note-\cl@ss\endcsname % make \th@cl@ss be a synonym for the current note class
  % if the noteclass has content to output ...
  \ifvoid\th@cl@ss\TRACE{no note}\else%
    \iff@rstnote\ifvoid\verybottomins\vfill\fi\fi % ignore depth of body text; fill space
    \iff@rstnote\footnoterule\global\f@rstnotefalse\else\kern\InterNoteSpace\fi
    \hbox{\vbox{\ifkeepn@tes\unvcopy\else\unvbox\fi \th@cl@ss}}\fi}% output notes

% insert a note class in which each note is in the same paragraph
\def\parains@rtn@tecl@ss#1{%
  \def\cl@ss{#1\g@tndstat}%
  \trace{n}{parains@rtn@tecl@ss \cl@ss}%
  \x@\let\x@\th@cl@ss\csname note-\cl@ss\endcsname % make \th@cl@ss be a synonym for the current note class
  % if the noteclass has content to output ...
  \ifvoid\th@cl@ss\else
    \iff@rstnote\ifvoid\verybottomins\vfill\fi %\kern-\lastd@pth\vfil % ignore depth of body text; fill space
      \footnoterule\global\f@rstnotefalse % output rule before first note
    \else\kern\InterNoteSpace\fi
    %\ifnum\pagenumber=8\showbox\th@cl@ss\fi
    {\maken@tepara{\th@cl@ss}{#1}}%
  \fi}

% insert a note class which is two column study notes style
\def\studyins@rtn@tecl@ss#1{%
  \def\cl@ss{#1\g@tndstat}%
  \trace{n}{studyins@rtn@tecl@ss \cl@ss \space (from #1)}%
  \x@\let\x@\th@cl@ss\csname note-\cl@ss\endcsname % make \th@cl@ss be a synonym for the current note class
  % if the noteclass has content to output ...
  \ifvoid\th@cl@ss\trace{n}{no note}\else
    \dimen10=0.4\ht\th@cl@ss
    \setbox10=\box1
    \splittopskip=0pt
    \@@LOOP\temptrue
      \setbox1=\copy\th@cl@ss
      \trace{n}{splitting study box to \the\dimen10}%
      \setbox11=\vsplit1 to \dimen10%
      \setbox12=\vsplit1 to \dimen10%
      \trace{n}{split box remainder \the\ht1}%
      \ifdim\ht1>0pt
        \ifdim\ht1 < \baselineskip \advance\dimen10 by \baselineskip\else \advance \dimen10 by 0.5\ht1\fi
      \else\tempfalse\fi \iftemp\@@REPEAT
    \setbox1=\box10
    \iff@rstnote
      \ifvoid\verybottomins\vfill\fi % ignore depth of body text; fill space
      \noterule{\AboveStudyNoteSpace}{\BelowStudyNoteRuleSpace}{\textwidth}{%
      \StudyNoteRuleLeftIndent \vrule height \StudyNoteRuleThickness width \StudyNoteRuleWidth\textwidth \StudyNoteRuleRightIndent}%
      \global\f@rstnotefalse
    \fi%\else\kern\InterNoteSpace\fi
    \dimen11=\ht11\ifdim\dimen11<\ht12 \dimen11=\ht12\fi
    \dimen12=\dp11\ifdim\dimen12<\dp12 \dimen12=\dp12\fi
    \trace{n}{Study notes (\the\ht11, \the\ht12)=\the\dimen11}%
    \hbox to \textwidth {\box\ifRTL 12\else11\fi
      \trace{o}{StudyGutterRule\ifStudyGutterRule true\else false\fi, ColumnGutterRule\ifColumnGutterRule true\else false\fi}%
      \makestudycolumngutter{\dimen11}{\dimen11}{\dimen12}{\dimen11}{0}\box\ifRTL 11\else12\fi}%
    \ifkeepn@tes\else
      \global\setbox\th@cl@ss\box\voidb@x
    \fi
  \fi}

\def\BelowFootNoteRuleSpace{0.5\AboveNoteSpace}
\def\FootNoteRuleThickness{0.4pt}
\def\FootNoteRuleWidth{1}
\def\FootNoteRuleLeftIndent{\hskip 0.00 mm}
\def\FootNoteRuleRightIndent{\hss}
\def\AboveStudyNoteSpace{\AboveNoteSpace}
\def\BelowStudyNoteSpace{0.5\AboveNoteSpace}
\def\StudyNoteRuleThickness{0.4pt}
\def\StudyNoteRuleWidth{1}
\def\StudyNoteRuleLeftIndent{\hskip 0.00 mm}
\def\StudyNoteRuleRightIndent{\hss}
\newif\ifStudyGutterRule

\newif\iff@rstnote
\f@rstnotetrue
\def\footnoterule{\noterule{\AboveNoteSpace}{\BelowFootNoteRuleSpace}{\n@tewidth}{%
  \FootNoteRuleLeftIndent \vrule height \FootNoteRuleThickness width \FootNoteRuleWidth\n@tewidth
  \FootNoteRuleRightIndent}}

\def\noterule#1#2#3#4{{\dimen1=#1\kern 0pt \vskip\dimen1 minus 0.5 \dimen1
  \dimen1=#2 % what's this for? \kern -\dimen1
  \setbox0=\hbox to #3{#4}\dimen0=0.5\ht0\kern-\dimen0\box0\kern-\dimen0
  \vskip \dimen1 minus 0.5\dimen1\kern 0pt}\prevdepth=-1000pt}


\def\EndNoteRuleWidth{0.5}
\def\EndNoteRuleThickness{0.4pt}
\def\EndNoteRuleLeftIndent{\hss}
\def\EndNoteRuleRightIndent{\hss}
\newdimen\AboveEndNoteSpace \AboveEndNoteSpace=14pt
\def\BelowEndNoteRuleSpace{10pt}

\def\endn@terule{\relax
  \@ndnotesfoundfalse
  \let\\=\ch@ckendnotes\the\endn@teclasses
  \if@ndnotesfound\zendnoterule\fi
}
\def\EndNoteSeparator{\hbox to \hsize{\EndNoteRuleLeftIndent
    \vrule width \EndNoteRuleWidth \hsize  height \EndNoteRuleThickness
    \EndNoteRuleRightIndent}%
  \kern-\EndNoteRuleThickness %
} 
\def\zendnoterule{\noterule{\AboveEndNoteSpace}{\BelowEndNoteRuleSpace}{\hsize}{%
  \EndNoteRuleLeftIndent\vrule width \EndNoteRuleWidth\hsize height \EndNoteRuleThickness
  \EndNoteRuleRightIndent}}

\def\zpostendnoterule{\if@ndnotesfound{%
    \toks0=\everypar\everypar={}\parskip=0pt \parindent=0pt \let\par=\endgraf\parfillskip=0pt %
%  \noindent\hbox{\noindent\kern\columnshift\vbox{\hrule width \dimen0}}\par
    \par
    \dimen0=\prevdepth\ifdim \dimen0>0pt \kern-\dimen0\fi
    \vbox{\kern 0.5\baselineskip
      \EndNoteSeparator
      \kern 0.5\baselineskip}%
    \everypar=\toks0 
  }\fi%
}

% determine if a given note class is to be paragraphed
%+c_paragraphedNotes
\def\ParagraphedNotes#1{\Par@gr@phedNotes{#1\g@tndstat}}
\def\Par@gr@phedNotes#1{\TRACE{Par@gr@phedNotes #1}\x@\let\csname paranotes-#1\endcsname=1}
\def\StudyNotes#1{\x@\let\csname studynotes-#1\g@tndstat\endcsname=1}
\newif\ifp@ranotes
\newif\ifColNotes \ColNotesfalse
\def\checkp@ranotes#1{%
  \edef\tmp{#1}%
  \ifnum\ifColNotes\ifx\tmp\XrefNotes 0\else 1\fi\else 1\fi =1
    \st@dynotesfalse
    \x@\ifx\csname paranotes-#1\endcsname\relax
      \x@\ifx\csname studynotes-#1\endcsname\relax\else\st@dynotestrue\fi
      \p@ranotesfalse
    \else\p@ranotestrue\fi
  \else\p@ranotestrue\fi}
%-c_paragraphedNotes

% Notewidth for diglots depends on settings and column. 
%+c_setnotewidth
\newdimen\n@tewidth
\def\setn@tewidth#1{%
  \x@\let\x@\th@sins\csname note-no-insert-#1\endcsname
  \ifx\th@sins\relax % Normal footnote
   \ifdiglot
     \trace{d}{setn@tewidth (\show@dstat)}%
     \ifdiglotSepNotes
        \ifnum 1=\ifx\c@rrdstat\empty 0\else \ifcsname column\c@rrdstat width\endcsname 1 \else 0\fi\fi
          \n@tewidth=\csname column\c@rrdstat width\endcsname
        \else
          \trace{d}{c@rrdstat currently set to '\c@rrdstat'.  No column width}%
          \n@tewidth=\csname columnLwidth\endcsname
        \fi
     \else\n@tewidth=\textwidth\fi
   \else
     \ifst@dynotes\global\n@tewidth=0.5\textwidth
       \dimen9=\StudyGutterFactor\FontSizeUnit
       \advance\n@tewidth by -0.5\dimen9
     \else
       \ifColNotes\ifnum\c@rrentcols=2\global\n@tewidth=\colwidth\else\onecolwidth{\n@tewidth}\fi
       \else\n@tewidth=\textwidth\fi
     \fi
   \fi
  \else
    %Assume that the note will be inserted using the current text width.
    \ifnum\c@rrentcols=1 \onecolwidth{\n@tewidth}\else
      \n@tewidth=\ifcsname endn@te-#1\endcsname \hsize \else
        \ifdiglot\csname column\c@rrdstat width\endcsname
        \else\ifColNotes\colwidth\else\textwidth\fi
        \fi
      \fi
    \fi
    %\advance\n@tewidth by -\columnshift
  \fi
  \trace{f}{setn@tewidth #1: \the\n@tewidth}%
}
%-c_setnotewidth

% reformat the contents of a note class insertion into a single paragraph.
% this is usually done for \x. It is sometimes done for \f.
% (based on code from the TeXbook, appendix D)
% #1 is vbox containing notes as individual paragraphs.
%+c_makenotepara
\newif\ifNoteTracing \NoteTracingfalse
\newskip\internoteskip \internoteskip=15pt plus 12pt minus 7pt
\newskip\noteRag\noteRag=0pt plus 36pt
\def\maken@tepara#1#2{%
  \setn@tewidth{#2}%
  \hsize=\n@tewidth%\advance\hsize by -\columnshift % width is full page size
  \let\par=\endgraf\x@\let\csname ch@pter\g@tdstat waiting\endcsname\relax
  \everypar={}% don't do body text formatting
  \ifkeepn@tes\unvcopy\else\unvbox\fi #1 % open up the vbox of notes to get at the list of individual note boxes
  \makehboxofhboxes % make a single hbox for all notes of this class
  \unskip\unskip
  \setbox0=\hbox{\unhbox0 \removehboxes}% add internote space                   %(1)
  \trace{j}{maken@tepara #2 \c@rrdstat: baseline was \the\baselineskip}%
  \s@tbaseline{#2}{#2}%
  % Enable justification according to marker.
  \trace{j}{maken@tepara #2 \c@rrdstat: baseline now \the\baselineskip}%
  \lineskiplimit=-10pt \leftskip=0pt \rightskip=0pt \parskip=0pt \parfillskip=0pt plus 1fil%\lineskip=10pt
  \getp@ram{justification}{#2}{#2}%
  \ifx\p@ram\c@nter
     \leftskip=\noteRag\rightskip=\noteRag
  \else\ifx\p@ram\l@ft
     \rightskip=\noteRag
  \else\ifx\p@ram\r@ght
     \leftskip=\noteRag
  \fi\fi\fi
  \edef\@seglyphmetrics{\the\XeTeXuseglyphmetrics}%
  \XeTeXuseglyphmetrics=3
  \noindent % starting making new paragraph
  \ifRTL\beginR\fi % respect directionality
  \ifNoteTracing\tracingparagraphs=1\fi
  %\ifnum\pagenumber=8\showbox0\fi
  \unhbox0 % unbox the text so it can be line-wrapped
  \unskip\unpenalty\unskip\unskip % remove internote skip info after last note
  % set penalty which allows breaking between notes unless this would cause
  % an extra line to be created.
  \linepenalty50
  \trace{f}{Note #2 baselineskip=\the\baselineskip, useglyphmetrics=\the\XeTeXuseglyphmetrics}%
  \par\leftskip=0pt \XeTeXuseglyphmetrics=\@seglyphmetrics\ifNoteTracing\tracingparagraphs=0\fi}
%-c_makenotepara
% make box0 = an hbox contining all the contents of this class
%+c_makehboxofhboxes
\def\makehboxofhboxes{%
  \setbox0=\hbox{}%
  \loop\unskip\setbox2=\lastbox\unskip \ifhbox2\setbox0=\hbox{\box2\unhbox0}\repeat} 
%
% remove inside level of boxing and adding inter note space after each
%     [[a][b][c]] --> [a \internotespace b \internotespace c \internotespace]
\def\removehboxes{%
  \unskip\setbox0=\lastbox\unskip
  \ifhbox0{\removehboxes}\unhbox0\internotespace\fi}
%
% skip between notes in paragraph. skip is good place to break.
\def\internotepenalty{-10}
%\def\internotespace{\hfil\hskip\intern@teskip\penalty\internotepenalty\hfilneg}
\def\internotespace{\penalty\internotepenalty\hskip\internoteskip}%\kern 0pt after \hskip, to force gap
%\def\internotespace{\hfil\hskip\intern@teskip\penalty-10\hfilneg}

\def\r@versevb@x#1{\setbox#1=\vbox{}\loop
  \unskip\unpenalty\setbox0=\lastbox\ifdim\ht0>0pt
    \setbox#1=\vbox{\unvbox#1\box0}\unskip\unpenalty\unskip\repeat
}
\newbox\xr@fbox
\newbox\XrefB@x
\newdimen\XrefSkip \XrefSkip=0pt
\def\m@ke@te@mp#1{
  \dimen0=\ht\xr@fbox \advance\dimen0 by \baselineskip
  \advance\dimen0 by -\dimen2\advance\dimen0 by -\dp\xr@fbox
  \setbox0=\hbox{}\trace{o}{current height=\the\dimen0}%
  \ifdim\dimen0>#1\loop
    \setbox1=\vbox{\unvbox1\unskip\unpenalty\global\setbox3=\lastbox\unskip}%
    \ifdim\ht3>0pt \setbox0=\hbox{\ifRTL\beginR\fi\unhbox3\unskip\unpenalty\space\unhbox0}\repeat
    \setbox1=\box\voidb@x
  \else\temptrue\loop
    \setbox1=\vbox{\unvbox1 \unskip\unpenalty\global\setbox3=\lastbox\unskip}%
    \ifdim\ht3>0pt \setbox0=\hbox{\ifRTL\beginR\hskip-\leftskip\fi\unhbox3\unskip\unpenalty\unskip\unpenalty\space\unhbox0}%
    \else\ifvoid3\tempfalse\else\trace{o}{m@kexrefbox: Weird nonempty box3 \the\ht3+\the\dp3}\fi\fi
    \dimen1=\dimen0 \advance\dimen1 by \ht1 \advance\dimen1 by \dp1
    \advance\dimen1 by 0pt % This is redundant, but makes it work by I think separating the \fi\ifdim
    \advance\dimen1 by \baselineskip \advance\dimen1 by -\dimen2 \advance\dimen1 by \XrefSkip
    \trace{o}{\space now height is \the\dimen1}%
    \ifdim\dimen1<#1\tempfalse\fi
    \iftemp\repeat
  \fi}
\def\m@kexrefbox#1#2#3#4{
% #1: height, #2: xrefbox width, #3: note marker/style, returns vbox of remaineder and results in xr@fbox
% #4: justification (only r is significant)
  \setbox\xr@fbox=\ifXrefTopfill\vbox{\vfil}\else\box\voidb@x\fi
  \temptrue
  \let\par=\endgraf\x@\let\csname ch@pter\g@tdstat waiting\endcsname\relax
  \everypar={}% don't do body text formatting
  \x@\let\x@\th@cl@ss\csname note-#3\endcsname % make \th@cl@ss be a synonym for the current note class
  \hsize=#2%\advance\hsize-\XrefNotesMargin\advance\hsize-\XrefNotesMargin
  \if#4r\relax \leftskip=0pt plus 0.8\hsize
    \rightskip=0pt \parfillskip=0pt \else
    \leftskip=0pt \rightskip=0pt plus 0.8\hsize \parfillskip=0pt\fi
  \ifRTL\skip0=\rightskip \rightskip=\leftskip \leftskip=\skip0\fi
  \trace{o}{Inside m@kexrefbox leftskip=\the\leftskip, rightskip=\the\rightskip}%
  % Empty the xref notes so that they don't crop up elsewhere
  \global\setbox\XrefB@x=\copy\th@cl@ss
  \unvbox\th@cl@ss \r@versevb@x{3}\unvbox3
  \s@tbaseline{#3}{#3}%
  \advance\baselineskip 0pt plus 1pt minus 0.1pt
  \loop
    \setbox0=\lastbox
    \dimen2=\ht0
    \ifdim\ht0>0pt
      \dimen1=\ht0
      \setbox1=\vbox{\ifRTL\beginR\fi\unhbox0}%
      \dimen0=\ht\xr@fbox \advance\dimen0 by \ht1 \advance\dimen0 by \baselineskip
      \advance\dimen0 by -\dimen1 \advance\dimen0 by -\dp\xr@fbox
      \ifdim\dimen0>#1
        \trace{o}{Bottom of xref box aiming for \the #1 but currently \the\dimen0. Lastbox \the\ht1}%
        \tempfalse % we're done with the main loop
        \setbox0=\hbox{}%
        \m@ke@te@mp{#1}%
        \ifdim\ht0>0pt \hbox{\unhbox0\unskip\unskip\unpenalty\unskip\unpenalty}\fi
      \fi
      \dimen1=\baselineskip\advance\dimen1-\dp\xr@fbox\advance\dimen1-\dimen2\advance\dimen2\XrefSkip
      \ifdim\ht1>0pt \global\setbox\xr@fbox=\vbox{\unvbox\xr@fbox\vskip\dimen1\unvbox1}\fi
    \else \ifvoid0\else\trace{o}{Unexpected 0 height box in m@kexrefbox}\fi\tempfalse
    \fi
  \iftemp\repeat
  %\ifdim\ht\xr@fbox=0pt \message{empty xref box}\setbox\xr@fbox=\vbox{\hbox to \hsize{\space}}\fi
% completed the main box. Now capture the rest as a normal paragraphed note
  \setbox1=\vbox{}%
  \loop
    \setbox0\lastbox
    \ifdim\ht0>0pt \setbox1=\vbox{\unvbox1\box0}\repeat
  \trace{o}{m@kexrefbox(\the#1, \the#2, marker #3): othernotes=\the\ht1, dimensions of xr@fbox= \the\ht\xr@fbox x \the\wd\xr@fbox, copy \the\ht\XrefB@x}%
  \prevdepth=-10000pt \maken@tepara{1}{#3}%
}
%-c_makehboxofhboxes

% don't allow stretching between notes.

%%%%%%%%%%%%%% OUTPUT ROUTINES %%%%%%%%%%%%%% 
% default output routine is single-column, somewhat based on Plain TeX output routine (see TeXbook)
%+c_onecol
\global\output={\onecol}
\global\holdinginserts=1
\def\onecol{%
  \global\setbox\galley=\copy255                                        %(1)
  \trace{b}{BALANCE pagebuild: cols=1: textheight=\the\textheight\space with ht=\the\ht255 \space and partial=\the\ht\partial}%
  % tempoarily split and see if there are marks in this text
  \bgroup\setbox0=\copy255 \setbox1=\vsplit0 to \maxdimen\egroup        %(+)
  \edef\t@mp{\splitbotmark}%
  \ifx\t@mp\empty\trace{H}{no marks found in onecol}\else\global\m@rksonpagetrue\trace{H}{Found mark \splitbotmark}\fi
  \global\galleypenalty=\outputpenalty                                  %(+)
  \s@ttrialheight
  \global\output={\onecoltrial}%
  % No marks on the page, and it didn't fit, so add a blank mark
  \ifm@rksonpage\else\gdef\p@gefirstmark{}\trace{H}{No marks found. Setting empty mark}\fi
  \global\holdinginserts=0
  \unvbox255
  \penalty\ifnum\outputpenalty=10000 0 \else \outputpenalty \fi         %(+)
}
%-c_onecol
%+c_onecoltrial_intro
\def\testXrefSideL{\xdef\tmp{0}\if\XrefNotes\relax\else\ifb@dy
  \ifnum\pageno<1\else
    \ifnum\XrefSide=1\xdef\tmp{1}\else
      \ifnum\XrefSide=3\ifodd\pageno \xdef\tmp{1}\fi
      \else\ifnum\XrefSide=4
        \ifodd\pageno\else \xdef\tmp{1}\fi
  \fi\fi\fi\fi\fi\fi}
\def\testXrefSideR{\xdef\tmp{0}\if\XrefNotes\relax\else\ifb@dy
  \ifnum\pageno<1\else
    \ifnum\XrefSide=2 \xdef\tmp{1}\else
      \ifnum\XrefSide=4\ifodd\pageno \xdef\tmp{1}\fi
      \else\ifnum\XrefSide=3
        \ifodd\pageno\else \xdef\tmp{1}\fi
  \fi\fi\fi\fi\fi\fi}

\def\p@geendcontent{%
  \traceifset{p@geendcontent}%
  \ifdim\ht\bottomins>0pt \vfil\kern-\lastd@pth
    \lastd@pth=0pt \vskip\skip\bottomins \hbox{\hbox to \columnshift{}\box\bottomins} \fi % ouput bottom spanning pictures
  \trace{b}{BALANCE pageouttxt: notes=\the\ht\n@tesbox , \the\dp\n@tesbox. Lastdepth=\the\lastd@pth \space Leaving \the\ht255, \the\dp255}%
  \unvbox\n@tesbox %\kern-\lastd@pth\lastd@pth=\dp2\unvbox2
  \noteseenfalse
  \trace{o}{verybottomins(onecoltrial): height=\the\ht\verybottomins , availht=\the\availht, textheight=\the\textheight, htpartial=\the\dimen9}
  \ifdim\ht\verybottomins>0pt \ifdim\availht > 0pt \vfil % \kern-\dimen0
    \lastd@pth=0pt \vskip\skip\verybottomins \hbox{\hbox to \columnshift{}\vbox{\unvbox\verybottomins}}%
  \fi\fi
  \traceifcheck{p@geendcontent}%
}
\def\onecoltrial{% single-column version of \twocoltrial (see below)
  \traceifset{onecoltrial}%
  \tracingparagraphs=0
  \vfuzz=\PaperHeight
  %\tracingall=1\tracingoutput=0\tracingpages=0\tracingparagraphs=0\tracingassigns=0\tracingscantokens=0
  \if\XrefNotes\relax\else
    \x@\let\x@\th@cl@ss\csname note-\XrefNotes\endcsname % make \th@cl@ss be a synonym for the current note class
    \setbox13=\copy\th@cl@ss
  \fi
  \pr@pinserts
  \s@ttrialheight
  \setbox14=\makefootbox
  \ifnoinkinmargin \advance\trialheight -\ht14\else\setbox14=\vbox to 0pt{\box14\vss}\fi
  \trace{i}{1c TRIAL with ht=\the\trialheight, depth=\the\dp255, vsize=\the\vsize, hIns=\the\holdinginserts, vfuzz=\the\vfuzz}%
  \keepn@testrue\c@lcavailht\keepn@tesfalse
  \ifColNotes\advance\availht by -\ht\coln@tebox\fi
  \setbox\s@vedpage=\copy255
  % split the galley to the actual size available
  \ifdim\availht<0pt                                                        %(2)
    \MSG{Page overfull with inserts. Perhaps a little more text and less pictures would help}%
  \fi
  %\setbox\colA=\vsplit255 to \availht
  % Marginal verses (and other things with 0 height vadjusts) need us to split to avail+lastdepth
  \setbox\colA=\vsplit255 to \dimexpr \availht + \lastdepth\relax
  \if\splitbotmark\relax\else\xdef\p@gebotmark{\splitbotmark}\fi
  \setbox\colA\vbox{\unvbox\colA\ifColNotes\vfil\box\coln@tebox\fi}%
  \ifvoid255 \fitonpagetrue\else\fitonpagefalse\fi
  \decr{\availht}{\colA}
%-c_onecoltrial_intro
%+c_onecoltrial_pagecontents
  \iffitonpage
    \dimen3=\ht\colA \advance\dimen3\lastd@pth
    \if\XrefNotes\relax\else
      \trace{o}{enter reheightcolnotes, colA=\the\ht\colA, for colnotes \the\ht13}%
      \reheightcolnotess{13}%
      \trace{o}{exit reheightcolnotes, colA=\the\ht\colA, xr@fbox=\the\ht\xr@fbox}%
      \ifdim\ht\xr@fbox>\dimen3 \dimen3=\ht\xr@fbox\fi
    \fi
    \trace{i}{1 SUCCEEDED, shipping page hIns=\the\holdinginserts}%
    \trace{o}{onecoltrial width=\the\wd\colA, hsize=\the\hsize, height=\the\ht\colA, partial=\the\ht\partial, verybottomins=\the\ht\verybottomins, botmark=\splitbotmark}%
    \def\pagecontents{%                                                     %(3)
      \dimen1=\textwidth% \advance\dimen1 -\ExtraRMargin
      \trace{b}{BALANCE pageoutins: cols=1: text=\the\ht\colA, \the\dp\colA: partial=\the\ht\partial, \the\dp\partial: topins=\the\ht\topins: bottomins=\the\ht\bottomins, width=\the\dimen1}%
      \dimen9=\ht\partial
      \ifvoid\partial\else \vbox{%      
          \ifpartialfr@med\else
            \doTextB@rder{\partial}%2HERE
          \fi
          \hbox to \dimen1{\vbox{\unvbox\partial}}} \fi
      \ifvoid\topins\else \vbox{\hbox to \columnshift{}\box\topins} \vskip\skip\topins \fi
      \setbox1=\vbox{\unvcopy\colA\unskip}%
      \iflastpage\setbox\colA\vbox{\unvbox\colA}\fi
      \lastd@pth=\dp\colA
      \ifColNotes\ifdim\dp\xr@fbox>\lastd@pth \lastd@pth=\dp\xr@fbox\fi\fi
      \locs@startstop{\colA}{1}%
      \trace{o}{colwidth=\the\wd\colA}%
      %\trace{o}{onecoltrial: dimen=\the\dimen3, lastd@pth=\the\lastd@pth}%
      \ifColNotes\ifdim\ht\xr@fbox>\dimen3 \dimen3=\ht\xr@fbox\fi\fi
      \hbox to \dimen1{\noindent\ifb@dy\testXrefSideL\ifnum\tmp=1
          \makecolumngutter{\the\dimen3}{\the\dimen3}{\the\lastd@pth}{\the\dimen3}{2}%
        \fi\fi
        \ifb@dy\iftob@dy\else\lshiftc@lumn{\the\pageno}\fi\fi
        \vbox to\dimen3{%
          \ifdim\ht\colA>1\onel@neunit
            \doTextB@rder{\colA}%3
          \fi
          \box\colA\kern0pt \vfil}%
        \ifb@dy\testXrefSideR\ifnum\tmp=1
          \trace{o}{onecoltrial rcolgutter makecolumngutter}%
          \iftob@dy\else\rshiftc@lumn{\the\pageno}\fi
          \makecolumngutter{\the\dimen3}{\the\dimen3}{\the\lastd@pth}{\the\dimen3}{1}%
        \fi\fi
        \hbox to \ExtraRMargin{}}%
        \p@geendcontent
        \box14
    }%
%-c_onecoltrial_pagecontents
%+c_onecoltrial_pagefit
    \ifnum\ifdim\ht\colA>\baselineskip 1\else\ifdim\ht\partial>0pt 1\else 0\fi\fi =1
      \plainoutput\trace{p}{plainoutput from onecoltrial}%    %(+)
      \notst@dynotesfalse\clearn@tes
      \s@tpage
    \else\setbox0=\box255\deadcycles=0
      \notst@dynotesfalse\clearn@tes
    \fi % dump empty pages (typically at end)
    \resetvsize
    \fin@lverybottom
    \global\holdinginserts=1
    \ifrerunsavepartialpaged\trace{o}{onecoltrial: rerunsavepartialpage}%
      \global\holdinginserts=0
      \global\output={\savepartialpage}\global\rerunsavepartialpagedfalse\unvbox255% leave flag set to allow caller to trigger extra \eject   %(+)
    \else
      \global\holdinginserts=1
      \global\output={\onecol}\unvbox255\fi
%-c_onecoltrial_pagefit
%+c_onecoltrial_fail
  \else % the contents of the "galley" didn't fit into the actual page,
        % so reduce \vsize and try again with an earlier break
    \trace{i}{1c REDUCING VSIZE \the\vsize \space by -\the\baselineskip, hIns=\the\holdinginserts}%
    \global\advance\vsize by -\baselineskip
    \trace{b}{BALANCE 1col go round vsize=\the\vsize}%
    \res@tpage
    \global\let\whichtrial=\onecoltrial
    \global\output={\backingup}%
    \unvbox\galley \penalty\ifnum\galleypenalty=10000 0 \else \galleypenalty \fi
  \fi
  \traceifcheck{onecoltrial}%
}
%-c_onecoltrial_fail

%+c_pagebody
\def\pagebody{\vbox to\textheight{\boxmaxdepth\maxdepth \pagecontents}\box14}
\def\makeheadline{%
  \s@tfont{h}{h}%
  %\nomsg@tfontname{h}%
  %\trace{h}{makeheadline: textwidth \the\textwidth}%
  \vbox to 0pt{\kern-\topm@rgin
   \vbox{\kern\HeaderPosition\MarginUnit
    \setbox0=\vbox{\hbox to \textwidth{\the\headline}}%
    \ht0=0.7\fontdimen6\csname font<\f@ntstyle>\endcsname \dp0=0pt \box0
    \ifrhr@le\ifdim\RHruleposition=\maxdimen\else
      \kern\RHruleposition\leftskip=0pt\rightskip=0pt
      \begingroup\everypar={}\noindent
        \dimen1=\textwidth \setbox1=\hbox{\lshiftc@lumn{\ifnum\BodyColumns=2 \ifRTL 1\else 0\fi\else\the\pageno\fi}}\advance\dimen1 by -\wd1
        \setbox1=\hbox{\rshiftc@lumn{\ifnum\BodyColumns=2 \ifRTL 0\else 1\fi\else\the\pageno\fi}}\advance\dimen1 by -\wd1
        \trace{h}{hrule width=\the\dimen1 \space from \the\textwidth, rshift=\the\wd1, cols=\the\BodyColumns}%
        \hbox{\lshiftc@lumn{\ifnum\BodyColumns=2 \ifRTL 1\else 0\fi\else\the\pageno\fi}\vrule width \dimen1 height 0.4pt}%
      \endgroup
    \fi\fi}\vss}\nointerlineskip}
\newif\ifrhr@le
\def\makefootline{%
  \vbox to 0pt{\dimen0=\textwidth\advance\dimen0 by -\columnshift
    \kern\bottomm@rgin
    \kern-\FooterPosition\MarginUnit
     \hbox to \textwidth{\the\footline}\vss}}
\newdimen\RHruleposition \RHruleposition=\maxdimen
\newif\ifnoinkinmargin \noinkinmargintrue
\def\makefootbox{%
  \vbox{
    \setbox8=\hbox{\the\footline}%
    \ifdim\ht8>0pt
      \dimen0=\textwidth\advance\dimen0 by -\columnshift
      %\dimen1=\bottomm@rgin\advance\dimen1 by -\FooterPosition\MarginUnit
      \dimen1=\FooterPosition\MarginUnit
      %\advance\dimen1 by -\ht8
      \prevdepth=-1000pt % no interline glue
      \kern\dimen1
      \hbox to \textwidth{\unhbox8}%
    \else
      \box\voidb@x
    \fi
  }
}
%-c_pagebody

%+c_doublecolumns
\def\PageFullFactor{0.9}
\gdef\layoutstylebreak@doublecolumn{\trace{o}{LAYOUTSTYLE break (double col)}\global\output={\savepartialpagedbounce}\par\eject\eject}
\def\doublecolumns{
  \traceifset{doublecolumns}%
  \ifnum\c@rrentcols=1
    % make switch from single to double column
    \vskip-\lastdepth
    \ifhe@dings\endhe@dings\fi % if headings in process, end headings
    \penalty-100\vskip\baselineskip % ensure blank line between single and double column material
    \holdinginserts=0
    \global\output={\savepartialpage}\eject
    \holdinginserts=1
    \layoutstylebreak
    \global\let\layoutstylebreak\layoutstylebreak@doublecolumn
	% if single column material already fills 3/4 page, go to next page to start double columns
    \dimen0=\ht\partial
    \trace{o}{DoubleColumns: partial=\the\dimen0\space > \PageFullFactor\space * \the\textheight}%
    \ifdim\dimen0>\PageFullFactor\textheight
      \global\output={\onecol} % but output it immediately if 75% full
      \loop\ifdim\dimen0>\textheight \advance\dimen0-\textheight\repeat
      \unvbox\partial
      \ifdim\dimen0>\PageFullFactor\textheight \vfill\eject
      \else \holdinginserts=0 \global\output={\savepartialpage}\eject \holdinginserts=1
      \fi
    \fi
    % reset parameters for 2-column formatting
    \global\hsize=\colwidth\trace{o}{doublecolumn vsize doubles}%
    \global\vsize=2\textheight % in 2 col mode you can put twice the height of text
	\global\advance\vsize by -2\ht\partial % subtract height of 1 column material
	\global\advance\vsize by 2\baselineskip % don't get caught short by 1/2 line or so
	% make a macro reset vsize for remaining pages which do not have 1 column material
    \trace{o}{Going to 2 col, textheight \the\textheight - \the\ht\partial}%
    \gdef\resetvsize{\global\vsize=2\textheight \global\advance\vsize by \baselineskip\trace{o}{resetvsize2 vsize=\the\vsize}} 
    \global\output={\twocols}%
    \global\c@rrentcols=2
    % top and bottom inserts effectively use twice their height
	% (\count of an insertion class is a scaling factor)
    \count255=2000
    \global\count\topins=\count255
    \global\count\bottomins=\count255
    \global\count\verybottomins=100%\count255
    \let\\=\s@tn@tec@unt \the\n@tecl@sses % reset \count for each note class
    \global\holdinginserts=1 % don't pull out inserts yet, we are still adjusting page
  \fi\traceifcheck{doublecolumns}}
%-c_doublecolumns

%+c_setnotecount
% iterate over note classes, set \count for each class  
\def\s@tn@tec@unt#1{%
  \x@\let\x@\th@cl@ss\csname note-#1\endcsname
  \checkp@ranotes{#1}%
  % notes made into paragraphs (e.g. \x) are counted as 0 height for now,
  % later we will backup making page size smaller until things fit
  % Diglot prefers to get things closer 1st time round...
  %\global\count\th@cl@ss=\ifp@ranotes \ifdiglot 500\else 0\fi \else 0\fi %\count255 \fi
  \global\count\th@cl@ss=\ifdiglot 100\else 0\fi
  \global\skip\th@cl@ss=0pt} %\AboveNoteSpace }
%-c_setnotecount

%+c_resetvsize
\def\resetvsize{\global\vsize=\textheight\trace{o}{resetvsize vsize=\the\vsize}} 
%-c_resetvsize

\def\msg#1{\immediate\write16{#1}}

%+c_savepartialpage
\newif\ifMidPageFootnotes % Should footnotes go before a single-double column transition
\MidPageFootnotesfalse

\def\reheightcolnotess#1{%
  \trace{i}{height=\the\dimen3, xr@fbox=\the\ht\xr@fbox, baselineskip=\the\baselineskip}%\tracingassigns=1\tracingmacros=1\tracingifs=1
  \traceifset{reheightcolnotes}%
  \ifdim\ht\xr@fbox>\dimen3 % reduce height of notes for smaller page
    \keepn@testrue\dimen4=\availht
    \count255=10
    \@LOOP
      \setbox\th@cl@ss=\copy#1
      \bestavailht=\dimen3
      \dimen5=\ht\xr@fbox \advance\dimen5 -\dimen3
      \temptrue\ifdim\dimen5<0pt \ifdim\dimen5>-\baselineskip \tempfalse\fi\fi
      \ifdim\dimen5>2000pt \tempfalse\fi
      \trace{o}{reheighting \the\ht\th@cl@ss, diff=\the\dimen5, will reheight=\iftemp true\else false\fi}%
      \iftemp
        \trace{o}{Retrying colnotes \the\ht\th@cl@ss, at \the\bestavailht, versus \the\availht, from=\the\dimen3, diff=\the\dimen5, baselineskip=\the\baselineskip}%
        \c@lcavailht
        \setbox255=\copy\s@vedpage
        \global\setbox\colA=\vsplit255 to \availht
        \ifColNotes\global\setbox\colA\vbox{\unvbox\colA\vfil\copy\coln@tebox}\fi
        \ifnum \ifvoid255 \ifdim\ht\colA<\availht 1\else 0\fi \else 0\fi =1
          \global\dimen3=\ht\colA
          \dimen6=\ht\xr@fbox\advance\dimen6 by -\dimen3
          \ifdim\dimen6<\dimen5 \ifdim-\dimen6>-\dimen5 \temptrue\else\tempfalse \fi\else\tempfalse \fi
          %\ifdim\dimen3>\dimen4\tempfalse\fi
        \else\tempfalse \fi
        \iftemp\else
          \setbox\th@cl@ss=\copy#1
          \bestavailht=\dimen3 \c@lcavailht
          \setbox255=\copy\s@vedpage
          \global\setbox\colA=\vsplit255 to \availht
          \ifColNotes\global\setbox\colA\vbox{\unvbox\colA\vfil\copy\coln@tebox}\fi
        \fi
        \global\dimen3=\ht\colA\advance\dimen3 by \dp\colA
        \trace{o}{New availht(1)=\the\availht, measured height=\the\dimen3, olddiff=\the\dimen5, newdiff=\the\dimen6, limit=\the\dimen4}%
      \fi
      \advance\count255 by -1
      \ifnum\count255=0 \tempfalse\fi
    \iftemp
      \multiply\dimen5 by 32
      \divide\dimen5 by \hsize\multiply\dimen5 by \XrefNotesWidth \divide\dimen5 by 32
      \ifdim\dimen5<\baselineskip \dimen5=\baselineskip\fi
      \advance\bestavailht by \dimen5
      \@REPEAT
  \fi%\tracingassigns=0\tracingmacros=0\tracingifs=0
  \traceifcheck{reheightcolnotes}%
}
\newdimen\t@xttrialheight
\def\savepartialpage{% save a partially-full page when switching to 2-column format,
  \xdef\p@gebotmark{\botmark}%
  \trace{o}{savepartialpage: hIns=\the\holdinginserts\space partial=\the\ht\partial}%
  \if\XrefNotes\relax\else
    \x@\let\x@\th@cl@ss\csname note-\XrefNotes\endcsname % make \th@cl@ss be a synonym for the current note class
    \setbox13=\copy\th@cl@ss
  \fi
  \s@ttrialheight
  \setbox14=\makefootbox\ifnoinkinmargin\advance\trialheight by -\ht14\else\setbox14=\vbox to 0pt{\box14\vss}\fi
  \trace{o}{savepartialpage: b@dy\ifb@dy true\else false\fi\space for trialheight \the\trialheight}%
  \setbox0=\vbox{\unvcopy255}%
  %\global\t@xttrialheight=\ht0
  \keepn@testrue\c@lcavailht\keepn@tesfalse
  \s@veallnotes{1}%
  \setbox\s@vedpage=\copy255
  % split the galley to the actual size available
  \setbox\colA=\vsplit255 to \availht
  \ifColNotes\setbox\colA\vbox{\unvbox\colA\vfil\copy\coln@tebox}\else\setbox\colA\vbox{\unvbox\colA}\fi
  \edef\t@mp{\splitbotmark}%
  \ifx\t@mp\empty\else
    \ifm@rksonpage\else
      \edef\prev@avh{\the\availht}%
      \global\m@rksonpagetrue\trace{H}{Found mark \splitbotmark}%
      \ifdim\dimexpr\pretextb@rderskip + \posttextb@rderskip\relax=0pt \else
        \trace{eb}{Bother. Need to remeasure and resplit the page because of the new-found border}%
        \keepn@testrue\c@lcavailht\keepn@tesfalse
        \setbox255\copy\s@vedpage
        \setbox\colA=\vsplit255 to \availht
        \ifColNotes\setbox\colA\vbox{\unvbox\colA\vfil\copy\coln@tebox}\else\setbox\colA\vbox{\unvbox\colA}\fi
      \fi
    \fi
    \if\splitbotmark\relax\else\xdef\p@gebotmark{\splitbotmark}\fi
  \fi
  \dimen1=\textwidth\advance\dimen1 by -\ExtraRMargin\ifb@dy\iftob@dy\else\advance\dimen1 by -\columnshift\fi\fi
  \dimen3=\ht\colA\advance\dimen3 by \dp\colA
  \if\XrefNotes\relax\else
    \reheightcolnotess{13}%
    \ifdim\ht\xr@fbox>\dimen3 \dimen3=\ht\xr@fbox\fi
  \fi
  % and check if it all fit; if not, we'll have to back up and try again
  \trace{o}{savepartialpage \the\dimen3 \space rem=\the\ht255}%
  \ifvoid255 \fitonpagetrue \else \fitonpagefalse \fi
  \iffitonpage
    \locs@startstop{\colA}{1}%
    \global\setbox\partial=\vbox{%
      \ifvoid\partial\else \vbox{\hbox to \dimen1{%
        \vbox{\unvbox\partial}}} \fi
      \ifvoid\topins\else \unvbox\topins \vskip\skip\topins \fi
      \lastd@pth=\dp\colA
      \ifColNotes\ifdim\dp\xr@fbox<\lastd@pth \lastd@pth=\dp\xr@fbox\fi\fi
      \dimen1=\textwidth \advance\dimen1 -\ExtraRMargin \ifb@dy\iftob@dy\else\advance\dimen1 by -\columnshift\fi\fi
      \hbox to \dimen1{\noindent
        \ifb@dy\testXrefSideL\ifnum\tmp=1
          \makecolumngutter{\the\dimen3}{\the\dimen3}{\the\lastd@pth}{\the\dimen3}{2}%
        \fi\fi
        \ifb@dy\iftob@dy\else\lshiftc@lumn{\the\pageno}\fi\fi
        %\special{"testme1"}
        \vbox to \dimen3{\unvbox\colA\vfil}%\special{"testme2"}%
        \ifb@dy\testXrefSideR\ifnum\tmp=1
          \iftob@dy\else\rshiftc@lumn{\the\pageno}\fi
          \makecolumngutter{\the\dimen3}{\the\dimen3}{\the\lastd@pth}{\the\dimen3}{1}%
        \fi\fi
        \hbox to \ExtraRMargin{}}%
      \ifvoid\bottomins\else%
        \kern-\lastd@pth \lastd@pth=0pt \vskip\skip\bottomins \unvbox\bottomins \fi
      \ifMidPageFootnotes
        %\f@rstnotetrue\m@kenotebox
        \trace{b}{BALANCE pageouttxt: notes=\the\ht2 , \the\dp2}%
        \unvbox\n@tesbox %\kern-\lastd@pth\lastdepth=\dp2\unvbox2
        \ifdim\ht\verybottomins>0pt \ifdim\availht > 0pt %
          \trace{o}{verybottomins(savepartialpage): height=\the\ht\verybottomins , availht=\the\availht, textheight=\the\textheight}
          \kern-\lastd@pth \lastd@pth=0pt \vskip\skip\verybottomins \unvbox\verybottomins
        \fi\fi
        \iff@rstnote\else \vskip\baselineskip \fi
        \s@veallnotes{1}% save the state after notes are used.
      \fi
    }%
    \ifdim\ht\partial>0.5\baselineskip
      \dimen1=\ht\partial %\advance\dimen1\dp\partial
      \dimen0=\baselineskip
      \m@d\advance\dimen1 by -\ht\partial\ifdim\dimen1 < -0.01pt
        \dimen1=\ht\partial \advance\dimen1\dimen0
        \trace{o}{Adjusting single partial ht from \the\ht\partial to \the\dimen1 with baselineskip=\the\baselineskip}%
        \global\ht\partial=\dimen1
      \fi
    \fi
    \ifColNotes\notst@dynotestrue\clearn@tes\notst@dynotesfalse\fi
    \resetvsize
    \ifdim\ht\colA>\baselineskip\plainoutput\fin@lverybottom\trace{p}{plainoutput of finalverybottom from savepartialpage}\s@tpage
    \else\trace{o}{Page empty cols=\the\ht\colA, \the\ht\colB, partial=\the\ht\partial}%
      \setbox0=\box255\deadcycles=0
    \fi % dump empty pages (typically at end)
    \tempfalse%
    \ifdim\ht\partial>\PageFullFactor\textheight \temptrue\fi%
    \trace{o}{outputpenalty = \the\outputpenalty}%
    \ifnum\outputpenalty<-10000 \temptrue\fi
    \iftemp\trace{o}{savepartial eject partial=\the\ht\partial, bottom=\the\ht\verybottomins}%
      \def\pagecontents{%
        \doTextB@rder{\partial}%
        \box\partial
        \p@geendcontent
        \ifnoinkinmargin\box14\fi
      }\plainoutput\trace{p}{plainoutput for large partial in savepartialpage}\s@tpage \clearn@tes
    \fi
    \fin@lverybottom
    \global\holdinginserts=1
  \else % the contents of the "galley" didn't fit into the actual page,
        % so reduce \vsize and try again with an earlier break
    \trace{i}{2c REDUCING VSIZE hIns=\the\holdinginserts}%
    \res@tpage
    %\global\advance\vsize by -\baselineskip
    \global\let\whichtrial=\onecoltrial
    \global\output={\backingup}%
    \global\rerunsavepartialpagedtrue
    \ifm@rksonpage\else\gdef\p@gefirstmark{}\trace{H}{No marks found. Setting empty mark}\fi%No marks on the page, and it didn't fit, so add a blank mark
    \unvbox\s@vedpage \penalty\ifnum\galleypenalty=10000 0 \else \galleypenalty \fi
  \fi
  \global\t@xttrialheight=0pt
}%
  % save 1 column material since we are switching to 2 columns
\newbox\partial
%-c_savepartialpage

%+c_twocols
\def\twocols{% primary output routine in 2-col mode
  \trace{i}{TWOCOLS @ \ch@pter:\v@rse, txtht=\the\textheight, partial=\the\ht\partial, hIns=\the\holdinginserts}%
  % save copy of current page so we can retry with different heights
  \trace{b}{BALANCE pagebuild: cols=2: textheight=\the\textheight}%
  \global\setbox\galley=\copy255
  \bgroup\setbox0=\copy255 \setbox1=\vsplit0 to \maxdimen\egroup
  \edef\t@mp{\splitbotmark}%
  \ifx\t@mp\empty\else\global\m@rksonpagetrue\trace{H}{Found mark \splitbotmark}\fi
  \global\galleypenalty=\outputpenalty % save current penalty so we can restore it at (A)
  \s@ttrialheight
  \global\output={\twocoltrial}%
  \global\holdinginserts=0 % when doing trial place insertions into boxes
  \unvbox255 % force invoking \twocoltrial
  \penalty\ifnum\outputpenalty=10000 0 \else \outputpenalty \fi % (A) restore output penalty
  }
\newbox\galley
\newcount\galleypenalty
\newdimen\trialheight
\newdimen\lastd@pth
%-c_twocols

% for measuring the space needed for each class of notes;
% this will decrease \availht by the space needed for the given class
%+c_reduceavailht
\def\reduceavailht#1{%
  \checkp@ranotes{#1}%
  %\TRACE{av: \the\availht =>}
  \ifp@ranotes\let\n@xt=\reduceavailht@para
    \ifdiglot\ifdiglotSepNotes\ifdiglotBalNotes\let\n@xt=\reduceavailht@para@bal\fi\fi\fi
  \else\let\n@xt=\reduceavailht@sep
    \ifdiglot\ifdiglotSepNotes\ifdiglotBalNotes\let\n@xt=\reduceavailht@sep@bal\fi\fi\fi
  \fi
  \iff@rstnote\else\global\noteseentrue\fi\n@xt{#1}}
%-c_reduceavailht

%+c_reduceavailht_para
\def\reduceavailht@para#1{
  \def\cl@sss{#1\g@tndstat}%
  \trace{n}{class \cl@sss}%
  \x@\let\x@\th@cl@ss\csname note-\cl@sss\endcsname
  \ifvoid\th@cl@ss\trace{f}{ParaNotes[#1] empty}\else
    \setbox0=\copy\th@cl@ss
    \setbox0=\vbox{\maken@tepara{0}{#1}}%
    \global\advance\availht by -\ht0
    \global\advance\availht by -\dp0
    \iff@rstnote
      \global\advance\availht by -\AboveNoteSpace
      \f@rstnotefalse
      \trace{f}{ParaNotes[#1] cost \the\ht0+\the\dp0 + \the\AboveNoteSpace + \BelowFootNoteRuleSpace}%
    \else
      \global\advance\availht by -\InterNoteSpace
      \trace{f}{ParaNotes[#1] cost \the\ht0+\the\dp0 + \the\InterNoteSpace}%
    \fi
    %\getp@ram{baseline}{#1}\ifx\p@ram\relax\else\advance\availht-\p@ram\fi
  \fi}
%-c_reduceavailht_para

\def\max@boxpara#1{%paragraph notes of type \v@lpfx#1 and find the max height amongst them
  \trace{n}{max@boxpara \v@lpfx\if #1L\else #1\fi}%
  \x@\let\x@\t@st\csname \v@lpfx\if #1L\else #1\fi\endcsname
  \ifvoid\t@st \else
    \setbox0=\copy\t@st
    \let\t@mpdstat\c@rrdstat
    \edef\c@rrdstat{#1}% make sure that maken@tepara can work with the right width
    \setbox0=\vbox{\maken@tepara{0}{\v@lpfx}}%
    \dimen0=\ht0 \advance\dimen0 by \dp0
    \ifdim\dimen1<\dimen0
      \dimen1=\dimen0
    \fi
    \let\c@rrdstat\t@mpdstat %don't forget to restore value
    \trace{n}{\the\dimen0, \the\dimen1}%
  \fi
}

\def\reduceavailht@para@bal#1{%
  \TRACE{reduceavailht@para@bal #1}%
  \bgroup%leave dimen0 etc unchanged
  \dimen0=0pt
  \edef\v@lpfx{note-#1}\edef\v@lsfx{}%
  \let\col@do=\max@boxpara\dimen1=0pt
  \x@\each@col\diglot@list\E % max@boxsz sets dimen1 to max ht+dp
  \ifdim\dimen1>0pt
    \global\advance\availht by -\dimen1
    \iff@rstnote\global\advance\availht by -\AboveNoteSpace \f@rstnotefalse
    \else \global\advance\availht by -\InterNoteSpace\fi
    \trace{n}{reduced by=\the\dimen1}%
  \fi\egroup}

%+c_reduceavailhtsep
\def\reduceavailht@sep#1{
  \def\cl@sss{#1\g@tndstat}%
  \TRACE{class \cl@sss}%
  \x@\let\x@\th@cl@ss\csname note-\cl@sss\endcsname
  \ifvoid\th@cl@ss\trace{f}{Notes[#1] empty}\else
    \f@rstnotefalse
    \global\advance\availht by -\ht\th@cl@ss
    \global\advance\availht by -\dp\th@cl@ss
    \iff@rstnote\advance\availht by -\AboveNoteSpace
      \global\advance\availht by -\BelowFootNoteSpace
      \trace{f}{Notes[#1] cost \the\ht\th@cl@ss+\the\dp\th@cl@ss + \dp\pstr@t + 2*\the\AboveNoteSpace}%
     \global\advance\availht -\dp\pstr@t \f@rstnotefalse % ensure we don't impinge on the depth
    \else
      \global\advance\availht by -\InterNoteSpace
      \trace{f}{Notes[#1] cost \the\ht\th@cl@ss0+\the\dp\th@cl@ss + \the\InterNoteSpace + \the\AboveNoteSpace}%
    \fi
  \fi}
%-c_reduceavailhtsep

\def\reduceavailht@sep@bal#1{%
  \TRACE{reduceavailht@sep@bal}%
  \edef\v@lpfx{note-#1}\edef\v@lsfx{}%
  \let\col@do=\max@boxszB\dimen1=0pt
  \x@\each@col\diglot@list\E % max@boxsz sets dimen1 to max ht+dp
  \ifdim\dimen1>0pt 
    \advance\availht by -\dimen1
    \iff@rstnote\advance\availht by -\AboveNoteSpace \f@rstnotefalse
      \advance\availht -\dp\pstr@t  % ensure we don't impinge on the depth
    \else \advance\availht by -\InterNoteSpace\fi
  \fi}

%+c_clearnoteclass

\def\cle@rn@tecl@ss#1{%
  \ifdiglot
    \ifdiglotSepNotes
       \diglotcle@rn@tecl@ss{#1}%
    \else\cle@rn@t@cl@ss{#1}\fi
  \else\cle@rn@t@cl@ss{#1}\fi
}

\def\cle@rn@t@cl@ss#1{%
  \ifnum \ifnotst@dynotes\x@\ifx\csname studynotes-#1\endcsname\relax 1\else 0\fi\else 1\fi =1
    \x@\let\x@\th@cl@ss\csname note-#1\endcsname
  \fi
  \trace{n}{cle@rn@t@cl@ss{#1}->\the\th@cl@ss}%
  \global\setbox\th@cl@ss=\box\voidb@x
}
\def\clearn@tes{\trace{o}{clearn@tes}\let\\=\cle@rn@tecl@ss \the\n@tecl@sses}

\def\s@venotes#1#2{\trace{nS}{Saving footnotes}\edef\n@tesavelevel{#2}\edef\n@t@sfx{#1}\let\\=\s@ven@te \the\n@tecl@sses}
\def\s@veallnotes#1{%
  \ifdiglot
    \ifdiglotSepNotes
      \let\sncdst@t=\c@rrdstat
      \def\col@do##1{\edef\c@rrdstat{##1}\s@venotes{\g@tndstat}{#1}}%
      \x@\each@col\diglot@list\E
      \let\c@rrdstat\sncdst@t
    \else
      \s@venotes{}{#1}%
    \fi
  \else
    \s@venotes{}{#1}%
  \fi
}
  
\def\r@storenotes#1#2{\trace{nS}{Restoring footnotes}\edef\n@tesavelevel{#2}\edef\n@t@sfx{#1}\let\\=\r@storen@te \the\n@tecl@sses}
\def\tw@{2}
\def\s@ven@te#1{% Level2 save saves to notesave1-X to notesave2-X
  \ifx\tw@\n@tesavelevel
    \x@\let\x@\tmp@note\csname notesave1-#1\n@t@sfx\endcsname
  \else
    \x@\let\x@\tmp@note\csname note-#1\n@t@sfx\endcsname
  \fi
  \x@\let\x@\tmp@save\csname notesave\n@tesavelevel-#1\n@t@sfx\endcsname
  \global\setbox\tmp@save=\copy\tmp@note
  \trace{n}{#1\n@t@sfx(\n@tesavelevel): \the\ht\tmp@save+\the\dp\tmp@save}%
}
\def\r@storen@te#1{% either notesavelevel restores to note-X
  \x@\let\x@\tmp@note\csname note-#1\n@t@sfx\endcsname
  \x@\let\x@\tmp@save\csname notesave\n@tesavelevel-#1\n@t@sfx\endcsname
  \trace{n}{#1\n@t@sfx(\n@tesavelevel): \the\ht\tmp@note+\the\dp\tmp@note -> \the\ht\tmp@save+\the\dp\tmp@save }%
  \global\setbox\tmp@note=\copy\tmp@save
}
%-c_clearnoteclass

% increment or decrement a given \dimen by the height of a given \box, unless void
% round to baselineskip assuming 0.5\baselineskip already added
%+c_incrdecr
\def\incr#1#2{\ifvoid#2\else\advance#1 by \skip#2\advance#1 by \ht#2\fi}
\def\decr#1#2{\ifvoid#2\else\advance#1 by -\skip#2\advance#1 by -\ht#2\fi}
%-c_incrdecr

% allocate some named registers for the \trial routine to use
\newbox\s@vedpage % to save the page contents for re-splitting columns
\newdimen\availht % overall available height
\newdimen\shortavail % shortened version of \availht for re-balancing loop
\newdimen\colhtA \newdimen\colhtB % dimen registers for calculating available ht for each col
\newbox\colA \newbox\colB % box registers to hold contents of the columns
\newcount\loopcount
\newif\ifrem@inder \rem@indertrue % if true will return non fitting page if unbalanced, else will allow unbalanced
\newif\ifunbalanced \unbalancedfalse% \unbalancedtrue
\newif\ifnoteseen \noteseenfalse% A trigger to help alert user that note might have gone missing
\newif\ifinextended \inextendedfalse % Test for sidebars (ptx extended)

%+c_setcolhts
\def\s@tcolhts#1{
    \colhtA=#1 \decr{\colhtA}{\topleftins}\decr{\colhtA}{\bottomleftins}%
    \colhtB=#1 \decr{\colhtB}{\toprightins}\decr{\colhtB}{\bottomrightins}%
    \ifColNotes
      \ifx\XrefNotes\empty\relax\else
        \x@\let\x@\th@cl@ss\csname note-\XrefNotes\endcsname
        \setbox\th@cl@ss=\copy\XrefB@x
      \fi
      \doColNotes{#1}\advance\colhtB by -\ht\coln@tebox
    \fi
    \ifdim\colhtA<0pt \colhtA=0pt \fi
    \ifdim\colhtB<0pt \colhtB=0pt \fi
    \trace{o}{s@tcolhts: colhtA=\the\colhtA, colhtB=\the\colhtB, toprightins=(\the\skip\toprightins, \the\ht\toprightins), bottomrightins=(\the\skip\bottomrightins, \the\ht\bottomrightins), colnotebox=\the\ht\coln@tebox}%
}
%-c_setcolhts

%+c_getcolhts
\def\g@tcolhts{
    \colhtA=\ht\colA \incr{\colhtA}{\topleftins}\incr{\colhtA}{\bottomleftins}%
    \colhtB=\ht\colB \incr{\colhtB}{\toprightins}\incr{\colhtB}{\bottomrightins}%
    %\ifColNotes\advance\colhtB by \ht\coln@tebox\fi
    \trace{o}{g@tcolhts: colhtA=\the\colhtA, colhtB=\the\colhtB}%
}
%-c_getcolhts

%+c_splitcols
\newif\ifswapcol
\def\spl@tcols#1{
    % even if \vsplit to 0pt, TeX will always pull one line from the input box over
    %\trace{o}{split params maxdepth=\the\splitmaxdepth, topskip=\the\splittopskip, colhtA=\the\colhtA, colhtB=\the\colhtB}
    \splittopskip=\topskip
    \setbox9=\copy#1
    \setbox\colA=\vsplit#1 to \colhtA
    \ifnum\badness>999999\setbox#1=\box9\setbox\colA\box\voidb@x\else\setbox\colA=\vbox{\unvbox\colA}\fi
    \setbox9=\copy#1
    \ifdim\colhtB<0pt \setbox#1=\box9\setbox\colB\box\voidb@x\else
      \setbox\colB=\vsplit#1 to \colhtB
      \ifnum\badness>999999\setbox#1=\box9\setbox\colB\box\voidb@x\else\setbox\colB=\vbox{\unvbox\colB}\fi
    \fi
    \xdef\p@gebotmark{\splitbotmark}%
    % swap boxes if either, colA is empty and colB full, or colB is empty and colA is over full.
    \ifvoid\colA\ifvoid\colB\else\ifswapcol\ifdim\colhtA<\ht\colB\else
      \trace{o}{splitcols swapping B->A}\setbox\colA=\box\colB\fi\fi\fi\fi
    \ifdim\ht\colA>\colhtA \ifdim\ht\colB<1pt \ifdim\ht\colA<\colhtB \setbox\colB=\box\colA\fi\fi\fi
}
%-c_splitcols

%+c_balanced_intro
\newdimen\availA \newdimen\availB
\newif\ifcolbfull \colbfulltrue
\newif\ifhascolnotes
\newdimen\b@lbestavail
\newdimen\b@lbestdiff
\def\BalanceThreshold{0.95}

\def\balanced{
  \vfuzz=\PaperHeight
  \setbox0=\copy\s@vedpage
  \s@tcolhts{\availht}
  \spl@tcols0                                               %(1)
  % and check if it all fit; if not, we'll have to back up and try again
  \ifvoid0 \global\fitonpagetrue \else \global\fitonpagefalse \fi
  \rebalancefalse
  \trace{o}{first \the\availht . col=\the\ht\colA, \the\colhtA . second col=\the\ht\colB, \the\colhtB . rem=\the\ht0}%
  \ifdim\ht\colA<.5\baselineskip \ifdim\ht\colB<.5\baselineskip% \message{I can't break this page!}    %(2)
    \iffitonpage\message{Abandoning ship with nothing on the page}%\loop\ifnum\currentgrouplevel>0 \egroup\repeat
    \else\setbox\colA=\box0\setbox\s@vedpage=\box\voidb@x \trace{o}{Everything to colA}% dump it all in colA and bail
  \fi\fi\fi
  \hascolnotesfalse
  \iffitonpage
    \colbfulltrue
    \g@tcolhts
    %\shortavail=\colhtB \advance\shortavail by -\ht\colB                        %(3)
    %\ifColNotes\advance\shortavail by -\ht\coln@tebox\fi
    %\advance\shortavail by \colhtA \advance\shortavail by -\ht\colA
    \dimen0=\colhtB \advance\dimen0 by -\colhtA
    \ifdim\dimen0<0pt \dimen0=-\dimen0\fi
    \ifdim\dimen0>5\baselineskip
      \shortavail=\colhtA \advance\shortavail\colhtB
      \divide\shortavail 2% \advance\shortavail by -0.5\ht\colA \advance\shortavail by -0.5\ht\colB
      \trace{o}{Rebalance trying from average difference \the\shortavail\space of
                \the\colhtA =\the\ht\colA, \the\colhtB =\the\ht\colB, colnotes=\the\ht\coln@tebox}%
      \loop\setbox0=\copy\s@vedpage
        \dimen0=\availht\advance\dimen0-\shortavail
        \s@tcolhts{\dimen0}
        \spl@tcols0
        \trace{o}{Rebalance average loop(\the\shortavail): \the\ht\colA=\the\colhtA,
                  \the\ht\colB=\the\colhtB\space remainder \the\ht0}%
        \ifdim\ht0>0pt \temptrue\else\tempfalse\fi
        \ifdim\shortavail<\baselineskip\tempfalse\fi
        \iftemp\advance\shortavail -\baselineskip\repeat
      \trace{o}{Rebalance starting with \the\colhtA=\the\ht\colA, \the\colhtB=\the\ht\colB}%
    \fi
    \g@tcolhts                                                                  %(4)
    \b@lbestdiff=\colhtA \advance\b@lbestdiff -\colhtB\ifdim\b@lbestdiff<0pt \b@lbestdiff=-\b@lbestdiff\fi
    \shortavail=\ifdim\colhtA>\colhtB \colhtA\else\colhtB\fi
    \b@lbestavail=\shortavail
    \ifunbalanced\shortavail=\availht\else\rebalancetrue\fi
    \ifColNotes\ifdim\ht\coln@tebox>0pt \hascolnotestrue\relax\fi\fi % pass back whether there are end col notes
    \advance\shortavail \baselineskip % anticipate gap reduction at start of rebalance loop
    \dimen9=\BalanceThreshold\baselineskip \advance\dimen9 0.5\baselineskip
%-c_balanced_intro
%+c_balanced_loop
    \loopcount=0
    \ifrebalance
      \loop                                                                     %(5)
        \advance\loopcount by 1
        \advance\shortavail by -\baselineskip
        \setbox0=\copy\s@vedpage
        \s@tcolhts{\shortavail}\spl@tcols0
		% if something left in box0, it didn't fit, quit loop
        \trace{o}{re-balancing cols=\the\ht\colA =\the\colhtA, \the\ht\colB =\the\colhtB, from \the\shortavail, bests: \the\b@lbestdiff @\the\b@lbestavail, rem \the\ht0}%
        \ifdim\ht0>0pt
          \trace{o}{rebalance found extra back to \the\b@lbestavail, diff=\the\b@lbestdiff \space against \the\dimen9}\rebalancefalse
          \ifrem@inder\ifdim\b@lbestdiff>\dimen9 \fitonpagefalse\fi\fi
          \shortavail=\b@lbestavail
        \fi
		% if second column longer than first column by less than .3*line height, quit loop
        \ifrebalance
          \ifdim\colhtA>0pt \ifdim\colhtB>0pt
            \g@tcolhts
            \ifColNotes\advance\colhtB\ht\coln@tebox\fi
            \dimen0=\colhtB \advance\dimen0 by -\colhtA
            \ifdim\dimen0<0pt \dimen0=-\dimen0\fi
            \trace{o}{Testing column difference of \the\dimen0 \space against threshold of \BalanceThreshold x\the\baselineskip}%
            \ifdim\dimen0<\b@lbestdiff \b@lbestdiff=\dimen0 \b@lbestavail=\shortavail \fi
            \ifdim\dimen0<0.5\baselineskip \rebalancefalse\fi
          \else\rebalancefalse\fi \else\rebalancefalse\fi
        \fi
		% give up if target size less than 2 lines (should not happen)
        \ifdim\shortavail<\baselineskip \MSG{Rebalancing bailed for short block \the\shortavail}%   %(8)
          \ifdim\b@lbestdiff>\dimen9 \fitonpagefalse\fi
          \ifdim\b@lbestdiff<\baselineskip \fitonpagefalse \shortavail=\availht
          \else
            \shortavail=\b@lbestavail\rebalancefalse
        \fi\fi
        \ifnum\loopcount>20 
          \ifrem@inder\ifdim\b@lbestdiff>\dimen9 \fitonpagefalse\fi\fi
          \shortavail=\b@lbestavail
          \rebalancefalse\MSG{Rebalancing loop count bail}\fi
        \ifrebalance\repeat
    \fi
    %\advance\shortavail by \ifdim\dp\colA>\dp\colB \dp\colA\else\dp\colB\fi     %(9)
    \s@tcolhts{\shortavail}\spl@tcols\s@vedpage\g@tcolhts
    \setbox\colA=\vbox{\unvbox\colA\unskip}%
    \setbox\colB=\vbox{\unvbox\colB\unskip}%
    \ifColNotes\advance\colhtB\ht\coln@tebox\ifdim\ht\coln@tebox>0pt \hascolnotestrue\relax
      \setbox\colB=\vbox{\unvbox\colB \vfil\unvbox\coln@tebox\unkern}
    \fi\fi
    \trace{o}{A = \the\colhtA=\the\ht\colA , B = \the\colhtB=\the\ht\colB , (from \the\shortavail) remaining = \the\ht\s@vedpage, hascolnotes\ifhascolnotes true\else false\fi}%
    \colhtA=\ht\colA \colhtB=\ht\colB
  \else
    \ifColNotes\ifrebalance
      \trace{o}{balanced: ColNotes A=\the\colhtA, \the\ht\colA, B=\the\colhtB, \the\ht\colB }
      \fitonpagetrue\setbox\s@vedpage=\box\voidb@x
    \fi\fi % come from colnotes
  \fi
}
%-c_balanced_loop

%+c_calcboxheights
\def\c@lcboxheights{%
  \g@tcolhts
  \trace{b}{BALANCE pageout: cols=2: texta=\the\ht\colA , \the\dp\colA : textb=\the\ht\colB , \the\dp\colB : partial=\the\ht\partial, \the\dp\partial : topins=\the\ht\topins+\the\skip\topins, \the\ht\topleftins+\the\skip\topleftins, \the\ht\toprightins+\the\skip\toprightins : bottomins=\the\ht\bottomins+\the\skip\bottomins, \the\ht\bottomleftins+\the\skip\bottomleftins, \the\ht\bottomrightins+\the\skip\bottomrightins}%
  \dimen9=\textheight
  \decr{\dimen9}{\partial}\decr{\dimen9}{\topins}\decr{\dimen9}{\bottomins}\decr{\dimen9}{\verybottomins}%
  \ifdim\dimen9<\baselineskip
    \message{No space for text on page!}%
  \else
    \ifdim\ht\topleftins>\dimen9
      \setbox5=\vsplit\topleftins to \dimen9
      \trace{i}{topleft split to \the\dimen9 (\the\ht\topleftins)}%
    \else
      \setbox5=\box\topleftins
    \fi
    \ifdim\ht\toprightins>\dimen9
      \setbox6=\vsplit\toprightins to \dimen9
      \trace{i}{topleft split to \the\dimen9 (\the\ht\toprightins)}%
    \else
      \setbox6=\box\toprightins
    \fi
    \dimen8=\dimen9
    \advance\dimen9 by -\ht5
    \ifdim\ht\bottomleftins>\dimen9
      \setbox7=\vsplit\bottomleftins to \dimen9
      \trace{i}{topleft split to \the\dimen9 (\the\ht\bottomleftins)}%
    \else
      \setbox7=\box\bottomleftins
    \fi
    \dimen9=\dimen8
    \advance\dimen9 by -\ht6
    \ifdim\ht\bottomrightins>\dimen9
      \setbox8=\vsplit\bottomrightins to \dimen9
      \trace{i}{topleft split to \the\dimen9 (\the\ht\bottomrightins)}%
    \else
      \setbox8=\box\bottomrightins
    \fi
    \ifdim\ht8>0pt \colbfullfalse\fi
  \fi
  \dimen9=\availht\advance\dimen9 by \bmg@p
  \trace{o}{fitting columns to \the\dimen9, coln@tes=\the\ht\xr@fbox}%
  \lastd@pth=\dp\colA \setbox\colA=\vbox to \dimen9{\unvbox5\unvbox\colA\ifdim\ht7>0pt \vfil\unvbox7\fi}\dp\colA=\lastd@pth
  \ifdim\ht\colB<1pt \setbox\colB\vbox{\noindent\par}\else
    \ifcolbfull\ifdim\lastd@pth<\dp\colB \lastd@pth=\dp\colB\fi\else\lastd@pth=\dp\colB\fi
  \fi
  \setbox\colB=\vbox to \dimen9{\unvbox6\unvbox\colB\ifdim\ht8>0pt \vfil\unvbox8\fi}\dp\colB=\lastd@pth
  \ifdim\colhtA>\colhtB \dimen3=\colhtA
    \ifcolbfull\lastd@pth=\dp\colA\dp\colB=\lastd@pth\fi
  \else
    \dimen3=\colhtB\lastd@pth=\dp\colB\dp\colA=\lastd@pth
  \fi
  \advance\dimen3 by \lastd@pth
}
%-c_calcboxheights

\def\@mptyinserts{%
  \trace{i}{Emptying Inserts}%
  \global\setbox\topins=\box\voidb@x
  \global\setbox\bottomins=\box\voidb@x
  \global\setbox\topleftins=\box\voidb@x
  \global\setbox\toprightins=\box\voidb@x
  \global\setbox\bottomleftins=\box\voidb@x
  \global\setbox\bottomrightins=\box\voidb@x
  \global\setbox\verybottomins=\box\voidb@x
}
\def\res@tpage{%Restore page to post-setup state
  \trace{i}{Restoring inserts}%
  \r@storeinserts{chunk}%
  \r@storenotes{}{1}%
  \global\setbox255=\box\voidb@x
  \global\holdinginserts=1
  %\let\\=\cle@rn@tecl@ss \the\n@tecl@sses
}

\def\s@tpage{%Save post-shipout state / reset other stuff that needs to happen after shipout 
  \trace{i}{Saving inserts}%
  \s@veinserts{chunk}%
  \s@veallnotes{1}%
  \xdef\p@gefirstmark{}\xdef\p@gebotmark{}%
  \nextshipout
  \ifcsname page-\the\pageno\endcsname
    \csname page-\the\pageno\endcsname
  \fi
}

%+c_calcavailht
\xdef\p@gebotmark{}% No guarantee this will be universally set
\newbox\coln@tebox\newbox\n@tesbox
\def\XrefNotes{}
\def\NoXrefNotes{\gdef\XrefNotes{}%
 \advance\colwidth by 0.5\XrefNotesWidth
 \advance\colwidth by \XrefNotesMargin}

\newdimen\bestavailht \bestavailht=0pt
\def\c@lcavailht{
  \traceifset{c@lcavailht}%
  \edef\t@st{\p@gefirstmark}%
  \ifx\t@st\empty\xdef\p@gefirstmark{\firstmark}\trace{H}{Loading firstmark \p@gefirstmark}\fi% remember first, if not already set
  \edef\t@st{\p@gebotmark}%
  \ifx\t@st\empty\xdef\p@gebotmark{\botmark}\fi
  %NOT HERE! \global\trialheight=\textheight \global\advance\trialheight by -\ht\partial
  \availht=\trialheight % amount of space we think is available
  \trace{i}{C@lcavailht partial:\the\ht\partial \space available:\the\availht}%
  \trace{o}{c@lcavailht = \the\availht, depth = \the\dp\pstr@t}%
  \decr{\availht}{\topins}% and by the space needed for spanning pictures
  \decr{\availht}{\bottomins}%
  \decr{\availht}{\verybottomins}%
  \trace{i}{availht before textborderadj = \the\availht}%
  \ch@ckiftextborderadj
  \iftemp
    \advance\availht -\pretextb@rderskip  % Space reserved for text border.
    \advance\availht -\posttextb@rderskip 
  \fi
  \trace{i}{availht after textborderadj = \the\availht, pre=\pretextb@rderskip, post=\posttextb@rderskip}%
  \f@rstnotetrue
  \ifColNotes\onlyst@dynotestrue\else\onlyst@dynotesfalse\fi
  \m@kenotebox\advance\availht by -\ht2\global\setbox\n@tesbox\box2
  \onlyst@dynotesfalse
  \trace{i}{Notes: ht=\the\ht\n@tesbox, dp=\the\dp\n@tesbox}%
  %\let\\=\reduceavailht\the\n@tecl@sses % reduce it by the space needed for each note class
  \trace{i}{after inserts: \the\availht}%
  % Round to lines accurate to 1/8pt in lineskip. 1/10pt causes overflow on longer pages. Support two column legal.
  \bmg@p=\availht
  \dimen0=8\baselineskip
  \ifdim\dimen0>0.5pt
      \trace{i}{dimen0=\the\dimen0, availht=\the\availht,  lastdepth=\the\lastdepth}%
      \multiply\availht by 8 \divide\availht by\dimen0 \multiply\availht by\dimen0 \advance\availht 1023sp\divide\availht by 8
  \fi
  \advance\bmg@p by -\availht
  %\advance\availht \dp\pstr@t
  %\advance\availht by \dp255 % split includes depth so give it space for that
  \ifdim\bestavailht>0pt \ifdim\bestavailht<\availht \availht=\bestavailht\fi\fi
  \trace{o}{new availht=\the\availht, best availht=\the\bestavailht, texttrialheight=\the\t@xttrialheight, topins=\the\ht\topins, bottomins=\the\ht\bottomins, baselineskip=\the\baselineskip}%
  \doColNotes{\availht}%
  \splittopskip=\topskip
  \traceifcheck{c@lcavailht}%
}
\newif\ifXrefSideAlign
\def\doColNotes#1{
  % creates column notes in coln@tebox with side effect of perhaps xr@fbox. Uses boxes 0,2,5
  \ifColNotes\ifb@dy
    \ifx\XrefNotes\empty\relax\setbox5=\box\voidb@x\else
      \ifXrefSideAlign\testXrefSideL \ifnum\tmp=1 \edef\tmp{r}\else\edef\tmp{l}\fi\else\edef\tmp{l}\fi
      \setbox5=\vbox{\m@kexrefbox{\ifdim\t@xttrialheight>0pt \ifdim\t@xttrialheight<#1 \t@xttrialheight\else #1\fi\else #1\fi}{\XrefNotesWidth}{\XrefNotes}\tmp}%
      %\setbox\xr@fbox\vbox{\unvbox\xr@fbox\setbox0=\lastbox\dimen0=\dp0\box0\vskip\dimen0}%
    \fi
    \keepn@testrue\notst@dynotestrue\m@kenotebox\notst@dynotesfalse\keepn@tesfalse
    \setbox\coln@tebox=\vbox{%
      \trace{o}{coln@tebox in=\the\ht5+\the\dp5, notes=\the\ht2+\the\dp2}%
      \ifdim\ht2>0pt
        \unvbox2\ifdim\ht5>0pt \setbox0\lastbox\dimen0=\dp0\box0\vskip\dimen0
                  \vskip\InterNoteSpace \unvbox5\setbox0\lastbox\dimen0=\dp0\box0\vskip\dimen0\fi
      \else\ifdim\ht5>0pt
          %\ifvoid\verybottomins\vfil\fi %\kern-\lastd@pth\vfil % ignore depth of body text; fill space
          \footnoterule\prevdepth=-10000pt
          \unvbox5\setbox0\lastbox\dimen1=\ht0\dimen0=\dp0\box0\ifdim\dimen1>0pt \kern-\dimen0\fi%\vskip\bmg@p
      \fi\fi}%
    \trace{o}{coln@tebox out=\the\ht\coln@tebox+\the\dp\coln@tebox, xrefbox=\the\ht\xr@fbox+\the\dp\xr@fbox}%
  \else
    \setbox\coln@tebox=\box\voidb@x\setbox\xr@fbox=\box\voidb@x
  \fi\fi
}
%-c_calcavailht
\newif\ifXrefTopfill \XrefTopfillfalse

\def\s@ttrialheight{
  \trace{o}{s@ttrialheight \the\textheight - \the\ht\partial (\the\t@xttrialheight)}%
  \global\trialheight=\textheight \global\advance\trialheight by -\ht\partial %\advance\trialheight by -\bmg@p
}

\def\reheightcolnotesd#1{%
  \ifdim\ht\xr@fbox>\dimen3 % reduce height of notes for smaller page
    \keepn@testrue\dimen4=\availht
    \@LOOP
      \setbox\th@cl@ss=\copy#1
      \dimen3=\ifdim\colhtA>\colhtB \colhtA\else\colhtB\fi
      \bestavailht=\dimen3
      \dimen5=\ht\xr@fbox \advance\dimen5 -\dimen3 \multiply\dimen5 by 32
      \divide\dimen5 by \colwidth\multiply\dimen5 by \XrefNotesWidth \divide\dimen5 by 32
      \advance\bestavailht by \dimen5
      \temptrue\ifdim\dimen5<\baselineskip\ifdim\dimen5>-\baselineskip\tempfalse\fi\fi
      \iftemp
        \trace{o}{Retrying colnotes(2) at \the\bestavailht, versus \the\availht, diff=\the\dimen5, baselineskip=\the\baselineskip. iflevel=\the\currentiflevel}%
        \c@lcavailht
        %\trace{o}{after c@lc iflevel=\the\currentiflevel}%
        \setbox\s@vedpage=\vbox{\unvcopy255}\balanced
        %\trace{o}{after balanced iflevel=\the\currentiflevel}%
        \iffitonpage\global\dimen3=\ht\colA \ifdim\dimen3<\ht\colB \global\dimen3=\ht\colB\fi 
          \dimen6=\ht\xr@fbox\advance\dimen6 by -\dimen3
          \ifdim\dimen6<\dimen5 \temptrue\else\tempfalse\fi
          \ifdim\dimen3>\textheight \tempfalse\fi
          \ifdim\dimen6>0pt \global\advance\dimen3 by \dimen6\fi
        \else\tempfalse\fi
        \iftemp\else
          \bestavailht=\dimen3
          \c@lcavailht \global\setbox\s@vedpage=\vbox{\unvcopy255}\balanced
        \fi
        \trace{o}{New availht(2)=\the\availht, measured height=\the\dimen3, olddiff=\the\dimen5, newdiff=\the\dimen6, limit=\the\dimen4. iflevel=\the\currentiflevel}%
      \fi
    \iftemp\@REPEAT
    \c@lcboxheights\keepn@tesfalse
  \fi
}
%+c_savepartialpaged_intro
\newif\ifrerunsavepartialpaged
\def\savepartialpaged{%Double columnn version of save partial page
  \trace{o}{savepartialpaged. hIns=\the\holdinginserts, iflevel=\the\currentiflevel}%
  \traceifset{savepartialpaged}%
  \ifnum\holdinginserts=0 \else\message{INTERNAL ERROR! Foonotes/figures shouldn't be held, or they'll get lost!}\fi
  \s@ttrialheight
  \setbox14=\makefootbox\ifnoinkinmargin\advance\trialheight -\ht14\else\setbox14=\vbox to 0pt{\box14\vss}\fi
  \if\XrefNotes\relax\else
    \x@\let\x@\th@cl@ss\csname note-\XrefNotes\endcsname % make \th@cl@ss be a synonym for the current note class
    \setbox13=\copy\th@cl@ss
  \fi
  \keepn@testrue\c@lcavailht\keepn@tesfalse
  \setbox\s@vedpage=\vbox{\unvcopy255}%
  \trace{o}{balancing from savepartialpaged, height=\the\ht\s@vedpage, hIns:\the\holdinginserts, vsz=\the\vsize, av=\the\availht, op:\the\outputpenalty}%
  \ifdim\ht\partial>\baselineskip \swapcolfalse\else\swapcoltrue\fi
  \rem@inderfalse\balanced
  \ifdim\availht<\baselineskip \fitonpagefalse \fi
  \iffitonpage
    \c@lcboxheights
    \setbox\colA=\vbox{\unvbox\colA\unskip}\setbox\colB=\vbox{\unvbox\colB\unskip}%
    \s@veallnotes{1}%
    \lastd@pth=\dp\colA\ifdim\dp\colB<\lastd@pth \lastd@pth=\dp\colB\fi
    \ifColNotes\ifdim\dp\xr@fbox<\lastd@pth \lastd@pth=\dp\xr@fbox\fi\fi
    \dimen3=\colhtA
    \ifdim\dimen3>\availht \dimen3=\availht\fi
    \if\XrefNotes\relax\else
      \trace{o}{reheighting columns. xr@fbox: \the\ht\xr@fbox, colsheight: \the\dimen3. iflevel=\the\currentiflevel}%
      \reheightcolnotesd{13}\ifdim\ht\xr@fbox>\dimen3 \dimen3=\ht\xr@fbox\fi
      %\c@lcboxheights
      \setbox\colA=\vbox{\unvbox\colA\unskip}\setbox\colB=\vbox{\unvbox\colB\unskip}%
      \colhtA=\ht\colA \colhtB=\ht\colB
      \space\keepn@tesfalse
      \trace{o}{after reheightening to \the\dimen3, iflevel=\the\currentiflevel}%
    \fi
    \dimen3=\ht\colA
    \ifdim\ht\colB>\dimen3
      \setbox1=\copy\colB
      \setbox2=\vsplit1 to \dimen3
      \ifvoid1 \ht\colB=\dimen3\if\XrefNotes\relax\else\ht\xr@fbox=\dimen3\fi\else \dimen3=\ht\colB\fi
    \fi
    \ifdim\dimen3>\availht \dimen3=\availht\fi
    %\setbox255=\box\voidb@x
    %\advance\dimen3\lastd@pth
    % strip off full page height from c@lcboxheights and grab the depth
    \locs@startstop{\colA}{\colA}\locs@startstop{\colB}{\colB}%
    \trace{o}{\the\ht\colA+\the\dp\colA//\the\ht\colB+\the\dp\colB}%
    \dimen8=\dp\colA \dimen9=\dp\colB
    \ifdim\ht\colA>0pt
      \setbox\colA=\vbox to \dimen3{\unvbox\colA\unskip\setbox0\lastbox\ifvoid0\else\global\dimen8=\dp0\box0\fi}%
    \else \setbox\colA=\vbox to \dimen3{\hbox to \hsize{}}\dimen8=0pt \fi
    \ifdim\ht\colB>0pt
      \setbox\colB=\vbox to \dimen3{\unvbox\colB\unskip\setbox0\lastbox\ifvoid0\else\global\dimen9=\dp0\box0\fi}%
    \else \setbox\colB=\vbox to \dimen3{\hbox to \hsize{}}\dimen9=0pt \fi
    \trace{o}{Setting lastdepth to max of (\the\dimen8,\the\dimen9)}%
    \lastd@pth=\ifdim\dimen8>\dimen9 \dimen8\else\dimen9\fi
    \ifColNotes\ifdim\dp\xr@fbox>\lastd@pth \lastd@pth=\dp\xr@fbox\fi\fi
    % do this after we have the final boxes
    \dimen4=\ht\partial \dimen5=\dp\partial
    \global\setbox\partial=\vbox{
      \dimen1=\textwidth\advance\dimen1 by -\ExtraRMargin
      \dimen2=\availht\advance\dimen2\bmg@p
      % bodge the partial box which should really have been properly textually merged
      % the problem is that savepartialpaged can get called twice in succession because
      % not all the output text (after a \singlecolumn) was passed the first time
      \ifvoid\partial\else \vbox{\hbox to \dimen1{\noindent\lshiftc@lumn{0}\vbox{\unvbox\partial}}\vskip -2\dimen5} \fi
      \ifvoid\topins\else \hbox{\noindent\lshiftc@lumn{0}\box\topins} \vskip\skip\topins \fi % output top spanning pictures
      \dimen5=\textwidth\advance\dimen5 by -\ExtraRMargin% \advance\dimen5 by -\columnshift %
      \dimen2=\dimen3% \advance\dimen2\bmg@p
      \trace{o}{texboxheight=\the\dimen2, htA=\the\ht\colA, htB=\the\ht\colB}%
      \setbox\colA=\vbox{\hbox to \dimen5{\noindent
          \ifRTL \lshiftc@lumn{1}\hbox to \ExtraRMargin{}\vbox to \dimen2{\box\colB\vfil}\rshiftc@lumn{1}%
            \makecolumngutter{\the\dimen3}{\the\dimen3}{\the\lastd@pth}{\the\dimen2}{3}\lshiftc@lumn{0}%
            \vbox to \dimen2{\box\colA\vfil}\rshiftc@lumn{0}%
          \else \lshiftc@lumn{0}\vbox to \dimen2{\box\colA\vfil}\rshiftc@lumn{0}%
            \makecolumngutter{\the\dimen3}{\the\dimen3}{\the\lastd@pth}{\the\dimen2}{3}\lshiftc@lumn{1}%
            \vbox to \dimen2{\box\colB\vfil}\hbox to \ExtraRMargin{}\rshiftc@lumn{1}\fi}
        \kern\lastd@pth}%
      \doTextB@rder{\colA}%
      \box\colA
      }%
    \global\partialfr@medtrue
%-c_savepartialpaged_intro
%+c_savepartialpaged
    \trace{o}{sppd: saved partial, ht:\the\ht\partial\space rem:\the\ht\s@vedpage}%
    %\iflastptxfile\showbox\partial\fi
    \global\holdinginserts=1
    \global\setbox255=\box\voidb@x
    \s@ttrialheight
    \keepn@testrue\c@lcavailht\keepn@tesfalse
    \trace{o}{spdd: partial=\the\ht\partial, textheight=\the\textheight, availht=\the\availht}%
    \if\XrefNotes\relax\else\trace{o}{IAFFM}\global\setbox\th@cl@ss=\box\voidb@x\global\setbox\xr@fbox=\box\voidb@x\fi
    \ifdim\availht<2\baselineskip \twocolp@geout \clearn@tes\else
      \ifColNotes\notst@dynotestrue\clearn@tes\notst@dynotesfalse\fi
    \fi
    \unvbox\s@vedpage\penalty\ifnum\outputpenalty=10000 0 \else \outputpenalty \fi
  \else
    \trace{o}{sppd: Not fit on page}%
    \trace{i}{sppd: emptyinserts hIns=\the\holdinginserts, \the\ht\partial, vs:\the\vsize}%
    \res@tpage
    %\global\advance\vsize by -\baselineskip
    %\global\let\whichtrial=\savepartialpaged
    \global\let\whichtrial=\twocoltrial
    \global\rerunsavepartialpagedtrue
    \global\output={\backingup}%
    \tempfalse%\ifvoid\partial\else\temptrue\fi%
    % remember \holdingInserts=1 from \@emptyinserts
    \ifvoid\galley\temptrue\fi
    \ifm@rksonpage\else\gdef\p@gefirstmark{}\trace{H}{No marks found. Setting empty mark}\fi%No marks on the page, and it didn't fit, so add a blank mark
    \iftemp
     \trace{i}{Using savedpage.pen=\the\outputpenalty}%
     \unvbox\s@vedpage\penalty\ifnum\outputpenalty=10000 0 \else \outputpenalty \fi
    \else%
     \trace{i}{Using galley. pen=\the\galleypenalty}%
     \unvbox\galley\penalty\ifnum\galleypenalty=10000 0 \else \galleypenalty \fi
    \fi%
  \fi%
  \traceifcheck{savepartialpaged}%
  \trace{o}{endof savepartialpaged: iflevel=\the\currentiflevel}%
}
%-c_savepartialpaged

%+c_makenotebox
% Make a box of all the notes
\def\m@kenotebox{
  \global\setbox2=\vbox{
  \f@rstnotetrue
  \let\\=\ins@rtn@tecl@ss \the\n@tecl@sses % output all note classes
  \iff@rstnote % no notes actually occurred!
    \trace{i}{No notes}%
    \ifnoteseen\MSG{Page \the\pageno\space is being printed without any footnotes/xrefs, etc. But at least one was seen earlier. Maybe it's ended up on the previous page or moved to the next one, or maybe it's vanished. Human checking is needed.}\fi
    \vfil
  \else
    %Removed because depth is descended not true box depth
    \setbox0=\lastbox
    \trace{i}{Inserted notes, ht \the\ht0, dp \the\dp0 (\the\lastd@pth)}%
    \lastd@pth=\dp0 \dp0=0pt \prevdepth=-1000pt \box0 \kern\lastd@pth
  \fi}}%\iff@rstnote\else\showbox2\fi}
%-c_makenotebox

\def\pr@pinserts{
  \ifdim\ht\topins>0pt
    \baselineskip=\onel@neunit%\s@tbaseline{p}{p}
    \dimen1=\ht\topins\advance\dimen1\dp\topins \dimen0=\baselineskip
    \m@d \ifdim\dimen0<\baselineskip
      \dimen1=\ht\topins\advance\dimen1\dimen0 \ht\topins=\dimen1 \dp\topins=0pt
    \fi
  \fi}
  
%+c_twocoltrial_intro
\def\twocoltrial{% trial formatting to see if current contents will fit on the page
  \traceifset{twocoltrial}%
  \tracingparagraphs=0
  \if\XrefNotes\relax\else
    \x@\let\x@\th@cl@ss\csname note-\XrefNotes\endcsname % make \th@cl@ss be a synonym for the current note class
    \setbox13=\copy\th@cl@ss
  \fi
  \pr@pinserts
  \s@ttrialheight
  \setbox14=\makefootbox
  \ifnoinkinmargin\advance\trialheight -\ht14\else\setbox14=\vbox to 0pt{\box14\vss}\fi
  \keepn@testrue\c@lcavailht\keepn@tesfalse
  \ifdim\availht<0pt
    \MSG{Page overfull with inserts. Perhaps a little more text and less pictures would help}%
  \fi
  \setbox\s@vedpage=\copy255
  \trace{o}{balance from twocoltrial height \the\ht\s@vedpage, availht=\the\availht, markrange=\firstmark ,\botmark}%
  \ifdim\ht\partial>\baselineskip \swapcolfalse\else\swapcoltrue\fi
  \rem@indertrue\balanced
  \trace{o}{twocoltrial: return from balanced with fitonpage\iffitonpage true\else false\fi}
  \iffitonpage\ifvoid\s@vedpage\else\fitonpagefalse\fi
  \else\ifdim\vsize<\baselineskip \fitonpagetrue\fi % no fit so dump the page we have now and try again with a new one.
  \fi
  \trace{o}{twocoltrial: recalculate fitonpage\iffitonpage true\else false\fi}
%-c_twocoltrial_intro
%+c_twocoltrial
  \iffitonpage
    \c@lcboxheights
    \setbox1=\vbox{\unvcopy\colA\unskip}%
    \setbox2=\vbox{\unvcopy\colB\unskip}%
    \locs@startstop{\colA}{1}%
    \locs@startstop{\colB}{2}%
    \ifdim\ht\colB<1pt \setbox\colB\vbox{\noindent\par}\fi
    %\ht\colA=\the\availht \ht\colB=\the\availht                             %(1)
    \lastd@pth=\dp\colA\ifdim\dp\colB<\lastd@pth \lastd@pth=\dp\colB\fi
    \ifColNotes\ifdim\dp\xr@fbox<\lastd@pth \lastd@pth=\dp\xr@fbox\fi\fi
    \trace{p}{twocoltrial: calculate dimen3 as min(\the\availht, max(\the\ht\colA, \the\ht\colB, \the\ht\xr@fbox))}
    \if\XrefNotes\relax\else
      \reheightcolnotesd{13}\ifdim\ht\xr@fbox>\dimen3 \dimen3=\ht\xr@fbox\fi
      \c@lcboxheights
      \keepn@tesfalse
      \setbox\colA=\vbox{\unvbox\colA\unskip}\setbox\colB=\vbox{\unvbox\colB\unskip}%
    \fi
    \dimen3=\ht\colA
    \ifdim\ht\colB>\dimen3
      \setbox1=\copy\colB
      \setbox2=\vsplit1 to \dimen3
      \ifvoid1 \ht\colB=\dimen3\else \dimen3=\ht\colB\fi
    \fi
    \ifdim\dimen3>\availht \dimen3=\availht\fi
    \setbox\colA=\vbox to \dimen3{\unvbox\colA} \setbox\colB=\vbox to \dimen3{\unvbox\colB}%
    \advance\dimen4 by \lastd@pth
    \trace{o}{box height + gap = \the\availht \space + \the\bmg@p}%
    \def\pagecontents{%
      \trace{o}{Topins=\the\ht\topins, Bottomins=\the\ht\bottomins, tlins=\the\ht\topleftins, blins=\the\ht\bottomleftins, trins=\the\ht\toprightins, brins=\the\ht\bottomrightins, boxA=\the\ht\colA, boxB=\the\ht\colB, colbox=\the\ht\xr@fbox}%
      \dimen1=\textwidth\advance\dimen1 by -\ExtraRMargin
      \dimen2=\availht\advance\dimen2\bmg@p
      \ifvoid\partial\else %\msg{2colout partial: \the\ht\partial}\hskip\columnshift
        \vbox{\hbox to \dimen1{\vbox{%
          \unvbox\partial}}}\fi % output partial page % \vskip -2\dimen5
      \ifvoid\topins\else \hbox{\lshiftc@lumn{0}\box\topins} \vskip\skip\topins \fi % output top spanning pictures
      \iflastpage\setbox\colA=\vbox{\unvbox\colA}\setbox\colB=\vbox{\unvbox\colB}%
        \ifdim\ht\colA>\ht\colB \ht\colB=\ht\colA\else\ht\colA=\ht\colB\fi\fi
      \trace{p}{Text depth = \the\dimen3}%
      \setbox\colA\hbox to \dimen1{\noindent
        \ifRTL \lshiftc@lumn{1}\hbox to \ExtraRMargin{}\vbox to \dimen2{\box\colB\vfil}\rshiftc@lumn{1}%
          \makecolumngutter{\the\dimen3}{\the\availht}{\the\lastd@pth}{\the\dimen2}{3}\lshiftc@lumn{0}%
          \vbox to \dimen2{\box\colA\vfil}\rshiftc@lumn{0}%
        \else \lshiftc@lumn{0}\vbox to \dimen2{\box\colA\vfil}\rshiftc@lumn{0}%
          \makecolumngutter{\the\dimen3}{\the\availht}{\the\lastd@pth}{\the\dimen2}{3}\lshiftc@lumn{1}%
          \vbox to \dimen2{\box\colB\vfil}\hbox to \ExtraRMargin{}\rshiftc@lumn{1}\fi}
      \doTextB@rder{\colA}%
      \box\colA
      \p@geendcontent
      \ifnoinkinmargin\box14\fi
    }%
    \global\partialfr@medfalse
    \global\setbox255=\box\voidb@x                                          %(+)
    \plainoutput\trace{p}{plainoutput twocoltrial}%
    \notst@dynotesfalse\clearn@tes
    \s@tpage
    \resetvsize % reset size of next page since it will not have any 1 column material
    \global\holdinginserts=1
    \ifrerunsavepartialpaged
      \global\output={\savepartialpagedbounce}\global\rerunsavepartialpagedfalse\unvbox\s@vedpage
    \else\global\output={\twocols}\unvbox\s@vedpage\fi
    \ifnum\interactionmode=2
      \showlists
    \fi
  \else % the contents of the "galley" didn't fit into the columns,         %(+)
        % so reduce \vsize and try again with an earlier break
    \trace{o}{Reducing vsize \the\vsize, by \the\baselineskip}%
    \global\advance\vsize by -\baselineskip
	% clear insertions
    \trace{i}{2coltrial: emptyinserts hIns=\the\holdinginserts, vsz:\the\vsize}%
    \res@tpage
    \global\let\whichtrial=\twocoltrial                                     %(+)
    \global\output={\backingup}%
    \ifm@rksonpage\else\gdef\p@gefirstmark{}\trace{H}{No marks found. Setting empty mark}\fi%No marks on the page, and it didn't fit, so add a blank mark
    \unvbox\galley \penalty\ifnum\galleypenalty=10000 0 \else \galleypenalty \fi
  \fi
  \traceifcheck{twocoltrial}}
\newif\iffitonpage
\newif\ifrebalance
%-c_twocoltrial

%+c_makecolumngutter
\newdimen\columnshift \columnshift=0pt
\edef\colshiftmode{\l@ft}
\def\rshiftc@lumn#1{%
 \ifdim\columnshift>0pt \count255=#1 \ifnum
    \ifx\colshiftmode\r@ght 1
    \else\ifodd\count255 \ifRTL\ifx\colshiftmode\@nner 1\else 0\fi\else\ifx\colshiftmode\@uter 1\else 0\fi\fi
      \else\ifRTL\ifx\colshiftmode\@uter 1\else 0\fi\else\ifx\colshiftmode\@nner 1\else 0\fi\fi\fi
     \fi =1 \hbox to \columnshift {}\fi\fi
}
\def\z@ropt{0pt}
\def\lshiftc@lumn#1{%
 \ifdim\columnshift>0pt \count255=#1 \ifnum
    \ifx\colshiftmode\l@ft 1
    \else\ifodd\count255 \ifRTL\ifx\colshiftmode\@uter 1\else 0\fi\else \ifx\colshiftmode\@nner 1\else 0\fi\fi
      \else\ifRTL\ifx\colshiftmode\@nner 1\else 0\fi\else\ifx\colshiftmode\@uter 1\else 0\fi\fi\fi
  \fi =1 \trace{M}{lshifting}\hbox to \columnshift{}\else\trace{M}{Not lshifting}\fi\fi
  \trace{M}{for \colshiftmode \ifx\colshiftmode\@nner ==\else !=\fi \@nner \space at #1 is \ifodd\count255 odd\else even\fi}%
}
\newdimen\ColumnGutterRuleSkip \ColumnGutterRuleSkip=0pt
\newif\ifColNotesRule \ColNotesRulefalse
\newdimen\RuleThickness \RuleThickness=0.4pt

\def\makecolumngutter#1#2#3#4#5{\ifnum#5=1\hfil\else\ifnum#5=3\hfil\else\ifnum#5=0\hfil\fi\fi\fi
  \trace{o}{makecolumngutter: rule height=#1, box height=#2, box depth=#3, total height=#4, lineside=#5 bits
            ColumnGutterRuleSkip=\the\ColumnGutterRuleSkip, baselineskip=\the\onel@neunit}%
% #1: Height of gutter rule, #2: height of gutter rule box, #3: depth of gutter rule box, #4: total height
% #5: line sides bit 0=left, 1=right (0=centre line, no XrefNotes)
  \dimen4=#1\advance\dimen4-\ColumnGutterRuleSkip%\advance\dimen4 by #3
  \setbox4=\ifdim#2>\onel@neunit\vbox to #4{%       don't rule empty single column stuff at end of 2 col
    \vskip\ColumnGutterRuleSkip
    \hbox to 1pt{\hfil\vrule width \RuleThicknessheight \the\dimen4 \hfil}%\dp5=#3 \box5
    \vfil}\else\box\voidb@x\fi
  \ifx\XrefNotes\empty
    \ifColumnGutterRule \ifnum#5=3\box4\else\ifnum#5=0\box4\fi\fi\fi
  \else\ifnum#5>0
    \trace{o}{makecolumngutter: xr@fbox=\the\ht\xr@fbox}%
    \ifnum\ifnum#5=1 1\else\ifnum#5=3 1\else 0\fi\fi =1
      \ifColNotesRule\copy4\fi\fi
    \dimen0=#1
    % todo: ensure empty box doesn't collapse
    % todo: allow user defined inner margin also on grid
    \ifnum#5=1\hbox to \XrefNotesMargin{}\else\ifnum#5=3\hbox to \XrefNotesMargin{}\fi\fi
    \ifdim\ht\xr@fbox>0pt
      \vbox to #4{\vbox to #1{\ifdim\ht\xr@fbox>0.9\dimen0 \unvbox\xr@fbox\unpenalty\unskip
        \else\ifdim\ht\xr@fbox=0pt \message{empty xrefbox}\vbox{\hsize=\XrefNotesWidth\hskip\XrefNotesWidth\hbox{}}%
          \else\trace{o}{underful xr@fbox=\the\ht\xr@fbox \space against \the\dimen0}\unvbox\xr@fbox\unpenalty\unskip\fi
          \ifXrefTopfill\else\vfil\fi
        \fi}\vfil}%
    \else \hbox to \XrefNotesWidth{}%
    \fi
    \ifnum#5>1\hbox to \XrefNotesMargin{}\ifColNotesRule\box4\fi\fi
  \fi\fi
  \ifnum#5>1\hfil\else\ifnum#5=0\hfil\fi\fi}

\newif\ifColumnGutterRule
\newdimen\StudyColumnGutterRuleSkip
\def\makestudycolumngutter#1#2#3#4#5{\ifnum#5=1\hfil\else\ifnum#5=3\hfil\else\ifnum#5=0\hfil\fi\fi\fi
  \trace{o}{makestudycolumngutter: rule height=#1, box height=#2, box depth=#3, total height=#4, lineside=#5 bits
            ColumnGutterRuleSkip=\the\StudyColumnGutterRuleSkip, baselineskip=\the\onel@neunit}%
% #1: Height of gutter rule, #2: height of gutter rule box, #3: depth of gutter rule box, #4: total height
% #5: line sides bit 0=left, 1=right (0=centre line, no XrefNotes)
  \dimen4=#1\advance\dimen4-\ColumnGutterRuleSkip%\advance\dimen4 by #3
  \setbox4=\ifdim#2>\onel@neunit\vbox to #4{%       don't rule empty single column stuff at end of 2 col
    \vskip\StudyColumnGutterRuleSkip
    \hbox to 1pt{\hfil\vrule width \RuleThickness height \the\dimen4 \hfil}%\dp5=#3 \box5
    \vfil}\else\box\voidb@x\fi
  \ifStudyGutterRule \ifnum#5=3\box4\else\ifnum#5=0\box4\fi\fi\fi
  \ifnum#5>1\hfil\else\ifnum#5=0\hfil\fi\fi
  \hbox to \columnshift{}}
\newif\ifColumnGutterRule
%-c_makecolumngutter

%+c_backingup
\def\backingup{% this output routine is used when we reduce \vsize;
               % it will cause a new page break to be found, and then the \trial routine is called again \trace{o}{backingup hIns=\the\holdinginserts(==1)}%
  \global\deadcycles=0
  \global\setbox\galley=\copy255
  \global\galleypenalty=\outputpenalty
  \global\output={\whichtrial}%
  \global\holdinginserts=0
  \unvbox255% eject
  \penalty\ifnum\outputpenalty=10000 0 \else \outputpenalty \fi
}
%-c_backingup

%+c_savepartialpagedbounce
\xdef\p@gefirstmark{}
\def\savepartialpagedbounce{
  % This output routine gets the partial galley (input text) before a 2->1 column transition
  % so that it can be re-run with holdinginserts=0, so footnotes etc can be seen.
  \bgroup\setbox0=\copy255 \setbox1=\vsplit0 to \maxdimen\egroup
   \edef\t@mp{\splitbotmark}%
   \ifx\t@mp\empty\else\global\m@rksonpagetrue\trace{H}{Found mark \splitbotmark}\fi
  \s@ttrialheight
  \keepn@testrue\c@lcavailht\keepn@tesfalse
  \vsize=2\availht
  \advance\vsize by \baselineskip
  \global\output={\savepartialpaged}%
  \global\setbox\galley=\copy255 \global\setbox\galley=\vbox{\unvbox\galley}%
  \trace{o}{sppb hIns=\the\holdinginserts, pt:\the\ht\partial, g:\the\ht255, \the\ht\galley, op:\the\outputpenalty}%
  \global\holdinginserts=0
  \unvbox255\eject
}
%-c_savepartialpagedbounce


% to switch back to diglot columns, first set the page  so far 
% and then change the rest.

\def\diglotcolumns{
  \trace{o}{DiglotColumns: partial:\the\ht\partial. Pff:  \PageFullFactor\space\the\textheight}%
  \ifx\@netimesetup\relax%Skip ending diglot if there's no possibily we've started.
    \ifdiglot
      \enddigl@t
    \fi
  \fi
  \global\let\layoutstylebreak\enddigl@t
  \global\diglottrue\diglots@tup
}


\def\diglots@tup{%
  % reset parameters for diglot-column formatting
  \kill@PossParamCache
  \global\hsize=\columnLwidth % always start with left column
  \global\vsize=\textheight
  \global\pagefullfalse%
  \global\BodyColumns=2
  \global\advance\vsize by \baselineskip % don't get caught short by 1/2 line or so
  \global\advance\vsize by -\ht\partial % Don't mis-inform TeX
  % make a macro reset vsize for remaining pages which do not have 1 column material
  \gdef\resetvsize{\global\vsize=\textheight \global\advance\vsize by -\ht\partial \global\advance\vsize by 0.5\baselineskip\trace{o}{resetvsize vsize=\the\vsize}} 
  \global\availht=\vsize
  \global\c@rrentcols=2
  \global\output={\diglotCollect}%
  \count255=1000
  \global\count\topins=\count255
  \global\count\bottomins=\count255
} 


%+c_twocolpageout
\def\twocolp@geout{%This gets used a few times
  \trace{o}{twocolp@geout cols=1: partial=\the\ht\partial , \the\dp\partial : topins=\the\ht\topins : bottomins=\the\ht\bottomins}%
  \def\pagecontents{
    \trace{b}{BALANCE pageout: cols=1: partial=\the\ht\partial , \the\dp\partial : topins=\the\ht\topins : bottomins=\the\ht\bottomins}%
    \vbox{\dimen1=\textwidth \advance\dimen1 -\ExtraRMargin
      \doTextB@rder{\partial}%1
      \ifvoid\partial\else \vbox{\hbox to \dimen1{\box\partial}} \fi
	  \lastd@pth=\dp\partial
      \ifvoid\bottomins\else % \kern-\dimen0
        \lastd@pth=0pt \vskip\skip\bottomins \hbox{\hbox to \columnshift{}\vbox{\unvbox\bottomins}}\fi
      \p@geendcontent
    }}\plainoutput\trace{p}{plainoutput for twocolp@geout}%
    \s@tpage
    \fin@lverybottom
}
\def\fin@lverybottom{%
  \ifdim\ht\verybottomins>0pt
      %\setbox0=\vbox{\unvbox\verybottomins\setbox1=\lastbox}
    \loop
      \trace{o}{verybottom(fin@lverybottom) height=\the\ht\verybottomins, textheight=\the\textheight, vsize=\the\vsize,
                verybottomins=\the\count\verybottomins, \the\dimen\verybottomins}
      \setbox0=\vsplit\verybottomins to \textheight
      \setbox15=\copy14
      \def\pagecontents{\vfil\hbox{\hbox to \columnshift{}\vbox{\unvbox0}}\ifnoinkinmargin\box14\fi}
      \plainoutput\trace{plainoutput for fin@lverybottom}%
      \s@tpage
      \ifdim\ht\verybottomins>0pt \temptrue\else\tempfalse\fi\iftemp\setbox14=\copy15\repeat
  \fi
}
%-c_twocolpageout

%+c_singlecolumn
% to switch back to single column, we reset the page size parameters
\def\layoutstylebreak@singlecolumn{
  \ifhe@dings\endhe@dings\fi
  \par
  \ifrerunsavepartialpaged\trace{o}{rerunsavepartialpaged}\eject\rerunsavepartialpagedfalse\fi}
\let\layoutstylebreak\layoutstylebreak@singlecolumn

\def\singlecolumn{%
  %\tracingmacros=0\tracingassigns=0
  \trace{o}{singlecolumn hIns=\the\holdinginserts, cols=\the\c@rrentcols, partial=\the\ht\partial \iflastpage LASTPAGE\fi}%\tracingmacros=1
  \ifnum\c@rrentcols>1
    %\temptrue
    \layoutstylebreak % save any partially-full page into \partial
    %LOTS happens before this line is reached!
    %\tempfalse
    \trace{o}{SingleColumn: \the\ht\partial > \PageFullFactor\space x \the\textheight ?\iflastpage (ignored as this is last page)\fi}%
    \ifdim\ht\partial>\PageFullFactor\textheight 
      \iflastpage\else %Don't force a break before a colophon
        \setbox14=\makefootbox
        \twocolp@geout %Set up pagecontents and call \plainoutput
        \clearn@tes
      \fi
    \fi
    \onecolwidth{\hsize}%
    \global\vsize=\textheight
    \global\let\layoutstylebreak\layoutstylebreak@singlecolumn
    \global\c@rrentcols=1
  \else\ifdim\ht\partial<\baselineskip
    \global\holdinginserts=0
    \global\output={\savepartialpage}%
    \eject
    \global\holdinginserts=1
  \fi\fi
  \trace{o}{resetting resetvsize for singlecolumn. Partial ht=\the\ht\partial}%
  \def\resetvsize{\global\vsize=\textheight\trace{o}{resetvsize vsize=\the\vsize}}%
  \global\output={\onecol}%
  \global\holdinginserts=1
  \count255=1000
  \global\count\topins=\count255
  \global\count\bottomins=\count255
  \let\\=\s@tn@tec@unt \the\n@tecl@sses % reset \count for each note class
  \iftob@dy\else\ifdim\ht\partial>\baselineskip% \vbox{\unvbox\partial}%
    \ifendbooknoeject\ifx\p@gefirstmark\t@tle \else\vskip\baselineskip\iflastptxfile\else\hskip-\columnshift\hrule\vskip2\baselineskip\fi\fi\fi
  \fi\fi
  %\ifnum\pageno>10\tracingmacros=1\tracingassigns=1\fi
}%
%-c_singlecolumn

\newread\t@mpfile
\def\includeifpresent#1{%
  \openin\t@mpfile="#1"
  \ifeof\t@mpfile\closein\t@mpfile
    \immediate\write-1{Optional file "#1" Cannot be found}%
  \else
    \closein\t@mpfile
    \immediate\write-1{Reading optional file "#1"}%
    \input "#1"
  \fi}


\def\includepdf{\@netimesetup % in case \ptxfile hasn't been used yet
  \ifx\XeTeXpdfpagecount\undefined
    \MSG{*** sorry, \string\includepdf\space requires XeTeX 0.997 or later}%
    \let\n@xt\relax
  \else \let\n@xt\incl@depdf \fi
  \n@xt}
\def\incl@depdf{\begingroup
  \m@kedigitsother \catcode`\[=12 \catcode`\]=12
  \futurelet\n@xt\t@stincl@de@ptions}
\def\t@stincl@de@ptions{\ifx\n@xt[\let\n@xt\incl@de@ptions
  \else\let\n@xt\incl@deno@ptions\fi\n@xt}
\def\incl@deno@ptions#1{\incl@de@ptions[]{#1}}
\def\incl@de@ptions[#1]#2{%
  \totalp@ges=\XeTeXpdfpagecount "#2"\relax
  \whichp@ge=0
  \msg{includepdf #2 pages=\the\totalp@ges}
  \loop \ifnum\whichp@ge<\totalp@ges
    \advance\whichp@ge by 1
    \setbox0=\hbox{\mapimagefile{\XeTeXpdffile}{"#2"}{page \whichp@ge #1}}%
    \ifdim\wd0>\pdfpagewidth
      \setbox0=\hbox{\mapimagefile{\XeTeXpdffile}{"#2"}{page \whichp@ge #1 width \pdfpagewidth}}%
    \fi
    \ifdim\ht0>\pdfpageheight
      \setbox0=\hbox{\mapimagefile{\XeTeXpdffile}{"#2"}{page \whichp@ge #1 height \pdfpageheight}}%
    \fi
    {\def\c@rrID{#2 #1 page \number\whichp@ge}% for lower crop-mark info, if requested
      \shipcompletep@gewithcr@pmarks{\vbox{\box0}}}%
    \advancepageno
    \repeat
  \endgroup % begun in \incl@depdf
}
\newcount\totalp@ges
\newcount\whichp@ge
\newcount\im@gecount

\newif\ifendbooknoeject \endbooknoejectfalse
\def\pagebreak{\message{pagebreak}\ifhmode\par\fi
  \ifsk@pping \egroup \fi%
  \ifhe@dings\endhe@dings\fi%
  \ifdiglot \vbox to 0pt{}\fi%
  \vfill\eject%
}
\def\bookendpagebreak{
  \ifsk@pping \egroup \fi%
  \ifhe@dings\endhe@dings\fi%
  \ifendbooknoeject\else%
  \vfill\eject\fi%
}
\let\pb=\pagebreak

\def\columnbreak{\vfill\eject}

\newcount\badspacepenalty \badspacepenalty=100
\tolerance=9000
\hbadness=10000
% how much space to happily insert before allowing overfull lines. Must be >27pt for emergencypass to run
\emergencystretch=1in
\vbadness=10000
\vfuzz=2pt
\maxdepth=\maxdimen
\frenchspacing

\XeTeXdashbreakstate=1 % allow line-break after en- and em-dash even if no space

%% various Unicode characters that we handle in TeX... and protect in TOC and PDF bookmarks
\def\intercharspace{\hskip0pt}
\def\SFTHYPHEN{\-}
\def\NBSP{\nobreak\space} % make Unicode NO-BREAK SPACE into a no-break space
\def\ZWSP{\penalty0\intercharspace\relax} % ZERO WIDTH SPACE is a possible break
\def\WJ{\leavevmode\nobreak} % WORD JOINER
\def\ZWNBSP{\WJ} % ZERO WIDTH NO-BREAK SPACE
\def\NBHYPH{\leavevmode\hbox{-}} % NON-BREAKING HYPHEN
\def\NQUAD{\penalty\the\badspacepenalty\relax\space} % subvert Unicode En-Quad as BAD BREAKING SPACE
\def\MQUAD{\penalty\the\badspacepenalty\hskip 1em plus .2em minus .2em}
\def\MSPACE{\hskip 1em}
\def\NSPACE{\space}
\def\THREEPEREMSPACE{\hskip .333em}
\def\FOURPEREMSPACE{\hskip .25em}
\def\SIXPEREMSPACE{\hskip .1666em}
\def\THINSPACE{\hskip .2em plus .1em minus .1em}
\def\HAIRSPACE{\hskip 0.042em\intercharspace} % 1/24 em
\def\HYPHEN{\-}
\def\LINESEP{\break}
\def\GOODBREAK{\penalty-\OptionalBreakPenalty}
\def\BADBREAK{\penalty\the\badspacepenalty}
\def\emdash{\char"2014\relax}
%endash defined in ptx-references.tex

\let\pr@tect=\relax
\def\pr@tectspecials{%
  \let\SFTHYPHEN=\relax
  \let\NBSP=\relax
  \let\ZWSP=\relax
  \let\ZWNBSP=\relax
  \let\WJ=\relax
  \let\NBHYPH=\relax
  \let\NQUAD=\relax
  \let\MQUAD=\relax
  \let\MSPACE=\relax
  \let\NSPACE=\relax
  \let\THREEPEREMSPACE=\relax
  \let\FOURPEREMSPACE=\relax
  \let\SIXPEREMSPACE=\relax
  \let\THINSPACE=\relax
  \let\HAIRSPACE=\relax
  \let\HYPHEN=\relax
  \let\LINESEP=\relax
  \let\GOODBREAK=\relax
  \let\BADBREAK=\relax
}

\catcode"A0=12
\catcode"AD=12
\catcode"200B=12
\catcode"2060=12
\catcode"FEFF=12
\catcode"2000=12
\catcode"2001=12
\catcode"2002=12
\catcode"2003=12
\catcode"2004=12
\catcode"2005=12
\catcode"2006=12
\catcode"2009=12
\catcode"200A=12
\catcode"2010=12
\catcode"2063=12
\catcode"2064=12
\def\liter@lspecials{%
  \def\pr@tect{}%
  \def\NBSP{^^a0}%
  \def\SFTHYPHEN{^^ad}%
  %\def\ZWSP{^^^^200b}%
  \def\WJ{^^^^2060}%
  \def\ZWNBSP{^^^^feff}%
  \def\NBHYPH{^^^^2011}%
  \def\NQUAD{^^^^2000}%
  \def\MQUAD{^^^^2001}%
  \def\NSPACE{^^^^2002}%
  \def\MSPACE{^^^^2003}%
  \def\THREEPEREMSPACE{^^^^2004}%
  \def\FOURPEREMSPACE{^^^^2005}%
  \def\SIXPEREMSPACE{^^^^2006}%
  \def\THINSPACE{^^^^2009}%
  \def\HAIRSPACE{^^^^200a}%
  \def\HYPHEN{^^^^2010}%
  \def\LINESEP{^^^^2028}%
  \def\GOODBREAK{^^^^2063}%
  \def\BADBREAK{^^^^2064}%
}

\catcode"A0=\active   \def^^a0{\pr@tect\NBSP}
\catcode"AD=\active   \def^^ad{\pr@tect\SFTHYPHEN}
\catcode"200B=\active \def^^^^200b{\pr@tect\ZWSP}
\catcode"2060=\active \def^^^^2060{\pr@tect\WJ}
\catcode"FEFF=\active \def^^^^feff{\pr@tect\ZWNBSP}
\catcode"2011=\active \def^^^^2011{\pr@tect\NBHYPH}
\catcode"2000=\active \def^^^^2000{\pr@tect\NQUAD}
\catcode"2001=\active \def^^^^2001{\pr@tect\MQUAD}
\catcode"2002=\active \def^^^^2002{\pr@tect\NSPACE}
\catcode"2003=\active \def^^^^2003{\pr@tect\MSPACE}
\catcode"2004=\active \def^^^^2004{\pr@tect\THREEPEREMSPACE}
\catcode"2005=\active \def^^^^2005{\pr@tect\FOURPEREMSPACE}
\catcode"2006=\active \def^^^^2006{\pr@tect\SIXPEREMSPACE}
\catcode"2009=\active \def^^^^2009{\pr@tect\THINSPACE}
\catcode"200A=\active \def^^^^200a{\pr@tect\HAIRSPACE}
\catcode"2010=\active \def^^^^2010{\pr@tect\HYPHEN}
\catcode"2063=\active \def^^^^2063{\pr@tect\GOODBREAK}
\catcode"2064=\active \def^^^^2064{\pr@tect\BADBREAK}
\lccode"2010="2010
\catcode"2028=\active \def^^^^2028{\pr@tect\LINESEP}
\catcode`\@=12

\def\asterisk{*}
\catcode`\%=12 \def\percent{%}\catcode`\%=14
\catcode`\$=12 \def\dollar{$}   %$ keep editor syntax highlight happy
\catcode`\#=12 \def\hash{#}\catcode`\#=6
\catcode`\&=12 \def\ampersand{&}\catcode`\&=4
\catcode`\^=12 \def\circumflex{^}\catcode`\^=7
\catcode`\|=12 \def\pipe{^^7c}

\parskip=0pt
\lineskip=0pt
% NOTE: The baselineskip is set negative here. Later on it will
% be set to the default of 14pt unless the user has specified
% another setting. If that is the case, that setting will be
% used and text can then move off the baseline which makes
% balancing easier but causes text to move off the grid in
% some instances.
\baselineskip=-14pt

\widowpenalty=10000
\clubpenalty=10000
\brokenpenalty=50

\endinput
