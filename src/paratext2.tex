%%%%%%%%%%%%%%%%%%%%%%%%%%%%%%%%%%%%%%%%%%%%%%%%%%%%%%%%%%%%%%%%%%%%%%%
% Part of the ptx2pdf macro package for formatting USFM text
% copyright (c) 2007 by SIL International
% written by Jonathan Kew
%
% Permission is hereby granted, free of charge, to any person obtaining  
% a copy of this software and associated documentation files (the  
% "Software"), to deal in the Software without restriction, including  
% without limitation the rights to use, copy, modify, merge, publish,  
% distribute, sublicense, and/or sell copies of the Software, and to  
% permit persons to whom the Software is furnished to do so, subject to  
% the following conditions:
%
% The above copyright notice and this permission notice shall be  
% included in all copies or substantial portions of the Software.
%
% THE SOFTWARE IS PROVIDED "AS IS", WITHOUT WARRANTY OF ANY KIND,  
% EXPRESS OR IMPLIED, INCLUDING BUT NOT LIMITED TO THE WARRANTIES OF  
% MERCHANTABILITY, FITNESS FOR A PARTICULAR PURPOSE AND  
% NONINFRINGEMENT. IN NO EVENT SHALL SIL INTERNATIONAL BE LIABLE FOR  
% ANY CLAIM, DAMAGES OR OTHER LIABILITY, WHETHER IN AN ACTION OF  
% CONTRACT, TORT OR OTHERWISE, ARISING FROM, OUT OF OR IN CONNECTION  
% WITH THE SOFTWARE OR THE USE OR OTHER DEALINGS IN THE SOFTWARE.
%
% Except as contained in this notice, the name of SIL International  
% shall not be used in advertising or otherwise to promote the sale,  
% use or other dealings in this Software without prior written  
% authorization from SIL International.
%%%%%%%%%%%%%%%%%%%%%%%%%%%%%%%%%%%%%%%%%%%%%%%%%%%%%%%%%%%%%%%%%%%%%%%

% Paratext formatting macros, spanning footnotes version

%+c_pt_intro
% Here we declare generic useful stuff. Including log output routines.
\TeXXeTstate=1 % enable the eTeX bidi extensions, in case we need RTL support
\catcode`\@=11
\count11=20 \count14=20 % need more temporary dimens and boxes
\let\x@=\expandafter
\newif\iftemp \tempfalse
\def\stripqu@tes#1"#2"#3\relax{#2}% Strip quotes if present
\edef\t@mp{\jobname}%
\xdef\j@bname{\x@\x@\x@\stripqu@tes\x@\t@mp\x@"\t@mp"\relax}

%There is no equivalent definition to maxdimen for counts
\def\m@xcount{2147483647}
\def\m@ncount{-2147483647} % Technically ...8, but TeX moans about athat.

\def\MSG{\immediate\write16 } % shorthand to write a message to the terminal
\def\TRACE#1{}%\let\TRACE=\MSG % default - consume messages

% Version .999992 (? or 3) changed pdfsavepos page position (+cur_h_offset) to use pdftex's definition (+473628)
\def\XeTeXvcheck#1.#2\E{\message{This is xetex "#1" "#2"}\ifnum #2 > 999991
    \global\let\adjustpdfXsavepos\empty\else
    \gdef\adjustpdfXsavepos{\advance\dimen0 1in}%
    \gdef\adjustpdfYsavepos{\advance\dimen1 1in}%
  \fi
}
\x@\x@\x@\XeTeXvcheck\x@\the\x@\XeTeXversion\XeTeXrevision\E


% This lot has to be early
\newif\ifinn@te\inn@tefalse
\newif\ifm@rksonpage % Try to keep track of marks, so we can kill them in end-sections. 
\newif\ifm@rksinchunk % Try to keep track of marks, so we know what needs a textborder
\m@rksonpagefalse
%-c_pt_intro

%+c_define-hooks
% `\addtoendhooks` collects macros to be executed at the end of the job
\def\addtoendhooks#1{\x@\global\x@\@ndhooks\x@{\the\@ndhooks#1}}
\def\addtoendhookstr#1#2{\addtoendhooks{\trace{x}{Hook:#2 (}#1}\trace{x}{)}}
\def\addtoendptxhooks#1{\x@\global\x@\@ndptxhooks\x@{\the\@ndptxhooks #1}}
\def\addtostartptxhooks#1{\x@\global\x@\st@rtptxhooks\x@{\the\st@rtptxhooks #1}}
\newtoks\@ndhooks
\newtoks\st@rtptxhooks
\newtoks\@ndptxhooks
\let\s@ve@nd=\end
\def\end{\ifsk@pping\egroup\fi\par\vfill\supereject\ifvoid\partial\else\box\partial\eject\fi \the\@ndhooks \s@ve@nd}

%: `\addtoinithooks` is for stuff we do during one-time-setup before the first PTX file
\def\addtoinithooks#1{\x@\global\x@\@nithooks\x@{\the\@nithooks #1}}
\newtoks\@nithooks

%: `\addtoidhooks` is for stuff we do at an id or a periph
\def\addtoidhooks#1{\x@\global\x@\@idhooks\x@{\the\@idhooks #1}}
\newtoks\@idhooks

%: `\addtoeveryparhooks` is for the start of every paragraph
\def\addtoeveryparhooks#1{\x@\global\x@\@veryparhooks\x@{\the\@veryparhooks #1}}
\newtoks\@veryparhooks

\def\addtoeveryparstarthooks#1{\x@\global\x@\@veryparstarthooks\x@{\the\@veryparstarthooks #1}}
\newtoks\@veryparstarthooks

%: `\addtoparstylehooks` is for stuff to do at each new parstyle marker
\def\addtoparstylehooks#1{\x@\global\x@\p@rstylehooks\x@{\the\p@rstylehooks #1}}
\newtoks\p@rstylehooks
%-c_define-hooks

%+c_timestamp
% initialize a \timestamp macro for the cropmarks etc to use
\edef\timestamp{\number\year.% print the date and time of the run
  \ifnum\month<10 0\fi \number\month.%
  \ifnum\day<10 0\fi \number\day\space :: }%
\count255=\time \divide\count255 by 60
\edef\hrsmins{\ifnum\count255<10 0\fi \number\count255:}%
\multiply\count255 by 60 \advance\count255 by -\time
\count255=-\count255
\edef\hrsmins{\hrsmins
  \ifnum\count255<10 0\fi \number\count255}%
\edef\timestamp{\timestamp \hrsmins}

%-c_timestamp

%+c_evenpage
% Some things need to start on an odd-numbered page, others on an even numbered page
\def\TPILB{}
\def\zEmptyPage{\message{Blank page for \the\pageno}\ifp@geone\temptrue\else\tempfalse\fi\bgroup\m@rksonpagefalse\def\pagecontents{\vfill\TPILB}\plainoutput\egroup\iftemp\global\p@geonetrue\fi}
\def\need@evenpage{\endgraf\ifodd\pageno\zEmptyPage\fi}
\def\need@oddpage{\endgraf\ifodd\pageno\else\zEmptyPage\fi}
\def\need@quadpage{\endgraf\ifodd\pageno\zEmptyPage\need@quadpage\fi \ifodd\numexpr \pageno/2\relax \zEmptyPage\zEmptyPage\fi}
\let\zNeedEvenPage\need@evenpage
\let\zNeedOddPage\need@oddpage
\let\zNeedQuadPage\need@quadpage
%-c_evenpage
\def\ptxversion{0}
\def\ifversion#1#2#3{\ifnum\ptxversion=0\relax #2\else
    \ifnum\ptxversion < #1\relax #3\else #2\fi\fi}

%+c_imports
\input ptx-constants.tex
\input ptx-utility.tex
\input ptx-tracing.tex
\input ptx-diglot.tex
\input ptx-para-style.tex
\input ptx-char-style.tex
\input ptx-milestone-style.tex
\input ptx-note-style.tex
\input ptx-stylesheet.tex % must come after the ptx-*-style.tex macros
\input ptx-periph.tex % 
\input ptx-attribute.tex %Must come after ptx-stylesheet
\input ptx-references.tex
\input ptx-cropmarks.tex
\input ptx-toc.tex
\input ptx-tables.tex
\input ptx-triggers.tex % must come before adj-list and pic-list
\input ptx-adj-list.tex % must come after ptx-stylesheet.tex
\input ptx-pic-list.tex % must come after ptx-stylesheet.tex
\input ptx-cutouts.tex
\input ptx-callers.tex % must come after ptx-note-style.tex
\input ptx-figure.tex % figure-handling
% Additional modules that are not part of the normal ptx2pdf module
\input ptxplus-character-kerning.tex
\input ptx-unicode.tex
% We want to import this module here, but it has to come after the stylesheet is loaded.
%\input ptxplus-marginalverses.tex
\input ptx-extended.tex % \esb and friends.
\input ptx-borders.tex % Borders for esb etc
\input ptx-labels.tex % \zlabel and friends
\input ptx-output.tex % output routines
\input ptx-plugins.tex % Load optional plugins based on contents of \pluginlist, if defined
%-c_imports

%+c_fonts-basic
% default font names (override in setup file)
\ifx\regular\undefined   \def\regular{"Times New Roman"}      \fi
\ifx\bold\undefined      \def\bold{"Times New Roman/B"}       \fi
\ifx\italic\undefined    \def\italic{"Times New Roman/I"}     \fi
\ifx\bolditalic\undefined\def\bolditalic{"Times New Roman/BI"}\fi
%-c_fonts-basic

\let\newb@x=\newbox
\let\newc@unt=\newcount
\let\newt@ks=\newtoks
\let\newdim@n=\newdimen
\let\newins@rt=\newinsert
\let\n@wif=\newif

% set baseline appropriately for the given style (may be using much smaller font than body)
% baselineskip = leading
\def\s@tbaseline#1#2{% Setbaseline-normal
  \trace{F}{s@tbaseline #1\c@rrdstat /\noexpand#2=#2}%
  \def\s@urce{no data}%
  \def\f@ntstyle{#1\c@rrdstat}%
  \@s@tbaseline{#1}{#2}%
}

\def\s@tbaseline@#1#2{% Setting-Side specific already, without looking in the tree
  \trace{F}{s@tbaseline@ #1#2}%
  \def\s@urce{no data}%
  \def\f@ntstyle{#1#2}%
  \@s@tbaseline{#1#2}{#1#2}%
}

\def\@s@tbaseline#1#2{%
  \getp@ram{baseline}{#1}{#2}\ifx\p@ram\relax
    \getp@ram{fontsize}{#1}{#2}\ifx\p@ram\relax\else
      \def\s@urce{calcn:\f@ntstyle}%
      \dimen0=\p@ram\le@dingunit
      \multiply\dimen0 by \LineSpaceBase \divide\dimen0 by 12 % default .75 shift of .85ex (ex=.5 fontsize) against 14/12
      \trace{F}{baseline [\f@ntstyle] (#1,#2) = \the\dimen0 = \p@ram * \the\le@dingunit  * \LineSpaceBase / 12}%
      \setp@ram{baseline}{\f@ntstyle}{\the\dimen0}\baselineskip=\dimen0
      \should@xist{}{\f@ntstyle}%
    \fi
  \else
    \def\s@urce{store:\f@ntstyle}%
    \baselineskip=\p@ram\fi
  \trace{j}{set baselineskip=\the\baselineskip (\s@urce)}}

\def\n@ffin#1{} %  Utility function

\def\includeifpresent#1{%
  \openin\t@mpfile="#1"
  \ifeof\t@mpfile\closein\t@mpfile
    \immediate\write-1{Optional file "#1" Cannot be found}%
    \let\n@xt\relax
  \else
    \closein\t@mpfile
    \immediate\write-1{Reading optional file "#1"}%
    \def\n@xt{\traceifset{#1}\input "#1"\traceifcheck{#1}}%
  \fi
  \n@xt}

\def\includepdf{\@netimesetup % in case \ptxfile hasn't been used yet
  \ifx\XeTeXpdfpagecount\undefined
    \MSG{*** sorry, \string\includepdf\space requires XeTeX 0.997 or later}%
    \let\n@xt\relax
  \else \let\n@xt\incl@depdf \fi
  \n@xt}
\def\incl@depdf{\begingroup
  \m@kedigitsother \catcode`\[=12 \catcode`\]=12
  \futurelet\n@xt\t@stincl@de@ptions}
\def\t@stincl@de@ptions{\ifx\n@xt[\let\n@xt\incl@de@ptions
  \else\let\n@xt\incl@deno@ptions\fi\n@xt}
\def\incl@deno@ptions#1{\incl@de@ptions[]{#1}}
\def\incl@de@ptions[#1]#2{%
  \pdfp@ges=\XeTeXpdfpagecount "#2"\relax
  \whichp@ge=0
  \msg{includepdf #2 pages=\the\pdfp@ges}
  \loop \ifnum\whichp@ge<\pdfp@ges
    \advance\whichp@ge by 1
    \setbox0=\hbox{\mapimagefile{\XeTeXpdffile}{"#2"}{page \whichp@ge #1}}%
    \ifdim\wd0>\pdfpagewidth
      \setbox0=\hbox{\mapimagefile{\XeTeXpdffile}{"#2"}{page \whichp@ge #1 width \pdfpagewidth}}%
    \fi
    \ifdim\ht0>\pdfpageheight
      \setbox0=\hbox{\mapimagefile{\XeTeXpdffile}{"#2"}{page \whichp@ge #1 height \pdfpageheight}}%
    \fi
    {\def\c@rrID{#2 #1 page \number\whichp@ge}% for lower crop-mark info, if requested
      \shipcompletep@gewithcr@pmarks{\vbox{\box0}}}%
    \ifholdpageno\else\advancepageno\fi
    \repeat
  \endgroup % begun in \incl@depdf
}
\newcount\pdfp@ges %Pages in the pdf
\newcount\whichp@ge
\newcount\im@gecount
\newcount\totalp@ges % Page count for this entire job, (excluding cover)
\totalp@ges=0
\newif\ifendbooknoeject \endbooknoejectfalse
\newif\iflastbooknoeject \lastbooknoejectfalse
\def\pagebreak{\leavevmode\message{pagebreak}\endp@ge} %A pagebreak that forces a blank page to be output if there's nothing on the page yet.
\def\endpage{\message{endpage}\endp@ge} %A pagebreak that doesn't force a blank page to be output
\def\endp@ge{\ifhmode\par\fi
  \ifsk@pping \egroup \fi%
  \ifhe@dings\endhe@dings\fi%
  \ifdiglot \vbox to 0pt{}\fi%
  \vfill\eject%
}
\def\bookendpagebreak{
  \trace{o}{bookendpagebreak endbooknoeject\ifendbooknoeject true\else false\fi}%
  \ifsk@pping \egroup \fi
  \ifhe@dings\endhe@dings\fi
  \ifendbooknoeject
      \trace{o}{NOT Ejecting page}%
  \else\ifx\PageAlign\val@MULTI
      \trace{o}{NOT Ejecting page (align=\PageAlign)}%
    \else
      \trace{o}{Ejecting page}%
      \vfill\eject\fi\fi
}

\def\columnbreak{\vfill\eject}

\newcount\badspacepenalty \badspacepenalty=100
\tolerance=9000
\hbadness=10000
% how much space to happily insert before allowing overfull lines. Must be >27pt for emergencypass to run
\emergencystretch=1in
\vbadness=10000
\vfuzz=2pt
\maxdepth=\maxdimen
\frenchspacing

\XeTeXdashbreakstate=1 % allow line-break after en- and em-dash even if no space

%% various Unicode characters that we handle in TeX... and protect in TOC and PDF bookmarks
\def\intercharspace{\hskip0pt}
\def\SFTHYPHEN{\-}
\def\NBSP{\nobreak\space} % make Unicode NO-BREAK SPACE into a no-break space
\def\ZWSP{\penalty0\intercharspace\relax} % ZERO WIDTH SPACE is a possible break
\def\WJ{\leavevmode\nobreak} % WORD JOINER
\def\ZWNBSP{\WJ} % ZERO WIDTH NO-BREAK SPACE
\def\NQUAD{\BADBREAK\relax\space} % subvert Unicode En-Quad as BAD BREAKING SPACE
\def\MQUAD{\BADBREAK\hskip 1em plus .2em minus .2em}
\def\MSPACE{\hskip 1em} % Fixed width
\def\NSPACE{\hbox{\space}} % Fixed width
\def\THREEPEREMSPACE{\hskip .333em}
\def\FOURPEREMSPACE{\hskip .25em}
\def\SIXPEREMSPACE{\hskip .1666em}
\def\THINSPACE{\hskip .2em plus .1em minus .1em}
\def\HAIRSPACE{\nobreak\hskip 0.042em\nobreak\intercharspace} % 1/24 em
\def\HYPHEN{\char"2010\discretionary{}{}{}}% Contrastive hyphen with hyphen break
\def\LINESEP{\break}
\def\GOODBREAK{\penalty-\OptionalBreakPenalty}
\def\BADBREAK{\penalty\the\badspacepenalty}
\def\emdash{\char"2014\relax}
\chardef\|"7C
\chardef\bslash"5C%
\chardef\tilde"7E%
%endash defined in ptx-references.tex

\let\pr@tect=\relax
\def\pr@tectspecials{%
  \let\SFTHYPHEN=\relax
  \let\NBSP=\relax
  \let\ZWSP=\relax
  \let\ZWNBSP=\relax
  \let\WJ=\relax
  \let\NBHYPH=\relax
  \let\NQUAD=\relax
  \let\MQUAD=\relax
  \let\MSPACE=\relax
  \let\NSPACE=\relax
  \let\THREEPEREMSPACE=\relax
  \let\FOURPEREMSPACE=\relax
  \let\SIXPEREMSPACE=\relax
  \let\THINSPACE=\relax
  \let\HAIRSPACE=\relax
  \let\HYPHEN=\relax
  \let\LINESEP=\relax
  \let\GOODBREAK=\relax
  \let\BADBREAK=\relax
}

\catcode"A0=12
\catcode"AD=12
\catcode"200B=12
\catcode"2060=12
\catcode"FEFF=12
\catcode"2000=12
\catcode"2001=12
\catcode"2002=12
\catcode"2003=12
\catcode"2004=12
\catcode"2005=12
\catcode"2006=12
\catcode"2009=12
\catcode"200A=12
\catcode"2010=12
\catcode"2063=12
\catcode"2064=12
\catcode"202A=12
\catcode"202B=12
\catcode"202C=12
\def\liter@lspecials{%
  \def\pr@tect{}%
  \def\NBSP{^^a0}%
  \def\SFTHYPHEN{^^ad}%
  %\def\ZWSP{^^^^200b}%
  \def\WJ{^^^^2060}%
  \def\ZWNBSP{^^^^feff}%
  \def\NQUAD{^^^^2000}%
  \def\MQUAD{^^^^2001}%
  \def\NSPACE{^^^^2002}%
  \def\MSPACE{^^^^2003}%
  \def\THREEPEREMSPACE{^^^^2004}%
  \def\FOURPEREMSPACE{^^^^2005}%
  \def\SIXPEREMSPACE{^^^^2006}%
  \def\THINSPACE{^^^^2009}%
  \def\HAIRSPACE{^^^^200a}%
  \def\HYPHEN{^^^^2010}%
  \def\LINESEP{^^^^2028}%
  \def\GOODBREAK{^^^^2063}%
  \def\BADBREAK{^^^^2064}%
}

\catcode"A0=\active   \def^^a0{\pr@tect\NBSP}
\catcode"AD=\active   \def^^ad{\pr@tect\SFTHYPHEN}
\catcode"200B=\active \def^^^^200b{\pr@tect\ZWSP}
\catcode"2060=\active \def^^^^2060{\pr@tect\WJ}
\catcode"FEFF=\active \def^^^^feff{\pr@tect\ZWNBSP}
\catcode"2000=\active \def^^^^2000{\pr@tect\NQUAD}
\catcode"2001=\active \def^^^^2001{\pr@tect\MQUAD}
\catcode"2002=\active \def^^^^2002{\pr@tect\NSPACE}
\catcode"2003=\active \def^^^^2003{\pr@tect\MSPACE}
\catcode"2004=\active \def^^^^2004{\pr@tect\THREEPEREMSPACE}
\catcode"2005=\active \def^^^^2005{\pr@tect\FOURPEREMSPACE}
\catcode"2006=\active \def^^^^2006{\pr@tect\SIXPEREMSPACE}
\catcode"2009=\active \def^^^^2009{\pr@tect\THINSPACE}
\catcode"200A=\active \def^^^^200a{\pr@tect\HAIRSPACE}
\catcode"2010=\active \def^^^^2010{\pr@tect\HYPHEN}
\catcode"2063=\active \def^^^^2063{\pr@tect\GOODBREAK}
\catcode"2064=\active \def^^^^2064{\pr@tect\BADBREAK}
% Default lowercases for hyphenation
\lccode"2010="2010
\lccode"2011="2011
\catcode"2028=\active \def^^^^2028{\pr@tect\LINESEP}
\catcode`\@=12

\def\asterisk{*}
\catcode`\%=12 \def\percent{%}\catcode`\%=14
\catcode`\$=12 \def\dollar{$}   %$ keep editor syntax highlight happy
\catcode`\#=12 \def\hash{#}\catcode`\#=6
\catcode`\&=12 \def\ampersand{&}\catcode`\&=4
\catcode`\^=12 \def\circumflex{^}\catcode`\^=7
\catcode`\|=12 \def\pipe{^^7c}

\parskip=0pt
\lineskip=0pt
% NOTE: The baselineskip is set negative here. Later on it will
% be set to the default of 14pt unless the user has specified
% another setting. If that is the case, that setting will be
% used and text can then move off the baseline which makes
% balancing easier but causes text to move off the grid in
% some instances.
\baselineskip=-14pt

\widowpenalty=10000
\clubpenalty=10000
\brokenpenalty=50

\endinput
