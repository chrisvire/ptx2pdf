%%%%%%%%%%%%%%%%%%%%%%%%%%%%%%%%%%%%%%%%%%%%%%%%%%%%%%%%%%%%%%%%%%%%%%%
% Part of the ptx2pdf macro package for formatting USFM text
% copyright (c) 2007-2008 by SIL International
% written by Jonathan Kew
%
% Permission is hereby granted, free of charge, to any person obtaining  
% a copy of this software and associated documentation files (the  
% "Software"), to deal in the Software without restriction, including  
% without limitation the rights to use, copy, modify, merge, publish,  
% distribute, sublicense, and/or sell copies of the Software, and to  
% permit persons to whom the Software is furnished to do so, subject to  
% the following conditions:
%
% The above copyright notice and this permission notice shall be  
% included in all copies or substantial portions of the Software.
%
% THE SOFTWARE IS PROVIDED "AS IS", WITHOUT WARRANTY OF ANY KIND,  
% EXPRESS OR IMPLIED, INCLUDING BUT NOT LIMITED TO THE WARRANTIES OF  
% MERCHANTABILITY, FITNESS FOR A PARTICULAR PURPOSE AND  
% NONINFRINGEMENT. IN NO EVENT SHALL SIL INTERNATIONAL BE LIABLE FOR  
% ANY CLAIM, DAMAGES OR OTHER LIABILITY, WHETHER IN AN ACTION OF  
% CONTRACT, TORT OR OTHERWISE, ARISING FROM, OUT OF OR IN CONNECTION  
% WITH THE SOFTWARE OR THE USE OR OTHER DEALINGS IN THE SOFTWARE.
%
% Except as contained in this notice, the name of SIL International  
% shall not be used in advertising or otherwise to promote the sale,  
% use or other dealings in this Software without prior written  
% authorization from SIL International.
%%%%%%%%%%%%%%%%%%%%%%%%%%%%%%%%%%%%%%%%%%%%%%%%%%%%%%%%%%%%%%%%%%%%%%%

% ptx-toc.tex
% Table of Contents generation

% Usage: \GenerateTOC[title]{filename}
% where [title] is optional, default is "Table of Contents"
% Writes TOC entries from \toc1, \toc2, \toc3 markers
% to the given file, which can be used with \ptxfile{...}
% or renamed and edited as needed to customize the TOC.

\def\GenerateTOC{\edef\s@vebracketcat{\the\catcode`\[}%
  \catcode`\[=12 \futurelet\n@xt\g@ntoc}
\def\g@ntoc{\ifx\n@xt[\let\n@xt\g@ntoc@rg
  \else\let\n@xt\g@ntocn@@rg\fi\n@xt}
\def\g@ntocn@@rg#1{\g@ntoc@rg[Table of Contents]{#1}}
\def\g@ntoc@rg[#1]#2{\def\t@ctitle{#1}\def\t@cfilename{#2}%
  \catcode`\[=\s@vebracketcat\relax}
\let\t@cfilename\empty

\newwrite\t@cfile

\addtoinithooks{\set@ptocmarkers}
\addtoendhooks{\closet@cfile}
\newif\ifuseTOCthree\useTOCthreetrue
\newif\ifTOCthreetab\TOCthreetabtrue
\newif\ifuseTOCtwo\useTOCtwotrue
\newif\ifuseTOCone\useTOConetrue


\m@kedigitsletters
\def\set@ptocmarkers{
%  \def\toc1{\bgroup\deactiv@tecustomch@rs\obeylines\t@cA}
%  \def\toc2{\bgroup\deactiv@tecustomch@rs\obeylines\t@cB}
%  \def\toc3{\bgroup\deactiv@tecustomch@rs\obeylines\t@cC}
  \def\toc1{\endlastp@rstyle{toc1}\bgroup\obeylines\t@cA}
  \def\toc2{\endlastp@rstyle{toc2}\bgroup\obeylines\t@cB}
  \def\toc3{\endlastp@rstyle{toc3}\bgroup\obeylines\t@cC}
  \def\zthumbtab{\bgroup\obeylines\th@mt@b} %Manual control of thumbtab text, from usfm file.

}
\m@kedigitsother

{\obeylines%
 \gdef\t@cA #1^^M{\ifuseTOCone\gdef\toc@A{#1}\fi\egroup}%%\message{TOC1 defined \toc@A}}%
 \gdef\t@cB #1^^M{\ifuseTOCtwo\gdef\toc@B{#1}\fi\egroup}%
 \gdef\t@cC #1^^M{\ifuseTOCthree\gdef\toc@C{#1}\fi\ifTOCthreetab\ifx\b@okShort\empty\gdef\b@okShort{#1}\fi\fi\egroup}%
 \gdef\th@mt@b #1^^M{\gdef\b@okShort{#1}\ifx\b@okShort\empty\gdef\b@okShort{\relax}\fi\egroup}%
}

\gdef\toc@A{}%
\gdef\toc@B{}%
\gdef\toc@C{}%
\newcount\tocc@l
\newif\ift@copen
\t@copenfalse

\catcode`\[=1 \catcode`\]=2 \catcode`\}=12 \catcode`\{=12
\gdef\maket@cline[%
   %\message[Writing toc]%
   \ifx\t@cfilename\empty
   \else
     \ift@copen\else
       \immediate\openout\t@cfile=\t@cfilename \global\t@copentrue
       %\write\t@cfile[\string\id \space FRT -- autogenerated --]
       %\write\t@cfile[\string\unprepusfm \string\catcode `\string\@ =11]
       \immediate\write\t@cfile[\string\defTOC{main}{]%
     \fi
     \begingroup
       \pr@tectspecials % protect \ZWSP etc from expansion in the TOC text
       \liter@lkerns
	   \ifdiglot\edef\tmp[\id@@@\c@rrdstat]\else\edef\tmp[\id@@@]\fi
       \xdef\t@cline[\string\doTOCline {\tmp }{\toc@A }{\toc@B }{\toc@C }{\the\pageno }]%
       \trace[t][\t@cline]\x@\write\x@\t@cfile\x@[\t@cline]%
     \endgroup
     \gdef\toc@A[]%
     \gdef\toc@B[]%
     \gdef\toc@C[]%
   \fi
]
\def\closet@cfile[%
  \ift@copen
      \immediate\write\t@cfile[}]
      %\immediate\write\t@cfile[\string\catcode `\string\@ =12 \string\prepusfm]
      %\immediate\write\t@cfile[\string\id \space FRT -- autogenerated --]
      %\immediate\write\t@cfile[\string\is \space\t@ctitle]
      %\immediate\write\t@cfile[\string\p]
      %\immediate\write\t@cfile[\string\zTOC]
      %\immediate\write\t@cfile[\string\p]
    \immediate\closeout\t@cfile
    \global\t@copenfalse
  \fi
]
\catcode`\{=1 \catcode`\}=2 \catcode`\[=12 \catcode`\]=12

\def\checkfort@c{%
  \let\n@xt\relax
  \ifx\toc@A\empty \else \let\n@xt\maket@cline \fi
  \ifx\toc@B\empty \else \let\n@xt\maket@cline \fi
  \ifx\toc@C\empty \else \let\n@xt\maket@cline \fi
  \n@xt}

\def\fnname{doTOCline}
\x@\def\csname\fnname\endcsname#1#2#3#4#5{%
% ID, cell1, cell2, cell3, pagenum
  \trace{t}{\fnname\space #1 #2 #3 #4 #5}%
  \def\tablecategory{toc}\tr
  \edef\t@mpa{#2}%
  \edef\t@mpb{#3}%
  \edef\t@mpc{#4}%
  \global\count252=1
  \ifx\t@mpa\empty\else
    \ifcsname tocalign\the\count252\endcsname
      \edef\t@mpd{tc\csname tocalign\the\count252\endcsname\the\count252}%
    \else
      \edef\t@mpd{tc\the\count252}%
    \fi
    \trace{t}{\t@mpd \space \t@mpa}%
    \csname \t@mpd\endcsname \t@mpa
    \ifcsname cat:toc|tc\the\count252 :fill\endcsname\else\d@ftoc{tc\the\count252 :fill}{\hfill}\fi
    \global\advance\count252 by 1
  \fi
  \ifx\t@mpb\empty\else
    \ifcsname tocalign\the\count252\endcsname
      \edef\t@mpd{tc\csname tocalign\the\count252\endcsname\the\count252}%
      \trace{t}{Found tocalign\the\count252}%
    \else
      \edef\t@mpd{tc\the\count252}%
    \fi
    \trace{t}{\t@mpd \space \t@mpb}%
    \csname \t@mpd\endcsname \t@mpb
    \ifcsname cat:toc|tc\the\count252 :fill\endcsname\else\d@ftoc{tc\the\count252 :fill}{\hfill}\fi
    \global\advance\count252 by 1
  \fi
  \ifx\t@mpc\empty\else
    \ifcsname tocalign\the\count252\endcsname
      \edef\t@mpd{tc\csname tocalign\the\count252\endcsname\the\count252}%
    \else
      \edef\t@mpd{tc\the\count252}
    \fi
    \trace{t}{\t@mpd \space \t@mpc}
    \csname \t@mpd\endcsname \t@mpc
    \ifcsname cat:toc|tc\the\count252 :fill\endcsname\else\d@ftoc{tc\the\count252 :fill}{\hfill}\fi
    \global\advance\count252 by 1
  \fi
  \edef\t@mpd{tc\ifRTL\else r\fi\the\count252}
  \csname\t@mpd\endcsname #5\@ndcell\@ndrow
  \trace{t}{\noexpand\tcr\the\count252 \space \t@mpd}
  \ifcsname l@drs\endcsname
    \trace{t}{Refactor for last column \the\count252}
    \deftocalign{\the\count252}{r}
    \d@ftoc{tc\the\count252 :fill}{}
    \d@ftoc{leaders-tc\ifRTL\else r\fi\the\count252}{}
    \d@ftoc{leaders-tc\ifRTL r\fi\the\count252}{\l@drs}
    \ifRTL\d@ftoc{tc\the\count252 :leftmargin}{0}
    \else\d@ftoc{tc\the\count252 :rightmargin}{0}\fi
  \fi
  \gdef\tablecategory{}
}

\addtoeveryparhooks{\checkfort@c}

\let\tocf@@t=\empty
\def\d@ftoc#1#2{\should@xist{cat:toc|#1}\expandafter\xdef\csname cat:toc|#1\endcsname{#2}}
\def\defTOC#1#2{\x@\def\csname ms:ztoc=#1\endcsname{\gentoche@dfoot\global\keeptriggertrue \toch@ad #2 \tocf@@t \endt@ble}}
\def\addt@tmptok#1#2{\x@\tmpt@ks\x@{\the\tmpt@ks \csname #1\endcsname #2}}%
\def\gentoche@dfoot{\let\toch@ad\empty\let\tocf@@t\empty
  \tmpt@ks{\tr }%
  \tempfalse
  \get@ttribute{h1}\ifx\relax\attr@b\let\attr@b\empty\else\temptrue\edef\tmp{{th1}{\attr@b}}\x@\addt@tmptok\tmp\fi
  \get@ttribute{h2}\ifx\relax\attr@b\let\attr@b\empty\else\temptrue\edef\tmp{{th2}{\attr@b}}\x@\addt@tmptok\tmp\fi
  \get@ttribute{h3}\ifx\relax\attr@b\let\attr@b\empty\else\temptrue\edef\tmp{{th3}{\attr@b}}\x@\addt@tmptok\tmp\fi
  \get@ttribute{h4}\ifx\relax\attr@b\let\attr@b\empty\else\temptrue\edef\tmp{{th4}{\attr@b}}\x@\addt@tmptok\tmp\fi
  \iftemp
    \x@\def\x@\toch@ad\x@{\the\tmpt@ks}%
  \fi
  \tmpt@ks{\tr }\tempfalse%
  \get@ttribute{f1}\ifx\relax\attr@b\let\attr@b\empty\else\temptrue\edef\tmp{{th1}{\attr@b}}\x@\addt@tmptok\tmp\fi
  \get@ttribute{f2}\ifx\relax\attr@b\let\attr@b\empty\else\temptrue\edef\tmp{{th2}{\attr@b}}\x@\addt@tmptok\tmp\fi
  \get@ttribute{f3}\ifx\relax\attr@b\let\attr@b\empty\else\temptrue\edef\tmp{{th3}{\attr@b}}\x@\addt@tmptok\tmp\fi
  \get@ttribute{f4}\ifx\relax\attr@b\let\attr@b\empty\else\temptrue\edef\tmp{{th4}{\attr@b}}\x@\addt@tmptok\tmp\fi
  \iftemp
    \x@\def\x@\tocf@@t\x@{\the\tmpt@ks}%
  \fi
}
\def\tocleaders#1#2{
  \if#1\hrule
    \gdef\l@drs{\xleaders #1 \hfill}
  \else
    \gdef\l@drs{\xleaders\hbox{#1}\hfill}
  \fi
  \Marker cat:toc|tc1\relax
  \StyleType Character\relax
  \Marker cat:toc|tc2\relax
  \StyleType Character\relax
  \Marker cat:toc|tc3\relax
  \StyleType Character\relax
  \Marker cat:toc|tc4\relax
  \StyleType Character\relax
  \dimen0=\IndentUnit \dimen1=#2 \divide\dimen0 by \dimen1
  \ifdim\dimen0>1sp \dimen1=1pt \divide\dimen1 by \dimen0\else\dimen0=0pt\fi
  \edef\rm@{\strip@pt{\dimen1}}
  \count252=1
  \loop
    \ifcsname tocalign\the\count252\endcsname
      \x@\let\x@\the@lign\csname tocalign\the\count252\endcsname
    \else
      \def\the@lign{}
    \fi
    \d@ftoc{tc\the\count252 :rightmargin}{\rm@}
    \d@ftoc{tc\the\count252 :leftmargin}{\rm@}
    \ifnum \ifx\the@lign\cell@r 1\else\ifx\the@lign\cell@c 1\else 0\fi\fi =1
      \d@ftoc{leaders-tc\ifRTL r\fi\the\count252}{\leftskip=0pt\rightskip=0pt\l@drs}
    \fi
    \ifx\the@lign\cell@r \else
      \d@ftoc{leaders-tc\ifRTL\else r\fi\the\count252}{\leftskip=0pt\rightskip=0pt\l@drs}
    \fi
    \d@ftoc{tc\the\count252 :fill}{\l@drs}
    \advance\count252 by 1
  \ifnum\count252<5\repeat
}
\def\deftocalign#1#2{\x@\def\csname tocalign#1\endcsname{#2}}
