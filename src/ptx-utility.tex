%:skip
%%%%%%%%%%%%%%%%%%%%%%%%%%%%%%%%%%%%%%%%%%%%%%%%%%%%%%%%%%%%%%%%%%%%%%%
% Part of the ptx2pdf macro package for formatting USFM text
% copyright (c) 2022 by SIL International
% written by David Gardner
%
% Permission is hereby granted, free of charge, to any person obtaining  
% a copy of this software and associated documentation files (the  
% "Software"), to deal in the Software without restriction, including  
% without limitation the rights to use, copy, modify, merge, publish,  
% distribute, sublicense, and/or sell copies of the Software, and to  
% permit persons to whom the Software is furnished to do so, subject to  
% the following conditions:
%
% The above copyright notice and this permission notice shall be  
% included in all copies or substantial portions of the Software.
%
% THE SOFTWARE IS PROVIDED "AS IS", WITHOUT WARRANTY OF ANY KIND,  
% EXPRESS OR IMPLIED, INCLUDING BUT NOT LIMITED TO THE WARRANTIES OF  
% MERCHANTABILITY, FITNESS FOR A PARTICULAR PURPOSE AND  
% NONINFRINGEMENT. IN NO EVENT SHALL SIL INTERNATIONAL BE LIABLE FOR  
% ANY CLAIM, DAMAGES OR OTHER LIABILITY, WHETHER IN AN ACTION OF  
% CONTRACT, TORT OR OTHERWISE, ARISING FROM, OUT OF OR IN CONNECTION  
% WITH THE SOFTWARE OR THE USE OR OTHER DEALINGS IN THE SOFTWARE.
%
% Except as contained in this notice, the name of SIL International  
% shall not be used in advertising or otherwise to promote the sale,  
% use or other dealings in this Software without prior written  
% authorization from SIL International.
%%%%%%%%%%%%%%%%%%%%%%%%%%%%%%%%%%%%%%%%%%%%%%%%%%%%%%%%%%%%%%%%%%%%%%%
%
% scantokens triggers eof, and might be used to pull in code that makes other scantokens calls.
% Therefore it should be wrapped in these: 
\newcount\s@vedeofdepth
\def\p@sheof{\advance\s@vedeofdepth by 1 \x@\def\csname eofsave\the\s@vedeofdepth\x@\endcsname\x@{\the\everyeof}\everyeof{}}
\def\p@peof{\x@\x@\x@\everyeof\x@\x@\x@{\csname eofsave\the\s@vedeofdepth\endcsname}\x@\let\csname eofsave\the\s@vedeofdepth\endcsname\undefined\advance\s@vedeofdepth by -1 }% 1st expansion makes the csname into a macro name, second gets its contents
% Utility macros for rotation, etc.
\def\geom@xform@u{{1}{0}{0}{1}} % Normal
\def\geom@xform@d{{-1}{0}{0}{-1}} %Rotate 180 / Flip both
\def\geom@xform@l{{0}{-1}{1}{0}} % Rotate +90
\def\geom@xform@r{{0}{1}{-1}{0}} % Rotate -90
\def\geom@xform@h{{-1}{0}{0}{1}} % Flip Horizontally 
\def\geom@xform@v{{1}{0}{0}{-1}} % Flip vertically
\def\geom@xform@L{{0}{-1}{-1}{0}} % Rotate +90 and Flip vertically
\def\geom@xform@R{{0}{1}{1}{0}} % Rotate -90 and Flip vertically


\def\setgeomtransform#1{%
  \x@\let\x@\geom@xform\csname geom@xform@#1\endcsname
  \ifx\geom@xform\relax
    \let\geom@xform\geom@xform@u
  \fi
  \x@\s@tgeomtransform@m\geom@xform % Calculate pdf@aa, etc.
} 

\def\s@tgeomtransform@m#1#2#3#4{%
  \def\pdf@aa{#1}\def\pdf@ab{#2}\def\pdf@ba{#3}\def\pdf@bb{#4}%
}

\setgeomtransform{u}

\def\@pprox#1#2#3{% #1 = number #2 =  significant 3-bits. #3 = where to save the value.
  \bgroup \dimen0=#1 \count255=0
    \loop \unless\ifdim \dimen0=0pt 
      \divide \dimen0 by 8
      \advance \count255 by 1
      \repeat
    %\message{Removing #1 has \the\count255 significant 3-bits}%
    \advance \count255 by -#2
    \dimen0=#1 \tmpcount=0
    \loop \ifnum\count255>0
      \divide \dimen0 by 8
      \advance \count255 by -1
      \advance \tmpcount by 1 \relax
      %\message{After removing \the\tmpcount  digits from #1: \the\dimen0}%
      \repeat
    \loop \ifnum\tmpcount>0
      \multiply \dimen0 by 8
      \advance \tmpcount by -1
      \repeat
    \xdef#3{\the\dimen0}%
    %\message{#1 to #2 significant 3-bits is #3}%
  \egroup
}
    
\long\def\addto@macro#1#2{%
    \x@\x@\x@\def\x@\x@\x@#1\x@\x@\x@{\x@#1#2}}%

\def\stripqu@te#1{\x@\@stripqu@te #1""\E\relax}
\def\@stripqu@te #1"#2"#3\E{\def\@@result{#2}\ifx\@@result\empty\def\@@result{#1}\fi}

\def\ch@cktypecs#1{% Check type of named control sequence
 \ifcsname \m@rker\endcsname
   \ifcsname stylet@pe-#1\endcsname
     \global\let\ch@ck@result=\dt@marker % More detail from interrogating the stylet@pe
   \else
     \x@\ch@cktype\x@{\csname \m@rker\endcsname}%
   \fi
 \else
   \global\let\ch@ck@result\dt@undefined
 \fi
}
\def\ch@cktype#1{% Check type of giVEN WOTsit
  \bgroup\edef\tmp{\meaning#1}%
  \ifx\tmp\dt@undefined \global\let\ch@ck@result\dt@macro
  \else
    \x@\x@\x@\ch@ktypeA\x@\tmp\dt@macro\ENDIT
  \fi
  \egroup \tracingassigns=0
}
\x@\def\x@\ch@ktypeA\x@#\x@1\dt@macro#2\ENDIT{%
  \def\tmpB{#2}\ifx\tmpB\empty 
    \global\let\ch@ck@result\oth@r% Some kind of primitive.
  \else 
    \global\let\ch@ck@result\dt@macro % Some kind of macro
  \fi
}

