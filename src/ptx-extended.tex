%:skip
%%%%%%%%%%%%%%%%%%%%%%%%%%%%%%%%%%%%%%%%%%%%%%%%%%%%%%%%%%%%%%%%%%%%%%%
% Part of the ptx2pdf macro package for formatting USFM text
% copyright (c) 2007 by SIL International
% written by Jonathan Kew
%
% Permission is hereby granted, free of charge, to any person obtaining  
% a copy of this software and associated documentation files (the  
% "Software"), to deal in the Software without restriction, including  
% without limitation the rights to use, copy, modify, merge, publish,  
% distribute, sublicense, and/or sell copies of the Software, and to  
% permit persons to whom the Software is furnished to do so, subject to  
% the following conditions:
%
% The above copyright notice and this permission notice shall be  
% included in all copies or substantial portions of the Software.
%
% THE SOFTWARE IS PROVIDED "AS IS", WITHOUT WARRANTY OF ANY KIND,  
% EXPRESS OR IMPLIED, INCLUDING BUT NOT LIMITED TO THE WARRANTIES OF  
% MERCHANTABILITY, FITNESS FOR A PARTICULAR PURPOSE AND  
% NONINFRINGEMENT. IN NO EVENT SHALL SIL INTERNATIONAL BE LIABLE FOR  
% ANY CLAIM, DAMAGES OR OTHER LIABILITY, WHETHER IN AN ACTION OF  
% CONTRACT, TORT OR OTHERWISE, ARISING FROM, OUT OF OR IN CONNECTION  
% WITH THE SOFTWARE OR THE USE OR OTHER DEALINGS IN THE SOFTWARE.
%
% Except as contained in this notice, the name of SIL International  
% shall not be used in advertising or otherwise to promote the sale,  
% use or other dealings in this Software without prior written  
% authorization from SIL International.
%%%%%%%%%%%%%%%%%%%%%%%%%%%%%%%%%%%%%%%%%%%%%%%%%%%%%%%%%%%%%%%%%%%%%%%

%+c_ext_intro
% Declare things we need.

\newbox\extb@x % Box in which extended study matter is placed
\newbox\extchunkb@x % Box in which extended study matter is placed
\newbox\sid@barnotes % Box in which notes are placed in a sidebar
\newtoks\old@bx %store old definition of \everyhbox
\def\c@tegory{}
\let\DebugBorderCat\undefined %  Set this to showbox the specified category just after the border is added
\newif\ifBoxLikeBorder\BoxLikeBordertrue % Set to false for old behaviour: the box does not affect the spacing of boxes.

\def\S@tCat#1{\trace{sc}{Started Category #1}%
  \edef\c@tegory{#1}%
  \s@tc@tpr@fix%
}
\gdef\sb@rmarker{esb} % future-proofing. What's the current sidebar marker?

\def\setsbp@ram#1#2{\trace{sc}{Styling \m@rker:#1 #2}\setp@ram{#1}{\m@rker}{#2}}
\def\getsbp@ram#1{\getp@ram{#1}{\sb@rmarker}{\sb@rmarker}\let\cp@ram\p@ram\ifx\cp@ram\empty\global\let\cp@ram\relax\fi\trace{sc}{Got result for (\c@tegory)\sb@rmarker:#1: \cp@ram}}
\def\getsbp@ramNoInh#1{\edef\d@sf{\g@tdstat}%
  \getp@r@m{#1}{\c@tprefix\sb@rmarker\d@sf}\ifx\p@ram\relax\getp@ram{#1}{\c@tprefix\sb@rmarker}{\c@tprefix\sb@rmarker}\fi
  \let\cp@ram\p@ram
  \ifx\cp@ram\empty\global\let\cp@ram\relax\fi
  \ifx\cp@ram\relax
    \trace{sc}{No result for \c@tprefix\sb@rmarker:#1}\else
    \trace{sc}{Got result for \c@tprefix\sb@rmarker:#1: \cp@ram}%
  \fi
}
%\setsbp@ram{hascol}{T}
%\setsbp@ram{bgcolour}{0 1 0}
%\setsbp@ram{alpha}{0.2}% 1=solid 0=invisible
\def\m@rker{esb}
\setsbp@ram{bgfigscale}{1}%
\setsbp@ram{fgfigsize}{box}%What is image scaled to?
\setsbp@ram{fgfigscale}{0.2}%
\setsbp@ram{fgfigscaleto}{text}%
\setsbp@ram{fgfigpos}{cl}%Default is cutout left
\setsbp@ram{bgfigscale}{1}%Full size
\setsbp@ram{bgfigscaleto}{text}%
\setsbp@ram{bgfigpos}{pc}%Horizontally centred
\setsbp@ram{hascol}{F}%No default colour.
\setsbp@ram{alpha}{0.2}% 1=solid 0=invisible
\setsbp@ram{borderwidth}{0.5}% in pt
\setsbp@ram{posn}{b}
\setsbp@ram{fgfigspec}{}%Normally no picture.
\setsbp@ram{bgfigspec}{}%Normally no picture.
\setsbp@ram{bgfigalpha}{1.0}%Normally solid colour
\setsbp@ram{bgfigcolour}{0 0 0}%Normally black
%\setsbp@ram{borderfillcolour}{NONE} % For double. (Should match fill@none)
\def\m@rker{textborder}
\setsbp@ram{properties}{nonpublishable}
\setsbp@ram{borderlinewidth}{0.05}
\setsbp@ram{borderwidth}{8}
\setsbp@ram{borderrpadding}{2}
\setsbp@ram{borderlpadding}{2}
\setsbp@ram{bordertpadding}{2}
\setsbp@ram{borderbpadding}{2}
\setsbp@ram{borderoddleft}{T}\setsbp@ram{borderevenleft}{T}
\setsbp@ram{borderoddright}{T}\setsbp@ram{borderevenright}{T}
\setsbp@ram{bordertop}{T}\setsbp@ram{borderbottom}{T}
\S@tCat{}
\let\c@tegory\empty
\def\fb@dfltpad{1}%Default
\edef\fb@lpadding{\fb@dfltpad pt}%Added left of feintbox content
\edef\fb@rpadding{\fb@dfltpad pt}%Added right of feintbox content
\edef\fb@tpadding{\fb@dfltpad pt}%added above and below feintbox content
\edef\fb@bpadding{\fb@dfltpad pt}%added above and below feintbox content
\newif\ifNoTransparency
\NoTransparencyfalse
\newcount\feintb@xnum
\feintb@xnum=1
\def\remove@transpancy#1 #2 #3|#4\E{\@lphacalc{#4}{#1} \@lphacalc{#4}{#2} \@lphacalc{#4}{#3}}
\def\@lphacalc#1#2{\strip@pt{\dimexpr #2\dimexpr #1 pt \relax + 1pt -\dimexpr #1 pt \relax\relax} }

\addtoendhooks{\ifinextended\errmessage{Reached end of book without finding \\esbe.}\fi}
\def\f@intthing#1#2#3{%
  % 1 - alpha
  % 2 - colour (r g b)
  % 3 - item to set
  \trace{e}{f@intthing #1 #2}%
  \edef\thisalpha{#1}%
  \edef\thisRGB{#2}%
  \ifNoTransparency
    \ifdim 1pt = #1 pt \relax\else
      \edef\thisRGB{\x@\remove@transpancy\thisRGB|\thisalpha\noexpand\E }%
      \edef\thisalpha{1}%
      \MSG{* Transparency is disabled, anything behind transparent item will be hidden. #2 -> \thisRGB}%
    \fi
  \fi
  \ifdim 1pt = \thisalpha pt \relax %Not feint
    \x@\special\x@{pdf:code q \thisRGB\space rg}%
  \else 
    \edef\thisalpha{#1 }% Important space
    \x@\special\x@{pdf:put @resources << /ExtGState << /GS0\the\feintb@xnum\space << /Type /ExtGState /CA \thisalpha /ca \thisalpha /AIS false >>  >> >>}%
    \x@\special\x@{pdf:code q /GS0\the\feintb@xnum\space gs \thisRGB\space rg}%
    \global\advance\feintb@xnum by 1
  \fi
  #3%
  \special{pdf:code Q}%
}%

\newif\ifble@d
\def\feintb@x#1#2#3#4#5{%
  % 1 - alpha
  % 2 - colour (r g b)
  % 3 4 5 height, depth, width of colour block
  \trace{e}{feintbox #1 #2 #3, #4, #5}%
  \traceifset{feintb@x}%
  \bgroup
  \hbox to 0pt{%
    \dimen0=#3\advance\dimen0 by \fb@tpadding
    \dimen1=#4\advance\dimen1 by \fb@bpadding
    \dimen2=#5\dimen3=\fb@lpadding
    \advance\dimen2 by \fb@rpadding
    \ifble@d 
      \advance \dimen2 by \rble@d 
      \advance \dimen3 by \lble@d 
      \advance \dimen0 by \tble@d 
      \advance \dimen1 by \bble@d 
    \fi
    \advance\dimen2 by \dimen3
    \trace{e}{Feint box dims: \the\dimen0 +\the\dimen1 *\the\dimen2 , \the\dimen3, skip:\fb@tskip}%
    \hskip -\dimen3 % Backup so the expanded box is centred properly
    \def\c@de{\vrule height \dimen0 depth \dimen1  width \dimen2 }%should be possible to get this working in pdf code, but this works.
    \f@intthing{#1}{#2}{\c@de}%
    \hss
  }%
  \egroup
  \traceifcheck{feintb@x}%
}
\def\fb@htadjust{0pt}
\def\feintbox#1#2#3{% Set a (transparent) coloured box as background for box given in #3. #1: alpha, #2: {r g b}  (all numbers in range 0-1.0)
  \kern-\fb@tskip %Relative positioning of box and border
  \setbox0#3%
  \trace{e}{pre-feintb@x, ht=\the\ht0 - \fb@htadjust, dp=\the\dp0}%
  \setbox0\hbox{\feintb@x{#1}{#2}{\the\dimexpr \ht0 - \fb@htadjust\relax}{\the\dp0}{\the\wd0}\box0}%
  \trace{e}{post-feintb@x, ht=\the\ht0, dp=\the\dp0}%
  \box0
  }

\def\minip@rs@two#1#2\end{%
   \edef\tempc{#1}\edef\tempd{#2}}%
\def\minip@rsepos#1#2\mid#3\end{%Parse just enough of the FgImagePos and pos parameters to determine the \hsize
   \edef\tempa{#1}\edef\tempb{#2}%
   \x@\minip@rs@two#3\end
   \trace{e}{parsed: \tempa(\tempb), \tempc(\tempd)}%
}

\def\p@rseBorder#1 #2\relax{%
  \let\n@xtB=\p@rseBorder
  \def\t@st{#1}%
  \trace{e}{Parsing border parameter '#1' '#2'}%
  \ifx\t@st\empty
    \let\n@xtB\endP@rseBorder
  \else
    \uppercase{\def\uc@ption{#1}}%
    \x@\let\x@\t@st\csname B@rder\uc@ption\endcsname
    \ifx\t@st\relax
      \message{Could not parse Border #1}%
    \else
      \t@st \relax
    \fi
  \fi
  \def\t@st{#2}%
  \ifx\t@st\empty
    \let\n@xtB\endP@rseBorder
  \fi
  \n@xtB#2 \relax
}

\def\endP@rseBorder#1 \relax{%
  \trace{e}{Finished parsing border}% 
}
\def\sb@leftedge{}\def\sb@rightedge{}% Contents of left/right edge boxes
\edef\sb@leftshift{0pt}\edef\sb@rightshift{0pt}%
\newif\ifsbar@rotated
\newdimen\sb@rwidth
\def\@@setsbp@cinswid{% get the width of the insert (or column) that the figure will go into
  \dimen1=\ifnum \c@rrentcols=1 \textwidth \else \colwidth\fi
  %we need to distingiush between things like c,p,h which have sub-specifications and t, tl, tr, etc (which don't)
  \x@\let\csname sb@onlypure-t\endcsname\tr@e
  \x@\let\csname sb@onlypure-b\endcsname\tr@e
  \let\@t@st\tempc
  \ifcsname sb@onlypure-\tempc\endcsname
    \x@\ifx\csname sb@onlypure-\tempc\endcsname\tr@e
      \ifx\relax\tempd\else
        \edef\@t@st{\tempc\tempd}%
      \fi
    \fi
  \fi
  \trace{sc}{Initial width set to \the\dimen1 '\@t@st' (\tempd)}%
  \x@\let\csname sb@height-t\endcsname\textheight
  \x@\let\csname sb@height-b\endcsname\textheight
  \x@\let\csname sb@height-B\endcsname\textheight
  \x@\let\csname sb@height-P\endcsname\textheight
  \x@\let\csname sb@height-F\endcsname\PaperHeight %Good luck with that!
  \x@\let\csname sb@width-t\endcsname\textwidth
  \x@\let\csname sb@width-b\endcsname\textwidth
  \x@\let\csname sb@width-B\endcsname\textwidth
  \x@\let\csname sb@width-P\endcsname\textwidth
  \x@\let\csname sb@width-F\endcsname\PaperWidth %Good luck with that!
  \ifcsname sb@width-\@t@st\endcsname
    \x@\let\x@\sb@height\csname sb@height-\@t@st\endcsname
    \x@\dimen1\csname sb@width-\@t@st\endcsname
    \trace{sc}{Revised width to \the\dimen1}%
  \else
    \let\sb@height\textheight
    \temptrue
    \ifnum\c@rrentcols>1\else
      \tempfalse
    \fi
    \iftemp
      \if\tempc t\relax\dimen1=\colwidth\else
        \if\tempc b\relax\dimen1=\colwidth
      \fi\fi
    \fi
  \fi
  %\ifsbar@rotated
    %\p@cinswid=\sb@height
    %\edef\sb@height{\the\dimen1}%
  %\else
    \p@cinswid=\dimen1
  %\fi
}
\def\ch@ckr@tation{%
  \getsbp@ram{sbarrotation}\ifx\cp@ram\relax\else
    \trace{e}{Side bar rotation is '\cp@ram'}%
    \ifcsname geom@xform@\cp@ram\endcsname
      \setgeomtransform{\cp@ram}%
      \ifnum \pdf@aa=0 
	\sbar@rotatedtrue
	\trace{e}{Side bar is rotated}%
      \fi
    \fi
  \fi
}
\def\s@tsb@rwidth{%Set default / specific width of esb box
  %h*,c*,p* take the current page width as their base
  %tX,bX take the column width 
  %t,b,B take the text width
  %F* takes the whole paper.
  %F* and P* might need to fill the vertical dimension.
  \trace{e}{setsb@rwidth c:\the\colwidth, t:\the\textwidth, h:\the\hsize, \the\c@rrentcols, \m@rker}%
  \sbar@rotatedfalse
  \ch@ckr@tation
  \gdef\sb@leftedge{}\gdef\sb@rightedge{}%
  \gdef\sb@rtopedge{}\gdef\sb@rbotedge{}%
  \xdef\sb@leftshift{0pt}\xdef\sb@rightshift{0pt}%
  \getsbp@ram{borderwidth}\let\b@drwidth\cp@ram
  \@get@Hborder@style@params
  \@get@Vborder@style@params
  \getsbp@ram{posn}%
  \global\advance\im@gecount by 1 % A side bar is a strange sort of image, but there are similarities
  \ifcsname fig\the\im@gecount p@ge\endcsname
    \x@\let\x@\pgn@\csname fig\the\im@gecount p@ge\endcsname
  \else
    \let\pgn@\relax
  \fi
  \ifx\pgn@\relax
    \whichp@ge=\pageno
    \trace{e}{Sidebar \the\im@gecount may not be aligned properly}%
  \else
    \whichp@ge=\pgn@
    \trace{e}{Sidebar \the\im@gecount was on page \pgn@ last time. [\the\pageno]}%
  \fi
  \ifx\cp@ram\relax
    \gdef\sb@rpos{b}%
    \setsbp@ram{posn}{\sb@rpos}%
  \else
    \global\let\sb@rpos\cp@ram
  \fi
  \let\localsb@rpos\sb@rpos
  \getsbp@ram{fgfigpos}\let\fgfigp@s\cp@ram
  \getsbp@ram{fgfigspec}\let\fgfigsp@c\cp@ram
  \getsbp@ramNoInh{w@dth}% Width must NOT be inherited.
  \global\let\fgfig@cutout\false
  \ifx\cp@ram\relax
    \x@\minip@rsepos\fgfigp@s\mid\sb@rpos\end% sets up temp[abcd] 
    \@@setsbp@cinswid% sets p@cinswid
    % Now apply scale..
    \getsbp@ram{scale}\ifx\cp@ram\relax\else
      \dimen1=\cp@ram\dimen1
      \trace{e}{scaled width to \cp@ram x\sb@rpos: \the\dimen1}%
    \fi
    \getsbp@ram{\ifodd\pageno borderoddleft\else borderevenleft\fi}\let\b@drleft\cp@ram
    \getsbp@ram{\ifodd\pageno borderoddright\else borderevenright\fi}\let\b@drright\cp@ram
    \getsbp@ram{borderwidth}\let\b@drwidth\cp@ram
    % And shrink further if s has been specified for the fg picture
    \ifx\fgfigsp@c\relax\else
      \p@rsefigscale{fg}%
      \dimen0=\fgfig@dim@wd
      \edef\fgfigsp@c{\fgfigsp@c \fgfig@scale@wd}%
      %\getsbp@ram{fgfigscale}%
      %$\ifx\cp@ram\relax
        %$\dimen0=0.2\dimen1
      %$\else
        %$\dimen0=\cp@ram\dimen1
      %$\fi
      \x@\x@\x@\p@rseLoc\x@\tempa\tempb\end
      \ifx\tempa \pos@Side
        \trace{e}{Forground picture will be \fgfigsp@c}%
        \advance\dimen1 by -\dimen0
        \ifx\tempb\empty\message{No side defined for foreground image in sidebar class '\c@tegory'. Assuming outer.}%
          \let\tempb\@lignOuter
        \fi
        \def\t@pedge{\vfil}%
        \def\b@tedge{\vfil}%
        \ifx\l@cspec@b\@lignLeft
          \ifx\l@cspec@c\@lignTop\def\t@pedge{}\fi
          \ifx\l@cspec@c\@lignBot\def\b@tedge{}\fi
          \xdef\sb@leftedge{\t@pedge\noexpand\sb@ins@rtpic\fgfigsp@c|\the\dimen0 \noexpand\E\noexpand\let\noexpand\sb@leftedge\noexpand\empty\b@tedge}%
          \trace{e}{leftedge defined to sb@ins@rtpic\fgfigsp@c...}%
          \advance\dimen0 by \fb@lpadding
          \xdef\sb@leftshift{\the\dimen0}%
        \else
          \ifx\l@cspec@c\@lignTop\def\t@pedge{}\fi
          \ifx\l@cspec@c\@lignBot\def\b@tedge{}\fi
          \xdef\sb@rightedge{\t@pedge\noexpand\sb@ins@rtpic\fgfigsp@c|\the\dimen0 \noexpand\E\noexpand\let\noexpand\sb@rightedge\noexpand\empty\b@tedge}%
          \advance\dimen0 by \fb@rpadding
          \xdef\sb@rightshift{\the\dimen0}%
        \fi
        \let\fgfigsp@c\empty
      \else% tempa!=s
        \def\l@ftedge{\hfil}%
        \ifx\l@cspec@b\@lignLeft\def\l@ftedge{}\else\ifx\l@cspec@b\@lignRight\def\l@ftedge{\hfill}\fi\fi
        \ifx\tempa\@lignTop
          \xdef\sb@rtopedge{\l@ftedge\noexpand\sb@ins@rtpic\fgfigsp@c|\the\dimen0 \noexpand\E}%
        \fi
        \ifx\tempa\@lignBot
          \xdef\sb@rbotedge{\noexpand\@@sb@rbotedge{\l@ftedge\noexpand\sb@ins@rtpic\fgfigsp@c|\the\dimen0 \noexpand\E}}%
        \fi
        \x@\x@\x@\p@rseLoc\x@\tempa\tempb\end
        \ifx\tempa\loc@Cut
          \prevdepth=1000pt % Ignore the presence of the figure
          \let\w@tsit\gr@phic
          \def\w@tsit{graphicInSidebar}%
          \trace{e}{Cutout for esb category '\c@tegory' (\tempa\tempb) will be: \fgfigsp@c}%
          \xdef\sb@rboxname{\w@tsit|\c@tegory}%
          \x@\let\x@\this@pic\csname\sb@rboxname\endcsname
          \ifx\this@pic\relax
            \trace{e}{Creating a new box}% 
            \g@nbox{\sb@rboxname}%
          \fi
          \x@\let\x@\this@pic\csname\sb@rboxname\endcsname
          \global\setbox\this@pic\hbox{\x@\sb@ins@rtpic\fgfigsp@c|\the\dimen0 \E}%
          \trace{e}{Saved box \the\wd\this@pic x\the\ht\this@pic top:\b@drtop}%
          \global\let\fgfig@cutout\tr@e
          \xdef\l@cspec{\l@cspec@b}%
          \ifx\l@cspec@c\empty\gdef\c@tskip{0}\else\xdef\c@tskip{\l@cspec@c}\fi%
          \x@\xdef\x@\c@tskip{\strip@pt{\dimexpr \c@tskip pt + 1 pt \relax}}% floating pt addition
          \x@\c@tskipparse\c@tskip..\E
          \trace{e}{Adjusted c@tskip \c@tskip (\c@tskipfrac)}%
          \gdef\sb@rtopedge{%
            \dimen0=\ht\this@pic \advance\dimen0 by \dp\this@pic %
            \count255=\dimen0
            \getsbp@ram{spacebeside}\ifx\cp@ram\relax\let\sp@cebeside=\DefaultSpaceBeside\else\let\sp@cebeside\cp@ram\fi
            \divide\count255 by \baselineskip
            \advance\count255 by 1
            \trace{e}{Cutout should be \the\count255 lines}%
            \def\thisw@tsitslop{{0}{0}}%
            \d@figureCut\this@pic\unvbox\voidb@x}%
        \fi
      \fi
    \fi
    \advance\dimen1 by -\dimexpr \fb@lpadding + \fb@rpadding\relax
    \ifx\b@drleft\tr@e\ifdim \border@lpadding>-\b@drwidth pt
        \advance\dimen1 by -\b@drwidth pt 
        \advance\dimen1 by -\border@lpadding
    \fi\fi
    \ifx\b@drright\tr@e\ifdim \border@rpadding>-\b@drwidth pt   
        \advance\dimen1 by -\b@drwidth pt 
        \advance\dimen1 by -\border@rpadding
    \fi\fi
    \if h\tempc\relax\else% Don't store width for class h or class p sidebars, as they may appear full width or single column
      \if p\tempc\relax\else
        \setsbp@ram{w@dth}{\the\dimen1}%
      \fi
    \fi
    \let\t@mploc\tempc % Will be overwriting tempc soon
    \ifx\tempd\empty\else
      \tempfalse
      \ifx\t@mploc\loc@Full\temptrue\else
        \ifx\t@mploc\loc@Page\temptrue\fi
      \fi
      \iftemp
        \x@\minip@rs@two\tempd\end
        \if f\tempd\relax
          \setsbp@ram{outerh@ight}{\sb@height}%
        \fi
      \fi
    \fi  
  \else
   \dimen1=\cp@ram
  \fi
  \sb@rwidth=\dimen1
  \trace{e}{text box width is \the\sb@rwidth}%
  \hsize=\sb@rwidth
}

% Sidebar 
\def\@sb{\let\oldc@tegory\c@tegory\let\oldc@prefix\c@tprefix
  \@@sbar{esb}%
}

\newif\ifble@dthisbar
\def\@@sbar#1{%
  \trace{e}{Starting side bar (\sb@rmarker) from \m@rker, \the\lastdepth}%
  \ble@dthisbarfalse
  %\endlastp@rstyle{\ss@Sbar}% Apply spaceafter, etc. NO, because we might be triggered from a verse
  \ifsk@pping \egroup 
    \endlastp@rstyle{\ss@Sbar}% Apply spaceafter, etc. 
  \fi % if we were skipping nonpublishable text, end that mode
  \xdef\oldlastdepth{\the\lastdepth}\lastdepth=0pt
  \xdef\stylet@pe{\ss@Sbar}%                                                                   %(2)
  \ifhe@dings\endhe@dings\fi% Sidebars should close a heading section.
  \ifdoingt@ble\endt@ble\fi% Sidebars should close a table, too
  % kill cached possParam-\m@rker values  
  \kill@PossParamCache
  \edef\hs@ze{\the\hsize}% Size of outer content
  \s@tc@tpr@fix
  \edef\sb@numcuts{\the\@numcuts}%
  \trace{e}{Saving cutouts: \sb@numcuts}%
  \ifnum\sb@numcuts>0
    \save@cutouts{esb}%
  \fi
  \mcpush{\ss@Sbar}{\sb@rmarker}\trace{e}{ESB}\setbox\extb@x\vbox\bgroup\cancelcutouts
    \edef\sb@rmarker{#1}%
    \let\m@rker\sb@rmarker
    \s@tsb@rwidth\inextendedtrue
    \ifble@d\ble@dfalse\ble@dthisbartrue\fi
    \mcpush{\ss@Para}{\if@ntro ip\else p\fi}%
    \textwidth=\hsize
    \sb@rchunkh@ight=\maxdimen
    \prevdepth=1000pt
    \c@th@@ks{\st@rtsb@r}%
}
\newdimen\sb@rchunkh@ight
\newdimen\sb@rus@dheightl
\newdimen\sb@rus@dheightr
\def\st@rtsb@r{%Gets run by \cat..\cat*, if one is used. 
  \c@th@@ks{}%Once per box is enough!
  \ifble@dthisbar
    \ble@dtrue
    \s@tsb@rwidth
    \ble@dfalse
  \else
    \s@tsb@rwidth
  \fi
  \trace{e}{startESB \the\sb@rwidth,\the\hsize (ble@dthisbar\ifble@dthisbar true\else false\fi)}%
  \kill@PossParamCache
  \sb@rchunkh@ight=\maxdimen
  \getsbp@ram{break}\ifx\cp@ram\relax\else \if\cp@ram F\else
   \trace{e}{break:\cp@ram}%
   \ifx\cp@ram\tr@e
     \sb@rchunkh@ight=0.2\textheight
   \else
     \sb@rchunkh@ight=\cp@ram\textheight\relax
   \fi
   \floatingpenalty=500\insertpenalties=0
  \fi\fi
  %\egroup\setbox\extb@x\vbox\bgroup
  \sb@rtopedge%\ifhmode\endgraf\fi
  \trace{e}{sidebar really starting \the\hsize, \the\textwidth, \the\sb@rwidth}%
}
\newif\ifdoIntskip % Skip should be applied in the text area, not 

\def\@@sb@rbotedge#1{\bgroup
  \setbox0\hbox{#1}\p@cheight=\ht0
  \loop 
    \advance\p@cheight by -\baselineskip 
  \ifdim\p@cheight>0pt \repeat
 % \advance\p@cheight by \baselineskip
  \vskip-\p@cheight
  \vskip 0pt
  \unhbox0
  \egroup}


\def\@p@rsef@gsc@le#1|#2|#3\E{% \v@lpfx  fg or bg, params scale which might be of form Xscale x Yscale
  \edef\tmp{#2}\ifx\tmp\empty
    \getsbp@ram{\v@lpfx figscaleto}%
    \ifx\cp@ram\relax 
      \x@\edef\csname \v@lpfx @scale@ref\endcsname{text}%
    \else
      \trace{e}{Figure will be scaled relative to \v@lpfx}%
      \lowercase{\x@\edef\csname \v@lpfx @scale@ref\endcsname{\cp@ram}}%
    \fi
    \p@rsef@gsc@le #1xx\E
  \else
    \trace{e}{Figure will be scaled relative to #1}%
    \lowercase{\x@\edef\csname \v@lpfx @scale@ref\endcsname{#1}}%
    \p@rsef@gsc@le #2xx\E
  \fi
}

\def\fg@scale@ref{text} % Defaults
\def\bg@scale@ref{text}

\def\f@grefdim#1{%
  \traceifset{@f@grefdim}%
  \traceifcheck{@f@grefdim}%
  \x@\let\x@\scale@ref\csname #1@scale@ref\endcsname
  \ifcsname \scale@ref @ref@w@dth\endcsname
    \trace{e}{Scaling #1 image to \scale@ref}%
    \x@\let\x@\scale@height\csname\scale@ref @ref@h@ight\endcsname
    \x@\let\x@\scale@width\csname\scale@ref @ref@w@dth\endcsname
  \else
    \message{Supplied scale reference for (\c@tprefix esb) #1 image ('\scale@ref')  not recognised!}%
    \x@\let\x@\scale@height\csname text@ref@h@ight\endcsname
    \x@\let\x@\scale@width\csname text@ref@w@dth\endcsname
  \fi
  \trace{e}{Scale size: \scale@width x\scale@height}%
  \ifcsname @@\scale@ref @ofs\endcsname
    \x@\edef\x@\fig@lskip{\the\dimexpr -\csname @@\scale@ref @ofs\endcsname \relax plus 1 fil \ifble@d minus \lble@d\fi}%
    \x@\edef\x@\fig@rskip{\the\dimexpr \csname @@\scale@ref @ofs\endcsname \relax plus 1 fil \ifble@d minus \rble@d\fi}%
    \trace{e}{Set skips for #1 (\scale@ref) image to <\fig@lskip|\fig@rskip>}%
  \else
    \x@\edef\x@\fig@lskip{\the\dimexpr -\csname @@text@ofs\endcsname \relax plus 1 fil}%
    \x@\edef\x@\fig@rskip{\the\dimexpr \csname @@text@ofs\endcsname \relax plus 1 fil}%
    \trace{e}{Set skips for #1 (unknown:\scale@ref) image to <\fig@lskip|\fig@rskip>}%
  \fi
}
% ch@ckfigdim needs defining. It gets called after the image trial load with dimen3=\wd figure box, dimen4=\ht figurebox
% (set by sb@ins@rtpic) 
%This p@rsef@gsc@le gets called with dimen1=\wd sidebar , \dimen0=scratch, \dimen2=scratch.
% f@grefdim should do something more sane
\def\p@rsef@gsc@le#1x#2x#3\E{% \v@lpfx  fg or bg, params scale which might be of form Xscale x Yscale
  \f@grefdim{\v@lpfx}%
  \edef\fig@h@dim{#1}\edef\fig@v@dim{#2}%
  \trace{e}{p@rsef@gscale:#1,#2,#3}%
  \ifdim\ht\extb@x=0pt
    \def\fig@v@dim{}% 
  \fi
  \def\ch@ckfigdim{\tempfalse}%
  \ifx\fig@v@dim\empty
    \x@\def\csname \v@lpfx fig@scale@ht\endcsname{}%
    \def\ch@ckfigdim{\trace{e}{Checking image height \the\dimen4 <\scale@height?}\ifdim\dimen4>\scale@height \def\resc@le{ height \scale@height}\temptrue\fi}%
  \else
    \dimen0=\fig@v@dim\dimexpr \scale@height \relax
    \x@\edef\csname \v@lpfx fig@scale@ht\endcsname{ height \the\dimen0}%
    \trace{e}{Set \v@lpfx fig@scale@ht to \the\dimen0}%
  \fi
  \ifx\fig@h@dim\empty
    \x@\def\csname \v@lpfx fig@scale@wd\endcsname{}%
    \x@\edef\csname \v@lpfx fig@dim@wd\endcsname{\scale@width}%
    \def\ch@ckfigdim{\trace{e}{Checking width \the\dimen3 <\scale@width?}\ifdim\dimen3>\scale@width \def\resc@le{ width \scale@width}\temptrue\fi}%
  \else
    \dimen0=\fig@h@dim\dimexpr \scale@width\relax
    \x@\edef\csname \v@lpfx fig@scale@wd\endcsname{ width \the\dimen0}%
    \x@\edef\csname \v@lpfx fig@dim@wd\endcsname{\the\dimen0}%
    \trace{e}{Set \v@lpfx fig@scale@wd to \the\dimen0}%
  \fi
}

\def\p@rsefigscale#1{%interpret bacground figure scale
  \ifdim\wd\extb@x>0pt
    \dimen1=\wd\extb@x
  \fi
  \getsbp@ram{#1figscale}%
  \edef\v@lpfx{#1}%
  \x@\@p@rsef@gsc@le\cp@ram ||\E
  \trace{e}{parsed figure scale: \csname #1fig@scale@wd\endcsname \csname #1fig@scale@ht\endcsname}%
}

\x@\def\csname \sb@rb@x @warning\endcsname#1#2{\msg{converted sidebar placement "#1" to "#2" in single-column layout}}

\def\localsb@rpos{}
\edef\sb@lastpen{0}
\def\@sbe{%
  \trace{e}{@sbe: \sb@rmarker|\localsb@rpos}%
  \endlastp@rstyle{esbe}% Fire any triggers, etc.
  \xdef\sb@lastpen{\the\lastpenalty}%
  \trace{e}{@sbe: ld:\lastdepth, lp:\sb@lastpen}%
  \let\m@rker\sb@rmarker
  \global\let\sb@rpos\localsb@rpos
  \ifhe@dings\endhe@dings\fi
  \ifdoingt@ble\endt@ble\fi% Sidebar end should close a table, too
  \let\@fter@sbe\relax
  \ifinextended
    \let\@fter@sbe\@@sbe
  \else
    \let\c@t@gory=\empty \trace{e}{ebse called without esb}%
  \fi
  \@fter@sbe
}

\newif\ifsb@rfillpage

\def\figbleed{5pt}
\def\@@sbe{%The real meat of an esbe
   \getsbp@ram{posn}%
   \gdef\pic@lign{c}% How does the sidebar align within the box?
   \gdef\picV@lign{c}% How does the sidebar align vertically on the page (Page/Full only)
   \edef\l@gstring{{\the\im@gecount}{Sidebar-\c@tegory}{\cp@ram}{\c@rref}{\the\hsize}}% Fully expand figure parameters, but leave page no. to be expanded later.
   \x@\writefigp@gelog\x@{\l@gstring}%
   \global\let\t@st\false
   \end@llpoppedstyles{\ss@Sbar*}% May also close groups
   \trace{e}{Stack now \mcstack}%
   \kill@PossParamCache
   \ifhmode\par\fi
   \ifx\sb@rbotedge\empty\else
     \ifdim\prevdepth<\baselineskip
       \vskip-\prevdepth
     \fi
     \sb@rbotedge\ifhmode\endgraf\fi
   \fi
   %Pass some values out of the group (which won't affect enclosing side-bars)
   \trace{e}{Ending group. hsz=\the\hsize , \the\sb@rwidth}%
   \gdef\@fteresb{}%
   \ifdim\hsize>\sb@rwidth
     \message{**  WARNING: something has changed the text width in the sidebar to be larger than the value calculated earlier}%  
   \fi
   \global\edef\sb@rchunkheight{\the\sb@rchunkh@ight}%
   \global\p@ppingtrue%suppress another end@llpoppedstyles
   \global\let\c@t@gory=\c@tegory\egroup 
  \message{\the\ht\extb@x+\the\dp\extb@x * \the\wd\extb@x}%
     \global\tracingassigns=0
  % ^^^ END OF GROUP! ^^^
  %
  \global\p@ppingfalse%
  \trace{e}{ESBE, chunking every \sb@rchunkheight}%
  \ch@ckr@tation
  \ifx\fgfig@cutout\tr@e
    \cancelcutouts
  \fi
  \let\p@stpenalty=\sb@lastpen %recover value from group
  \def\prepenalty{10}%recover value from group
  \let\c@tegory=\c@t@gory %recover value from group, without making global changes.
  \s@tc@tpr@fix
  \x@\minip@rs@two\sb@rpos\end % Parse the location, ready to get the size of this sidebar's container
  \@@setsbp@cinswid 
  \trace{e}{ESBE \c@tprefix(\the\ht\extb@x*\the\wd\extb@x) in \sb@rpos: wd=\the\p@cinswid, \the\sb@rwidth}%
  %\showbox\extb@x
  %\inextendedfalse
  \let\os@vedtpadding=\fb@tpadding%%Preserve old value
  \let\os@vedbpadding=\fb@bpadding%%Preserve old value
  \let\os@vedlpadding=\fb@lpadding%
  \let\os@vedrpadding=\fb@rpadding%
  \get@border@style@params
  \getsbp@ram{bgfigspec}\let\bgf@gsp@c\cp@ram
  \getsbp@ram{bgfiglow}\let\bgf@gl@w\cp@ram%
  \getsbp@ram{hascol}\let\h@scol\cp@ram%
  \getsbp@ram{bgfigalpha}\let\bgf@galpha\cp@ram
  \getsbp@ram{bgfigcolour}\let\bgf@gcol\cp@ram
  \getsbp@ram{spacebeside}\ifx\cp@ram\relax\let\sp@cebeside=\DefaultSpaceBeside\else\let\sp@cebeside\cp@ram\fi
  \let\s@vedtpadding=\fb@tpadding%%Preserve old value
  \let\s@vedbpadding=\fb@bpadding%%Preserve old value
  \prepb@rderdims
  \tempfalse
  \trace{e}{= \bgf@gsp@c}%
  %Does the box have a defined size? 
  %
  % Reference dimensions
  \edef\flble@d{\ifdim \lble@d > \figbleed \figbleed\else\lble@d\fi}% Figure bleeds
  \edef\frble@d{\ifdim \rble@d > \figbleed \figbleed\else\rble@d\fi}%
  \edef\ftble@d{\ifdim \tble@d > \figbleed \figbleed\else\tble@d\fi}%
  \edef\fbble@d{\ifdim \bble@d > \figbleed \figbleed\else\bble@d\fi}%
  \edef\text@ref@w@dth{\the\wd\extb@x}%
  \edef\text@ref@h@ight{\the\dimexpr \ht\extb@x + \dp\extb@x\relax}%
  \edef\text@ref@lskip{0pt plus 1fil minus 1fil}%
  \edef\text@ref@rskip{0pt plus 1fil minus 1fil}%
  % Fall back to text lskip/rskip for inner
  \edef\inner@ref@w@dth{\the\wd\extb@x}%
  \edef\outer@ref@w@dth{\the\dimexpr \p@cinswid\relax}%
  \edef\border@ref@w@dth{\the\dimexpr\wd\extb@x + \b@dr@left@pad + \b@dr@right@pad\relax}%
  \edef\box@ref@w@dth{\the\dimexpr \wd\extb@x + \fb@lpadding + \fb@rpadding\relax}%
  \def\@@text@ofs{0pt}%
  \def\@@box@ofs{\fb@lpadding}%
  \def\@@border@ofs{\b@dr@left@pad}%
  \def\@@outer@ofs{\the\dimexpr \ifdim \fb@lpadding < \b@dr@left@pad \b@dr@left@pad \else \fb@lpadding\fi\relax}%
  \def\@@colbleed@ofs{\the\dimexpr \ifdim \fb@lpadding < \b@dr@left@pad \b@dr@left@pad \else \fb@lpadding\fi + \lble@d\relax}%
  \def\@@bleed@ofs{\the\dimexpr \ifdim \fb@lpadding < \b@dr@left@pad \b@dr@left@pad \else \fb@lpadding\fi + \flble@d\relax}%
  %\edef\bleed@ref@rskip{\the\dimexpr - \outer@ref@lskip - \lble@d\relax plus 1 fil}%
  \getsbp@ramNoInh{outerh@ight}% must NOT be inherited.
  \ifx\cp@ram\relax
    \let\sb@rheightspec\empty
    \edef\outer@ref@h@ight{\the\textheight}%
    \edef\inner@ref@h@ight{\the\dimexpr \textheight - \b@dr@bot@edge - \b@dr@top@edge\relax}%
    \edef\box@ref@h@ight{\the\dimexpr \ht\extb@x+\dp\extb@x +\fb@tpadding + \fb@bpadding\relax}%
    \edef\border@ref@h@ight{\the\dimexpr\ht\extb@x + \b@dr@top@pad + \b@dr@bot@pad\relax}%
    \sb@rfillpagefalse
  \else
    \sb@rfillpagetrue
    \edef\outer@ref@h@ight{\the\dimexpr \cp@ram \relax}%
    \edef\inner@ref@h@ight{\the\dimexpr \cp@ram - \b@dr@bot@edge - \b@dr@top@edge\relax}%
    \edef\border@ref@h@ight{\the\dimexpr\cp@ram + \border@tpadding + \border@tpadding +\b@drwidth pt * \b@drvsides \relax}%
    \edef\box@ref@h@ight{\the\dimexpr \cp@ram\relax}%
    \ifble@d
      \edef\colbleed@ref@h@ight{\the\dimexpr \cp@ram + \tble@d + \bble@d\relax}%
      \edef\colbleed@ref@w@dth{\the\dimexpr \outer@ref@w@dth + \lble@d + \rble@d\relax}%
      \edef\bleed@ref@h@ight{\the\dimexpr \cp@ram + \ftble@d + \fbble@d\relax}%
      \edef\bleed@ref@w@dth{\the\dimexpr \outer@ref@w@dth + \flble@d + \frble@d\relax}%
    \else
      \let\colbleed@ref@h@ight\outer@ref@h@ight
      \let\colbleed@ref@w@dth\outer@ref@w@dth
      \let\bleed@ref@h@ight\outer@ref@h@ight
      \let\bleed@ref@w@dth\outer@ref@w@dth
    \fi
    \edef\sb@rheightspec{to \inner@ref@h@ight}%
    \trace{e}{Sidebar will be \sb@rheightspec, paper:\the\PaperHeight, text:\the\textheight}%
  \fi
  %
  \earlybgimagefalse
  \ifx\bgf@gsp@c\relax
    \let\t@mpfigspec\relax
  \else
    \trace{e}{Sidebar has image}%
    \p@rsefigscale{bg}%
    \edef\t@mpfigspec{\bgf@gsp@c \bgfig@scale@ht \bgfig@scale@wd|\scale@width}%
    \setbox\newpicb@x=\hbox{\x@\sb@ins@rtpic\t@mpfigspec\E}%
    \ch@ckfigdim
    \iftemp% It doesn't fit
      \trace{e}{Image oversize}%
      \getsbp@ram{bgfigoversize}%
      \ifx\cp@ram\Img@Shrink\edef\t@mpfigspec{\bgf@gsp@c \resc@le|\scale@width}\else
        \ifx\cp@ram\Img@Distort\edef\t@mpfigspec{\bgf@gsp@c \bgfig@scale@ht \bgfig@scale@wd \resc@le|\scale@width}\else
          \ifx\cp@ram\Img@Crop\message{Image crop not supported}\tempfalse\else
            \tempfalse % No action
          \fi
        \fi
      \fi
      \iftemp
        \setbox\newpicb@x=\hbox{\x@\sb@ins@rtpic\t@mpfigspec\E}%
      \fi
    \fi
    \ifdim\wd\newpicb@x >\wd\extb@x
      \setbox\newpicb@x\hbox to 0pt{\box\newpicb@x\hss}%
    \fi
    \def\sb@rchunkheight{\the\maxdimen}% Background images and page-broken chunks are incompatible 
    \tempfalse
    \if\bgf@gl@w F%
      \earlybgimagetrue
    \fi
    \if\h@scol F%
      \earlybgimagetrue
    \fi
  \fi
  \let\w@tsit\sb@rb@x % Make sure the item is identified properly - used in debugging and .delayed file
  \let\loc@ption\sb@rpos
  \picUsesInstrue \picNarrowfalse
  \ifdim\wd\extb@x=\p@cinswid\else
    \picNarrowtrue
  \fi
  \x@\p@rsePicUse\loc@ption\end  %set loc@ption and discover if there's to be an insert. (sets \picUsesInsfalse if not)
  \doIntskipfalse% Upper baseline-preserving skip should be outside the border, not in the text area.
  \ifpicUsesIns
    \doIntskiptrue %Skip inside the frame, not outside it, so that frame lines up with page boundaries
    \dimen4=-\b@dr@top@edge
  \else
    \dimen4=-\b@dr@top@pad
  \fi
  \advance\dimen4 by -\ht\extb@x
  \loop\ifdim\dimen4> \onel@neunit
    \advance\dimen4 by -\onel@neunit
  \repeat
  \loop\ifdim\dimen4<0pt
    \advance\dimen4 by \onel@neunit
  \repeat
  \edef\b@xgridadj{\the\dimen4}%
  \ifdoIntskip
    \trace{e}{hts: \the\headingtopspace, b@xintadj:\b@xintadj, b@xgridadj\b@xgridadj}%
    \setbox\extb@x\vbox{%\dbghrule
      \kern\b@xgridadj
      \unvbox\extb@x}%
  \fi
  \ifearlybgimage
    %modify the esbbox to include the picture
    \trace{e}{Including background image below contents: \t@mpfigspec}%
    \ifx\sb@rheightspec\empty
      \edef\sb@rheightspec{to \the\ht\extb@x}%
    \fi
    \setbox\extb@x\vbox\sb@rheightspec{\lineskiplimit=0pt \baselineskip=0pt
      \vbox to 0pt{\vbox \sb@rheightspec{\vss \hbox to \wd\extb@x{\f@intthing{\bgf@galpha}{\bgf@gcol}{\hbox{\hskip-\sb@leftshift\unhbox\newpicb@x\hskip-\sb@rightshift}}\hss}\vss}\vss}%
      \vbox\sb@rheightspec{\unvbox\extb@x}%
    }%
    \let\bgf@gsp@c\relax %used it, so delete it. If it were still there later, we'd use it.
  \fi
  \ifx\h@scol\tr@e
    \getsbp@ram{alpha}\global\let\pdf@lpha=\cp@ram
    \getsbp@ram{bgcolour}\let\pdfBGc@l=\cp@ram
  \fi
  \let\fstb@drtop=\b@drtop % Top border for the first chunk? 
  \let\lstb@drbot=\b@drbottom % Bottom border for the last chunk?
  \ifdim\ht\extb@x > 0.9\textheight
    \message{*** WARNING: Sidebar or colophon might not print on page \folio. (\the\ht\extb@x\space high, and  page is \the\textheight).}%
  \fi
  \def\lstch@nkdp{0pt}%
  \loop %Breakable box:
    \ifdim\ht\extb@x>\sb@rchunkheight % to enable page splitting, the box should be chunked.
      \dimen1=\sb@rchunkheight
      {% 
        \@LOOP
          \setbox2=\copy\extb@x
          \global\setbox\extchunkb@x=\vsplit\extb@x to \dimen1
          \ifnum\badness>999999
            \setbox\extb@x=\box2
            \advance\dimen1 by \baselineskip %Couldnt split, make the chunk bigger.
            \trace{e}{Chunk grown to \the\dimen1}%
        \@REPEAT
      }%
      \setbox\extchunkb@x=\vbox\sb@rheightspec{\unvbox\extchunkb@x}% Reset to natural size
      \let\b@drbottom=\relax
    \else
      \setbox\extchunkb@x=\vbox\sb@rheightspec{\unvbox\extb@x}%
      \let\fb@tpadding=\s@vedtpadding
      \let\fb@bpadding=\s@vedbpadding
      \let\b@drbottom=\lstb@drbot
    \fi
    %\showbox\extchunkb@x
    % Shift text for side image(s) 
    \ifdim\sb@leftshift>0pt
      \setbox0\vbox to\ht\extchunkb@x{\sb@leftedge}%
      \trace{e}{Adding \sb@leftshift to left of box}%
      \setbox\extchunkb@x\vbox{\hbox{\hbox to \sb@leftshift{\hss\box0\hskip\fb@rpadding}\box\extchunkb@x}}\fi
    \ifdim\sb@rightshift>0pt
      \setbox0\vbox to\ht\extchunkb@x{\sb@rightedge}%
      \trace{e}{Adding \sb@rightshift to right of box}%
      \setbox\extchunkb@x\vbox{%
        \hbox{\box\extchunkb@x\hbox to \sb@rightshift{\hskip\fb@lpadding\box0\hss}}}\fi
    \trace{e}{Chunk of extbox is \the\ht\extchunkb@x+\the\dp\extchunkb@x*\the\wd\extb@x (\the\ht\extb@x*\the\wd\extb@x) in insert \the\p@cinswid}%
    \xdef\thsch@nkdp{\the\dimexpr \dp\extchunkb@x\relax}%
    \getsbp@ram{spaceafter}%
    \ifx\cp@ram\relax
      \edef\sp@ceafter{0pt}%
    \else
      \edef\sp@ceafter{\the\dimexpr\cp@ram\verticalsp@ceunit\relax}%
    \fi
    \trace{e}{Before bg: \the\ht\extchunkb@x+\the\dp\extchunkb@x * \the\wd\extchunkb@x}%
    \ifx\h@scol\tr@e
      \setbox\extchunkb@x\vbox{%
        \ifx\bgf@gsp@c\relax\else
         \trace{e}{Including background image below everything: \t@mpfigspec}%
          \vbox to 0pt{\vbox to\ht\extchunkb@x{\vss\f@intthing{\bgf@galpha}{\bgf@gcol}{\box\newpicb@x}\vss}\vss}%
        \fi
        \let\fb@htadjust=\lstch@nkdp
        \trace{cov}{Space removal around feintbox (\ifble@d bleed: \the\dimexpr \tble@d\relax,\the\dimexpr \bble@d\relax\else none\fi)}%
        \ifble@d
          %\hbox{\hskip-\lble@d\vrule width 1pt \vbox{\feintbox{\pdf@lpha}{\pdfBGc@l}{\box\extchunkb@x}}\hskip-\rble@d}%
          \vskip -\dimexpr \tble@d\relax
          \feintbox{\pdf@lpha}{\pdfBGc@l}{\box\extchunkb@x}%
          \vskip -\dimexpr \bble@d\relax
        \else
          \feintbox{\pdf@lpha}{\pdfBGc@l}{\box\extchunkb@x}%
        \fi
      }%
      %\ifdim\the\ht\extchunkb@x = 840.04684pt \showbox\extchunkb@x \fi
      \trace{e}{After bg: \the\ht\extchunkb@x+\the\dp\extchunkb@x}%
      \xdef\lstch@nkdp{\thsch@nkdp}%
    \else
      \ifsb@rfillpage
        \dp\extchunkb@x=\fb@bpadding
      \fi
    \fi
    \trace{e}{Chunk is (with box) \the\ht\extchunkb@x+\the\dp\extchunkb@x}%
    \let\b@drtop=\fstb@drtop
    \ch@ckhasb@rder
    \prepb@rderbox\extchunkb@x 
    \ifx\picl@c\loc@Inl
      \headingtopspace=0pt %-\oldlastdepth
    \else
      \headingtopspace=0pt
    \fi
    \ifble@d
      %\advance\headingtopspace -\tble@d
    \fi
    \ifhasb@rder\ifsb@rfillpage
        %\advance\headingtopspace \fb@tpadding
    \fi\fi
    \trace{e}{heading topspace=\the\headingtopspace}%
    \getsbp@ram{spacebefore}\ifx\cp@ram\relax\else
      \advance\headingtopspace by \cp@ram\verticalsp@ceunit
    \fi
    \ifhasb@rder % any border
      \dob@rder % Actually generate the border
      \dimen1=\dimexpr -\b@rderbox@depth\relax
      \advance\dimen1 by -\sp@ceafter
      \relax
      %\message{onel@neunit=\the\onel@neunit, dimen1=\the\dimen1}%
      \@LOOP\ifdim\dimen1 <0.1\onel@neunit
        \advance\dimen1 by \onel@neunit
      \@REPEAT
      \ifsb@rfillpage
       \trace{e}{Killing kerns/skips as box is stretch-to-fit (\sb@rheightspec)}%
        \dimen1=0pt \def\b@xextadj{0pt}\def\b@xintadj{0pt}%
      \fi
      \trace{e}{Border Box \the\ht\b@rderbox +\the\dp\b@rderbox x \the\wd\b@rderbox (\b@rderbox@height x\b@rderbox@width) is offset \the\headingtopspace (v) and <\b@dr@left@pad|\b@dr@right@pad> (h). Final adjust\the\dimen1}%
      \trace{e}{Box adj (\c@tegory):\the\headingtopspace, spacing: \b@xintadj / \b@xextadj,  \ifdoIntskip 1st \else 2nd\fi}%
      \x@\setbox\extchunkb@x\vbox{\lineskiplimit=0pt \lineskip=0pt \baselineskip=0pt
        \vskip\headingtopspace\ifdoIntskip\kern\b@xintadj\else\vskip\b@xextadj\fi
        \vbox to 0pt{\vskip\b@dr@top@pad\hbox to \b@rderbox@width{\hskip\b@dr@left@pad\box\extchunkb@x\hskip\b@dr@right@pad}\vss}\penalty10000\box\b@rderbox
        \kern\dimen1
        \vskip\sp@ceafter
      }%
    \else
      \ifBoxLikeBorder
        \dimen1=\dimexpr -\dp\extchunkb@x\relax
        \advance\dimen1 by -\sp@ceafter
      \else
        \dimen1=-\sp@ceafter
      \fi
      \relax
      %\message{onel@neunit=\the\onel@neunit, dimen1=\the\dimen1}%
      \@LOOP\ifdim\dimen1 <0.1\onel@neunit
        \advance\dimen1 by \onel@neunit
      \@REPEAT
      \trace{e}{Borderless Box \the\ht\extchunkb@x +\the\dp\extchunkb@x x \the\wd\extchunkb@x is offset \the\headingtopspace+\ifdoIntskip\b@xintadj\else\b@xextadj\fi+\b@dr@top@pad (v) \ifBoxLikeBorder and <\b@dr@left@pad|\b@dr@right@pad> (h)\fi. Final adjust\the\dimen1,\sp@ceafter}%
      \x@\setbox\extchunkb@x\vbox{\lineskiplimit=0pt \baselineskip=0pt \vskip\headingtopspace\ifdoIntskip\kern\b@xintadj\else\vskip\b@xextadj\fi%
          \ifBoxLikeBorder
            \vbox{\vskip\b@dr@top@pad
              \hbox{\hskip\b@dr@left@pad\box\extchunkb@x\hskip\b@dr@right@pad}}%
          \else
            \box\extchunkb@x
          \fi
          \ifsb@rfillpage\else\kern\dimen1\fi
          \vskip\sp@ceafter}%
    \fi
    \ifx\c@tegory\DebugBorderCat \bgroup\def\@im{\the\interactionmode}\interactionmode=2 \showboxdepth=99\showboxbreadth=99\showbox\extchunkb@x
       \interactionmode=\@im \egroup 
    \fi
    \getsbp@ram{firstindent}\ifx\cp@ram\relax\p@chshift=0pt\else\p@chshift=\cp@ram\IndentUnit\fi
    \let\sp@ceafter\empty
    \let\fstb@drtop=\relax
    \trace{e}{After border, chunk is  now \the\ht\extchunkb@x+\the\dp\extchunkb@x*\the\wd\extchunkb@x}%
    %\ifsbar@rotated
      %\tracingmacros=1
      %\tracingcommands=3
      %\setbox0\box\extchunkb@x
      %\ifnum \pdf@ab=1
	%\let\r@tSet\r@cwSet%FIXME
      %\else
	%\let\r@tSet\r@acwSet%
      %\fi
      %%\rot@tebz
      %\tracingmacros=0
      %\tracingcommands=0
    %\fi
    %\let\r@tSet\r@cwSet%FIXME
    %\setbox\extchunkb@x\hbox{\rot@tebz}%
    \ifpicUsesIns
      \x@\let\x@\@xtins\csname ins-\loc@ption\endcsname
      \trace{g}{Sidebar aligned into hbox \the\p@cinswid (\pic@lign)}%
      \ifdim\wd\extchunkb@x<\p@cinswid
        \setbox\extchunkb@x\vbox{%
          \ifx\pic@lign\@lignRight\hbox to \p@cinswid{\hss\box\extchunkb@x}%
          \else
            \ifx\pic@lign\@lignLeft \hbox to \p@cinswid{\box\extchunkb@x\hss}%
            \else\hbox to \p@cinswid{\hss\box\extchunkb@x\hss}\fi
        \fi}%
      \fi
      \trace{e}{using ins-\loc@ption}%
      \insert\@xtins{%\vskip\baselineskip
        \dimen0=-\dp\extchunkb@x
        \penalty10000
        \pdfsavepos
        \write\p@rlocs{\noexpand\@nontextstart{\the\pdflastxpos}{\the\pdflastypos}}%
        \penalty10000
        %\gridp@ctrue\gridb@x\extchunkb@x\gridp@cfalse% make sure we align to the page grid
        \unvbox\extchunkb@x\vskip\the\dimen0
        \penalty10000
        \pdfsavepos
        \write\p@rlocs{\noexpand\@nontextstop{\the\pdflastxpos}{\the\pdflastypos}}%
        \ifdim\ht\extb@x>0pt \penalty5000 \fi}%
    \else
      \ifx\picl@c\empty
        \message{Could not interpret position \loc@ption for \c@tegory:\sb@rmarker}%
      \else
        \dimen0=\ht\extchunkb@x \advance\dimen0 by \dp\extchunkb@x \advance\dimen0 by 0.5\baselineskip % Rounding correction
        \divide\dimen0 by \baselineskip \count255=\dimen0
        \advance\count255 by 1
        \trace{g}{Inline \w@tsit [\c@tegory:\sb@rmarker] (\picl@c [\l@cspec@b\l@cspec@c]) takes \the\count255 lines, ls:\l@cspec@b, oldlastsdepth:\oldlastdepth, hts=\the\headingtopspace}%
        \ifx\picl@c\loc@Inl
          \ifx\l@cspec@b\@lignLeft\picNarrowtrue\fi
          \ifx\l@cspec@b\@lignRight\picNarrowtrue\fi
          %\headingtopspace=\oldlastdepth
	  %\vskip -\oldlastdepth
          \lastdepth=\oldlastdepth
          \d@figureInl\extchunkb@x
          \ifhmode\else
           \lastdepth=\oldlastdepth
          \fi
        \fi
        \ifx\picl@c\loc@Par
          \ifx\l@cspec@b\@lignLeft\picNarrowtrue\fi
          \ifx\l@cspec@b\@lignRight\picNarrowtrue\fi
          \lastdepth=0pt
          \setbox\extchunkb@x\vbox{\unvbox\extchunkb@x\kern-1sp \kern 1sp}% Add a marker
          \d@figurePar\extchunkb@x
        \fi
        \ifx\picl@c\loc@Page
          \d@figurePage\extchunkb@x
        \fi
        \ifx\picl@c\loc@Full
          \trace{e}{Full page box \b@x@lunpad | \b@x@runpad}%
          \setbox\extchunkb@x\hbox {\penalty-10001\kern\b@x@lunpad\box\extchunkb@x\kern \b@x@runpad}%
          \d@figureFull\extchunkb@x
        \fi
        \ifx\picl@c\loc@Cut
          \unskip\unskip
          \d@figureCut\extchunkb@x
          %\showlists
          %\ifhmode |+\fi
        \fi
        \ifvoid\extchunkb@x\else
          \message{Could not understand / interpret position \loc@ption for \c@tegory:\sb@rmarker}%
        \fi
      \fi
    \fi
    \ifdim\ht\extb@x>0pt
      \def\fb@tpadding{0pt}%
      \def\fb@bpadding{0pt}%
  \repeat
  \let\c@tegory\oldc@tegory 
  \s@tc@tpr@fix
  \ifvoid\sid@barnotes\else
    \vadjust{\unvbox\sid@barnotes}%
  \fi
  \let\fb@lpadding\os@vedlpadding
  \let\fb@rpadding\os@vedrpadding
  \let\fb@tpadding\os@vedtpadding
  \let\fb@bpadding\os@vedbpadding
  \kill@PossParamCache
  \c@th@@ks{}% Just in case
  \ifx\c@tprefix\empty
    \dos@vedm@rk
  \fi
  \ifnum \sb@numcuts>0
    \@numcuts=\sb@numcuts
    \trace{e}{Re-doing cutouts for \sb@numcuts}%
    \restore@cutouts{esb}%
    \makecutouts
  \fi
}

\lowercase{\edef\msl@sh{|}}
% Category file parsing 
\def\Category   #1\relax{%Store name of current category and defaults
  \S@tCat{#1}%
  \xdef\m@rker{esb}%
}

\def\initc@t{%
  \ifx\c@tprefix\empty
    \edef\t@stname{\m@rker\ds@ffix:is_defined}%
  \else 
    \edef\t@stname{\c@tprefix \m@rker\ds@ffix:is_defined}%
  \fi
  \x@\xdef\csname m@rkerexists-\c@tprefix\m@rker\endcsname{t}%
  \x@\let\x@\t@st\csname \t@stname\endcsname
  \ifx\t@st\relax 
    \trace{e}{First definition of \c@tprefix\m@rker\ds@ffix}%
    \x@\xdef\csname \t@stname\endcsname{Done}%
  \fi
}
\def\Position   #1\relax{\initc@t\setsbp@ram{posn}{#1}}% Where does it go on the page? (like figures, but also B for below notes)
\def\Scale      #1\relax{\initc@t\setsbp@ram{scale}{#1}}% Where does it go on the page? (like figures, but also B for below notes)
\def\Breakable   #1\relax{\initc@t\tempfalse\setbox0\hbox{\ifcat 1#1 \global\temptrue\fi}\iftemp\setsbp@ram{break}{#1}\else\setsbp@ram{break}{\uppercase{#1}}\fi} %Can the box page-break?
\def\FgImage      #1\relax{\initc@t\stripqu@te{#1}\setsbp@ram{fgfigname}{\@@result}\GenC@tFig{fg}}
\def\FgImagePos   #1\relax{\initc@t\setsbp@ram{fgfigpos}{#1}\GenC@tFig{fg}}
\def\FgImageScale   #1\relax{\initc@t\setsbp@ram{fgfigscale}{#1}\GenC@tFig{fg}}
\def\FgImageScaleTo   #1\relax{\initc@t\setsbp@ram{fgfigscaleto}{#1}\GenC@tFig{fg}}
\def\BgImage      #1\relax{\initc@t\stripqu@te{#1}\setsbp@ram{bgfigname}{\@@result}\GenC@tFig{bg}}
\def\BgImageScale   #1\relax{\initc@t\setsbp@ram{bgfigscale}{#1}\GenC@tFig{bg}}
\def\BgImageScaleTo   #1\relax{\initc@t\setsbp@ram{bgfigscaleto}{#1}\GenC@tFig{bg}}
\def\BgImagePos   #1\relax{\initc@t\setsbp@ram{bgfigpos}{#1}\GenC@tFig{bg}}
\def\BgImageLow   #1\relax{\initc@t\uppercase{\edef\t@st{#1}}\setsbp@ram{bgfiglow}{\t@st}}% What's the right sequence for colour layer and background image
\def\BgImageColour       #1\relax{\initc@t\x@\checkh@x #1\end\setsbp@ram{bgfigcolour}{\rgb@out}}
\let\BgImageColor=\BgImageColour%for Americans
\def\BgImageAlpha       #1\relax{\initc@t\setsbp@ram{bgfigalpha}{#1}}
\def\BgImageOversize       #1\relax{\initc@t\setsbp@ram{bgfigoversize}{#1}}

\def\BgColour   #1\relax{\initc@t\tempfalse\uppercase{\edef\t@st{#1}}\setbox0\hbox{\if F\t@st \global\temptrue\trace{e}{Cancelling colour}\else\if\tr@e\t@st\global\temptrue\fi\fi}\iftemp\setsbp@ram{hascol}{\t@st}\else\setsbp@ram{hascol}{T}\x@\checkh@x #1\end\setsbp@ram{bgcolour}{\rgb@out}\fi}%For British/Australian/NZ/etc.
\def\SpaceBeside #1\relax{\initc@t\setsbp@ram{spacebeside}{#1}}
\let\BgColor\BgColour%For Americans
\def\Alpha   #1\relax{\initc@t\setsbp@ram{alpha}{#1}}% 1=solid 0=invisible
\def\BoxLPadding #1\relax{\initc@t\setsbp@ram{boxlpadding}{#1}}
\def\BoxRPadding #1\relax{\initc@t\setsbp@ram{boxrpadding}{#1}}
\def\BoxHPadding #1\relax{\initc@t\setsbp@ram{boxlpadding}{#1}\setsbp@ram{boxrpadding}{#1}}
\def\BoxTPadding #1\relax{\initc@t\setsbp@ram{boxtpadding}{#1}}
\def\BoxBPadding #1\relax{\initc@t\setsbp@ram{boxbpadding}{#1}}
\def\BoxVPadding #1\relax{\initc@t\setsbp@ram{boxtpadding}{#1}\setsbp@ram{boxbpadding}{#1}}
\def\BoxPadding #1\relax{\initc@t\setsbp@ram{boxlpadding}{#1}\setsbp@ram{boxrpadding}{#1}\setsbp@ram{boxtpadding}{#1}\setsbp@ram{boxbpadding}{#1}}
\def\Rotation  #1\relax{\initc@t\setsbp@ram{sbarrotation}{#1}}
\def\GenC@tFig#1{%
 \getp@r@m{#1figname}{\c@tprefix\m@rker}\ifx\p@ram\relax
   \trace{e}{No figname yet for \c@tprefix\m@rker}%
   \x@\setsbp@ram{#1figspec}{}%
 \else
   \ifx\p@ram\false
    \x@\setsbp@ram{#1figspec}{}%
   \else
     \let\c@tfigname\p@ram
     %\getbp@r@m{#1figscale}%
     \edef\tmp{\c@tfigname|}%
     \x@\setsbp@ram{#1figspec}{\tmp}%
     \trace{e}{figspec for \c@tprefix\m@rker:  \tmp}%
   \fi
 \fi
}
\def\sb@ins@rtpic#1|#2|#3\E{%
  \trace{e}{Figure #1|#2|#3}%
  \dimen0=#3\relax
  \openin\t@stread=#1 
   \ifeof\t@stread 
     \m@kepl@ceholder{\dimen0}{0.618 \dimen0}{#1}%
   \else\closein\t@stread
    \let\picfilecomm@nd=\XeTeXpicfile
    \expandafter\ch@ckpdf#1..\endf@lename
    \setbox0\hbox{\picfilecomm@nd #1 #2}%
    \global\dimen3=\wd0
    \global\dimen4=\ht0
    \trace{e}{Image dims: \the\dimen3 x \the\dimen4}%
    \hbox{\hskip\sb@leftshift\hbox to #3{\hskip\fig@lskip\box0\hskip\fig@rskip}\hskip\sb@rightshift}%
   \fi}

\def\EndCategory{\gdef\c@tegory{}\global\let\c@tprefix\empty}

\def\showTheCategory{\c@tegory\message{The category is \c@tegory}}

\def\StyleCategory#1#2{\S@tCat{#1}#2\EndCategory}
\def\categorysheet#1{%Just wrap \stylesheet with category-blanking commands
  \subc@lltrue
  \EndCategory\stylesheet{#1}\EndCategory
  \subc@llfalse
}
\def\B@rderTOP{\setsbp@ram{bordertop}{T}}
\def\B@rderBOTTOM{\setsbp@ram{borderbottom}{T}}
\def\B@rderLEFT{\setsbp@ram{borderoddleft}{T}\setsbp@ram{borderevenleft}{T}}
\def\B@rderRIGHT{\setsbp@ram{borderoddright}{T}\setsbp@ram{borderevenright}{T}}
\def\B@rderINNER{\ifBookOpenLeft\setsbp@ram{borderoddright}{T}\setsbp@ram{borderevenleft}{T}\else\setsbp@ram{borderoddleft}{T}\setsbp@ram{borderevenright}{T}\fi}
\def\B@rderOUTER{\ifBookOpenLeft\setsbp@ram{borderoddleft}{T}\setsbp@ram{borderevenright}{T}\else\setsbp@ram{borderoddright}{T}\setsbp@ram{borderevenleft}{T}\fi}
\def\B@rderALL{\setsbp@ram{borderoddleft}{T}\setsbp@ram{borderevenleft}{T}\setsbp@ram{borderoddright}{T}\setsbp@ram{borderevenright}{T}\setsbp@ram{bordertop}{T}\setsbp@ram{borderbottom}{T}}
\def\B@rderNONE{\setsbp@ram{borderoddleft}{F}\setsbp@ram{borderevenleft}{F}\setsbp@ram{borderoddright}{F}\setsbp@ram{borderevenright}{F}\setsbp@ram{bordertop}{F}\setsbp@ram{borderbottom}{F}}

\addtoinithooks{\let\@SB=\esb\let\@SBE=\esbe \let\esb=\@sb\let\esbe=\@sbe}
\def\g@nbox#1{%For some reason, this prefers to be in its own macro.
  \x@\newb@x\csname #1\endcsname
}
%\x@\def\csname d@code-t\endcsname{\edef\tmp{cat:\tmp}}
%\x@\def\csname d@code-T\endcsname{\edef\tmp{cat:\tmp}}
\x@\def\csname d@code-T\endcsname{\let\tmp\h@phen} % a paragraph-catergory does not alter the stystak
\x@\def\csname d@code-t\endcsname{\let\tmp\h@phen} % a non-paragraph-catergory does not alter the stystak either.

\def\dbghrule{\vbox to 0pt{\vrule width 0.5\hsize height 0.1pt \vss}}
\def\Override{\m@kedigitsother}
\def\EndOverride{\m@kedigitsletters}
