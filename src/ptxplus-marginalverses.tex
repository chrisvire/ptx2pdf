% This TeX code provides a way to have marginal verses
% Caution, it is still under heavy development

\catcode`\@=11
\input marginnotes.tex
%+cmargin_verse
\newdimen\VerseBoxWidth \VerseBoxWidth=0pt
{\catcode`\~=12 \lccode`\~=32 \lowercase{
\gdef\getverseboxwidth{\setbox2=\hbox{\ch@rstylepls{v}~\endash 000\ch@rstylepls{v}*}\global\VerseBoxWidth=\wd2}
}}

\def\mymarginpverse{%
  \ifdim\VerseBoxWidth=0pt\getverseboxwidth\fi
  \getp@ram{position}{v}{\styst@k}\ifx\p@ram\relax\let\p@ram\l@ft\fi
  \trace{v}{position is \p@ram}%
  \ifx\v@rsefrom\v@rseto\edef\t@mp{\id@@@\ch@pter.\v@rsefrom}\else\edef\t@mp{\id@@@\ch@pter.\v@rsefrom-\v@rseto}\fi
  \ifcsname marginnote-\t@mp @v:pageno\endcsname \count10=\csname marginnote-\t@mp @v:pageno\endcsname
    \trace{v}{from marginnote-\t@mp @v:pageno, count=\the\count10}\ifodd\count10\relax\t@mptrue\else\t@mpfalse\fi
  \else\trace{v}{no marginnote-\t@mp @v:pageno, pageno=\the\pageno}\ifodd\pageno\relax\t@mptrue\else\t@mpfalse\fi\fi%
  \trace{v}{pagenumber is \ift@mp odd\else even\fi, and side is \p@ram, value=\the\count10, mcstack=\mcstack}%
  \ifnum \ifx\p@ram\@nner\ift@mp 0\else 1\fi\else\ifx\p@ram\@uter\ift@mp 1\else 0\fi\else\ifx\p@ram\r@ght 1\else 0\fi\fi\fi =0
    \t@mptrue\else\t@mpfalse
  \fi
  \ifx\v@rsefrom\v@rseto\setbox2=\vbox{\hbox to \VerseBoxWidth{\ift@mp\hfil\fi\simpleprintv@rse\ift@mp\else\hfil\fi}}%
    \trace{v}{in mymarginpv@rse \v@rseto - \v@rsefrom, side is \ift@mp left\else right\fi, pad to \the\VerseBoxWidth}%
  \else
    \s@tsideskips{v}%
    \ift@mp \dimen9=\leftskip\leftskip=\rightskip\rightskip=\dimen9\fi
    \trace{v}{leftskip=\the\leftskip, rightskip=\the\rightskip}%
    \myh@ngprintv@rse{2}%
    \setbox2=\vbox{\hbox to \VerseBoxWidth{\ift@mp\hfil\fi\box2\kern\AfterVerseSpaceFactor\FontSizeUnit\ift@mp\else\hfil\fi}}%
  \fi
  \trace{v}{verse \t@mp \space width=\the\wd2, height=\the\ht2, depth=\the\dp2, newwidth=\the\VerseBoxWidth, \ift@mp left\else right\fi}%
  \d@marginnote{2}{\t@mp}{v}%
}

\let\defaultprintverse=\mymarginpverse
\let\printverse\defaultprintverse
\catcode`\~=12 \lccode`\~=32\lowercase{
\def\myh@ngprintv@rse#1{%
    \setbox1=\hbox{\ch@rstylepls{v}~\endash\adornv{\v@rseto}\ch@rstylepls{v}*}%
    \setbox#1=\vtop{\s@tbaseline{v}{v}\hbox to \wd1{\hfil\ch@rstylepls{v}~\adornv{\v@rsefrom}\ch@rstylepls{v}*}%
        \box1}}
}

%-cmargin_verse

% Example integration code:
% Basics
%\BookOpenLefttrue
%\def\AfterVerseSpaceFactor{0}
%\expandafter\def\csname v:position\endcsname{inner}

% Limiting one verse per paragraph
%\newif\iffirstinpara \firstinparatrue
%\let\mytv=\defaultprintverse
%\def\defaultprintverse{\iffirstinpara\global\firstinparafalse\mytv\fi}
%\def\paramarker#1{\expandafter\let\csname _#1\expandafter\endcsname \csname #1\endcsname
%    \expandafter\gdef\csname #1\endcsname{\global\firstinparatrue\csname _#1\endcsname}}
%\paramarker{p}\paramarker{q1}\paramarker{li}

\endinput

