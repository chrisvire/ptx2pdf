%:strip
%%%%%%%%%%%%%%%%%%%%%%%%%%%%%%%%%%%%%%%%%%%%%%%%%%%%%%%%%%%%%%%%%%%%%%%
% Part of the ptx2pdf macro package for formatting USFM text
% copyright (c) 2007-2020 by SIL International
% written by Jonathan Kew
%
% Permission is hereby granted, free of charge, to any person obtaining  
% a copy of this software and associated documentation files (the  
% "Software"), to deal in the Software without restriction, including  
% without limitation the rights to use, copy, modify, merge, publish,  
% distribute, sublicense, and/or sell copies of the Software, and to  
% permit persons to whom the Software is furnished to do so, subject to  
% the following conditions:
%
% The above copyright notice and this permission notice shall be  
% included in all copies or substantial portions of the Software.
%
% THE SOFTWARE IS PROVIDED "AS IS", WITHOUT WARRANTY OF ANY KIND,  
% EXPRESS OR IMPLIED, INCLUDING BUT NOT LIMITED TO THE WARRANTIES OF  
% MERCHANTABILITY, FITNESS FOR A PARTICULAR PURPOSE AND  
% NONINFRINGEMENT. IN NO EVENT SHALL SIL INTERNATIONAL BE LIABLE FOR  
% ANY CLAIM, DAMAGES OR OTHER LIABILITY, WHETHER IN AN ACTION OF  
% CONTRACT, TORT OR OTHERWISE, ARISING FROM, OUT OF OR IN CONNECTION  
% WITH THE SOFTWARE OR THE USE OR OTHER DEALINGS IN THE SOFTWARE.
%
% Except as contained in this notice, the name of SIL International  
% shall not be used in advertising or otherwise to promote the sale,  
% use or other dealings in this Software without prior written  
% authorization from SIL International.
%%%%%%%%%%%%%%%%%%%%%%%%%%%%%%%%%%%%%%%%%%%%%%%%%%%%%%%%%%%%%%%%%%%%%%%

%% Macros to read Paratext stylesheets and define TeX control sequences
%% for all markers present in the file

%
% Macros to process each marker used in a Paratext stylesheet,
% either ignoring it or saving the parameter as appropriate
%
%+csty_definition_setup
\def\ds@ffix{} % diglot suffix (L/R)
\def\m@rker{} % The currently being defined \Marker
\let\prev@rsemode\empty
\def\remember@xists#1{\x@\xdef\csname m@rkerexists-#1\ds@ffix\endcsname{t}}
\def\should@xist#1{\ifcsname m@rkerexists-#1\endcsname\else
  \trace{sC}{Remembering that #1 exists as a marker}%
  \remember@xists{#1}%
  \s@tpp@name{#1}\x@\global\x@\let\csname \pp@name\endcsname\relax%Invalidate old cache
\fi
}


% Remember the name of the marker currently being defined
\def \Marker         #1\relax{
  \m@kedigitsletters%
  \xdef\m@rker{\scantokens{#1\noexpand}}%
  \m@kedigitsother
  \remember@xists{\m@rker}% store name of marker currently being defined
  \ifx\c@tprefix\empty\else
    \remember@xists{\c@tprefix\m@rker}%
  \fi
  \expandafter\let\csname font<\m@rker>\endcsname\relax}
  %\expandafter\let\csname extrafont<#1>\endcsname\relax}
%-csty_definition_setup

%+csty_definitions
% Make sure we list everything that can occur in a .sty file. If a .sty file contains a
% missing sfm marker, then processing for that and all subsequent stylesheets will stop
% at that point.
% Ignore stylesheet markers that do not affect formatting 
\def \Name           #1\relax{}
\def \Description    #1\relax{}
\def \OccursUnder    #1\relax{}
\let \Occursunder\OccursUnder
\def \Rank           #1\relax{}
\def \NotRepeatable  #1\relax{}
\let \Notrepeatable\NotRepeatable
\def \Color          #1\relax{\defp@ram{color}{#1}}
\def \Background     #1\relax{\x@\checkh@x #1\end\edefp@ram{background}{\rgb@out}} %Store as r g b triplet
\let \color\Color
\def \ColorName      #1\relax{}

% For each property that affects formatting, define a macro to store the value
% of this field, these will be accessed via getp@ram.
% Constructs a macro name \<marker>:<parameter> to hold the value
\def \Endmarker      #1\relax{\defp@ram{endmarker}{#1}} % endmarker is not currently checked except for milestones
\def \TextType       #1\relax{\lowercase{\defp@ram{type}{#1}}}
\let \Texttype\TextType
\def \TextProperties #1\relax{\lowercase{\defp@ram{properties}{#1}}}
\let \Textproperties\TextProperties
\def \FontSize       #1\relax{\edef\t@mp{#1}\ifx\t@mp\empty\else\clearp@ram{fontscale}\defp@ram{fontsize}{#1}\fi}
\let \Fontsize\FontSize
\def \FontScale      #1\relax{\clearp@ram{fontsize}\defp@ram{fontscale}{#1}}
\let \Fontscale\FontScale
\def \FontFactor     #1\relax{\defp@ram{fontfactor}{#1}}
\def \FontName       #1\relax{\defp@ram{fontname}{#1}}
\let \Fontname\FontName
\def \FirstLineIndent #1\relax{\defp@ram{firstindent}{#1}}
\let \Firstlineindent\FirstLineIndent
\def \LeftMargin     #1\relax{\defp@ram{leftmargin}{#1}}
\let \Leftmargin\LeftMargin
\def \RightMargin    #1\relax{\defp@ram{rightmargin}{#1}}
\let \Rightmargin\RightMargin
\def \Italic         #1\relax{\defbinp@ram{italic}{#1}}
\def \Bold           #1\relax{\defbinp@ram{bold}{#1}}
\def \Superscript    #1\relax{\condp@ram{#1}{\defp@ram{raise}{\SuperscriptRaise}\defp@ram{fontfactor}{\SuperscriptFactor}}{\defp@ram{fontfactor}{1}}}
\def \Raise          #1\relax{\defp@ram{raise}{#1}}
\def \Underline      #1\relax{\defbinp@ram{underline}{#1}}
\let \underline\Underline
\def \Regular        {\defp@ram{regular}{true}\defp@ram{italic}{false}\defp@ram{bold}{false}\defp@ram{superscript}{false}\defp@ram{smallcaps}{false}}
\let \superscript\Superscript
\def \SpaceBefore    #1\relax{\defp@ram{spacebefore}{#1}}
\let \Spacebefore\SpaceBefore
\def \SpaceAfter     #1\relax{\defp@ram{spaceafter}{#1}}
\let \Spaceafter\SpaceAfter
\def \Justification  #1\relax{\lowercase{\defp@ram{justification}{#1}}}
\def \NonJustifiedFill #1\relax{\defp@ram{glue}{#1}} % ptx2pdf extension, glue portion of leftskip/rightskip (as fraction of hsize)
\def \CallerStyle    #1\relax{\defp@ram{callerstyle}{#1}} % ptx2pdf extension
\def \CallerRaise    #1\relax{\defp@ram{callerraise}{#1}}
\def \NoteCallerStyle    #1\relax{\defp@ram{notecallerstyle}{#1}} % ptx2pdf extension
\def \NoteCallerRaise    #1\relax{\defp@ram{notecallerraise}{#1}}
\def \NoteBlendInto      #1\relax{\defp@ram{notebase}{#1}}
\def \TEStyleName    #1\relax{} % Translation Editor extension
\def \SmallCaps      #1\relax{\defbinp@ram{smallcaps}{#1}}
\let \Smallcaps\SmallCaps
\def \BaseLine       #1\relax{\defp@ram{baseline}{#1}}
\def \LineSpacing    #1\relax{\defp@ram{baseline}{#1\onel@neunit}}
\def \ztexFontFeatures  #1\relax{\defp@ram{fontfeatures}{#1}}
\def \ztexFontGrSpace   #1\relax{\defp@ram{fontgrspace}{#1}}
\def \VerticalAlign     #1\relax{\defp@ram{verticalalign}{#1}} % for table cells

% Make a TeX macro based on the StyleType
\def \StyleType      #1\relax{\lowercase{\def\styl@type{#1}}\m@kestyle}     %(1)
\let \Styletype=\StyleType
% New extras that PTX supports
\def \XMLTag         #1\relax{}
\def \Encoding       #1\relax{}
\def \Rank           #1\relax{}
\def \TEStyleName    #1\relax{}
\def \NotRepeatble   #1\relax{}
\def \Attributes     #1\relax{}
%-csty_definitions

%+csty_defparam
%:
% \defp@ram: store the value of a parameter from the .sty file
% p | description
% --|------
% #1|name of parameter to store
% #2|value
% --|------
% Constructs a macro name \_<marker>_:_<parameter>_ to hold the value.
% `csname` is required here in order to build a macro name containing a colon and possibly a number.
%

\def\clearp@ram#1{\global\x@\let\csname\c@tprefix\m@rker\ds@ffix:#1\endcsname=\relax}
\def\defp@ram#1#2{\x@\def\csname\c@tprefix\m@rker\ds@ffix:#1\endcsname{#2}\trace{sa}{def \c@tprefix\m@rker\ds@ffix:#1 -> #2}}
\def\edefp@ram#1#2{\x@\edef\csname\c@tprefix\m@rker\ds@ffix:#1\endcsname{#2}\trace{sa}{def \c@tprefix\m@rker\ds@ffix:#1 -> #2}}
\def\condp@ram#1#2#3{\edef\t@mp{#1}\ifx\t@mp\h@phen\relax\def\t@mp{#3}\else\def\t@mp{#2}\fi\t@mp}
\def\defbinp@ram#1#2{\edef\t@mp{#2}\ifx\t@mp\h@phen\defp@ram{#1}{false}\else\defp@ram{#1}{true}\fi}
\def\setp@ram#1#2#3{
  \x@\xdef\csname\c@tprefix#2\ds@ffix:#1\endcsname{#3}\trace{sa}{set \c@tprefix#2\ds@ffix:#1 -> #3}
  \should@xist{#2\ds@ffix}%Ensure that the marker is in the exists list
  \ifx\c@tprefix\empty\else\should@xist{\c@tprefix#2\ds@ffix}\fi
} % expand now, not later 
%-csty_defparam

%+csty_getparam
%:
% \getp@aram: fetch the value of a style parameter into \p@ram
% --|------
% #1|name of parameter to fetch
% #2|name of marker
% --|------
% Sets temporary macro \p@ram to the value
% The macro was once just diglot aware, allowing all parameters to have L Left/Right variant or use
% default. (e.g when asked for paramter for v, check for vL|vR, default to v). Now it (potentially) needs to 
% take into account: diglot, categories and milestones.
% Rather than attempt to code \ifs for all possible 8 (16? more?) options, it
% now subcontracts s@tpossParams to build a comma-separated list of them, 
% and then finds the first match. This results in cleaner code.

\newif\ifnotracestacks \notracestacksfalse
\def\tracestack{,}
\def\pushtracing{\xdef\tracestack{\the\tracingmacros,\tracestack}\tracingmacros=0}
\def\poptracing{\tracingmacros=\x@\cstackpeek\tracestack\E\xdef\tracestack{\x@\cstackpop\tracestack\E}}
\def\stystacksplit#1+#2\E{\edef\t@mp{#2}\let\tmpstyst@kbody\empty\edef\tmpstyst@kfirst{#1}\ifx\t@mp\empty\else\styst@cksplit#2\E\fi\trace{sk}{styst@ksplit: '\tmpstyst@kfirst' , '\tmpstyst@kbody'}}
\def\styst@cksplit#1+\E{\edef\tmpstyst@kbody{#1}}
\def\stystack@pts#1+#2\E{%Given an style stack, recurse down it, turning it into an option stack, going a maximum of count255 levels deep
% If \AttrMilestoneMatchesUnattrtrue, then the option list should be as follows:
%   nd+Fred|qt-s+p -> (nd+Fred|qt-s+p, nd+qt-s+p, nd+Fred|qt-s, nd+qt-s, nd)
% If 2 milestones are active, then only the inner-most will be affected by this flag, 
% outer ones must be fully specified.
  \trace{sk}{stystack@pts '#1' + '#2' (out:\@ut)}%
  \ifAttrMilestoneMatchesUnattr
    \if@ttrMilestone
      \edef\t@mp{#1}%
      \ifx\t@mp\thisattrmil@stone % Need to match on unattributed milestone too.
        \ifx\mini@stackB\empty\let\mini@stackB\mini@stack\fi
        \trace{sko}{Adding \thismil@stone\space to possible params}%
        \edef\mini@stackB{\mini@stackB+\thismil@stone}%
      \else
        \edef\mini@stackB{\mini@stackB+#1}%
      \fi
    \fi
  \fi
  \edef\mini@stack{\mini@stack+#1}%
  \trace{sko}{'#1' -> \mini@stack\space(\mini@stackB)}%
  \ifx\mini@stackB\empty\else
    \xdef\@ut{\mini@stackB,\@ut}%
  \fi
  \xdef\@ut{\mini@stack,\@ut}%
  \edef\t@mp{#2}\advance\count255 by -1 
  \ifx\t@mp\empty\count255=-1 \relax\fi %signal end of list before (or at) truncation point.
  \trace{sko}{'#2',\the\count255---> \@ut}%
  \ifnum\count255>0 \x@\stystack@pts#2\E\fi
}

%returns \@@ut and \c@t@ut.  For expansions, (\@@ut,) should be used, \c@t@ut has cat:foo|* before the final entries.
\def\fullstack@xpand#1|#2\E{%Parameters: complexity|style stack
  \trace{sk}{fullstack@xpand}%
  \edef\exp@stystak{#2}%
  \ifx\exp@stystak\empty\else
    \x@\stystacksplit\exp@stystak+\E% Cut off the head entry of the stylestack
    \ifdiglot
       \edef\p@ssParams{\tmpstyst@kfirst\c@rrdstat,\tmpstyst@kfirst}%
    \else
       \edef\p@ssParams{\tmpstyst@kfirst}%
    \fi 
    \trace{sko}{#1::(\p@ssParams)+\tmpstyst@kbody}%
    \ifx\styst@kfirst\empty
      \let\styst@kfirst\tmpstyst@kfirst
      \let\styst@kbody\tmpstyst@kbody
    \fi
    \let\mini@stack\empty
    \let\mini@stackB\empty
    \ifx\tmpstyst@kbody\empty
      \xdef\@lso{\p@ssParams}%
      \let\c@talso\@lso
      \ifx\c@tprefix\empty\else % Prefix anything that ought to go before the bottom level here.
        \trace{sko}{type of \tmpstyst@kfirst\space is '\csname stylet@pe-\tmpstyst@kfirst\endcsname'}%
        \x@\ifx\csname stylet@pe-\tmpstyst@kfirst\endcsname\ss@Para
          \xdef\c@talso{\c@tprefix*,\p@ssParams}%
        \fi
      \fi
      \trace{sko}{no body...appending \@lso(\c@talso) (onto: \@@ut) }%
      %\let\@ut\empty
      %\let\d@\cs@ppend
      %\x@\cstackdown \p@ssParams,\E 
      %\trace{sk}{:\@ut ->(\@@ut)}%
      \ifx\@@ut\empty
        \global\let\@@ut\@lso
        \global\let\c@t@ut\c@talso
      \else
       \xdef\c@t@ut{\@@ut,\c@talso}%
       \xdef\@@ut{\@@ut,\@lso}%
      \fi
      \trace{sko}{fullstack@xpand #1|#2 : \@@ut. No more}%
    \else
      \count255=#1% Limit the options
      \def\@ut{}%
      \x@\stystack@pts\tmpstyst@kbody+\E%
      \let\@in\@ut
      \let\@ut\empty
      \ifnum\count255<0\else
        \edef\@in{+\tmpstyst@kbody,\@in}%  still allow full-stack definitions, even if the list is normally truncated
      \fi
      %\tracingmacros=1
      %\tracingassigns=1
      \def\d@##1\E{\trace{sko}{appending '##1' onto (\p@ssParams)}\edef\@lso{##1}\let\d@c\d@
        \let\@ut=\empty
        \let\d@=\cs@ppend\x@\cstackdown\p@ssParams,\E
        \trace{sko}{:\@ut (\@@ut) pp:\p@ssParams, spp:\simplep@ssParams}%
        \let\d@\d@c\xdef\@@ut{\ifx\@@ut\empty\else\@@ut,\fi\@ut}}%
      \x@\cstackdown \@in,\E 
      \xdef\@@ut{\@@ut,\p@ssParams}%
      \trace{sko}{fullstack@xpand #1|#2 : \@@ut. Continue (\tmpstyst@kbody)}%
      \x@\fullstack@xpand\x@#1\x@|\tmpstyst@kbody\E %Recurse
    \fi
  \fi
}

\def\StackComplexity{3}
\def\this@stystak{}
\def\periph@@@{} % Eventually will hold the current peripheral type

\def\s@tpossParams#1#2{\trace{sa}{s@tpossParams {#1}{#2}\m@rker (\styst@kfirst / \styst@k)}%
  \ifdiglot
     \edef\p@ssParams{#1\c@rrdstat,#1}%
  \else
     \edef\p@ssParams{#1}%
  \fi
  \let\simplep@ssParams\p@ssParams
  \edef\tmp{#1}% #1 is the main marker
  \edef\this@stystak{#2}%#2 is the stystack to be used for *this* call (normally \styst@k, but not *ALWAYS*). The stystack supplied should always include the present marker.
  \ifx\tmp\this@stystak
    \trace{sk}{No stack}%
    \edef\c@t@ut{\c@tprefix*,\p@ssParams}%
  \else % Assuming there's is a real stack to expand...
    \let\@@ut\empty
    \let\@ut\empty
    \x@\fullstack@xpand\x@\StackComplexity\x@|\this@stystak\E%
    \let\p@ssParams\@@ut
  \fi
  \trace{sk}{1. Checking list: \p@ssParams}%
  \ifx\mspr@fix\empty\else
     %If we expand our prefix list for special styling of milestone+marker, (rather than just relying on the stack) then:
     %\exp@ndmspr@fix{\let\@ut=\empty \let\d@=\cspr@pend\x@\cstackdown \p@ssParams,\E \edef\p@ssParams{\@ut,\p@ssParams}}%
  \fi
  \let\tmpd@=\d@
  \ifx\c@tprefix\empty\else
    \trace{sk}{1b. Checking list (\mspr@fix,\tmp=\m@rker?): \p@ssParams}%
    \edef\@lso{\c@tprefix}%
    \let\@ut=\empty \let\d@=\cspr@pend\x@\cstackdown \p@ssParams,\E 
    \edef\p@ssParams{\@ut,\c@t@ut}%  c@tout has cat:foo|* wildcard before final marker, if that's a parstyle. Note there's no L/R version of *!
  \fi
  \ifx\empty\periph@@@\else
    \edef\@lso{periph:\periph@@@|}%
    \trace{sk}{1c. Checking list (\@lso): \p@ssParams}%
    \let\@ut=\empty\let\d@=\cs@lsolist
    \x@\cstackdown \p@ssParams,\E 
    \edef\p@ssParams{\@ut}%
  \fi
  \edef\@lso{id:\id@@@|}%
  \trace{sk}{1d. Checking list (\@lso): \p@ssParams}%
  \let\@ut=\empty\let\d@=\cs@lsolist
  \x@\cstackdown \p@ssParams,\E 
  \edef\p@ssParams{\@ut}%
  \trace{sk}{2. Checking list: \p@ssParams}%
  \let\@ut\empty
  \let\d@=\cscheck@xist\x@\cstackdown \p@ssParams,\E
  \let\d@=\tmpd@
  \global\let\p@ssParams=\@ut
  \trace{sa}{3. Shrunk list: \p@ssParams}%
}

% Should a milestone with a defined attribute (e.g. \qt-s|Jesus\*) match the
% main stylesheet entry for the milestone? (without a defined attribute).
\newif\ifAttrMilestoneMatchesUnattr
\newif\if@ttrMilestone
\AttrMilestoneMatchesUnattrtrue% Default: yes.
\@ttrMilestonefalse

\def\kill@font#1\E{\x@\let\csname font<#1>\endcsname\relax}

\def\ClearCachedFonts{%Force font redefinition. 
  \ifx\f@ntlist\empty\else
    \let\d@\kill@font
    \x@\cstackdown\f@ntlist,\E
  \fi
  \global\let\f@ntlist\empty
}

\def\kill@cache #1\E{%Remove listed item
   \x@\global\x@\let\csname #1\endcsname\relax}

\def\p@ssParamCached{}

\def\kill@PossParamCache{\ifx\p@ssParamCached\empty\else% kill cached possParam-\m@rker values
    \let\d@kill@PossParamCache=\d@
    \let\d@=\kill@cache
    \x@\cstackdown\p@ssParamCached,\E
    \global\let\p@ssParamCached\empty
    \let\d@=\d@kill@PossParamCache
    \trace{sC}{Emptied p@ssParamCached}%
  \fi
}

\def\s@tpp@name#1{%
   \xdef\pp@name{pPrms-\c@tprefix\mspr@fix#1}%
}
\def\setpp@name#1{%
   \s@tpp@name{#1\ifdiglot\c@rrdstat\fi}%
}

\def\setpossParams#1#2{%
  \setpp@name{#2}%
  \x@\let\x@\g@tp@tmp\csname \pp@name\endcsname
  \ifx\g@tp@tmp\relax
    \s@tpossParams{#1}{#2}\x@\global\x@\let\csname \pp@name\endcsname\p@ssParams
    \ifx\p@ssParamCached\empty
      \xdef\p@ssParamCached{\pp@name}%
    \else
      \xdef\p@ssParamCached{\p@ssParamCached,\pp@name}%
    \fi
    \trace{s}{\pp@name\space defined (\p@ssParamCached) as '\p@ssParams'}%
  \else
    \global\let\p@ssParams\g@tp@tmp
    \trace{s}{\pp@name\space exists: '\p@ssParams'}%
  \fi
}

\newif\ifg@ttingspecificparam
\def\getp@ram#1#2#3{\trace{s}{getp@ram #2:#1 [#3](\mcstack)}\ifnotracestacks\pushtracing\fi
 \let\p@ram\relax
 \def\gp@test{#1}%
 \xdef\tmpstyst@k{#3}%
 \ifx\gp@test\empty\message{Eh? getp@ram called for #2 and empty parameter}\else 
   \setpossParams{#2}{#3}%
   {%protect outer version of \d@, etc.
   %\tracingassigns=1
   %\tracingmacros=1
  % Final version of the list might be cat:inl|ms:qt|pR,cat:inl|ms:qt|p,cat:inl|pR,cat:inl|p,ms:qt|pR,ms:qt|p,pR,p
   \ifcsname getthep@ram-#1\endcsname %call a parameter-specific routine. e.g. for fontsize.
     \trace{sa}{Specific 'getpar@m-#1'}%
     %\tracingmacros=0
     %\tracingassigns=1
     %\tracingifs=1
     \ifg@ttingspecificparam % Avoid lookup loops.
       \trace{sa}{No recursion}%
       \gettop@aram{#1}%
     \else
       \g@ttingspecificparamtrue
       \csname getthep@ram-#1\endcsname{#1}%
       \g@ttingspecificparamfalse
     \fi
     %\tracingifs=0
     %\tracingassigns=0
     %\tracingmacros=0
   \else
     \gettop@aram{#1}%
   \fi
   \global\let\p@@r@m=\p@ram % Need to return outside the brackets
   }%
   \let\p@ram=\p@@r@m
 \fi
 \trace{s}{\space\space (\c@tprefix)(\mspr@fix)#2\ifdiglot(\c@rrdstat)\fi :#1 is \p@ram}%
 \ifnotracestacks\poptracing\fi
}

\def\gettop@aram#1{%Check list of acceptable parameters and return the first one found.
   \edef\p@ramcheck{#1}%
   \ifx\p@ramcheck\empty\else
     \trace{sa}{Checking for #1 in \p@ssParams}%
     \let\d@gettopp@ram=\d@
     \let\d@=\stackg@tp@ram
     \x@\cstackdown \p@ssParams,\E
     \let\d@=\d@gettopp@ram
   \fi
}

\def\stackg@tp@ram#1\E{%\trace{sa}{tgp #1}%
  \getp@r@m{\p@ramcheck}{#1}\ifx\p@ram\relax\else\let\d@=\cstackrelax\fi}

\def\getp@r@m#1#2{\x@\let\x@\p@ram\csname#2:#1\endcsname \ifx\p@ram\relax\trace{spv}{#2:#1 undefined}\else\trace{sa}{#2:#1 found: \p@ram}\fi}
% prepend \@lso to #1 and build comma separated list of results
\def\cspr@pend#1\E{\ifx\@ut\empty\xdef\@ut{\@lso#1}\else\xdef\@ut{\@ut,\@lso#1}\fi}
% apply function do@it to #1, and build comma-separated list of results
\def\cs@apply#1\E{\ifx\@ut\empty\xdef\@ut{\do@it{#1}}\else\xdef\@ut{\@ut,\do@it{#1}}\fi}
% like cspr@pend, but also record unprepended item.
\def\cs@lsolist#1\E{\trace{sa}{expanding #1->\@lso#1,#1}\ifx\@ut\empty\xdef\@ut{\@lso#1,#1}\else\xdef\@ut{\@ut,\@lso#1,#1}\fi}
\def\cscheck@xist#1\E{\ifcsname m@rkerexists-#1\endcsname\ifx\@ut\empty\xdef\@ut{#1}\else\xdef\@ut{\@ut,#1}\fi\fi}
\def\cs@ppend#1\E{\ifx\@ut\empty\xdef\@ut{#1\@lso}\else\xdef\@ut{\@ut,#1\@lso}\fi}

%-csty_getparam

%+csty_getmcparam
%Main utility function for font parameters, stacking-styles aware
%\def\getmcp@ram#1{\trace{sa}{getmcp #1}\let\r@s=\relax\ifnotracestacks\pushtracing\fi
%  \def\d@##1+##2+##3\E{\d@code{##1}{##2}\ifx\tmp\empty\else\if##1\ss@Sbar\global\let\r@s\relax\else\if##1\ss@Note\global\let\r@s\relax\fi\fi
%    \getp@ram{#1}{\tmp}{\tmp}\ifx\p@ram\relax\else\global\let\r@s\p@ram\fi\fi}%
%  \mcup\global\let\p@ram\r@s\traceNum{s}{param #1 for \mcstack\space is \p@ram}\ifnotracestacks\poptracing\fi}    %(1)
%\def\getmcp@ramR#1{\trace{sa}{getmcpR #1}\let\r@s\relax\ifnotracestacks\pushtracing\fi
%  \def\d@##1+##2+##3\E{%
%    \trace{sa}{inside getmcp@ramR ##2 currently res=\r@s}%
%    \d@code{##1}{##2}%
%    \ifx\tmp\empty\else\if##1\ss@Sbar\global\let\r@s\relax\else\if##1\ss@Note\global\let\r@s\relax\fi\fi
%    \getp@ram{regular}{\tmp}{\tmp}\ifx\p@ram\relax\getp@ram{#1}{\tmp}{\tmp}\ifx\p@ram\relax\else\global\let\r@s\p@ram\fi
%                           \else\global\let\r@s\relax\fi\fi}%
%  \mcup\global\let\p@ram\r@s\traceNum{s}{paramR #1 for \mcstack\space is \p@ram}\ifnotracestacks\poptracing\fi}

\def\getthep@ramObeyRegular#1{%Called with p@ssParams set up. work up through possParam list, obeying \Regular in the stylesheet
  \let\d@gettheparam=\d@
  \trace{sa}{getthep@ramObeyRegular{#1}}%
  \let\r@s\relax\ifnotracestacks\pushtracing\fi
  %\tracingassigns=1
  \def\d@##1\E{\edef\tmpstylen@me{##1}\trace{sa}{inside getthep@ramObeyRegular ##1 currently res=\r@s}%
  \getp@r@m{regular}{\tmpstylen@me}\ifx\p@ram\relax\getp@r@m{#1}{\tmpstylen@me}\ifx\p@ram\relax\else\let\r@s\p@ram\fi
       \else\global\let\r@s\relax\let\t@preset\tmpstylen@me\fi}%
  \x@\cstackup\p@ssParams,\E
  %\tracingassigns=1
  \let\d@\d@gettheparam
  \let\p@ram\r@s
}

\def\getthep@ramObeyRegularTop#1{%Called with p@ssParams set up. work down through possParam list, stopping on value or \Regular in the stylesheet
  \let\d@gettheparam=\d@
  \trace{sa}{getthep@ramObeyRegularTop{#1}}%
  \def\d@##1\E{\edef\tmpstylen@me{##1}\trace{sa}{inside getthep@raObeyRegularTop ##1}%
    \getp@r@m{regular}{\tmpstylen@me}\ifx\p@ram\relax\getp@r@m{#1}{\tmpstylen@me}\ifx\p@ram\relax\else\let\d@\cstackrelax\fi\else\let\t@preset\tmpstylen@me\global\let\p@ram\relax\let\d@\cstackrelax\fi}%
  \x@\cstackdown\p@ssParams,\E
  \let\d@\d@gettheparam
}


\def\r@setparam{=[]}

\x@\def\csname getthep@ram-fontfeatures\endcsname#1{%
  \trace{sa}{getthep@ram-fontfeatures}%
  \let\d@gettheparam=\d@
  \let\r@s\empty
  \def\d@##1\E{\edef\tmpstylen@me{##1}\getp@r@m{regular}{\tmpstylen@me}%
    \ifx\p@ram\relax\getp@r@m{#1}{\tmpstylen@me}%
      \ifx\p@ram\relax\else
        \ifx\p@ram\r@setparam
          \let\d@\cstackrelax
        \else
          \ifx\p@ram\empty\else
            % Param might be in stacking (issue #575) [foo,bar] format or older :foo:bar 
            % Reprocess to stacking format:
            %\x@\ch@ckfontfeatures\p@ram\E % Initial colon test
            \x@\ch@ckfontfeaturesbrk\p@ram[]\E % Lack of square brackets test.
            \xdef\r@s{\p@ram\space\r@s\space}%Processed left to right, so top level (overrides) go *LAST*
          \fi
        \fi
      \fi
    \else\let\t@preset\tmpstylen@me\global\let\p@ram\relax\let\d@\cstackrelax
    \fi}%
    \x@\cstackdown\p@ssParams,\E
  \ifx\r@s\empty
    \let\p@ram\relax
  \else
    \trace{sa}{Found fontfeatures: '\r@s'}%
    \let\f@atureset\cstackempty
    \let\f@aturen@xt\n@xt
    \trace{sa}{Merging options}%
    \x@\fe@turemaths\r@s[] \E
    % Font-features should now be recorded in \f@atureset. Now Rebuid r@s based on its contents.
    \let\r@s\empty % MUST start with empy list!
    \def\d@##1\E{\x@\let\x@\t@st\csname ##1\endcsname\ifx\relax\t@st\else\edef\r@s{\ifx\r@s\empty:\else\r@s:\fi\t@st}\x@\let\csname ##1\endcsname\relax\fi}%
    \x@\cstackdown\f@atureset,\E
    \let\n@xt\f@aturen@xt
    \let\p@ram\r@s
  \fi
  \let\d@\d@gettheparam
}

\def\ch@ckfontfeatures#1#2\E{% Check for an initial colon
  \def\t@st{#1}%
  \ifx\t@st\c@lon
    \let\p@ram\empty
    \m@dernisefontfeatures#2:\E
    \ifx\p@ram\empty\else
      \edef\p@ram{=[\p@ram]}%
    \fi
  \fi
}

\def\ch@ckfontfeaturesbrk#1[#2]#3\E{% Check for nothing in brackets
  \def\t@st{#2#3}%
  \ifx\t@st\empty
    \let\p@ram\empty
    \m@dernisefontfeatures#1:\E
    \ifx\p@ram\empty\else
      \edef\p@ram{=[\p@ram]}%
    \fi
  \fi
}

\def\m@dernisefontfeatures#1:#2\E{%
  \def\t@st{#2}%
  \let\n@xtmff\m@dernisefontfeatures
  \ifx\t@st\empty
    \let\n@xtmff\endf@aturemaths
  \fi
  \def\t@st{#1}%
  \ifx\t@st\empty\else
    \ifx\p@ram\empty
      \edef\p@ram{#1}%
    \else
      \edef\p@ram{\p@ram,#1}%
    \fi
  \fi
  \n@xtmff #2\E
}

\def\endf@aturemaths#1\E{}
\def\fe@turemaths #1[#2] #3\E{%
  \trace{sa}{Feature maths #1.#2.#3}%
  \let\n@xt\fe@turemaths
  \edef\t@st{#2}\ifx\t@st\empty
    \let\n@xt\endf@aturemaths
    \trace{sa}{No more features #2}%
  \else
    \edef\t@st{\zap@space #1 \empty}\ifx\t@st\empty\let\t@st\pl@s\fi
    \if\t@st\m@nusLtr\relax
      \let\d@\rmf@ature
      \trace{sa}{Removing font features #2}%
    \else
      \let\d@\addf@ature\relax
      \if\t@st\equ@ls
        \trace{sa}{Exact font features #2}%
        \let\f@atureset\cstackempty
      \else\if\t@st\pl@sLtr\relax\else
          \message{Unrecognised control flag '\x@\meaning\t@st' \x@\the\x@\catcode\x@`\t@st  in fontfeature definition #1[#2], processing like '+' (valid: +/-/=)}%
        \fi
        \trace{sa}{Adding font features #2}%
      \fi
    \fi 
    \x@\cstackdown #2,\E
  \fi
  \edef\t@st{#3}\ifx\t@st\empty
    \let\n@xt\endf@aturemaths
  \fi
  \n@xt #3 []\E
}

\def\rmf@ature#1\E{%
  \x@\decodeff@ature #1==\E
  \x@\let\csname \@ffname\endcsname\relax
}
\def\addf@ature#1\E{%
  \x@\decodeff@ature #1==\E
  \x@\edef\csname \@ffname\endcsname{\@ffvalue}%
  \edef\f@atureset{\f@atureset,\@ffname}% Duplicate values don't matter, as we'll unset values as the list is generated.
}


\def\decodeff@ature#1#2=#3=#4\E{% [+foo] or [foo] adds '+foo' [-foo] removes '+foo',   [foo=bar] adds foo=bar
  \edef\t@st{#3}\ifx\t@st\empty
    \if#1+\edef\@ffname{@@ffname-#2}\edef\@ffvalue{+#2}%
    \else
      \if#1-\edef\@ffname{@@ffname-#2}\edef\@ffvalue{}%
      \else\edef\@ffname{@@ffname-#1#2}\edef\@ffvalue{+#1#2}%
      \fi
    \fi
  \else
    \edef\@ffname{@@ffname-#1#2}\edef\@ffvalue{#1#2=#3}\E
  \fi
}

%Which parameters need to obey the \Regular reset?
\x@\let\csname getthep@ram-raise\endcsname\getthep@ramObeyRegularTop
\x@\let\csname getthep@ram-underline\endcsname\getthep@ramObeyRegularTop
\x@\let\csname getthep@ram-bold\endcsname\getthep@ramObeyRegularTop
\x@\let\csname getthep@ram-italic\endcsname\getthep@ramObeyRegularTop
%\x@\let\csname getthep@ram-fontfeatures\endcsname\getthep@ramObeyRegularTop
\x@\let\csname getthep@ram-smallcaps\endcsname\getthep@ramObeyRegularTop
\x@\let\csname getthep@ram-fontfactor\endcsname\getthep@ramObeyRegularTop

\def\getmcp@ram#1{\getp@ram{#1}{\styst@kfirst}{\styst@k}}
\def\getmcp@ramR#1{\getp@ram{#1}{\styst@kfirst}{\styst@k}}

\def\getFp@ram#1#2#3{%\ifnotracestacks\pushtracing\fi
  \let\p@ram\relax
  \def\d@##1+##2+##3\E{\d@code{##1}{##2}\edef\tmpa{#2}\ifx\tmp\tmpa\getmcp@ram{#1}\else\getp@ram{#1}{#2}{#2}\fi}%
  \mctop\ifnotracestacks\poptracing\fi}
\def\getFp@ramR#1#2#3{\ifnotracestacks\pushtracing\fi
  \def\d@##1+##2+##3\E{\d@code{##1}{##2}\edef\tmpa{#2}\ifx\tmp\tmpa\getmcp@ramR{#1}\else\getp@ram{#1}{#2}{#2}\fi}%
  \mctop\ifnotracestacks\poptracing\fi}
%-csty_getmcparam

%+csty_getmcfontsize
%\def\getmcfonts@ze{\ifnotracestacks\pushtracing\fi
%  \getp@ram{fontsize}{p}{p}\ifx\p@ram\relax\edef\ps@ze{12}\else\edef\ps@ze{\p@ram}\fi
%  \trace{sa}{getmcfonts@ze starting with \ps@ze}%
%  \def\d@##1+##2+##3\E{\d@code{##1}{##2}\ifx\tmp\empty\else
%      \let\tmpstylen@me\tmp
%      \dimen0=\c@rrfontsize pt 
%      \ifnum\if##1\ss@Para 1\else\if##1\ss@Note 1\else0\fi\fi =1
%        \getp@ram{fontsize}{\tmpstylen@me}{\tmpstylen@me}\dimen0=\ifx\p@ram\relax 12\else\p@ram\fi pt%
%      \else\getp@ram{fontscale}{\tmpstylen@me}{\tmpstylen@me}\edef\sc@le{\p@ram}%                            %(1)
%        \ifx\p@ram\relax\edef\sc@le{1}\getp@ram{fontsize}{\tmpstylen@me}{\tmpstylen@me}%
%          \ifnum\ifx\p@ram\relax 1\else\ifx\p@ram\ps@ze 1\else 0\fi\fi =1
%            \dimen0=\sc@le\dimen0\else\dimen0=\p@ram pt\fi \else\dimen0=\sc@le\dimen0                                                  %(2)
%    \fi\fi\xdef\c@rrfontsize{\strip@pt{\dimen0}}\fi\trace{sa}{\space working c@rrfontsize from ##2 is \c@rrfontsize}}%
%  \mcup\trace{sa}{set c@rrfontsize for \mcstack\space is \c@rrfontsize}\ifnotracestacks\poptracing\fi}             %(3)

\x@\def\csname getthep@ram-fontsize\endcsname#1{%Called with p@ssParams set up. work up through possParam list
  \let\savep@ssParams\p@ssParams\let\savestyst@k\tmpstyst@k
  \getp@r@m{fontsize}{p}%
  \ifx\p@ram\relax \edef\ps@ze{12}\edef\tmpfr@m{default}\else \edef\ps@ze{\p@ram}\edef\tmpfr@m{p}\fi
  \trace{sa}{getthep@ram-fontsize starting with \ps@ze\space from \tmpfr@m}%
  \def\d@##1\E{\getp@r@m{fontsize}{##1}\ifx\p@ram\relax\else\edef\tmpfr@m{##1}\let\ps@ze\p@ram\fi}\cstackbottom\p@ssParams,\E
  \let\p@ssParams\savep@ssParams\let\tmpstyst@k\savestyst@k
  \trace{sa}{getthep@ram-fontsize found \ps@ze\space from \tmpfr@m}%
    \dimen0=\ps@ze pt 
  \def\d@##1\E{\edef\tmpstylen@me{##1}%
    \getp@r@m{fontscale}{\tmpstylen@me}\edef\sc@le{\p@ram}%                            %(1)
    \ifx\p@ram\relax\edef\sc@le{1}\getp@r@m{fontsize}{\tmpstylen@me}%
      \ifnum\ifx\p@ram\relax 1\else\ifx\p@ram\ps@ze 1\else 0\fi\fi =1
        %\dimen0=\sc@le\dimen0 % If this is reached, sc@le is 1, so it does nothing
      \else
        \dimen0=\p@ram pt%
        \let\tmpfr@m\tmpstylen@me
      \fi
    \else
      \dimen0=\sc@le\dimen0                                                  %(2)
      \let\tmpfr@m\tmpstylen@me
    \fi
    \xdef\c@rrfontsize{\strip@pt{\dimen0}}%
    \trace{sa}{\space working c@rrfontsize from ##1 is \c@rrfontsize}}%
  \x@\cstackup\p@ssParams,\E
  \edef\p@ram{\c@rrfontsize}%
  \trace{sa}{set c@rrfontsize for \tmpstyst@k\space from \tmpfr@m\space is \c@rrfontsize}%
}

%-csty_getmcfontsize

%+csty_makestyle
\def\m@kestyle{{\uccode`\|=`\\\uppercase{\message{|\m@rker}}}
 \x@\let\csname font<\m@rker>\endcsname=\relax 
 \ifx\styl@type\P@ra \x@\defp@rstyle\x@{\m@rker}
 \else\ifx\styl@type\Ch@r \x@\defch@rstyle\x@{\m@rker}
 \else\ifx\styl@type\N@te \x@\defn@testyle\x@{\m@rker}
 \else\ifx\styl@type\MSt@ne \x@\defmst@nestyle\m@rker\E
 \else \message{unknown style type '\styl@type'}
 \fi\fi\fi\fi}

%:
% \def*style: define a USFM marker as a paragraph, character or note style marker
% which will expand to \p@rstyle, \ch@rstyle or \n@testyle, with the marker name
% as its parameter
%
% \csname...\endcsname is used because the marker may contain numbers as well as letters
%
\def\defp@rstyle#1{\x@\def\csname#1\endcsname{\trace{M}{ParMarker: #1}\p@rstyle{#1}}\x@\global\x@\let\csname stylet@pe-#1\endcsname\ss@Para}
\edef\pl@scatcode{\the\catcode`+}
\catcode`+=11
\def\defch@rstyle#1{\x@\def\csname#1\endcsname{\trace{M}{CharMarker: #1}\ch@rstyle{#1}}\x@\def\csname +#1\endcsname{\ch@rstylepls{#1}}\x@\global\x@\let\csname stylet@pe-#1\endcsname\ss@Char}
\catcode`+=\pl@scatcode
\def\defn@testyle#1{\x@\def\csname#1\endcsname{\trace{M}{NoteMarker: #1}\n@testyle{#1}}\m@ken@tecl@ss{#1}\x@\global\x@\let\csname stylet@pe-#1\endcsname\ss@Note}
%Make a milestone marker if there isn't a character style with the same prefix
\def\defmst@nestyle#1\E{\trace{M}{MilestoneMarker: #1}\x@\let\x@\tmp\csname #1\endcsname\x@\global\x@\let\csname stylet@pe-#1\endcsname\ss@Mstn
  \ifx\tmp\relax
    \getp@r@m{endmarker}{\m@rker}%
    \ifx\p@ram\relax
      \x@\def\csname#1\endcsname{\def\milestoneOp{}\mst@nestyle{#1}}%
    \else
      \x@\def\csname#1\endcsname{\def\milestoneOp{s}\mst@nestyle{#1}}%
      \x@\def\csname\p@ram\endcsname{\def\milestoneOp{e}\mst@nestyle{#1}}%
    \fi
  \fi} 
%-csty_makestyle

%+csty_sethook
% Allow "hooking" custom TeX code at the beginning or ending of specific styles.
% e.g. `\sethook{start}{s1}{-}` put a hyphen at the start of every section heading
\def\sethook#1#2#3{\x@\gdef\csname #1-#2\endcsname{#3}}
\def\setbookhook#1#2#3{\x@\gdef\csname book#1-#2\endcsname{#3}}
%-csty_sethook

%+csty_testpublishability
% All types of marker use this to check for the "nonpublishable" property
\def\t@stpublishability#1{\n@npublishablefalse % check if "nonpublishable" occurred in the marker's \Properties
 \getp@ram{properties}{#1}{#1}%Should not deceend!
 \x@\t@stnonpub\p@ram nonpublishable!}
\def\t@stnonpub #1nonpublishable#2!{%
 \def\t@st{#2}\ifx\t@st\empty\else\n@npublishabletrue\fi}
\newif\ifn@npublishable
%-csty_testpublishability

% Patch in special handling for \c and \v.
%+csty_defv
\newif\ifhangingverse

\catcode`-=11
\def\setPrintVerse #1 #2.{\gdef\print@bleverse{#1-#2}}
\def\noPrintVerse{\global\c@ncelfirstversetrue}
\def\spl@tprintableverse{\x@\spl@tverses\print@bleverse --\relax}
\catcode`-=12
\def\pdfv@rsemark{%TODO: Not every verse.
  \special{pdf:dest (\id@@@\c@rrdstat.\ch@pter:\v@rsefrom) [@thispage /XYZ @xpos @ypos null]}%
}%

\def\MakeChapterLabel#1#2{#1\ #2}
\gdef\reference{}%

\def\@ddcvhooks{
 % called after \c and \v have been processed in stylesheet at beginning of typesetting file
 \let\@V=\v% remember \v defn for (A) below
 % Define a macro to read an actual space delimited verse number, and then check if the next token is \vp
 \def\@v@ ##1 {\gdef\v@rse{##1}\trace{V}{\c@rrdstat\ch@pter:\v@rse}\ifdiglot\x@\global\x@\let\csname @v@rse\g@tdstat\endcsname\v@rse\fi\global\let\print@bleverse\v@rse\futurelet\mayb@vp\@V@}%
 % If the next token IS vp, then call relevant code to interpret before dropping 
 \def\@V@{%
  \let\nxt\@@V@@
  \ifx\mayb@vp\vp\let\nxt\@@VP@@\fi\nxt}%
 \def\@@VP@@\vp ##1\vp*{\xdef\print@bleverse{##1}\@@V@@}
 \def\@@V@@{%
  \the\prev@rsehooks
  \spl@tprintableverse
  \pdfv@rsemark % make book.chapter:verse mark (plural if this is a range)
  %\ if not cancelling first verse, output verse number
  \ifc@ncelfirstverse\trace{V}{Cancelling \ch@pter:\v@rse}\else
   \ifrotate
    \setbox0=\hbox{\printverse}\dimen0=\wd0%
    \dimen1=\ht0\dimen2=0.5\dimen0\advance\dimen2 -0.5\ht0%
    \setbox0=\hbox{\lower\dimen2\hbox to \dimen1{\kern\ht0\special{x:gsave}\special{x:rotate 90}%
     \box0\special{x:grestore}\hss}}\ht0=0pt\dp0=0pt\@V\box0\@V*%
     \kern\AfterVerseSpaceFactor\FontSizeUnit
   \else
    \ifreftooltip\ch@ptooltipstart\fi\printverse\ifreftooltip\ch@ptooltipend\fi
    %\printverse
   \fi
  \fi
  \egroup
  \m@rkverse % generate a milestone for running headers
  % run the hooks for picture insertion, paragraph adjustments, etc.
  % these hooks were added via \addtoversehooks
  \the\v@rsehooks
  % remember current chapter:verse, not currently used. Could potentially be used
  % to insert chapter:verse in footnotes.
  \gdef\reference{\ch@pter:\v@rse}%
  \ifhangingverse
    \trace{hv}{Hanging verse \reference}%
    \hangingversefalse
  \fi
  \nobreak\hskip0.5sp\relax% let \x detect that there's "space" (or maybe drop-chapter) here
 }
 \let\C@T\cat %Preserve \cat definition from stylesheet in case that's ever useful, and then restore it to correct version
 \let\cat\c@t
%-csty_defv
 % define handling for usfm \v marker
%+csty_verse
 \def\v{%
  \ifvmode\xdef\prev@rsemode{\m@rker}\else
    \ifhangingverse 
      \trace{hv}{Horiz Mode:cancel hangingverse}%
      \hangingversefalse % Cancel
    \fi
  \global\let\prev@rsemode\empty\fi
  \leavevmode % ensure we are in horizontal mode to build paragraph
  % set c@ncelfirstversetrue if we are immediately following a drop cap chapter number
  % and the user has requested omission of the first verse number in a chapter
  \c@ncelfirstversefalse
  \dimen0=\lastkern
  \ifnum\spacefactor=\n@wchaptersf \kern0sp % override \hangversenumber here
   \ifOmitVerseNumberOne \c@ncelfirstversetrue \fi \fi
  % -1sp indicates that a hanging verse number has been requested here. But that does not seem as reliable as \ifhangingverse
  % When this happens we make ll@p use llap to right justify the verse number at the current position.
  % Otherwise make ll@p do nothing.
  \ifdim\lastkern=-1sp \hangingversetrue\fi
  \ifhangingverse\let\printverse\hangprintverse\else\let\printverse\defaultprintverse\fi
  %\trace{hv}{lastkern \the\dimen0}%
  \bgroup\@v@}%
 %{.. was: \ll@p\bgroup\m@kedigitsother\@v@}%
 % remember chapter macro for later
%-csty_verse
%+csty_chapdef
 \let\@C=\c
 % Define a macro to process space delimited chapter number
 % ## is necessary because this is a nested macro definition
 \def\@c@ ##1 {\ifsk@pping \egroup \fi % if we were skipping, stop it
   \gdef\ch@pter{##1}% remember chapter number
   \gdef\ch@ptert@xt{##1}% the chapter number we will actually output
   \gdef\v@rse{}% clear verse number
   \ifdiglot
     \trace{c}{Saving chapter number '\ch@pter' for side: \c@rrdstat}%
     \x@\global\x@\let\csname @ch@pter\g@tdstat\endcsname\ch@pter
     \x@\global\x@\let\csname @ch@ptert@xt\g@tdstat\endcsname\ch@ptert@xt
     \x@\global\x@\let\csname @v@rse\g@tdstat\endcsname\v@rse
   \fi
   \ifReenterIntroOK\else\global\intr@donetrue\fi% Are markers starting 'i' allowed to flip us back into one-column mode?
   \uppercase{\xdef\c@rref{\id@@@\ifdiglot\c@rrdstat\fi\ch@pter.0}}%
   \global\let\print@bleverse=\v@rse
   \op@ninghooks{start}{c}%
   \m@kedigitsletters
    % If omitting chapter number, skip this. 
    % We would have to set \OmitChapterNumbertrue on a book by book basis
    % in the configuration file in order to omit chapter numbers for single chapter books.
   \ifOmitChapterNumber
     \trace{c}{Chapter numbers ommitted}%
     % If we are mid paragraph remove the space caused by the \n before the \c.
     % Not sure why the damage is already done (e.g. end of ior*)
     %\ifhmode\unskip\fi
     \unskip
     \pdfb@@kmark \pdfch@ptermark
   \else
     % if chapter label not present, set \ch@pter[X]waiting to cause chapter number to be printed
     % at start of first text paragraph.
     \x@\global\x@\let\x@\ch@pl@bel\csname ch@plabel\g@tdstat\endcsname% global as p@rstyle{cl} may end a group.
     \ifx\ch@pl@bel\empty\relax \x@\global\x@\let\csname ch@pter\g@tdstat waiting\endcsname\ch@pter
       \trace{c}{No chapter label in use : \ch@pter}%
     \else % if \cl present, output chapter label here
       \trace{c}{Chapter label in use: \ch@pl@bel \ch@pter}%
       \p@rstyle{cl}\MakeChapterLabel{\ch@pl@bel}{\ch@pter}%
       \pdfb@@kmark \pdfch@ptermark
     \fi
   \fi}
 %
 \def\c{\m@kedigitsother\@c@}
 \def\cp ##1 {\gdef\ch@ptert@xt{##1}}
%-csty_chapdef
 % set default value for drop cap chapter number size if not already set
%+csty_dropchap
 \ifdim\dropnumbersize=0pt
   \calcdr@psize
 \fi
 \def\m@rker{c}\defp@ram{fontsize}{\dr@psize}%
 \ifdiglot \repeatsetdropc@p\fi%Can't do this here since things get lowercased
}
%-csty_dropchap
\def\bridgeVerses #1.#2-#3.{
  \addtrigger{#1.#2-preverse}{\setPrintVerse #2 #3.}
  \addtrigger{#1.#3-preverse}{\noPrintVerse}
}

\newbox\pstr@t
\newif\ifCalcChapSize \CalcChapSizetrue
\def\calcdr@psize{%
  \should@xist{c}%
  \ifdiglot 
    \trace{d}{Calculating drop: \c@rrdstat}%
    \should@xist{c\c@rrdstat}%
    \s@tbaseline{p\c@rrdstat}{p\c@rrdstat}%
  \else
    \s@tbaseline{p}{p}%
  \fi
  \edef\@seglyphmetrics{\the\XeTeXuseglyphmetrics}%
  \s@tfont{p}{p}%workaround for Lotus Elam Bold issue - ensure the \p font is loaded first
  \mkstr@t\setbox\pstr@t=\copy\str@tbox
  \XeTeXuseglyphmetrics=0\relax\setbox1=\hbox{0}% get font metrics          %(1)
  \XeTeXuseglyphmetrics=3\s@tfont{c}{c}\relax\setbox0=\hbox{0123456789}% set a box at the size specified by the \c marker
  \XeTeXuseglyphmetrics=\@seglyphmetrics% \tracingcommands=0\MSG{useglyphmetrics=\the\XeTeXuseglyphmetrics}
  \getp@ram{fontsize}{c}{c}\dropnumbersize=\ifx\p@ram\relax 0pt\else\p@ram\FontSizeUnit\fi \relax % C in following
  \getp@ram{fontsize}{p}{p}\dimen255=\p@ram\FontSizeUnit
  \advance\dimen255 by 0.5\dimen255
  \advance\dimen255 by 2sp
	% if font size specified for chapter number stylesheet is less than 1.5*paragraph font size ...
    %\ifdim\dropnumbersize<\dimen255
    \trace{c}{CalcChapSize: \the\dropnumbersize <  \the\dimen255}%
    \ifCalcChapSize\ifdim\dropnumbersize<\dimen255
      \ifdim\dropnumbersize<1pt
        \dropnumbersize=\baselineskip
      \fi
	  %clear font<c> so that it will get rebuilt on next setfont for \c (after we have figured out real size)
      \x@\let\csname font<c>\endcsname=\relax 
	  % Calculate drop cap height = base line height + (lower case) x height
      % get x height (appendix F) 5 is x-height of font X=x + baselineskip
      \s@tfont{p}{p}\dimen0=\fontdimen5\font\advance\dimen0\baselineskip           %(2)
      \dimen1=\dimen0 \multiply\dimen1 by 128 \divide\dimen1 by \dropnumbersize % X/C * 128
      \dimen2=\ht1 \advance\dimen2 by \dp1 \advance\dimen2 by \baselineskip % total height available: T
      \dimen3=128\dimen0 \divide\dimen3 by \dimen2 \multiply\dimen3 by \dropnumbersize \divide\dimen3 by 128 % CX/T
      \dimen4=\dp1 \multiply\dimen4 by \dimen1 \advance\dimen4 64sp\divide\dimen4 by 128 % scaled available
      \trace{h}{dimen0=\the\dimen0, dimen1=\the\dimen1, dimen2=\the\dimen2, dimen3=\the\dimen3,
                dimen4=\the\dimen4, dp0=\the\dp0, ht0=\the\ht0, dp1=\the\dp1, ht1=\the\ht1,
                FontSizeUnit=\the\FontSizeUnit, dropnumbersize=\the\dropnumbersize}%
      \ifdim\dimen3<\ht0 \dimen2=\dimen3\else\ifdim\dimen4<\dp0\dimen2=\dimen3\else\dimen2\ht0\fi\fi % min(CX/T, ht(0))
      \trace{h}{starting height=\the\dimen2}%
      % The following calculations are scaled by 128 in order to get better accuracy even though calculations are integer values
	  % divide desired height by current height
      \multiply\dimen0 by 128                                                   %(3)
      \divide\dimen0 by \dimen2
	  % multiply by original size from \c marker in stylesheet
      \trace{h}{font x-height \x@\the\fontdimen5\font, ht of digits \the\ht0, dp of digits \the\dp0, baselineskip \the\baselineskip, for dropcaps \the\dimen0\space multiply by \the\dropnumbersize}%
      \dimen3=\dimen0 % depth scale factor * 128
      \multiply\dimen0 by \dropnumbersize
      \dropnumbersize=\dimen0
      % divide by the \FontSizeUnit to get value for fontsize param of marker \c
      \divide\dropnumbersize by \FontSizeUnit
      \multiply\dropnumbersize by 512 % (2^16 / 128)
      \multiply\dimen3 by \dp0  % new depth
      \advance\dimen3 64sp\divide\dimen3 by 128
      \trace{h}{dropnumbersize=\the\dropnumbersize, dimen1=\the\dimen1, dimen3=\the\dimen3}%
      \ifdim\dimen3>\dp1 % would overflow downwards. Need to shift              %(4)
        \advance\dimen3 by -\dp1% \divide\dropnumberraise by 2
        \setp@ram{raise}{c\ifdiglot\c@rrdstat\fi}{\the\dimen3}%
      \fi
      \trace{h}{dropnumbersize = \the\dropnumbersize, dropnumberraise = \the\dropnumberraise}%
    \fi\fi
    \edef\dr@psize{\strip@pt\dropnumbersize}%
}

\def\repeatsetdropc@p{%The above code is in \lowercase which  breaks L and R
  \message{Setting \c@rrdstat drop-cap size to \dr@psize}%
  \setp@ram{fontsize}{c\c@rrdstat}{\dr@psize}%
  \g@tfontname{c}{c}\x@\let\csname font<\f@ntstyle>\endcsname=\relax
  \let\d@neit\c@rrdstat
  \def\col@do##1{
    \setc@rdstat{##1}%
    \ifx\c@rrdstat\d@neit\else% Avoid double-processing
      \setLRspecific
      \dropnumbersize=0pt
      \calcdr@psize
      \g@tfontname{c}{c}\x@\let\csname font<\f@ntstyle>\endcsname=\relax
      \message{ Setting \ drop-cap size to \dr@psize}%
      \setp@ram{fontsize}{c##1}{\dr@psize}%
    \fi
  }%i
  \x@\each@col\diglot@list\E
  %\x@\let\csname font<cL>\endcsname=\relax %
  %\x@\let\csname font<cR>\endcsname=\relax %
}
%-csty_dropchap

%+csty_strippt
\let\E=\empty
{\catcode`P=12 \catcode`T=12 \lowercase{\gdef\@strip@pt#1PT#2\E{#1} \xdef\@pt{PT}}}
\def\strip@pt#1{\x@\x@\x@\@strip@pt\x@\the\x@ #1\@pt\E}
%-csty_strippt

%+csty_hangverse
\def\AfterVerseSpaceFactor{2}
\newif\ifOmitChapterNumber
\newif\ifc@ncelfirstverse
\lowercase{
\def\simpleprintv@rse{\ch@rstylepls{v}~\plainv@rse\ch@rstylepls{v}*%
     \kern\AfterVerseSpaceFactor\FontSizeUnit}
}
\def\plainv@rse{\AdornVerseNumber{\v@rsefrom}\ifx\v@rsefrom\v@rseto       %(2)
                 \else\endash\AdornVerseNumber{\v@rseto}\fi}
\def\hangprintv@rse{\ifx\v@rsefrom\v@rseto\relax\defaultprintverse
    \else {\s@tsideskips{v} \h@ngprintv@rse{0}}\box0\kern\AfterVerseSpaceFactor\FontSizeUnit\fi
}
\def\hangprintverse{\llap\defaultprintverse}
\def\adornv#1{\AdornVerseNumber{#1}}
\def\ssf{\SuperscriptFactor}
\lowercase{
\def\h@ngprintv@rse#1{%
    \s@tbaseline{v}{\styst@k}%
    %\getp@ram{baseline}{v}\baselineskip=\ifx\p@ram\relax\baselineskip\else\p@ram\fi
    \dimen0=\ssf\baselineskip \advance\dimen0 by -\baselineskip
    \setbox1=\hbox{\ch@rstylepls{v}~\adornv{\v@rseto}\ch@rstylepls{v}*}%
    \global\setbox#1=\vbox to \ht1{\s@tbaseline{v}{v}\vtop{\rightskip=0pt\leftskip=0pt\hbox{\llap{\ch@rstylepls{v}~\endash\ch@rstylepls{v}*}%
                                 \vbox{\hbox to \wd1{\hskip\leftskip\ch@rstylepls{v}~\adornv{\v@rsefrom}\ch@rstylepls{v}*\hskip\rightskip}%
                                       %\kern\dimen0
                                        \box1}}}}%
    \trace{v}{hangprintverse height=\the\ht#1, width=\the\wd#1}}%
}
\let\defaultprintverse=\simpleprintv@rse
\let\printverse=\defaultprintverse                                               %(1)
\def\@ne{1}
\def\AdornVerseNumber#1{#1}                                                     %(3)

% put \hangversenumber into the <start> hook for a style such as \q1
% in order to 'hang' verse numbers into the paragraph indent of the style
\def\hangversenumber{\hangingversetrue\kern-1sp\relax}
%-csty_hangverse

% size to use for drop-cap style numbers; automatically calculated if not set
\newdimen\dropnumbersize
\newdimen\dropnumberraise

%+csty_define-hooks
% other modules can use \addtoversehooks to insert macros that will be executed at each verse
\def\addtoversehooks#1{\x@\global\x@\v@rsehooks\x@{\the\v@rsehooks #1}}
\def\addtopreversehooks#1{\x@\global\x@\prev@rsehooks\x@{\the\prev@rsehooks #1}}
\newtoks\v@rsehooks
\newtoks\prev@rsehooks
\def\lastc@rref{}\def\lastc@rradj{1}
% A CV hook is executed only for a particular CV when the verse milestone occurs.
\def\setcvhook#1#2{\x@\gdef\csname cv-#1\endcsname{#2}}
{\catcode`\1=11\xdef\@neletter{1}}
\def\docvhook{\edef\c@rref{\id@@@\ifdiglot\c@rrdstat\fi\ch@pter.\v@rse}\csname cv-\c@rref\endcsname
  %\trace{v}{verse: \v@rse, catcodes:\x@\getcatcodes\v@rse\E, and of one:\x@\getcatcodes\@neletter\E}%
  \ifnum\ifnum\lastc@rradj>1 1\else\ifx\v@rse\@neletter 1\else 0\fi\fi =1
    \xdef\lastc@rradj{1}\ifdiglot\xdef\lastc@rref{\dc@rref}\else\xdef\lastc@rref{\c@rref}\fi\fi
}
\addtoversehooks{\docvhook}
%-csty_define-hooks

\def\ch@plabel{}

% macros to switch digits between "letter" and "other" category; must be "letter" when reading USFM data,
% but "other" when we want to read numeric values
%+csty_fndigits
\def\m@kedigitsletters{\catcode`0=\el@ven \catcode`1=\el@ven \catcode`2=\el@ven \catcode`3=\el@ven
 \catcode`4=\el@ven \catcode`5=\el@ven \catcode`6=\el@ven \catcode`7=\el@ven
 \catcode`8=\el@ven \catcode`9=\el@ven \catcode`-=\el@ven \relax}
\def\m@kedigitsother{\catcode`0=\tw@lve \catcode`1=\tw@lve \catcode`2=\tw@lve \catcode`3=\tw@lve
 \catcode`4=\tw@lve \catcode`5=\tw@lve \catcode`6=\tw@lve \catcode`7=\tw@lve
 \catcode`8=\tw@lve \catcode`9=\tw@lve \catcode`-=\tw@lve \relax}
\def\el@ven{11}
\def\tw@lve{12}
%-csty_fndigits

\let\book\relax
\def\b@okShort{}%
\def\ch@pter{}
\def\v@rse{}
\def\m@rknum{1}
\let\@ch@pter=\ch@pter
\let\@ch@pterR=\ch@pter
\let\@v@rse=\v@rse
\let\@v@rseR=\v@rse
\def\m@rkverse{\ifdiglot\m@rkverseDglot\else\m@rkverseNml\fi}
\def\m@rkverseNml{\mark{\book:\ch@pter:\print@bleverse}\trace{H}{mark: \book:\ch@pter:\print@bleverse}} %normal verse
\def\m@rkverseDglot{\mark{\book:\ch@pter:\print@bleverse}\edef\cur@ds{\g@tdstat}\marks\m@rknum{\csname book\cur@ds\endcsname:\csname @ch@pter\cur@ds\endcsname:\print@bleverse}} % diglot verse

\def\n@xtparstart{}
\def\print@bleverse{}
%+csty_addspecialhooks
\begingroup
\obeylines%
\gdef\@ddspecialhooks{% there are a few special USFM markers that we give "magic" properties
 \let\pb=\pagebreak %pb isn't a paragraph style, 
 \let\@H=\h%
 \def\h{\bgroup\deactiv@tecustomch@rs\catcode`\/=12
        \obeylines\@h}% \h gets stored as the book name (for references)
 \def\@h ##1^^M{\x@\gdef\csname book\g@tdstat\endcsname{##1}\trace{ss}{heading book\g@tdstat=##1}\egroup}% store book id in \book / bookR
 \def\h@h{\bgroup\deactiv@tecustomch@rs\catcode`\/=12 \obeylines\@hh}% \h1 gets stored as the book name (for references)
 \def\@hh ##1^^M{\x@\gdef\csname b@ok1\g@tdstat\endcsname{##1}\trace{ss}{heading b@ok1\g@tdstat=##1}\egroup}% store book id in \b@ok1 / b@ok1R
 \def\bookalt{\csname b@ok1\g@tdstat\endcsname}%
 \x@\let\csname h1\endcsname=\h@h
 %
 \let\@CL=\cl%
 \def\do@CL{\x@\global\x@\let\csname ch@pter\g@tdstat waiting\endcsname\relax\gdef\n@xtparstart{\pdfch@ptermark}\@CL}%
 \def\cl{\x@\ifx\csname ch@pter\g@tdstat waiting\endcsname\relax \let\n@xt=\st@recl\else\let\n@xt=\do@CL\fi \n@xt}%
 \def\st@recl{\bgroup\obeylines\@cl}% \cl gets stored as chapter label and changes \c format
 \def\@cl ##1^^M{\x@\gdef\csname ch@plabel\g@tdstat\endcsname{##1}\egroup}%
 %
 \let\@ID=\id%
 \def\id{\bgroup\obeylines\unc@tcodespecials \m@kedigitsother
   \catcode32=10 \@id}% \id gets stored and printed in the margin with cropmarks
 \def\@id ##1^^M{\gdef\c@rrID{##1}\uppercase{\@@id##1ZZZ\end}\egroup
        \gdef\b@okShort{}%Prevent roll-over of short-form book name
        \csname bookstart-all\endcsname
        \csname bookstart-\id@@@\endcsname}%
 %
 \def\b{\endlastp@rstyle{b}\bgroup % End previous par before making a gap
   \getp@ram{baseline}{b}{\styst@k}\ifx\p@ram\relax
     \baselineskip=\onel@neunit%\s@tbaseline{p}{p}
     \vskip\baselineskip
    \else
      \vskip\p@ram
    \fi\egroup} 
 \def\fig{\trace{g}{In fig}\bgroup\m@kedigitsother\catcode`\:=12\catcode`\/=12\catcode`\\=12\ptx@fig}% \fig has its own special macro to parse the fields
 %
 \let\nb=\ptx@nb% \nb is a special code that suppresses a paragraph break across \c
 \global\let\@k=\k
 \def\setkr@f##1{\edef\kr@f{\zap@space ##1 \empty}}
 \catcode`\~=12 \lccode`\~=32\lowercase{
 \def\@@k##1\k*{%\k processing 2nd half
    \let\v@rse=\empty % Normally verses and keywords don't co-occur
    \edef\kr@f{\detokenize{##1}}%
    \x@\setkr@f\x@{\kr@f}%
    \trace{T}{keyword: \kr@f}%
    \runtrigg@rsother{k.\kr@f}%
    \activ@tecustomch@rs%   no CR here
    \@k~##1\endch@rstylen* %
    \runtrigg@rs{\id@@@ k.\kr@f}%
  }}
 \def\k{%\k processing
   \deactiv@tecustomch@rs% no CR here
   \@@k}
}%

\endgroup
\catcode`0=11\def\z@ro{0}\catcode`0=12
\newif\iftfmp \tfmpfalse
\def\unc@tcodespecials{\def\do##1{\catcode`##1=12 }\dospecials}
\def\@@id#1#2#3#4\end{\gdef\id@@@{#1#2#3}% get first 3 chars of the \id
  \global\let\ch@pter\z@ro\global\let\v@rse\@neletter
  \settrig@refcv\docvhook
  \ifdiglot\def\col@do##1{%
     \x@\global\x@\let\csname @ch@pter##1\endcsname\ch@pter
     \x@\global\x@\let\csname @ch@ptert@xt##1\endcsname\ch@pter
     \x@\global\x@\let\csname @v@rse##1\endcsname\v@rse}\x@\each@col\diglot@list\E\col@do{}%
  \fi}
\def\ptx@fig #1\fig*{\egroup\ifIncludeFigures\tfmptrue\else\tfmpfalse\fi\IncludeFigurestrue
    \trace{g}{In ptx@fig #1}\d@figure{#1}\m@kedigitsletters
    \iftfmp\else\IncludeFiguresfalse\fi}                           %(1)
%-csty_addspecialhooks

%+csty_stack
% Comma Separated stack
\def\E{}
\def\cstackpop#1,#2\E{#2}
\def\cstackpush#1#2{#2,#1}
\def\cstackpeek#1,#2\E{#1}
\def\cstackup#1,#2\E{\edef\cstack@tmp{#2}\ifx\cstack@tmp\empty\let\n@x=\cstackrelax\else\let\n@x=\cstackup\fi\n@x #2\E\edef\cstack@tmp{#1}\ifx\cstack@tmp\empty\else\d@ #1\E\fi}                  %(1)
\def\cstackbottom#1,#2\E{\edef\cstack@tmp{#2}\ifx\cstack@tmp\empty\d@ #1\E\let\n@x=\cstackrelax\else\let\n@x=\cstackbottom\fi\n@x #2\E}                  %(1)
\def\cstackdown#1,#2\E{\edef\cstack@tmp{#1}\ifx\cstack@tmp\empty\else\d@ #1\E\fi\edef\cstack@tmp{#2}\ifx\cstack@tmp\empty\let\n@x=\cstackrelax\else\let\n@x=\cstackdown\fi\n@x #2\E}
\def\cstacktop#1,#2\E{\edef\temp{#1}\ifx\temp\empty\else\d@ #1\E\fi}
\def\cstackrelax#1\E{}
%-csty_stack

%+csty_mstack
% marker (comma) stack operations (mcstack)
\let\mcstack\mcstack@mpty
\def\mcpopsimple{\traceNum{s}{popsimple stack: \mcstack}\xdef\mcstack{\x@\cstackpop\mcstack\E}\ifx\mcstack\empty\global\let\mcstack\mcstack@mpty\fi\ifp@pping\else\s@tstyst@k\fi}
\def\mcpush#1#2{%
  \ifcsname isb@rrier-#1\endcsname%stop here?
    \let\tmpstyst@k=\empty
  \else
    \gett@psty
  \fi
  \d@code{#1}{#2}\ifx\tmp\h@phen % Not evertying appears in the styst@k e.g. cat should not
    \xdef\styst@k{\tmpstyst@k}%
  \else
    \xdef\styst@k{\tmp\ifx\empty\tmpstyst@k\else+\tmpstyst@k\fi}%
  \fi
  \mcp@sh{#1+#2+\styst@k}}
\def\mcpushE#1+#2\E{\trace{sa}{mcpushE '#1'+'#2'}\if #1\relax\relax\else\mcpush{#1}{#2}\fi}
\def\mcp@sh#1{\traceNum{s}{push #1 onto \mcstack}\xdef\mcstack{#1,\mcstack}}
\def\mcup{\x@\cstackup\mcstack\E}
\def\mcdown{\x@\cstackdown\mcstack\E}
\def\mctop{\x@\cstacktop\mcstack\E}
\def\mctopnoms{\let\d@a=\d@\gdef\d@##1+##2\E{\if ##1m\else\d@a ##1+##2\E\let\d@=\cstackrelax\fi}\mcdown}
\def\mcstackdown#1,#2\E{\let\n@x=\mcstackdown\edef\tmp{#1}%Iftemp, then rebulild mcstack with the modified/saved portion (msst@ck) on top, then any insertion (inst@ck) and  the unprocesed parts of the stack, and stop
    \ifx\tmp\empty\let\n@x=\cstackrelax\else\d@b #1\E\iftemp\xdef\mcstack{#2}\rebuild@mcstack{\msst@ck\inst@ck}\let\n@x=\cstackrelax\fi\fi\n@x #2\E}

    
\def\rebuild@mcstack#1{\trace{sa}{rebuild@mcstack '#1' (on \mcstack)}\ifx\mcstack\empty\let\mcstack\mcstack@mpty\fi\edef\r@mclist{#1}\ifx\r@mclist\empty\else\let\d@=\mcpushE\x@\cstackup\r@mclist\E\fi\s@tstyst@k}
\def\doB@pop@nonms#1+#2+#3\E{\if #1m\xdef\msst@ck{#1+#2,\msst@ck}\else\global\temptrue\fi}% For mcstackdown. Push milestones onto msst@ck. lose the 1st non-ms
\def\doB@save@ms#1+#2+#3\E{\trace{sa}{[#1+#2+#3] (\msst@ck)}\if #1\relax\relax\xdef\inst@ck{\inst@ck\mcstack@mpty}\temptrue\else\if #1m\relax\xdef\msst@ck{#1+#2,\msst@ck}\else\xdef\inst@ck{\inst@ck#1+#2,}\global\temptrue\fi\fi}% For mcstackdown. Push milestones onto msstack. bail out on 1st non-ms, saving it.

\def\mcpop{\traceNum{s}{mcpop stack: '\mcstack'}\global\tempfalse\xdef\inst@ck{}\xdef\msst@ck{}% Pop the first non-milestone
    \let\d@b=\doB@pop@nonms
    \x@\mcstackdown\mcstack\E\ifx\mcstack\empty\global\let\mcstack\mcstack@mpty\fi}
    
\def\mcpushunderms#1#2{\traceNum{s}{mcpushunder stack(#1+#2+...): \mcstack}\global\tempfalse\edef\msst@ck{}\xdef\inst@ck{#1+#2,}% push #1 above the first non-milestone
    \let\d@b=\doB@save@ms
    \x@\mcstackdown\mcstack\E\trace{sa}{->\msst@ck [\iftemp t\else f\fi], '\mcstack'}%
    \iftemp % It's been dealt with
    \else\rebuild@mcstack{\inst@ck}\fi
    \trace{sa}{Stack now \mcstack}}

\def\s@ttmpsty#1+#2+#3\E{\xdef\tmpstyst@k{#3}}
\def\gett@psty{\ifx\mcstack\empty\let\tmpstyst@k\empty\else{\let\d@\s@ttmpsty\mctop}\fi}
\def\d@code#1#2{\edef\tmp{#2}\ifx\tmp\empty\else\csname d@code-#1\endcsname\fi}
% milestone stack (MScstack)
%\gdef\MScstack{+,}  
%\let\MScstack@mpty\MScstack
%\def\MScup{\x@\cstackup\MScstack\E}
%\def\MScdown{\x@\cstackdown\MScstack\E}
%\def\MScpush#1{\traceNum{s}{push #1 onto milestone stack \MScstack}\xdef\MScstack{#1,\MScstack}}

% Test to see if a new THING pops other things.
\def\mct@stname#1#2{#1-mcp@ps-#2}
\def\mcp@pCont#1#2{\edef\t@@st{\mct@stname{#1}{#2}}\x@\xdef\csname\t@@st\endcsname{Y}} % Pop and continue
\def\mcp@pEnd#1#2{\edef\t@@st{\mct@stname{#1}{#2}}\x@\xdef\csname\t@@st\endcsname{P}} % Pop and end.
\def\mcnp@pSt@re#1#2{\edef\t@@st{\mct@stname{#1}{#2}}\x@\xdef\csname\t@@st\endcsname{S}} % no pop, Store and continue
%defs for comparison purposes  Now in ptx-constants

\mcp@pCont{\ss@Para}{\ss@Para}% new P pops P
\mcp@pCont{\ss@Para}{\ss@ChrP}% new P pops c (\+char)
\mcp@pCont{\ss@Para}{\ss@Char}% new P pops C (\char)
\mcp@pCont{\ss@Char}{\ss@ChrP}% new C pops c
\mcp@pCont{\ss@Char}{\ss@Char}% new C pops C
\mcp@pCont{\ss@Sbar}{\ss@Char}% Side bars pop character styles.
\mcp@pCont{\ss@Sbar}{\ss@ChrP}% Side bars pop character styles.
\mcp@pCont{\ss@Ctgy}{\ss@Ctgy}% new cat pops old cat
\mcp@pCont{\ss@Para}{\ss@ParCtgy}% new paragraph pops old parcat
\mcnp@pSt@re{*}{\ss@Mstn}% Milestones are always stored.
%Style end pops:
\mcp@pCont{\ss@Sbar*}{\ss@Ctgy}% sidebar end pops caT
\mcp@pCont{\ss@Sbar*}{\ss@Para}% sidebar end pops Paragraph
\mcp@pCont{\ss@Sbar*}{\ss@ChrP}% sidebar end pops embedded charstyle
\mcp@pCont{\ss@Sbar*}{\ss@Char}% sidebar end pops Charstyle
\mcp@pEnd{\ss@Sbar*}{\ss@Sbar}% sidebar end pops ONE sidebar, then finish
\mcp@pEnd{t*}{\ss@Ctgy}% cat end pops one old cat, then finish
\mcp@pEnd{t*}{\ss@ParCtgy}% cat end pops one old cat, then finish
\mcp@pCont{N*}{\ss@ChrP}% Ending Note pops c
\mcp@pCont{N*}{\ss@Char}% Ending Note pops C
\mcp@pCont{N*}{\ss@Para}% Ending Note pops P
\mcp@pCont{tc*}{\ss@Char}% Ending table cell pops C (\char)
\mcp@pCont{tc*}{\ss@ChrP}% Ending table cell pops c (\+char)
\mcp@pCont{\ss@Tbl*}{\ss@Ctgy}
\mcp@pCont{\ss@Tbl*}{\ss@Para}
\mcp@pEnd{\ss@Tbl*}{\ss@Tbl}

%No point going below these for font styling, etc.
\def\mcb@rrier#1{\x@\let\csname isb@rrier-#1\endcsname\empty}
\mcb@rrier{\ss@Para}
\mcb@rrier{\ss@Note}
\mcb@rrier{\ss@Sbar}
%Are these style modifiers (category or milestone)  or actual styles?
\def\msm@dfyname#1{#1-is-modifyer}
\def\msm@difyer#1{\edef\t@@st{\msm@dfyname{#1}}\x@\xdef\csname\t@@st\endcsname{Y}}
\def\g@tmodifyer#1{\edef\t@@st{\msm@dfyname{#1}}\x@\let\x@\t@stme\csname\t@@st\endcsname}
\msm@difyer{\ss@Ctgy}
\msm@difyer{\ss@MstnS}
\msm@difyer{\ss@Mstn}

\newif\ifp@pping
\p@ppingfalse
\def\last@popped@parstyle{}
\def\itp@psstore{}
\def\endit@c{\endch@rstylepls*}
\def\endit@C{\endch@rstyle*}
\def\endit@T{\cl@singhooks{end}{cat}\par\mcpopsimple\egroup\@ndc@t\s@tc@tpr@fix} %Must call closing hooks before end of group, Must call mcpopsimple.
\def\endit@t{\cl@singhooks{end}{cat}\mcpopsimple\s@tc@tpr@fix}

\def\whileitp@ps#1+#2+#3\E{% This almost does the same as \end@llcharstyle, but uses the new mechanism
  \global\p@ppingtrue
  \global\let\wip@do=\d@
  \trace{sa}{whileitp@ps:\mcch@cking:#1+#2+#3 (\mcstack) [\itp@psstore]}%
  \x@\let\x@\tmp\csname\mct@stname{\mcch@cking}{#1}\endcsname
  \ifx\tmp\relax %No specific style pop, is there a generic?
    \x@\let\x@\tmp\csname\mct@stname{*}{#1}\endcsname
  \fi
  \trace{sa}{whileitp@ps result:\tmp}%
  \if S\tmp\relax % Store it?
     \xdef\itp@psstore{#1+#2,\itp@psstore}%only store the marker and type, restacking will recaclulate the stystak
     \mcpopsimple
  \else
    \if Y\tmp\relax  %end it and pop it
      \if #1\ss@Para\relax\xdef\last@popped@parstyle{#2}\fi
      \ifcsname endit@#1\endcsname
        \trace{sa}{Complex pop endit@#1}% endit@ macros are assumed(requrired) to call mcpop
        \csname endit@#1\endcsname
      \else
        \trace{sa}{Simple pop sufficient}%
        \mcpopsimple
      \fi
    \else
      \let\wip@do=\cstackrelax %stop here
      \if P\tmp\relax % ....but end it and pop first
        \ifcsname endit@#1\endcsname
          \trace{sa}{Complex pop endit@#1}% endit@ macros are assumed(requrired) to call mcpop 
          \csname endit@#1\endcsname
	    \else 
          \trace{sa}{Simple pop sufficient}%
          \mcpopsimple
        \fi
      \fi
    \fi
  \fi
  \trace{sa}{whileitp@ps END:\mcch@cking:#1+#2+#3 (\mcstack) [\itp@psstore]}%
  \global\p@ppingfalse 
  \let\d@=\wip@do
}
\def\end@llpoppedstyles#1{\trace{sa}{end@llpoppedstyles #1 \ifp@pping WHILE already popping??\fi}%
  \global\let\itp@psstore\empty\xdef\mcch@cking{#1}\let\d@=\whileitp@ps
  \mcdown\ifx\itp@psstore\empty\else\x@\rebuild@mcstack\x@{\itp@psstore}\fi
  \s@tstyst@k
  \trace{sa}{end@llpoppedstyles END: \mcstack}}

\def\s@tstyst@k{%Get a + separated list of styles that ought to uniquely define a font/etc.
  %List excludes marker #1 and stops at a barrier (e.g. a par/note/sidebar)
  %defines \styst@kfirst to first value.
  \xdef\styst@k{\empty}%
  \xdef\styst@kfirst{\empty}%
  \def\d@##1+##2+##3\E{\ifcsname isb@rrier-##1\endcsname\let\d@=\cstackrelax\fi%stop here?
    \xdef\styst@k{##3}% The real one
    \xdef\tmpstyst@k{##3}% This will get over-ridden if there are other calls.
    \d@code{##1}{##2}%
    \ifx\tmp\h@phen
    \else
      \xdef\styst@kfirst{\tmp}%
      \let\d@=\cstackrelax
    \fi
  }\mcdown
  \global\let\tmpstyst@k\styst@k
  \global\let\tmpstyst@kfirst\styst@kfirst
  \trace{sa}{styst@k -> (\styst@kfirst/\styst@k)}%
}
\let\styst@kfirst\empty
\let\styst@k\empty
\let\tmpstyst@k\empty
%-csty_mstack

%+csty_stylesheet
% \stylesheet{...} reads a Paratext stylesheet line by line
%
{% Horrible piece of parsing to ignore everything after # except when there's #! at the start of a line.
  \catcode`\~=6
  \catcode`\#=12
  \gdef\styl@sheetline~1{%
     \x@\styl@sh@@tline ~1#!#\relax
  }
  \gdef\styl@sh@@tline~1#~2!~3#~4\relax{%
     %\trace{sa}{\detokenize{ReadLine '~1' # '~2' ! '~3'}}%
     \catcode`\~=13
     \catcode`\#=6
     \def\tmpa{~1}\def\tmpb{~3}%
     \def\tmpc{~1~2}%
     \ifx\tmpc\empty%There's a line-start #!
       \ifx\tmpb\empty\else % and the line isn't empty
         ~3\relax
       \fi
     \else %A normal line
       \ifx\tmpa\empty\else %that's not empty
         ~1\relax
       \fi
     \fi
  }
}
\newif\ifc@ntinue
\def\c@tegory{}
\newread\styl@sheet
\newif\ifnewstyles
\newstylestrue

\def\stylesheet#1{
 \newstylestrue
 \let\oldc@tegory=\c@tegory
 \m@kedigitsother
 \let\save@regular=\regular \let\regular=\Regular
 \let\save@bold=\bold \let\bold=\Bold
 \let\save@italic=\italic \let\italic=\Italic
 \ifLogFonts
   \openout\f@ntl@g="\j@bname.fontusage"
 \fi
 \openin\styl@sheet="#1" \c@ntinuetrue                      %(1)
 \ifeof\styl@sheet\errmessage{Paratext stylesheet "#1" not found}\else
   \trace{sa}{New Stylesheet: clearing fonts: \f@ntlist}%
   \ClearCachedFonts
   \message{Reading Paratext stylesheet "#1" (\c@tegory)...}%
   \endlinechar=-1
   %\catcode`\#=5 % paratext comment char
   \loop
    \catcode`\#=12
    \read\styl@sheet to \th@line                            %(2)
    \styl@sheetline{\th@line}
   % \th@line \relax
    \ifeof\styl@sheet \c@ntinuefalse \fi
    \ifc@ntinue\repeat
   \endlinechar=13
   \closein\styl@sheet
   \catcode`\#=6 % restore default TeX catcode
 \fi
 \let\regular=\save@regular
 \let\bold=\save@bold
 \let\italic=\save@italic
 \ifx\oldc@tegory\c@tegory\else\ifsubc@ll\else\message{Error in stylsheet: Stylesheet changed category from '\oldc@tegory' to '\c@tegory'. Resetting to '\oldc@tegory'}\global\let\c@tegory=\oldc@tegory\s@tc@tpr@fix\fi\fi%
 \global\let\c@tegory=\oldc@tegory
}
\newif\ifsubc@ll \subc@llfalse
%-csty_stylesheet

\newif\iflastptxfile
\newif\iflastpage
\newdimen\origb@selineskip %Save the initial baselineskip here
\newif\ifinperiph
\def\codesforperiphgr@b{\catcode13=12 \catcode92=12 }
\def\endofperiph{\relax}
{\catcode13=12 %
\gdef\@@rem{\bgroup\catcode13=12 \unc@tcodespecials\catcode`\\=12\@@rem@}%
\gdef\@@rem@#1^^M{\egroup\@@realrem #1}%
%Code to read periph sections line by line.
\gdef\@periph{\ifinperiph\endp@riph\inperiphfalse\fi\bgroup\catcode13=12 \@@periph}%
\gdef\@@periph #1^^M{\x@\@@@periph #1||\E}%
\gdef\g@tp@riphline#1^^M{\gdef\p@riphtst{#1}\trace{T}{Got line '\p@riphtst'}\ifx\p@riphtst\endofperiph\let\getp@riphline\endp@riph\else\ifx\p@riphtst\empty\else\testp@riph{#1}\fi\fi\getp@riphline}%
%% DO NOT COMMENT END OF FOLLOWING LINE:
\gdef\getp@riphlines{\bgroup\inperiphtrue\codesforperiphgr@b\let\p@riphlines\empty\let\getp@riphline\g@tp@riphline\everyeof{\relax
\noexpand} \getp@riphline}%
\gdef\n@wline{
}%
}

\gdef\n@wperiph{\expandafter\newperiphb\p@riphtst\E}%
%Code to check periph sections line by line until the next periph section or end of file. Slash is made inactive
\catcode`\~=0 \catcode`\\=12 
  ~gdef~@@testendperiph{\zendperiph}%
  ~gdef~testp@riph #1{~edef~t@mp{#1}~ifx~t@mp~@@testendperiph~inperiphfalse~fi
    ~expandafter~def~expandafter~p@riphtst~expandafter{#1}%
    ~ifx~p@riphtst~relax~inperiphfalse~fi%
    ~ifinperiph~expandafter~@testp@riph#1\periph~E~else~let~getp@riphline~endp@riph~E~fi}%
{~catcode13=12 %
  ~gdef~@testp@riph #1\periph#2~E{% NOTE that line feeds are not spaces any more! Comment them!
    ~def~tstb{#1}~ifx~tstb~empty%
      ~trace{T}{New periph starting}%
      ~let~getp@riphline~n@wperiph%
    ~else%
       ~xdef~p@riphlines{~ifx~p@riphlines~empty~else~p@riphlines~n@wline~fi~p@riphtst}%
    ~fi}% DO NOT COMMENT THE END OF THE NEXT LINE!
  ~gdef~newperiphb #1\periph#2~E{~xdef~p@riphlines{~ifx~p@riphlines~empty~else ~p@riphlines~n@wline~fi #1}~endp@riph ~periph #2
}%
}
~catcode`\=0 
\catcode`\~=13

\def\StorePeriph#1{\x@\global\x@\let\csname storeperiph-#1\endcsname\relax}
\def\NoStorePeriph#1{\x@\global\x@\let\csname storeperiph-#1\endcsname\undefined} %A thing let to undefined does not ifcsname
\def\KeepPeriph#1{\x@\global\x@\let\csname keepperiph-#1\endcsname\relax}

\StorePeriph{intbible}
\StorePeriph{intnt}
\StorePeriph{intot}
\StorePeriph{intpent}
\StorePeriph{inthistory}
\StorePeriph{intpoetry}
\StorePeriph{intprophesy}
\StorePeriph{intdc}
\StorePeriph{intgospels}
\StorePeriph{intepistles}
\StorePeriph{intletters}
\StorePeriph{cover}
\StorePeriph{spine}


\def\endp@riph{\global\let\periphattr@b\attr@b\egroup\egroup\let\tmp\p@riphlines\x@\global\x@\def\csname ms:zgetperiph=\periphattr@b\x@\endcsname\x@{\x@\scantokens\x@{\tmp}\ifcsname keepperiph-\periphattr@b\endcsname\else\global\keeptriggerfalse\fi}\everyeof{}}
\def\zendperiph{\ifinperiph\endp@riph\inperiphfalse\fi\endlastp@rstyle{periph}\global\let\periph@@@\empty}

\newif\ifStorePeriph\StorePeriphtrue

\def\@@@periph #1|#2|#3\E{%
  \def\tmp{#2}%
  \unset@ttribs
  \def\thisdefault@ttrkey{id}%
  \ifx\tmp\empty
    \let\attr@b\empty
  \else\let\@ttributes\tmp
    \parse@ttribs{\@ttributes}%
    %\let\d@\apply@attr@specials
    %\x@\cstackdown \attribsus@d,\E
    \get@ttribute{id}%
    \ifx\attr@b\relax
      \let\attr@b\empty
    \fi
  \fi
  \global\let\periph@@@\attr@b
  \tempfalse
  \ifStorePeriph
    \ifcsname storeperiph-\attr@b\endcsname \temptrue\fi
  \fi
  \iftemp
    %\tracingassigns=1
    \let\n@xt\getp@riphlines
  \else 
    \egroup
    \let\n@xt\@@realperiph
  \fi
  \n@xt
}

%+csty_onetime
\def\@netimesetup{
 % stuff to execute after loading all stylesheets but (just) before processing the first USFM file
 \ifdiglot
  \trace{d}{Onetimesetup, (\c@rrdstat)}%
 \else
   \ifRTL
     \BookOpenLefttrue
   \fi
 \fi
 \origb@selineskip=\baselineskip
 \s@tupsizes
 \gdef\id@@@{UNK}%
 \defineheads
 \ifnum 1=\ifVisTrace 1 \else \ifVisTraceExtra 1 \else 0\fi\fi 
   \font\VisTracefont="Arial":color=3f7f3f at 6pt\fi
 \ifdiglot\setc@rdstat{L}%
 \fi
 \let\@@realrem\rem
 \let\rem\@@rem
 \let\@@realperiph\periph
 \let\periph\@periph
 
 %\global\lastptxfiletrue
 \opentriglist "\j@bname.triggers"
 \let\@netimesetup=\relax}                                                      %(1)

\def\@netimesetuplater{
 \@ddcvhooks
 \@ddspecialhooks
 \the\@nithooks
 \let\@netimesetuplater=\relax}                                                      %(1)
%-csty_onetime

\newif\iftildenbsp
\tildenbsptrue

\def\prepusfm{
 %\catcode`\$=12
 \catcode`\^=12 % make these printable
 \catcode`\&=12
 \iftildenbsp
  \catcode`\~=13 %Normal value
 \else
  \catcode`\~=13 %printable
 \fi
 \catcode`\#=12
 \catcode`\{=12 \catcode`\}=12
 \catcode`\_=11 % treat as letter for TE custom styles
 \catcode`\/=\active
 \catcode13=10
 \catcode`+=11 % treat as letter
 \activ@tecustomch@rs
 % The last comment on all matters
 \catcode`\%=12
 \m@kedigitsletters
}

\def\unprepusfm{
 \m@kedigitsother
 \deactiv@tecustomch@rs
 \catcode13=5
 \catcode`\/=12
 \catcode`\#=6 \catcode`\%=14 % restore TeX meanings for those we might use
 \catcode`\{=1 \catcode`\}=2
 \catcode`+=\pl@scatcode % don't treat as letter any more
}

% Read the files associated with this book: usfm, adjustment list, picture list
%+csty_ptxfile_intro
\def\ptxst@rtp@ge{1}
\def\preptxfile{
 \outputpenalty=0
 \saveLRspecificOrig%This becomes a \relax after first use. Always call it.
 \kill@PossParamCache
 \global\intr@donefalse
 \xdef\ptxst@rtp@ge{\the\pageno}%
 \ifdiglot
    \trace{d}{New ptxfile}%
    \gdef\c@rrdstat{L}%
    \setLRspecific% \@netimesetup needs actual values, in a diglot there might not be any use of generic FontSizeUnit, etc.
 \fi
 \ifcsname KeepMyBrokenAdjList\endcsname
  \global\let\LRspecificCount\LRspecificCountBroken %Use old broken paragraph numbering
 \else
  \global\let\LRspecificCount\LRspecificCountFixed %Properly track paragraph numbers in diglots.
 \fi
 \@netimesetup
 \let\periph@@@\empty
 \ifdiglot\bgroup%This must be outside of @onetimesetup as there's no guarantee that diglot will be true for all ptxfiles.
   \diglots@tup
   \ifx\PDFBookMarkColumn\relax
     \def\col@do##1{\xdef\PDFBookMarkColumn{##1}}%Very crude... set PDFBookMarkColumn to last column
     \x@\each@col\diglot@list\E
   \fi
   \dimen0=\baselineskip
   \global\topskip=2\baselineskip
   \global\splittopskip=0pt
   \message{Setting up column-specific values for columns: \diglot@list}%
   \global\baselineDelta=-\maxdimen
   \def\col@do##1{% Per-column setup
     \setc@rdstat{##1}%
     \showLRspecific
     \setLRspecific\s@tb@dyb@seline{##1}%
     \x@\global\csname baselineDelta##1\endcsname=\dimexpr \baselineskip - \topskip\relax
     \ifdim\baselineDelta<\csname baselineDelta##1\endcsname
       \global\baselineDelta=\csname baselineDelta##1\endcsname
     \fi
     \saveLRspecificSide{##1}%
     \bgroup 
       \let\mcstack\mcstack@mpty
       \s@tfont{p}{p}%
       \setbox\f@@tstrut=\hbox to 0pt{\XeTeXuseglyphmetrics=0\hbox{\char32}\s@tfont{v}{v+p}\getp@r@m{raise}{v}\setbox0\hbox{\char32}
            \ifx\p@ram\relax\else\dimen9=\p@ram\ifdim\dimen9<0pt\lower-\dimen9 \else\raise\dimen9 \fi\box0 \fi
       \hss}%
       %\showbox\f@@tstrut
       \trace{d}{Saving font ascent for ##1 as \the\ht\f@@tstrut}%
       \x@\xdef\csname gridding@scent##1\endcsname{\the\ht\f@@tstrut}%
       \ifdim\ht\f@@tstrut > \baselineskip
         \message{Paragraph font for ##1 including a verse number claims it is taller (\the\ht\f@@tstrut) than baseline (\the\baselineskip)}%
       \fi
     \egroup
   }
   \x@\each@col\diglot@list\E 
   \baselineskip=\dimen0
   \egroup
   \global\let\gridb@x\diglotgr@db@x
 \else
   \ifsquashgridbox\global\let\gridb@x\gr@db@@x\else\global\let\gridb@x\gr@db@x\fi % liberated from onetimesetup, as it might not be one-time.
 \fi
 \@netimesetuplater
 \gdef\ch@plabel{}
 \gdef\ch@plabelR{}
 \x@\def\csname b@ok1L\endcsname{}%
 \x@\def\csname b@ok1R\endcsname{}%
 \x@\def\csname b@ok1\endcsname{}%
 \global\let\pagenumberL=\folio\global\let\pagenumberR=\folio\global\let\pagenumber=\folio
 \initp@rastyles
 \initn@testyles
 \ifendbooknoeject\else\p@geonetrue\fi
}
%-csty_ptxfile_intro
%+csty_ptxfile_start
\def\ptxfile#1{
 \preptxfile
 \openadjlist "\the\AdjListPath#1.adj"
 \openpiclist "\the\PicListPath#1.piclist"
 \opentriglist "#1.triggers"
 \edef\tmp{\the\TrigListPath}\ifx\tmp\empty % Avoid opening same file twice 
 \else
   \opentriglist "\the\TrigListPath#1.triggers"
 \fi
 \multitriglists "\the\TrigListPath#1"
 \openpicpages "\j@bname.picpages"
 \ifcsname ifm@rginnotesopen\endcsname\openm@rginnotes "\j@bname.marginnotes"\fi
 \lastpagefalse
 \prepusfm
 \edef\t@st{#1}%
 \ifx\t@st\empty\else %opening empty file causes problems with some paths. 
   \openin\t@stread="\the\PtxFilePath#1"
     \ifeof\t@stread \def\n@xt{\MSG{USFM file "\the\PtxFilePath#1" not found -- ignored}}
     \else \closein\t@stread \def\n@xt{\input "\the\PtxFilePath#1"\relax}\fi
   \n@xt
 \fi
 \postptxfile
}
%-csty_ptxfile_start
%+csty_ptxfile_end
\def\postptxfile{%
 \trace{o}{postptxfile \iflastptxfile LAST ptx file\fi}%
 \ifhmode \unskip\end@llpoppedstyles{P}\par \fi
 \ifsk@pping\egroup\fi
 \ifhe@dings\endhe@dings\fi
 \endt@ble
 \ifdiglot\enddigl@t\fi
 \iflastptxfile \global\lastpagetrue\fi
 \ifx\b.ok\relax\else\csname bookend-\id@@@\endcsname\relax\fi
 \csname bookend-all\endcsname
 \the\@ndptxhooks
 \ifnum \ifnotesEachBook 1\else\iflastptxfile 1\else 0\fi\fi =1 \singlecolumn\zplaceallnotes\fi
 \unprepusfm
 \iflastptxfile\trace{o}{running bookend-final}\csname bookend-final\endcsname\par\fi
 \singlecolumn % must come after bookend-final, as that might include 2col content.
 \bookendpagebreak
 %\ifendbooknoeject\else\global\output={\shipout\vbox{\box\partial\box255}}\eject\xdef\p@gefirstmark{}\setbox\partial=\box\voidb@x\fi
 \global\output={\onecol}
 % no idea why we need \noindent here:
 \ifendbooknoeject\else\trace{o}{end of book eject with \the\ht\partial \space remaining}\ifdim\ht\partial>0pt \noindent\eject\fi
 \iflastptxfile\closepiclist\fi
 \closeadjlist
 \xdef\p@gefirstmark{}\rerunsavepartialpagedfalse\fi
}
%-csty_ptxfile_end
\newread\t@stread
\newtoks\AdjListPath % path to look for adjustment files
\newtoks\PicListPath % path to look for picture list files
\newtoks\TrigListPath % path to look for trigger list files
\newtoks\PtxFilePath % path to look for main ptx input files

% mechanism to define active chars within the ptx files
\catcode`\~=\active
\def\DefineActiveChar#1#2{\count255=`#1
  \trace{ch}{DefineActiveChar: \the\count255}%
  \do@ctive=\x@\x@\x@{\x@\the\x@\do@ctive
    \x@\d@something\the\count255 }%
  \lccode`\~=\count255
  \def\t@mp{\noexpand#2} \lowercase{\edef~{\relax\t@mp}}}
\newtoks\do@ctive

\def\@ctivate{\x@\edef\csname save-cat-\the\count255\endcsname{\the\catcode\count255}%
  \catcode\the\count255=\active\relax}
\def\de@ctivate{\catcode\the\count255=\csname save-cat-\the\count255\endcsname\relax}

\def\activ@tecustomch@rs{% activate all the chars configured with \DefineActiveChar
 \def\d@something{\afterassignment\@ctivate \count255=}\the\do@ctive}  
\def\deactiv@tecustomch@rs{% deactivate all the chars configured with \DefineActiveChar
 \def\d@something{\afterassignment\de@ctivate \count255=}\the\do@ctive}  

% use numbers to allow sequences to work in \fig .. \fig*
\def\openbracket{\char91\ignorespaces}
\def\openparen{\char40\ignorespaces}

\DefineActiveChar{[}{\openbracket} % make these skip following spaces
\DefineActiveChar{(}{\openparen}   % (useful at end-of-line before new verse)

%+csty_slash
\catcode`\/=\active
\def/{\futurelet\n@xt\sl@sh}
\def\sl@sh{\ifx\n@xt/\let\n@xt\sl@shbreak\else\let\n@xt\sl@shprint\fi\n@xt}
\def\OptionalBreakPenalty{300}
\let\oldlinepenalty\relax
\def\sl@shbreak/{\ifx\oldlinepenalty\relax\edef\oldlinepenalty{\the\linepenalty}\fi\linepenalty=\OptionalBreakPenalty\unskip\penalty-\OptionalBreakPenalty{} \ignorespaces}
\def\sl@shprint{\char`\/}
\catcode`\/=12
%-csty_slash
%+csty_category
% Define the current category
\def\s@tc@tprefix#1{%
  \def\c@tegory{#1}%global or NOT global? No apparent issue with either, so stick with local, as that allows nesting. Which we need.
  \s@tc@tpr@fix
}
\def\s@tc@tpr@fix{%
  \trace{sc}{c@tegory is '\meaning\c@tegory' empty is '\meaning\empty'}%
  \ifx\c@tegory\empty
    \let\c@tprefix\empty
  \else
    \edef\c@tprefix{cat:\c@tegory|}% See above...
  \fi
  \trace{sc}{Stylesheet prefix for '\c@tegory': '\c@tprefix'}%
}
\newif\ifn@npublishablec@t
\def\endc@tmarker{t*}
\def\@ndc@t{%
  \trace{sc}{@ndc@t(\c@tegory)}%
  \ifx\c@tegory\empty
    \trace{sc}{@ndc@t called with empty category. Eh?}%
  \else
      \ifp@pping\else
        \end@llpoppedstyles{\endc@tmarker}%pop any styles closed by an end of t
      \fi
      %\n@npublishablefalse\getmcp@ram{properties}%
      \ifn@npublishable\temptrue\else\tempfalse\fi
      \n@npublishablefalse\getp@ram{properties}{\styst@kfirst}{\styst@kfirst}%
      \trace{i}{cat \c@tegory, properties=\p@ram, from \styst@kfirst}%
      \x@\t@stnonpub\p@ram nonpublishable!
      \ifn@npublishable\global\n@npublishablec@ttrue\else\global\n@npublishablec@tfalse\fi
      \iftemp\n@npublishabletrue\else\n@npublishablefalse\fi
      \endgroup %End the group started between the cat groups     %(3)
      \cl@singhooks{after}{cat}%
      \s@tc@tprefix{}%
      \the\afterh@@ks
      \egroup % End the group from the containing code.             %(4)
  \fi
}
% remember as c@t, onetime setup will define \cat to this, protecting the definition from stylesheet
\def\nofin@lspace #1 #2\E{#1}%Eat a final space
\def\c@t #1\cat*{\doc@t{#1}}
\def\doc@t#1{%
  \trace{sc}{Starting category #1, with stylet@pe=\stylet@pe, mcstack=\mcstack}%
  \ifx\c@tegory\empty\else                                      %(5)
    \trace{sc}{Odd situation: two cat commands in one group or nested categories? Assuming nesting.}%
    \end@llpoppedstyles{\ss@Ctgy}%pop styles closed by a start of t, in case this IS 2 cat commands
    % Assume nesting and don't run any endhooks. Otherwise we'd do this:
    %\op@ninghooks{end}{cat}% 
    %\endgroup %End the group started between the cat groups  
    %\cl@singhooks{after}{cat}%
  \fi
  \ifx\stylet@pe\ss@Para
    %NOTE To potential vandals: This bit of code should ONLY be reached from a 
    % normal parargraph style. If you are trying to solve something else, fix
    % the something else, maybe stylet@pe is incorrectly set.
    \bgroup% DO NOT COMMENT THIS OUT! There is no containing group, so NEED to provide one.
    \let\stylet@pe\ss@ParCtgy
  \else 
    \let\stylet@pe\ss@Ctgy
  \fi
  \edef\tmp{\nofin@lspace #1 \E}%Ensure full expansion 
  \x@\s@tc@tprefix\x@{\tmp}%
  \the\c@th@@ks% Run anythingthe enclosing code wants
  \op@ninghooks{before}{cat}% 
  \ifhmode\par\fi
  \begingroup % This group will be closed between the end and after hooks of @ndc@t %(1)
  \mcpush{\stylet@pe}{#1}%
  \op@ninghooks{start}{cat}%
  \let\styst@k=\empty
  \s@tstyst@k%
  %\n@npublishablefalse\getmcp@ram{properties}%
  %\x@\t@stnonpub\p@ram nonpublishable!
  \bgroup % This group will be closed by the end-of footnote marker / other container.  %(2)
  %I.e. we're borrowing the egroup from other bits of code.  
  \ifx\stylet@pe\ss@Ctgy
    \aftergroup\@ndc@t %
  \fi
}
\global\let\c@tprefix\empty % category-specific prefix
\global\let\mspr@fix\empty % Milestone-specific prefix
%-csty_category

\def\mapinput "#1"{%
  \message{INPUTFILE: #1}%
  \input "#1"
}

\def\mapimagefile#1#2#3{%
% #1 command to load the file, #2 file, #3 rest of modifiers, etc.
  \message{INPUTFILE: #2}%
  #1 #2 #3\relax
}

\def\mapinputfile#1{\noexpand #1}

%+csty_hooks
%Hook-runners for paragraph and character styles
\def\op@ninghooks#1#2{% for <start> and <before> hooks
   \trace{sh}{op@ninghooks #1 #2}%
   \ifcsname #1-#2\endcsname
     \trace{sh}{#1-#2 exists}%
     \csname #1-#2\endcsname
   \fi
   \ifx\c@tegory\empty\else
      \ifcsname #1-\c@tprefix#2\endcsname
        \csname #1-\c@tprefix#2\endcsname
      \fi
   \fi
   \ifdiglot\setsid@ % \MSG{start-\thisch@rstyle\sid@}%
     \ifcsname #1-#2\sid@\endcsname\relax 
       \csname #1-#2\sid@\endcsname\relax % handle side-specific <before> hook
     \fi
     \ifx\c@tegory\empty\else
        \ifcsname #1-\c@tprefix#2\sid@\endcsname
          \csname #1-\c@tprefix#2\sid@\endcsname %side-specific and category-specific
        \fi
     \fi
   \fi
}
\newtoks\c@th@@ks
\newtoks\afterh@@ks
\def\cl@singhooks#1#2{% for <end> and <after> hooks. For <after> hooks, it's necessary to end the group after these are defined 
  \trace{sa}{cl@singhooks #1 #2}%
  \global\let\n@xt\relax
  \global\let\n@@xt\relax
  \global\let\n@@@xt\relax
  \global\let\n@@@@xt\relax
  %\global\afterh@@ks{}%
  %\tracingassigns=1
  \ifdiglot\setsid@ % \MSG{end-\thisch@rstyle\sid@}%
    \ifx\c@tegory\empty\else
      \x@\global\x@\let\x@\n@xt
        \csname #1-\c@tprefix#2\sid@\endcsname %side-specific and category-specific
      \x@\global\afterh@@ks{\n@xt}%
    \fi
    \x@\global\x@\let\x@\n@@xt
      \csname #1-#2\sid@\endcsname % handle side-specific <end> hook
    \x@\global\x@\afterh@@ks\x@{\the\afterh@@ks\n@@xt}%
  \fi
  \ifx\c@tegory\empty\else
    \x@\global\x@\let\x@\n@@@xt
      \csname #1-\c@tprefix#2\endcsname
    \x@\global\x@\afterh@@ks\x@{\the\afterh@@ks\n@@@xt}%
  \fi
  \x@\global\x@\let\x@\n@@@@xt
    \csname #1-#2\endcsname%
  \x@\global\x@\afterh@@ks\x@{\the\afterh@@ks\n@@@@xt\global\afterh@@ks{}}% Notice the self-clearing %(1)
  %\showthe\afterh@@ks
  \edef\@@@a{#1}%                                                  %(2)
  \ifx\@@@a\@fter\else\the\afterh@@ks\fi
}
\edef\@fter{after}
%-csty_hooks
% Create registers to hold configuration information, set defaults
\newdimen\FontSizeUnit		\FontSizeUnit=1bp % units for font sizes in Paratext stylesheet
\newdimen\IndentUnit		\IndentUnit=1in % units for indents
\newdimen\PaperWidth		\PaperWidth=210mm
\newdimen\PaperHeight		\PaperHeight=297mm
\newdimen\MarginUnit		\MarginUnit=1in % units for margins
\newdimen\BindingGutter		\BindingGutter=5mm
\def\LineSpacingFactor{1.0} % scale factor used for line spacing
\def\VerticalSpaceFactor{1.0} % scaling factor applied to space before/after styles
\def\TopMarginFactor{1.0}
\def\BottomMarginFactor{\TopMarginFactor}
\def\SideMarginFactor{1.0}
\def\ColumnGutterFactor{15} % relative to FontSizeUnit
\def\HeaderPosition{0.5} % pos of baseline from top of page, relative to MarginUnit
\def\FooterPosition{0.5} % pos of baseline from bottom of page, relative to MarginUnit
\def\DiglotLeftFraction{0.5} % In case texts fit better with unequal columns
\def\DiglotRightFraction{0.5} %

\newcount\TitleColumns \TitleColumns=1
\newcount\IntroColumns \IntroColumns=1
\newcount\BodyColumns  \BodyColumns=1
\def\mainBodyColumns{1}

\newif\ifBindingGutter % \BindingGutter won't be used unless this is set to true
\newif\ifDoubleSided \DoubleSidedtrue % binding gutter on alternate sides; use odd/even headers
\def\debugBL{\MSG{\f@ntstyle, Baseline: \the\baselineskip}} % for scattering in usfm as desired.
% calculate various dimensions based on the factors, units, etc that have been defined
%+csty_setupsizewas s
\def\s@tb@dyb@seline#1{%Parameter is {} for monoglot, L / R for diglot.
 \baselineskip=\origb@selineskip
 \global\le@dingunit=\LineSpacingFactor\FontSizeUnit
 \onel@neunit=\LineSpaceBase\le@dingunit
 \verticalsp@ceunit=\VerticalSpaceFactor\le@dingunit
 \trace{F}{le@dingunit(\the\le@dingunit)=\LineSpacingFactor * \the\FontSizeUnit, onel@neunit(\the\onel@neunit)=\LineSpaceBase * \the\le@dingunit}
 \ifdim\baselineskip < 1pt \baselineskip=\onel@neunit\fi % If the baselineskip has been specifically
                                                           % set it will remain. Otherwise, the default
                                                           % of 14 will be applied here and set text will
                                                           % hold to the grid.
 \getp@r@m{baseline}{p#1}\ifx\p@ram\relax
   \s@tbaseline{p#1}{p#1}\setp@ram{baseline}{p#1}{\the\baselineskip}%
   \x@\xdef\csname b@selineskip#1\endcsname{\the\baselineskip}%
   \MSG{Baselineskip for p#1 was undefined, now set to \the\baselineskip}%
 \else
   \MSG{Baselineskip for p#1 was \p@ram (orig: \the\origb@selineskip). now defined as \the\baselineskip}%
   \x@\xdef\csname b@selineskip#1\endcsname{\the\baselineskip}%
 \fi
}
\def\onecolwidth#1{%
  \dimen9=\textwidth
  \advance \dimen9 by -\columnshift
  \ifx\XrefNotes\empty\relax\else\ifx\ch@pter\empty\else\ifnum\BodyColumns=1
    \advance \dimen9 by -\XrefNotesWidth
    \advance \dimen9 by -\XrefNotesMargin
    \advance \dimen9 by -0.5\gutter
    \ifColNotesRule \advance \dimen9 by -1pt \fi  % width of rule
  \fi\fi\fi
  \global #1=\dimen9
  %\trace{o}{setonecolwidth textwidth=\the\textwidth, columnshift=\the\columnshift, with notes width=\the\XrefNotesWidth, margin=\the\XrefNotesMargin, gutter=\the\gutter, result=\the #1, xrefnotes=\XrefNotes, ch@pter=\ch@pter, body columns=\the\BodyColumns}%
}


\def\s@tupsizes{
 \s@tb@dyb@seline{}%
 %\tracingifs=1
 \lineskiplimit=-9\le@dingunit
 \lineskip=0pt
 \s@tbaseline{p}{p}%
 \NoteCallerWidth=1.1ex %Something sane?
 \global\topskip=\ifdiglot 2\fi\baselineskip
 \global\splittopskip=0pt
 \MSG{\baselineskip=\the\baselineskip,  topskip=\the\topskip}
 \gutter=\ColumnGutterFactor\FontSizeUnit
 \dimen0=\PaperWidth \dimen2=\SideMarginFactor\MarginUnit
 \advance\dimen0 by -2\dimen2
 \ifBindingGutter \advance\dimen0 by -\BindingGutter \fi
 \advance\dimen0 by -1\ExtraRMargin
 \textwidth=\dimen0
 \colwidth=0.5\textwidth \advance\colwidth by -0.5\gutter\advance\colwidth by -\columnshift
 \advance\colwidth by -0.5\XrefNotesWidth
 \advance\colwidth by -\XrefNotesMargin
% Diglot dimensions
 \ifcsname DiglotLFraction\endcsname\else
    \let\DiglotLFraction=\DiglotLeftFraction
 \fi
 \ifcsname DiglotRFraction\endcsname\else
    \let\DiglotRFraction=\DiglotRightFraction
 \fi
 \csname set@col@widths@\diglotlayout\endcsname
%
  %\onecolwidth{\hsize}%
  \hsize=\textwidth
  \trace{o}{hsize=\the\hsize, notes width=\the\XrefNotesWidth, margin=\XrefNotesMargin}%
 \topm@rgin=\TopMarginFactor\MarginUnit
 \advance\topm@rgin by -\topskip \advance\topm@rgin by 12\FontSizeUnit
 \bottomm@rgin=\BottomMarginFactor\MarginUnit
 \dimen0=\PaperHeight
 \advance\dimen0 by -\topm@rgin
 \advance\dimen0 by -\bottomm@rgin
 \textheight=\dimen0
 \advance\dimen0 by -\topskip
 \dimen1=\dimen0 \dimen0=\baselineskip \m@d
 \bmg@p=\baselineskip \advance\bmg@p-\dimen0
 \vsize=\textheight
 \iffalse % Code for testing breakable inserts. It breaks things!
   \dimen\topins=\textheight
   \dimen\bottomins=\textheight
   \dimen\topleftins=0.7\textheight
   \dimen\toprightins=0.7\textheight
   \dimen\bottomleftins=0.7\textheight
   \dimen\bottomrightins=0.7\textheight
   \dimen\bottomins=\textheight
   \dimen\verybottomins=\textheight
 \fi
 \ifrotate\pdfpagewidth=\PaperHeight \pdfpageheight=\PaperWidth\else
 \pdfpagewidth=\PaperWidth \pdfpageheight=\PaperHeight\fi
 \trace{p}{setupsizes: \the\pdfpagewidth, \the\pdfpageheight}
 \resetvsize
}

%-csty_setupsizes
\def\LineSpaceBase{14}
\newdimen\le@dingunit
\newdimen\onel@neunit
\newdimen\verticalsp@ceunit
\newdimen\topm@rgin
\newdimen\bottomm@rgin
\newdimen\bmg@p     % gap between last baseline and bottom margin
\newdimen\textwidth
\newdimen\textheight
\newdimen\colwidth
\newdimen\ExtraRMargin \ExtraRMargin=0pt
\newdimen\gutter \gutter=20pt
\newdimen\XrefNotesWidth \XrefNotesWidth=0pt
\newdimen\XrefNotesMargin \XrefNotesMargin=1pt
\endinput
