%%%%%%%%%%%%%%%%%%%%%%%%%%%%%%%%%%%%%%%%%%%%%%%%%%%%%%%%%%%%%%%%%%%%%%%
% Part of the ptx2pdf macro package for formatting USFM text
% copyright (c) 2007 by SIL International
% written by Jonathan Kew
%
% Permission is hereby granted, free of charge, to any person obtaining  
% a copy of this software and associated documentation files (the  
% "Software"), to deal in the Software without restriction, including  
% without limitation the rights to use, copy, modify, merge, publish,  
% distribute, sublicense, and/or sell copies of the Software, and to  
% permit persons to whom the Software is furnished to do so, subject to  
% the following conditions:
%
% The above copyright notice and this permission notice shall be  
% included in all copies or substantial portions of the Software.
%
% THE SOFTWARE IS PROVIDED "AS IS", WITHOUT WARRANTY OF ANY KIND,  
% EXPRESS OR IMPLIED, INCLUDING BUT NOT LIMITED TO THE WARRANTIES OF  
% MERCHANTABILITY, FITNESS FOR A PARTICULAR PURPOSE AND  
% NONINFRINGEMENT. IN NO EVENT SHALL SIL INTERNATIONAL BE LIABLE FOR  
% ANY CLAIM, DAMAGES OR OTHER LIABILITY, WHETHER IN AN ACTION OF  
% CONTRACT, TORT OR OTHERWISE, ARISING FROM, OUT OF OR IN CONNECTION  
% WITH THE SOFTWARE OR THE USE OR OTHER DEALINGS IN THE SOFTWARE.
%
% Except as contained in this notice, the name of SIL International  
% shall not be used in advertising or otherwise to promote the sale,  
% use or other dealings in this Software without prior written  
% authorization from SIL International.
%%%%%%%%%%%%%%%%%%%%%%%%%%%%%%%%%%%%%%%%%%%%%%%%%%%%%%%%%%%%%%%%%%%%%%%

% Note style macros

%+cnote_makenote
% We keep a list of all the note classes in \n@tecl@sses, each prefixed with \\ and enclosed in braces.
% Then we can define \\ on the fly, and execute the \n@tecl@sses list to apply it to all classes.
\newtoks\n@tecl@sses

% for each Note marker defined in the stylesheet, we allocate a "note class"
% with its own \insert number (see TeXbook!)
%
\def\m@ken@tecl@ss#1{%
  \newcl@sstrue
  \def\n@wcl@ss{#1}
  \let\\=\ch@ckifcl@ss \the\n@tecl@sses % check if this note class is already defined
  \ifnewcl@ss \allocatecl@ss{#1} \fi
}
\def\ch@ckifcl@ss#1{\def\t@st{#1}\ifx\t@st\n@wcl@ss\newcl@ssfalse\fi}
\newif\ifnewcl@ss

% new note class: append to the list in \n@tecl@sses, and allocate an \insert number
\def\allocatecl@ss#1{%
  \x@\n@tecl@sses\x@{\the\n@tecl@sses \\{#1}}%
  \ifcsname zplacenotes-#1\endcsname\else% zplacenotes-#1 is an endnote macro. (as is ztestnotes-#1) Trigger a warning rather than leave an undefined macro.
    \x@\gdef\csname zplacenotes-#1\endcsname{\write1{WARNING: p.\the\pageno: zplacenotes-#1 used in text when #1 is a footnote, not an endnote.}}%
    \x@\gdef\csname ztestnotes-#1\endcsname{\write1{WARNING: p.\the\pageno: ztestnotes-#1 used in text when #1 is a footnote, not an endnote.}\@ndnotesfoundfalse}%
  \fi
  \relax%
  \ifdiglot%
    \def\col@do##1{\n@wnoteins@rt{#1\@g@tdst@t{##1}}}%
    \x@\each@col\diglot@list\E
  \else%
    \n@wnoteins@rt{#1}%
  \fi%
}
\def\n@wnoteins@rt#1{
    \trace{n}{Creating note class #1}%
	\x@\n@winsert\csname note-#1\endcsname
	\x@\newb@x\csname notesave1-#1\endcsname
	\x@\newb@x\csname notesave2-#1\endcsname
}
\let\n@winsert=\newinsert % work around the \outer nature of \newinsert
%-cnote_makenote

%+cnote_initnotes
% set default insert parameters for a note class (updated if we switch to double-column)
\def\initn@testyles{\let\\=\s@tn@tep@rams \the\n@tecl@sses}
\def\s@tn@tep@rams#1{%
  \checkp@ranotes{#1}%
  \ifdiglot
    \ifdiglotSepNotes\s@tn@tep@r@ms{#1}\s@tn@tep@r@ms{#1R}\else\s@tn@tep@r@ms{#1}\fi
  \else\s@tn@tep@r@ms{#1}\fi
}
\def\s@tn@tep@r@ms#1{%
  \x@\count\csname note-#1\endcsname=\ifdiglot 100 \else \ifp@ranotes 100 \else 100 \fi\fi 
  \x@\skip\csname note-#1\endcsname=\AboveNoteSpace %Skip before footntoes
  \x@\dimen\csname note-#1\endcsname=\maxdimen %Max note skip
}
\newdimen\AboveNoteSpace \AboveNoteSpace=\medskipamount
\newdimen\InterNoteSpace \InterNoteSpace=3.5pt
\newdimen\NoteCallerWidth %Now set in 1timesetup \NoteCallerWidth=1.1ex %Minimum width for callers/callees, helps alignment.
\newdimen\NoteCallerSpace \NoteCallerSpace=.2em
%-cnote_initnotes

%
% Each USFM note marker is defined to call \n@testyle, with the marker name as parameter
%
%+cnote_notestyle
\def\n@testyle#1{\trace{n}{n@testyle:#1 \c@rrdstat}%
 \def\newn@testyle{#1}%
 \catcode32=12\relax % look ahead to see if space or * follows (like char styles)
 \futurelet\n@xt\don@testyle}
\def\don@testyle{\catcode32=10\relax 
 \if\n@xt*\let\n@xt\endn@testyle\else\let\n@xt\startn@testyle\fi
 \n@xt}
%-cnote_notestyle

%+cnote_startnote_start
\def\b@lance{BALANCE}
\lowercase{
 \def\startn@testyle~#1 {% get the caller code as a space-delimited parameter
  \trace{n}{startn@testyle #1}%
  \t@stpublishability{\newn@testyle}\ifn@npublishable
    \begingroup
    \let\aftern@te\relax
    \ifhe@dings\bgroup\fi
    \setbox0=\hbox\bgroup \skipn@testyletrue
    \global\n@tenesting=1\relax
    \ifdim\lastskip>0pt \sp@cebeforetrue \else \sp@cebeforefalse \fi % was there a preceding space?
%-cnote_startnote_start
%+cnote_startnote
  \else
    \leavevmode
    \getp@ram{notebase}{\newn@testyle}{\newn@testyle}\edef\n@tebase{\ifx\p@ram\relax\newn@testyle\else\p@ram\fi}%
    \csname before-\newn@testyle\endcsname
    \def\t@st{#1}%
    \ifx\t@st\pl@s % if it is + then generate an auto-numbering caller
      \inc@utonum{\newn@testyle}%
      \x@\gen@utonum\x@{\newn@testyle}%
    \else\ifx\t@st\min@s \def\them@rk{}% if it is - then there is no caller
    \else \trace{n}{the note parameter is \t@st \space and is \ifx\t@st\-\else not\fi\space \- \space catcode - is \the\catcode`\-}\def\them@rk{#1}% otherwise the caller is the parameter
    \fi\fi
    \begingroup
    \x@\let\x@\aftern@te\csname after-\newn@testyle\endcsname
    \ifdim\lastskip>0pt \sp@cebeforetrue \else \sp@cebeforefalse \fi % was there a preceding space?
    \resetp@rstyle                                                              %(1)
    \m@kenote{\n@tebase}{\newn@testyle}{%
        \everypar={}\cancelcutouts % begin a note insertion in the given style
        \getp@ram{callerstyle}{\newn@testyle}{\newn@testyle}% see if a caller style was defined
        \ifx\p@ram\relax\edef\c@llerstyle{v}\else\edef\c@llerstyle{\p@ram}\fi % if not, treat it like "v"
        \ifx\them@rk\empty \trace{n}{note caller is EMPTY}\setbox0=\box\voidb@x \else
          \trace{n}{note main caller}%
          \setbox0=\hbox{\x@\cstyle\x@{\c@llerstyle}{\them@rk}}\ht0=0pt\dp0=0pt
          \trace{n}{note caller is \the\wd0 \space wide}%
        \fi
        \getp@ram{callerraise}{\newn@testyle}{\newn@testyle}\ifx\p@ram\relax\box0\else\raise\p@ram\box0\fi % suppress height of caller
        \ifnum\pagetracing>0
          \edef\n@tetxt{\b@lance\space note \newn@testyle\space in \n@tebase. \id@@@\space \ch@pter.\v@rse}%
          \x@\x@\x@\write-1\x@{\n@tetxt}%
        \fi
      }{\getp@ram{notecallerstyle}{\newn@testyle}{\newn@testyle}\ifx\p@ram\relax\else
          \edef\c@llerstyle{\p@ram}%
          \ifx\them@rk\empty \trace{n}{note caller is EMPTY}\setbox0=\box\voidb@x \else
            \trace{n}{note caller}%
            \setbox0=\hbox{\hss\x@\cstyle\x@{\c@llerstyle}{\them@rk}\hss}\ht0=0pt\dp0=0pt
          \fi
          \getp@ram{notecallerraise}{\newn@testyle}{\newn@testyle}\ifx\p@ram\relax\box0\else\raise\p@ram\box0\fi
        \fi}\bgroup
    \global\n@tenesting=1\relax
    \trace{b}{BALANCE note: style=\newn@testyle}%
    \csname start-\newn@testyle\endcsname % execute the <start> hook, if defined
    \ignorespaces
  \fi
 }
}
%-cnote_startnote

%+cnote_endnote
\def\endn@testyle*{\trace{f}{endn@testyle}%
 \end@llpoppedstyles{N*}%end any character styles within the note
 \ifskipn@testyle \else
  \csname end-\newn@testyle\endcsname % execute <end> hook
 \fi
 \egroup % end the insert (started in \m@kenote) 
 \ifx\@wrap\empty\else\egroup\let\@wrap\empty\fi%\@wrap would have started a box
 \global\inn@tefalse
 \global\n@tenesting=0 
 \ifsp@cebefore \global\let\n@xt=\ignorespaces % ignore following spaces if there was a preceding one
   \else \global\let\n@xt\relax \fi
 \aftern@te
 \endgroup\n@xt}
\newif\ifsp@cebefore
\newif\ifskipn@testyle
%-cnote_endnote

%+cnote_autonum
\def\inc@utonum#1{\set@utonum{#1}\count255=0\csname \@utonum\endcsname\relax % increment note counter for class #1
 \advance\count255 by 1 \x@\xdef\csname \@utonum\endcsname{\number\count255}}

%% overridden by ptx-callers.tex
\ifx\gen@utonum\undefined
 \def\gen@utonum#1{%
  \set@utonum{#1}%
  \count255=0\csname \@utonum \endcsname
  \loop \ifnum\count255>26 \advance\count255 by -26 \repeat
  \advance\count255 by 96 \edef\them@rk{\char\count255}}
\fi

\def\resetautonum#1{\set@utonum{#1}\x@\xdef\csname \@utonum\endcsname{0}} % reset the current note counter (f)
\def\resetSpecAutonum#1{\x@\xdef\csname autonum #1\endcsname{0}} % reset the exact note counter f / fL / R
\def\resetAllAutonum#1{\def\col@do##1{\ifcsname autonum#1##1\endcsname\resetSpecAutonum{#1##1}\fi}%
   \x@\each@col\diglot@list\E} %reset all note counters.
%-cnote_autonum

%+cnote_declare
\edef\pl@scatcode{\the\catcode`+}
\catcode`+=11
\edef\pl@s{+}
\catcode`+=\pl@scatcode
\catcode`-=11
\def\min@s{-}
\catcode`-=12
%-cnote_declare
\endinput
