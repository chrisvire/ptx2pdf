%:strip
% Part of the ptx2pdf macro package for formatting USFM text
% copyright (c) 2020 by SIL International
% written by David Gardner 
%
% Permission is hereby granted, free of charge, to any person obtaining  
% a copy of this software and associated documentation files (the  
% "Software"), to deal in the Software without restriction, including  
% without limitation the rights to use, copy, modify, merge, publish,  
% distribute, sublicense, and/or sell copies of the Software, and to  
% permit persons to whom the Software is furnished to do so, subject to  
% the following conditions:
%
% The above copyright notice and this permission notice shall be  
% included in all copies or substantial portions of the Software.
%
% THE SOFTWARE IS PROVIDED "AS IS", WITHOUT WARRANTY OF ANY KIND,  
% EXPRESS OR IMPLIED, INCLUDING BUT NOT LIMITED TO THE WARRANTIES OF  
% MERCHANTABILITY, FITNESS FOR A PARTICULAR PURPOSE AND  
% NONINFRINGEMENT. IN NO EVENT SHALL SIL INTERNATIONAL BE LIABLE FOR  
% ANY CLAIM, DAMAGES OR OTHER LIABILITY, WHETHER IN AN ACTION OF  
% CONTRACT, TORT OR OTHERWISE, ARISING FROM, OUT OF OR IN CONNECTION  
% WITH THE SOFTWARE OR THE USE OR OTHER DEALINGS IN THE SOFTWARE.
%
% Except as contained in this notice, the name of SIL International  
% shall not be used in advertising or otherwise to promote the sale,  
% use or other dealings in this Software without prior written  
% authorization from SIL International.
%%%%%%%%%%%%%%%%%%%%%%%%%%%%%%%%%%%%%%%%%%%%%%%%%%%%%%%%%%%%%%%%%%%%%%%

\newif\ifin@ttrib



%\tracingall=1

\catcode`\|=\active
\def\use@ttrSlash{%
    \let|=\@ttrSlash %this one is easy (for use with milestones)
}
\def\use@ttrGrab{%
    \let|=\start@ttributegrab %have to build attributes char-by-char, sadly.
}
\def\use@ttrSpecific{%
   %\tracingmacros=1
   %\tracingassigns=1
   \setgr@bname{\thiswh@tstyle}%
   \trace{A}{grabname set to \gr@bname}%
   \x@\let\x@|\csname \gr@bname\endcsname
}
\catcode`\|=12
\def\kill@ttrib #1\E{%
  \trace{A}{removing (old) attribute '#1'}%
  \x@\global\x@\let\csname attr:#1\endcsname=\undefined
}
\def\apply@attr@specials #1\E{% Apply 'simple' attribute features - one feature=one atribute. markers that need multiple attributes should probably define \csname complex-foo\endcsname, like rb does.
  \x@\let\x@\tmpa\csname \Spec@lAttribName{#1}{start}\endcsname
  \x@\let\x@\tmpb\csname \Spec@lAttribName{#1}{end}\endcsname
  \temptrue
  \ifx\tmpa\relax
    \ifx\tmpb\relax\tempfalse
    \fi
  \fi
  \iftemp
    \get@ttribute{#1}%
    \trace{A}{Special '#1':'\attr@b'}%
    %\tracingmacros=1
    %\tracingassigns=1
    \let\exptmp\relax\let\tmp\relax
    \setbox0\hbox{%
        \begingroup% Ensure there are no nesting errors
        %\trace{A}{>>>}%
	\ifx\tmpa\relax\else
          \edef\exptmp{\tmpa}\exptmp\fi
        \relax
	\unhbox0
	\ifx\tmpb\relax\else
          \edef\exptmp{\tmpb}\exptmp\fi
        \relax
        %\trace{A}{<<<}%
        \endgroup
      }%
    %\tracingassigns=0
    %\tracingmacros=0
  \fi
}
\def\Spec@lAttribName#1#2{attr-spec-#1-#2}
\newif\ifattrspecial
\def\FlagSpecialAttrib#1{%The given attribute uses special handling, and thus charstyles with it box their values
  \x@\global\x@\let\csname attr-spec-#1\endcsname\tr@e
}
\def\CheckSpecial#1{\ifcsname attr-spec-#1\endcsname\attrspecialtrue\fi}
\def\SetOneSpecialAttrib#1#2#3{%
  \FlagSpecialAttrib{#1}%
  \S@tOneSpecialAttrib{#1}{#2}{#3}}

\def\S@tOneSpecialAttrib#1#2#3{%
  \def\tmp{#3}\def\tmpa{\Spec@lAttribName{#1}{#2}}\ifx\tmp\empty
    \x@\let\csname\tmpa\endcsname\relax
  \else
    \x@\let\csname\tmpa\endcsname\tmp
  \fi}
\def\SetSpeciallAttribs#1#2#3{%\tracingassigns=1\relax%
  \FlagSpecialAttrib{#1}%
  \S@tOneSpecialAttrib{#1}{start}{#2}%
  \S@tOneSpecialAttrib{#1}{end}{#3}\tracingassigns=0\relax}%

\def\unset@ttribs{% Remove any current attibute definitions
  \let\d@=\kill@ttrib
  \x@\cstackdown \attribsus@d,\E
  \xdef\attribsus@d{}%
  \xdef\@ttributes{}%
  \relax\relax
  \trace{A}{attributes list reset}%
}
\def\init@ttribs{%Attribute pre-setupcode, defines grabber, etc. 
  %\tracingassigns=1
  \edef\thiswh@tstyle{\ifmst@nestyle\thismil@stone\else\thisch@rstyle\fi}%
  \@@init@ttribs % Processing config. Might be used without this wrapper when attributes are have been grabbed already.
  \edef\attrid{\milestoneOp id}% id / sid / eid, depending.
  \catcode`\|=\active
  \ifmst@nestyle
    \use@ttrSlash
  \else
    \use@ttrSpecific % based on style
    %\use@ttrGrab % char-by-char 
  \fi
  \in@ttribfalse
  \xdef\@ttributes{}%
}

\def\@@init@ttribs{%Most generic-case attribute initialisiation. 
  \trace{A}{InitAttribs \thiswh@tstyle (\csname thisch@rstyle\endcsname,\csname thismil@stone\endcsname)}%
  \ifx\attribsus@d\empty\else
    \unset@ttribs% Clear any old attributes.
  \fi
  \let\attrkey\relax
  %\x@\let\x@\attrkey\csname defaultattrkey-\thisch@rstyle\endcsname
  \x@\let\x@\t@st\csname thiswh@tstyle\endcsname
  \ifx\t@st\relax\else
%
    \x@\let\x@\attrkey\csname defaultattrkey-\thiswh@tstyle\endcsname
  \fi
  \ifx\attrkey\relax
    \let\thisdefault@ttrkey=\default@ttrkey
  \else
    \let\thisdefault@ttrkey=\attrkey
    \trace{A}{Attributes with no keyname will be \attrkey}%
  \fi
}
\def\GOTOLinkBorderstyle{/S/U/W 1} % Underline
\def\URLLinkBorderstyle{/S/U/W 1} % Underline
\def\GOTOLinkBorderCol{.9 .5 .5}% pale red
\def\URLLinkBorderCol{.5 .5 .5}% Gry

%\def\LinkBorderstyle{}

\def\pdfURL#1{\trace{A}{Appyling PDFurl}\special{pdf:bann<</Type/Annot/Subtype/Link/Border[0 0 1]\ifx\URLLinkBorderstyle\empty\else/BS<<\URLLinkBorderstyle>>\fi/H/I\ifx\GOTOLinkBorderCol\empty\else/C[\URLLinkBorderCol]\fi\ifActions/A<</S/URI/URI(#1)>>\fi>>}}

\edef\h@shcatcode{\the\catcode`\#}
\catcode`\#=12
\def\h@sh{#}
\catcode`\#=\h@shcatcode

\def\pdfG@T@#1#2 #3\E{\edef\tmp{#1}\ifx\tmp\h@sh \edef\tmp{#2}\else\edef\tmp{#3}\ifx\tmp\empty\edef\tmp{#1#2}\else\edef\tmp{\zap@space #1#2.#3 \empty}\fi\fi
  \special{pdf:bann<</Type/Annot/Subtype/Link/Border[0 0 1]\ifx\GOTOLinkBorderstyle\empty\else/BS<<\GOTOLinkBorderstyle>>\fi/H/I\ifx\GOTOLinkBorderCol\empty\else/C[\GOTOLinkBorderCol]\fi\ifActions/A<</S/GoTo/D(\tmp)>>\fi>>}%
  \trace{A}{link is GOTO to '\tmp'}%
}
\def\pdfGOTO#1{\x@\pdfG@T@#1 \E}
\def\pdfURLorGOTO#1:#2\end{\lowercase{\edef\tmp{#1:schema}}\x@\ifcsname\tmp\endcsname\csname\tmp\endcsname{\attr@b}\else\pdfGOTO{\attr@b}\fi}
\def\pdfPRJ#1{}

\x@\let\csname http:schema\endcsname\pdfURL
\x@\let\csname https:schema\endcsname\pdfURL
\x@\let\csname prj:schema\endcsname\pdfPRJ

\def\PDFposfromattr{\ifActions\raise\the\baselineskip\hbox{\special{pdf:dest (\attr@b) [@thispage /XYZ @xpos @ypos 1.4]}}\fi}
\SetSpeciallAttribs{link-id}{\noexpand\maybePDFp@sfromattr\noexpand\makelabelfromattr}{}
\SetSpeciallAttribs{link-href}{\ifx\attr@b\empty\else\noexpand\trace{A}{Applying link-href:'\attr@b'}\noexpand\pdfURLorGOTO \attr@b:\noexpand\end\fi}{\ifx\attr@b\empty\else\special{pdf:eann}\fi}
\SetSpeciallAttribs{outline-entry}{\noexpand\m@keoutlinefromattr{\attr@b}}{}
\SetSpeciallAttribs{link-title}{\noexpand\maybem@keoutlinefromattr{\attr@b}}{}
\def\maybem@keoutlinefromattr#1{\testt@ttrn@me{link-href}\if@ttrexists\else\m@keoutlinefromattr{#1}}
\def\maybePDFp@sfromattr#1{\testt@ttrn@me{outline-entry}\if@ttrexists\else\PDFposfromattr\fi}

\def\m@keoutlinefromattr#1{\get@ttribute{link-id}\ifx\attr@b\relax
    \message{Unable to make outline entry for #1, no link-id field specified}
  \else
    \PDFposfromattr
    \let\tmpattrb\attr@b
    \get@ttribute{outline-level}\ifx\attr@b\relax \def\attr@b{2}\fi% Outline level:0:book, 1:chapter
    \special{pdf:outline \attr@b\space << /Title (#1) /A << /S /GoTo /D (\tmpattrb) >> >>}%
  \fi
}
%%%%%%%%%
%
\def\def@endm@rker#1+#2+#3\E{\edef\end@marker{\if #1\ss@ChrP +\fi#2}}
% make something that swallows everything up to a given end-marker
\def\s@tend@marker#1{\let\end@marker\relax\let\d@\def@endm@rker\mctop\ifx\end@marker\relax\edef\end@marker{#1}\fi}
\def\setgr@bname#1{%
  \s@tend@marker{#1}%
  \edef\gr@bname{@attr@grab-\end@marker}%
  \trace{A}{Grab@name: \gr@bname}%
  \x@\ifcsname \gr@bname\endcsname\else
    \trace{A}{New grabber: \gr@bname\space about to be defined, endmarker: \end@marker}%
    \x@\setup@ttr@grab\x@{\csname \end@marker\endcsname}%
  \fi
}

\def\setup@ttr@grab#1{%
  \x@\xdef\csname \gr@bname\endcsname{\catcode`\/=12\relax % FIXME other active chars too?
    \csname @@\gr@bname\endcsname}%
  \x@\gdef\csname @@\gr@bname\endcsname##1#1{\xdef\@ttributes{\noexpand{##1}}\trace{A}{\gr@bname:found \@ttributes}#1}}

\def\proc@ttribs{%
  \trace{A}{Processing attributes (\ifin@ttrib true\else false\fi)}%
  \ifin@ttrib
    \trace{A}{Attributes specified:\attrib@rgs}%
    \parse@ttribs{\attrib@rgs}%
    \in@ttribfalse
  \fi
  \catcode`\|=12
  \catcode`\/=\active
  \activ@tecustomch@rs
  }


\def\setdefaultattrib#1#2{\trace{A}{Default attrib for #1 is #2}\x@\xdef\csname defaultattrkey-#1\endcsname{#2}%
  \edef\@ttrlistname{@ttriblist-#1}%
  \x@\xdef\csname \@ttrlistname\endcsname{#2}%
  \CheckSpecial{#1}\ifattrspecial\x@\let\csname boxed-\thiswh@tstyle\endcsname\tr@e\fi
}

\x@\def\csname nostore-attrib-eid\endcsname{}
\x@\def\csname nostore-attrib-sid\endcsname{}
\def\extend@ttriblist#1#2{%
  \edef\tmpa{#2}%
  \edef\@ttrlistname{@ttriblist-#1}%
  \x@\let\x@\tmp\csname\@ttrlistname\endcsname
  \ifx\tmp\relax
    \x@\let\x@\tmp\csname defaultattrkey-#1\endcsname
  \fi
  \x@\xdef\csname \@ttrlistname\endcsname{\tmp,#2}%
}

\edef\p@stattribcmd{}% Parse the attributes, run any code.

%\def\foo#1{Thatsit "#1"}

\def\oddc@tcodes{\xdef\sl@shcode{\the\catcode`\/}\catcode`"=11 \catcode`==11
\catcode`\ =11 \catcode`|=0\catcode`\/=12\catcode`\\=12}
\def\normalc@tcodes{\catcode`"=12 \catcode`==12
\catcode`\ =10\catcode`\\=0\catcode`\/=\sl@shcode\catcode`\|=12}
\trace{A}{CATCODES: "\the\catcode`",  =\the\catcode`=,  \the\catcode`\ , *\the\catcode`\*,  \the\catcode`\\,  \the\catcode`\|}%


\xdef\attribsus@d{}% List of used attributes.

%%%%%%%%%%%%
%
% Grabbing attribute values char-by-char 
%
\def\start@ttributegrab{\let\c@ntinue\isitslash\oddc@tcodes\futurelet\nxt\c@ntinue}
\def\end@ttributegrab{%
  \xdef\@ttributes{\x@\noexpand\x@{\@ttributes}}% Make it all strings.
  \trace{A}{Now executing \p@stattribcmd}%
  \csname\p@stattribcmd\endcsname}% 

\def\store@ttributes#1{\def\newbit{#1}\edef\@ttributes{\@ttributes\newbit}\futurelet\nxt\isitslash}
\def\startp@stattribcmd#1{\futurelet\nxt\isitcmd}
\def\storep@stattribcmd#1{\def\newbit{#1}\edef\p@stattribcmd{\p@stattribcmd\newbit}\futurelet\nxt\isitcmd}
\lccode`\~=32
\lowercase{
 \gdef\isitcmd{%
        \let\c@ntinue\storep@stattribcmd
        \ifcat a\nxt\else%\ifcat=\nxt \else
	\let\c@ntinue\end@ttributegrab\fi%\fi
	\c@ntinue}%
}
\oddc@tcodes%NO SPACES until |normalc@tcodes is called.
|gdef|isitslash{|let|c@ntinue|store@ttributes|if|nxt\|let|c@ntinue|startp@stattribcmd|normalc@tcodes|fi|c@ntinue}%
|def|zap@slash#1\#2{#1|ifx#2|empty|else|expandafter|zap@slash|fi#2}%
|def|deslash#1{|edef|tmp{|zap@slash#1\|empty}}%
|normalc@tcodes%
%
%
%
%%%%%%%%%%%%

%Parse key="value" or value.
\def\default@ttrkey{} % What unnamed attributes get saved 
\def\thisdefault@ttrkey{} % What do unnamed attributes get saved as?
\def\spaceval{ }
\def\s@tattr@val #1 #2\E{\edef\test{#2}\ifx\test\empty\else\edef\attr@val{\ifx\attr@val\empty\else\attr@val\space\fi#1}\x@\s@tattr@val \test\E\fi}% Remove trailing space(s)
\def\milestoneJoinChar{ } % While replacing ' ' with '_' is  nice for styling things like \qt-s |the crowd\*, this then breaks equivalence with ... who="the crowd"
\def\save@ttribute#1="#2" #3\E{%
  \trace{A}{save@ttribute '#1'="#2" #3}%
  \lowercase{\edef\attr@key{#1}}\edef\attr@val{#2}%
  \ifx\attr@val\relaxval
    \ifx\thisdefault@ttrkey\empty
      \edef\thisdefault@ttrkey{UnNamed}%
    \fi
    \ifx\attr@key\spaceval\else
      \ifx\attr@key\empty\else
      \get@ttribute{\thisdefault@ttrkey}%
      \edef\attr@val{}%
        \ifx\attr@b\relax
          \s@tattr@val #1 \E
        \else
          \ifmst@nestyle
            \s@tattr@val \attr@b\milestoneJoinChar#1 \E% Join words with milestoneJoinChar instead of space.
          \else
            \s@tattr@val \attr@b\space #1 \E% Join words with a space.
          \fi 
        \fi
        \trace{A}{Got unnamed(\thisdefault@ttrkey) attribute '#1' -> '\attr@val'}%
        \set@ttribute{\thisdefault@ttrkey}{\attr@val}%
      \fi
    \fi
  \else
    \edef\t@st{\zap@space #1 \empty}%
    \ifx\attr@key\t@st
      \trace{A}{Got named attribute \attr@key="\attr@val"}%
      \set@ttribute{\attr@key}{\attr@val}%
    \else
      \trace{A}{Got unnamed AND named attributes '#1'}%
      \x@\spl@tonspace #1\E
      \trace{A}{Got unnamed(\thisdefault@ttrkey) -> '\f@rstitem'}%
      \set@ttribute{\thisdefault@ttrkey}{\f@rstitem}%
      \trace{A}{Got named attribute \s@conditem="\attr@val"}%
      \set@ttribute{\s@conditem}{\attr@val}%
    \fi
  \fi
  \def\t@st{#3}%
  \ifx\t@st\empty\else
    \ifx\t@st\spaceval\else
      \save@ttribute #3\E
    \fi
  \fi
}


\def\@checkzvar#1#2::#3::#4\E{%if the value saves is zvar::something, then return that zvar
  \trace{A}{@checkzvar '#1' '#2' '#3'}%
  \edef\tmp{#2}\ifx\tmp\zv@r
    \edef#1{\csname ms:zvar=#3\endcsname}%
    \trace{A}{attrib '#1' is '\attr@b'}%
  \fi
}
\newif\if@ttrexists
\def\testt@ttrn@me#1{\ifcsname attr:#1\endcsname\@ttrexiststrue\else\@ttrexistsfalse\fi}%
\def\get@ttribute#1{\x@\let\x@\attr@b\csname attr:#1\endcsname \trace{A}{attrib '#1' is '\attr@b'}%
  \x@\@checkzvar\x@{\x@\attr@b\x@}\attr@b ::::\E} %simple access function

\def\get@ttributedef#1#2#3{%Gets a (potentially numeric) atttribute, with a default. #1 attribute; #2 csname ; #3 default if not defined
  \get@ttribute{#1}\ifx\attr@b\relax
    \ifx#2\undefined 
      \edef#2{#3}%
    \else\ifx#2\relax 
        \edef#2{#3}%
    \fi\fi
  \else
    \m@kenumber{\attr@b}%
    \let#2\@@result
  \fi
}

\def\set@ttribute#1#2{%
  \CheckSpecial{#1}\ifattrspecial\ifcsname boxed-\thiswh@tstyle\endcsname\else %
    \MSG{A special attribute (#1) was used for \thiswh@tstyle\space but it was not listed as an attribute in the stylesheet}%
  \fi\fi
  \x@\xdef\csname attr:#1\endcsname{#2}%
  \xdef\attribsus@d{\ifx\attribsus@d\empty\else\attribsus@d,\fi#1}% Keep a list so we can junk it later.
}

% Parse a space-separated list.
\def\relax@tem#1\E{}

\def\zap@space#1 #2{%borrowed from latex.ltx. Eats all spaces in arg, including mid-space
  #1%
  \ifx#2\empty\else\expandafter\zap@space\fi
  #2}
\def\zap@plus#1+#2{%borrowed from latex.ltx. Eats all pluses in arg, including mid-space
  #1%
  \ifx#2\empty\else\expandafter\zap@plus\fi
  #2}
\def\zap@comma#1,#2{%borrowed from latex.ltx. Eats all commas in arg, including mid-space
  #1%
  \ifx#2\empty\else\expandafter\zap@comma\fi
  #2}


\def\geton@item #1 #2\E{% Parse space separated list
  \edef\it@m{#1}%
  \edef\it@mtwo{\zap@space #2 \empty}%
  \trace{A}{getone pair (#1) (#2) -> (\it@m) (\it@mtwo)}%
  \ifx\it@m\empty\ifx\it@mtwo\empty\let\nxt@tem=\relax@tem\fi\else
    \D@first{#1}%
  \fi
  \nxt@tem #2 \E
}


% Parse an attribute="value" from the USFM code. May be wrapped in {}
\def\parse@ttribs#1{% Expand again to solve brackets
   \x@\@parse@ttribs #1 \E
}
\def\@parse@ttribs#1 \E{%
   \x@\save@ttribute #1 ="\relax" \E
}

%\def\parse@ttribs#1{%
  %\let\nxt@tem=\geton@item
  %\let\D@first=\s@ve@ttribute
  %\x@\geton@item #1 \E
%}

% Parse list of attributes from style file to read the default attribute
% and the complete list(comma separated).
\def\Attributes#1\relax{% Override empty \Attributes from ptx-stylesheets
  \trace{A}{Loading attr list for \m@rker (#1)}%
  \let\nxt@tem=\geton@item %only need the first one.
  \let\D@first=\interpr@tfirstattrib
  \let\save@ttrib=\setdefaultattrib
  \let\@ttrlistname\relax
  \x@\geton@item #1 \E
  \ifx\@ttrlistname\relax\else 
    \trace{A}{Attribute list for \m@rker\space is \csname\@ttrlistname\endcsname}%
  \fi
}
\def\spl@tonspace#1 #2\E{\trace{A}{Spl@tonspace{#1}{#2}}\edef\f@rstitem{#1}\edef\s@conditem{#2}}

\def\interpr@tfirstattrib#1{%
  \x@\interpr@t@ttrib #1??\E
  \let\D@first=\interpr@tother@ttrib
  \let\save@ttrib=\extend@ttriblist
}
\def\interpr@tother@ttrib#1{%
  \x@\interpr@t@ttrib #1??\E
}
\def\store@nattrib#1-#2|#3\E{%Ignore - in milestone markers
  \save@ttrib{#1}{#3}%
}


\def\interpr@t@ttrib#1?#2?#3\E{% There may be a preceding ? mark. Also setdefaultattrib needs to not have the -s / -e for milestones. 
  \def\tmp{#1}%
  \ifx\tmp\empty
     \trace{A}{Attribute '#2' is optional (#1,#2,#3)}%
     \x@\store@nattrib\m@rker-|#2\E
  \else
     \trace{A}{Attribute '#1' is officially required (#1,#2,#3)}%
     \x@\store@nattrib\m@rker-|#1\E
  \fi
}

\def\alignb@x#1#2{\getp@ram{justification}{#1}{#1+\styst@k}% Do we want a third parameter here?
    \ifx\p@ram\c@nter\hfil\else\ifx\p@ram\r@ght\hfil\else\ifx\p@ram\relax\hfil\fi\fi\fi
    \box#2%
    \ifx\p@ram\c@nter\hfil\else\ifx\p@ram\l@ftb@l\hfil\else\ifx\p@ram\l@ft\hfil\else\ifx\p@ram\relax\hfil\fi\fi\fi\fi}
\def\rubyb#1#2{\trace{A}{rubyb}\setbox1=\hbox{\s@tfont{#2|#1}{#2|#1+\styst@k}\op@ninghooks{start}{#2|#1}{#2|#1+}%
  \get@ttribute{#2}\attr@b\cl@singhooks{end}{#2|#1}{#2|#1+}}%
    \dimen0=\ifdim\wd0>\wd1 \wd0\else\wd1\fi
    \setbox0=\hbox{\vtop{\s@tbaseline{#1}{\styst@k}\hbox to \dimen0{\alignb@x{#1}{0}}%
        \hbox to \dimen0{\alignb@x{#2|#1}{1}}}}}
\def\rubyt#1#2{\trace{A}{rubyt}\setbox1=\hbox{\s@tfont{#2|#1}{#2|#1+\styst@k}\op@ninghooks{start}{#2|#1}{#2|#1+}%
  \get@ttribute{#2}\attr@b\cl@singhooks{end}{#2|#1}{#2|#1+}}%
    \dimen0=\ifdim\wd0>\wd1 \wd0\else\wd1\fi
    \setbox0=\hbox{\vbox{\s@tbaseline{#2|#1}{#2|#1+\styst@k}\hbox to \dimen0{\alignb@x{#2|#1}{1}}%
        \hbox to \dimen0{\alignb@x{#1}{0}}}}}

\expandafter\def\csname complex-rb\endcsname{\rubyt{rb}{gloss}} % USFM-3 suggests ruby goes above. 
 
%usfm 3.0 defaults - now loaded from stylesheet.
%\setdefaultattrib{qt}{who}
%\setdefaultattrib{rb}{gloss}
%\setdefaultattrib{w}{lemma}

