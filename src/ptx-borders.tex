%:skip
%%%%%%%%%%%%%%%%%%%%%%%%%%%%%%%%%%%%%%%%%%%%%%%%%%%%%%%%%%%%%%%%%%%%%%%
% Part of the ptx2pdf macro package for formatting USFM text
% copyright (c) 2022 by SIL International
% written by David Gardner
%
% Permission is hereby granted, free of charge, to any person obtaining  
% a copy of this software and associated documentation files (the  
% "Software"), to deal in the Software without restriction, including  
% without limitation the rights to use, copy, modify, merge, publish,  
% distribute, sublicense, and/or sell copies of the Software, and to  
% permit persons to whom the Software is furnished to do so, subject to  
% the following conditions:
%
% The above copyright notice and this permission notice shall be  
% included in all copies or substantial portions of the Software.
%
% THE SOFTWARE IS PROVIDED "AS IS", WITHOUT WARRANTY OF ANY KIND,  
% EXPRESS OR IMPLIED, INCLUDING BUT NOT LIMITED TO THE WARRANTIES OF  
% MERCHANTABILITY, FITNESS FOR A PARTICULAR PURPOSE AND  
% NONINFRINGEMENT. IN NO EVENT SHALL SIL INTERNATIONAL BE LIABLE FOR  
% ANY CLAIM, DAMAGES OR OTHER LIABILITY, WHETHER IN AN ACTION OF  
% CONTRACT, TORT OR OTHERWISE, ARISING FROM, OUT OF OR IN CONNECTION  
% WITH THE SOFTWARE OR THE USE OR OTHER DEALINGS IN THE SOFTWARE.
%
% Except as contained in this notice, the name of SIL International  
% shall not be used in advertising or otherwise to promote the sale,  
% use or other dealings in this Software without prior written  
% authorization from SIL International.
%%%%%%%%%%%%%%%%%%%%%%%%%%%%%%%%%%%%%%%%%%%%%%%%%%%%%%%%%%%%%%%%%%%%%%%
%
% Border processing
\newbox\b@rderbox
\newif\ifhorizb@rder
\newif\ifhasb@rder
\newif\ifearlybgimage
\def\borderstylelist{plain, double}

\let\b@drcol\relax

%Stylesheet 
\def\BorderWidth        #1\relax{\initc@t\setsbp@ram{borderwidth}{#1}}
\def\BorderColour       #1\relax{\initc@t\x@\checkh@x #1\end\setsbp@ram{bordercolour}{\rgb@out}}
\def\BorderStyle        #1\relax{\initc@t\ifcsname mkb@rder@#1\endcsname\setsbp@ram{borderstyle}{#1}\else\BorderRef #1\relax\fi}
\def\BorderStyleExtra   #1\relax{\initc@t\setsbp@ram{borderstyleextra}{#1}}
\def\BorderHPadding #1\relax{\initc@t\setsbp@ram{borderhpadding}{#1}}
\def\BorderVPadding #1\relax{\initc@t\setsbp@ram{bordervpadding}{#1}}
\def\BorderPadding #1\relax{\initc@t\setsbp@ram{borderhpadding}{#1}\setsbp@ram{bordervpadding}{#1}}
\let\BorderColor=\BorderColour%for Americans
\def\Border	#1\relax{\initc@t\x@\p@rseBorder#1 \relax \relax}
\def\BorderFillColour #1\relax{\initc@t\x@\checkh@x #1\end\setsbp@ram{borderfillcolour}{\rgb@out}}
\let\BorderFillColor\BorderFillColour
\def\BorderLineWidth #1\relax{\initc@t\setsbp@ram{borderlinewidth}{#1}} % linewidth for double border (or ornament)
\def\BorderRef #1\relax{\initc@t\setsbp@ram{borderref}{#1}\xdef\dep@ndlist{\ifx\empty\dep@ndlist\else\dep@ndlist,\fi\c@tprefix\m@rker}}%\x@\xdef\csname border@deps-#1\endcsname{\ifcsname border@deps-#1\endcsname\csname border@deps\endcsname,\fi\m@rker}}
\def\SetBorder #1\relax{\Marker bdr-#1\relax}

\def\mkb@rder{%Put a simple border around a box
  \ifx\b@drcol\relax
    \def\bc@l{}\def\endbc@l{}%
  \else
    \def\bc@l{\special{color push rgb \b@drcol}}\def\endbc@l{\special{color pop}}%
  \fi
  \dimen0=\b@rderbox@height \advance\dimen0 by \dimexpr -\b@drwidth pt  * \b@drvsides\relax
  \dimen1=0pt
  \setbox\b@rderbox\vbox to \b@rderbox@height{\baselineskip=0pt \lineskip=0pt \hsize=\b@rderbox@width\bc@l
    \ifx\b@drtop\tr@e\trace{eb}{Drawing Border:top \b@rderbox@width}\hbox{\vrule height\b@drwidth pt  width \b@rderbox@width}\fi%
    \hbox to \b@rderbox@width{%
      \ifx\b@drleft\tr@e\trace{eb}{Drawing Border:left \the\dimen0+\the\dimen1}%
        \vrule width \b@drwidth pt  height \dimen0 depth \dimen1 
      \fi
    \hss
      \ifx\b@drright\tr@e\trace{eb}{Drawing Border:right \the\dimen0+\the\dimen1}%
        \vrule width \b@drwidth pt  height \dimen0 depth \dimen1
      \fi
    }\vss
    \ifx\b@drbottom\tr@e\trace{eb}{Drawing Border:bottom}\hbox{\vrule height\b@drwidth pt  width \b@rderbox@width}\fi
\endbc@l}%
}

\def\ch@ckiftextborderadj{\tempfalse\ifm@rksonpage
  \ifdim \dimexpr \pretextb@rderskip + \posttextb@rderskip\relax>0pt
    \temptrue
  \fi
 \fi
}

\def\s@ttextb@rderdims{%
  \trace{et}{s@ttextb@rderdims}%
  \testm@rkerexists{textborder}%
  \iftemp
    \t@stpublishability{textborder}%
    \ifn@npublishable\else
      \temptrue
    \fi
  \fi
  \iftemp\bgroup
      \def\sb@rmarker{textborder}%
      \@get@Vborder@style@params
      \ch@ckhasb@rder
      \@prepVb@rderdims
      \ifhasb@rder
        \getsbp@ram{spacebefore}%
        \xdef\pretextb@rderskip{\the\dimexpr \ifx\p@ram\relax 0pt\else \p@ram\verticalsp@ceunit\fi + \b@dr@top@pad\relax}
        \getsbp@ram{spaceafter}%
        \xdef\posttextb@rderskip{\the\dimexpr \ifx\p@ram\relax 0pt\else \p@ram\verticalsp@ceunit\fi + \b@dr@bot@pad\relax}
      \else
        \xdef\pretextb@rderskip{0pt}%
        \xdef\posttextb@rderskip{0pt}%
      \fi
      \trace{et}{Set textborder pre/post dimensions to \pretextb@rderskip / \posttextb@rderskip}%
    \egroup
  \else
    \xdef\pretextb@rderskip{0pt}%
    \xdef\posttextb@rderskip{0pt}%
  \fi
}
\addtoidhooks{\s@ttextb@rderdims}% 
%\def\TextBorderStyle{}
%\addtoidhooks{%
%  \xdef\posttextb@rderskip{20pt}%
%  \xdef\pretextb@rderskip{20pt}%
%  }
\def\doTextB@rder#1{% Put border around box #1
  \testm@rkerexists{textborder}%
  \iftemp
    \trace{et}{textborder style exists}%
    \t@stpublishability{textborder}%
    \ifn@npublishable\else
    %\ifm@rksonpage\ifcsname TextBorderStyle\endcsname\ifx\TextBorderStyle\empty\else
      \ifnum\pageno>0\edef\t@st{\botmark}%
        \ifx\t@st\t@tle\ifnum\BodyColumns=\IntroColumns \else \tempfalse\fi\fi
        \ifm@rksonpage\else\tempfalse\fi
        \ifx\t@st\empty\tempfalse\fi
        \iftemp
          \trace{eb}{Making page border for box #1 (page \the\pageno [\t@st]), \the\ht#1+\the\dp#1x\the\wd#1}%
          \bgroup 
            \def\sb@rmarker{textborder}%
            \S@tCat{\id@@@}%
            \get@border@style@params
            %\let\b@drtop\tr@e\let\b@drbottom\tr@e
            %\let\b@drleft\tr@e\let\b@drright\tr@e
            %\hasb@rdertrue
            \prepb@rderdims
            \prepb@rderbox{#1}%
            \ifhasb@rder
              \dob@rder
              %\vbox to 0pt{\vskip-\b@dr@top@pad\hbox to 0pt{\hskip-\b@dr@left@pad\box\b@rderbox\hss}\vss}%
              \global\setbox#1\vbox to \ht#1{\baselineskip=0pt\kern\pretextb@rderskip\vbox to 0pt{\vskip-\b@dr@top@pad\hbox to 0pt{\hskip-\b@dr@left@pad\box\b@rderbox\hss}\vss}\box#1\vskip\posttextb@rderskip}%
            \fi
          \egroup
        \fi
      \fi
    \fi
  \fi
}

\let\mkb@rder@plain\mkb@rder
\def\mkb@rder@double{%
  \ifx\b@drcol\relax
    \def\bc@l{}\def\endbc@l{}%
  \else
    \def\bc@l{\special{color push rgb \b@drcol}}\def\endbc@l{\special{color pop}}%
  \fi
  \ifx\b@drlinewidth\relax
    \edef\b@drlinewidth{\strip@pt{\dimexpr \b@drwidth pt /3\relax}}% Use pt for division.
    \let\b@drfillwidth\b@drlinewidth
  \else
    \edef\b@drfillwidth{\strip@pt{\dimexpr \b@drwidth pt  - 2\dimexpr \b@drlinewidth pt  \relax\relax}}%
    %\edef\b@drwidth{\strip@pt{\dimexpr \b@drlinewidth pt \relax}}%
  \fi
  \ifdim \b@drfillwidth pt<0pt \let\b@drfillcolour\fill@none\fi
  \temptrue
  \ifx\b@drfillcolour\relax\tempfalse
  \else\ifx\b@drfillcolour\fill@none\tempfalse\fi
  \fi
  \iftemp 
    \def\fc@l{\special{color push rgb \b@drfillcolour}}\def\endfc@l{\special{color pop}}%
  \else \def\fc@l{}\def\endfc@l{}%
  \fi
  \dimen0=\b@rderbox@height \advance\dimen0 by \dimexpr -\b@drlinewidth pt  * \b@drvsides\relax %Inner vertical
  \dimen1=\dimen0 \advance\dimen1 by \dimexpr -\b@drfillwidth pt  * \b@drvsides\relax %Fill vertical
  \dimen2=\dimen1 \advance\dimen2 by \dimexpr -\b@drlinewidth pt  * \b@drvsides\relax %Outer vertical
  \dimen3=\b@rderbox@width\advance\dimen3 by \dimexpr -\b@drlinewidth pt  * \b@drhsides\relax % Fill width
  \dimen4=\dimen3\advance\dimen4 by \dimexpr -\b@drfillwidth pt  * \b@drhsides\relax %Inner horiz length
  \trace{eb}{Borders: \b@rderbox@width - \b@drlinewidth pt   =\the\dimen3, \b@rderbox@height - }%
  \dimen5=\dimexpr \b@drfillwidth pt  + \b@drlinewidth pt \relax
  \setbox\b@rderbox\vbox to \b@rderbox@height{\baselineskip=0pt \lineskip=0pt \lineskiplimit=0pt \hsize=\b@rderbox@width
    \ifx\b@drtop\tr@e
      \hbox to \b@rderbox@width{\bc@l\vrule height \b@drlinewidth pt  width \b@rderbox@width \endbc@l}%
      \ifx\b@drfillcolour\fill@none\kern\b@drfillwidth pt \else
        \hbox to \b@rderbox@width{\ifx\b@drleft\tr@e\kern\b@drlinewidth pt \fi
          \fc@l\vrule height \b@drfillwidth pt  width\dimen3\endfc@l
          \ifx\b@drright\tr@e\kern\b@drlinewidth pt \fi}%
      \fi
      \hbox{\ifx\b@drleft\tr@e\kern\dimen5\fi
        \bc@l
        \vrule height\b@drlinewidth pt  width\dimen4\endbc@l}%
      \kern-\dimen5
    \fi
    \hbox to \b@rderbox@width{%   
      \ifx\b@drleft\tr@e
        \bc@l\vrule width \b@drlinewidth pt  height \dimen0 \endbc@l
        \ifx\b@drfillcolour\fill@none\kern\b@drfillwidth pt \else\raise\b@drfillwidth pt \hbox{\fc@l \vrule width \b@drfillwidth pt  height \dimen1 \endfc@l}\fi%
        \raise\dimen5\hbox{\bc@l\vrule width \b@drlinewidth pt  height \dimen2 \endbc@l}%
      \fi
    \hss
      \ifx\b@drright\tr@e
        \raise\dimen5\hbox{\bc@l\vrule width \b@drlinewidth pt  height \dimen2 \endbc@l}%
        \ifx\b@drfillcolour\fill@none\kern\b@drfillwidth pt \else\raise\b@drfillwidth pt \hbox{\fc@l \vrule width \b@drfillwidth pt  height \dimen1 \endfc@l}\fi%
        \bc@l \vrule width \b@drlinewidth pt  height \dimen0 \endbc@l
      \fi
    }%
    \vss
    \ifx\b@drbottom\tr@e
      \kern -\dimen5
      \hbox to \b@rderbox@width{\ifx\b@drleft\tr@e\kern\dimen5\fi\bc@l\vrule height\b@drlinewidth pt  width\dimen4\endbc@l}%
      \ifx\b@drfillcolour\fill@none\kern\b@drfillwidth pt \else
        \hbox to \b@rderbox@width{\ifx\b@drleft\tr@e\kern\b@drlinewidth pt \fi\fc@l\vrule height \b@drfillwidth pt  width\dimen3\endfc@l}%
      \fi
      \hbox{\bc@l\vrule height \b@drlinewidth pt  width \b@rderbox@width \endbc@l}%
    \fi
  }%
}

\def\@get@Vborder@style@params{%Read the minimal set of border-related parameters for vertical spacing
  \trace{eb}{Reading border vertical params for \c@tprefix\sb@rmarker}%
  \getsbp@ram{bordervpadding}\ifx\cp@ram\relax\xdef\border@vpadding{0pt}\else\edef\border@vpadding{\the\dimexpr \cp@ram pt \relax}\fi
  \getsbp@ram{boxvpadding}\ifx\cp@ram\relax\else\xdef\fb@vpadding{\the\dimexpr \cp@ram pt \relax}\fi
  \getsbp@ram{bordertop}\let\b@drtop\cp@ram
  \getsbp@ram{borderbottom}\let\b@drbottom\cp@ram 
  \getsbp@ram{borderwidth}\let\b@drwidth\cp@ram
  \getsbp@ram{properties}\let\b@drprop\cp@ram
} 
\def\get@border@style@params{%Read the border-related parameters
  \trace{eb}{Reading border params for \c@tprefix\sb@rmarker}%
  \getsbp@ram{borderhpadding}\ifx\cp@ram\relax\xdef\border@hpadding{0pt}\else\edef\border@hpadding{\the\dimexpr \cp@ram pt \relax}\fi
  \getsbp@ram{boxhpadding}\ifx\cp@ram\relax\else\xdef\fb@hpadding{\the\dimexpr \cp@ram pt \relax}\fi
  \getsbp@ram{\ifodd\pageno borderoddleft\else borderevenleft\fi}\let\b@drleft\cp@ram
  \getsbp@ram{\ifodd\pageno borderoddright\else borderevenright\fi}\let\b@drright\cp@ram
  \@get@Vborder@style@params
  \getsbp@ram{borderlinewidth}\let\b@drlinewidth\cp@ram
  \getsbp@ram{borderfillcolour}\let\b@drfillcolour\cp@ram
  %\ifx\b@drfillcolour\relax\let\b@drfillcolour\fill@none\fi
  \getsbp@ram{borderstyle}\let\b@drstyle\cp@ram
  \getsbp@ram{bordercolour}\let\b@drcol\cp@ram
  \getsbp@ram{properties}\let\b@drprop\cp@ram
  \trace{eb}{Params finished}%
}
\def\b@rderparameters{borderstyle,borderwidth,borderfillcolour,bordercolour,borderlinewidth,bordertop,borderbottom,borderoddleft,borderoddright,borderevenleft,borderevenright,borderhpadding,bordervpadding,boxhpadding,boxvpadding,raise,verticalalign}
%raise and verticalalign are not used by borders, but is by zrules


\def\dob@rder{%
  \ifx\b@drstyle\relax
    \ifx\b@drfillcolour\relax
      \ifx\b@drlinewidth\relax
        \edef\b@drstyle{plain}%
      \else
        \trace{eb}{Border line width defined and style not specified, assume double border}%
        \edef\b@drstyle{double}%
      \fi
    \else
      \trace{eb}{Border fill defined to '\b@drfillcolour', and style not specified, assume double border}%
      \edef\b@drstyle{double}%
    \fi
  \fi
  \ifcsname mkb@rder@\b@drstyle\endcsname 
    \trace{eb}{making box for \b@drstyle}%
    \csname mkb@rder@\b@drstyle\endcsname
  \else
    \message{! Unknown borderstyle '\b@drstyle'. Known styles: \borderstylelist}%
  \fi 
}

\def\@prepVb@rderdims{% Minimal set of vertical border dimensions
  \ifhasb@rder
    \xdef\b@dr@top@pad{\the\dimexpr \fb@vpadding  \ifx\b@drtop\tr@e +\border@vpadding + \b@drwidth pt  \fi\relax}% Distance to back-skip to insert border
    \xdef\b@dr@bot@pad{\the\dimexpr \fb@vpadding  \ifx\b@drbottom\tr@e +\border@vpadding + \b@drwidth pt  \fi\relax}%
  \else
    \xdef\b@dr@top@pad{0pt}%
    \xdef\b@dr@bot@pad{0pt}%
  \fi
}
  
\def\ch@ckhasb@rder{%
  \hasb@rderfalse\horizb@rderfalse
  \ifx\b@drleft\tr@e\hasb@rdertrue\trace{eb}{Border: left}\fi
  \ifx\b@drright\tr@e\hasb@rdertrue\trace{eb}{Border: right}\fi
  \ifx\b@drtop\tr@e\hasb@rdertrue\horizb@rdertrue\trace{eb}{Border: Top}\fi
  \ifx\b@drbottom\tr@e\hasb@rdertrue\horizb@rdertrue\trace{eb}{Border: Bottom}\fi
  \ifx\b@drprop\N@nPublish@ble\hasb@rderfalse\trace{eb}{Border: nonpublishable}\else\trace{eb}{Border property: \b@drprop}\fi
}
\def\prepb@rderdims{% Prepare all b@rderbox dimensions
  \ch@ckhasb@rder
  \ifhasb@rder
    %Where does is the outer-edge of the  border or box,  relative to the contents?
    \edef\b@dr@top@edge{\the\dimexpr\ifx\b@drtop\tr@e \ifdim  \border@vpadding > -\b@drwidth pt  \border@vpadding + \b@drwidth pt  +\fi\fi  \fb@vpadding\relax}%
    \edef\b@dr@bot@edge{\the\dimexpr\ifx\b@drbottom\tr@e \ifdim  \border@vpadding > -\b@drwidth pt  \border@vpadding + \b@drwidth pt +\fi\fi  \fb@vpadding\relax}%
    %Where does the border go relative to the contents? (Subtliy different to 
    \@prepVb@rderdims
    \xdef\b@dr@left@pad{\the\dimexpr \fb@hpadding  \ifx\b@drleft\tr@e +\border@hpadding + \b@drwidth pt  \fi\relax}%
    \xdef\b@dr@right@pad{\the\dimexpr \fb@hpadding  \ifx\b@drright\tr@e +\border@hpadding + \b@drwidth pt  \fi\relax}%
    \edef\b@drhsides{\the\numexpr \ifx\b@drleft\tr@e 1 +\fi \ifx\b@drright\tr@e 1+\fi 0\relax}%
    \edef\b@drvsides{\the\numexpr \ifx\b@drtop\tr@e 1 +\fi \ifx\b@drbottom\tr@e 1+\fi 0\relax}%
    \edef\b@xintadj{\the\dimexpr \b@dr@top@edge - \b@dr@top@pad\relax}%
    \trace{eb}{BotGap:\b@dr@bot@edge, TopGap:\b@dr@top@edge}%
  \else
    \edef\b@drvsides{0}%
    \edef\b@drhsides{0}%
    \edef\b@dr@top@edge{\fb@vpadding}%
    \edef\b@dr@bot@edge{\fb@vpadding}%
    \xdef\b@dr@left@pad{\fb@hpadding}%
    \xdef\b@dr@right@pad{\fb@hpadding}%
    \xdef\b@dr@top@pad{\fb@vpadding}%
    \xdef\b@dr@bot@pad{\fb@vpadding}%
    \xdef\b@xintadj{0pt}%
  \fi
}

\def\prepb@rderbox#1{% Prepare b@rderbox
  \ifhasb@rder
    \setbox\b@rderbox\vbox to \ht#1{\hsize=\wd#1\hbox to \wd#1{\hss}\vss}%Placeholder
    \dp\b@rderbox=\dp#1\relax
    \edef\b@rderbox@width{\the\dimexpr \wd\b@rderbox +\fb@hpadding*2 + \b@drwidth pt  *\b@drhsides+ \border@hpadding * \b@drhsides\relax}%% Final width of border artwork
    \edef\b@rderbox@height{\the\dimexpr \ht\b@rderbox +\fb@vpadding +\dp\b@rderbox + \b@drwidth pt  *\b@drvsides + \border@vpadding *\b@drvsides\relax}% Final height of border artwork 
    \edef\b@xadj{\ifdim\border@vpadding< -\b@drwidth pt  \ifx\b@drtop\tr@e\the\dimexpr \fb@vpadding -\b@drwidth pt -\border@vpadding\relax\else 0pt \fi\else 0pt \fi}% b@xadj puts the top edge of the border or box  at the top of an insert.
    \edef\b@xextadj{\the\dimexpr 0pt - \ifdim\border@vpadding< -\b@drwidth pt  \ifx\b@drbottom\tr@e\the\dimexpr \b@drwidth pt  +\border@vpadding\relax\else 0pt \fi\else 0pt \fi\relax}% b@xminadj is for inserts, so that the top of the box is at exactly the page top.
    %\edef\b@xextadj{\ifdim\border@vpadding> -\b@drwidth pt  \ifx\b@drtop\tr@e 3pt \else 0pt \fi\else 0pt \fi}%b@xextadj puts some space at the top of a box that will be gridded. This protects it from descenders
    \edef\b@rderbox@depth{\the\dimexpr \dp\b@rderbox  \ifx\b@drbottom\tr@e + \border@vpadding +\b@drwidth pt  \fi \relax}% 
    %\trace{e}{Depth b@drbox=\the\dp\b@rderbox / \the\dp\extchunkb@x / \thsch@nkdp}%
    %\xdef\b@xintadj{10pt}%
  \else
    \setbox\b@rderbox\box\voidb@x
    \edef\b@xextadj{0pt}%
    \edef\b@xextadj{0pt}%
    %\edef\b@xadj{\the\dimexpr -\fb@vpadding\relax}%
  \fi
  \trace{eb}{b@rderbox: \the\ht\b@rderbox+\the\dp\b@rderbox x\the\wd\b@rderbox}%
}
\def\c@pyp@rams#1\E{% If the style has no parameter, copy the parent's one in
  \getp@r@m{#1}{\m@rker}%
  \ifx\p@ram\relax
    \getp@r@m{#1}{bdr-\p@rent}%
    \ifx\relax\p@ram
      \trace{ebi}{Parent has no #1 }%
    \else
      \trace{ebi}{Inheriting #1 (\p@ram) from parent}%
      %\tracingassigns=1
      \setp@ram{#1}{\m@rker}{\p@ram}%
      %\tracingassigns=0
    \fi
  \else
    \trace{ebi}{Keeping own version of #1 (\p@ram)}%
  \fi
}
\def\tryc@pyb@rders#1\E{%
  \def\m@rker{#1}%
  \getp@r@m{borderref}{\m@rker}%
  \trace{ebi}{Trying to process \m@rker <-\p@ram}%
  \ifx\p@ram\relax\else
    \ifx\p@ram\empty\else
      \let\p@rent\p@ram
      \testm@rkerexists{bdr-\p@rent}%
      \iftemp
        \getp@r@m{borderref}{bdr-\p@rent}%
        \trace{ebi}{Has parent got a parent? "\p@ram"}%
        \temptrue
        \ifx\p@ram\empty
          \tempfalse
        \else
          \ifx\p@ram\relax
            \tempfalse
          \fi
        \fi
        \iftemp
          \trace{ebi}{Postponing \m@rker, \p@rent has not been done yet}%
          \xdef\postp@nelist{\ifx\empty\postp@nelist\else\postp@nelist,\fi\m@rker}%
        \else
          \trace{ebi}{Processing \m@rker, inheriting borders from \p@rent}%
          \let\id@\d@
          \let\d@\c@pyp@rams
          \x@\cstackdown\b@rderparameters,\E
          \setp@ram{borderref}{\m@rker}{}%
          \let\d@\id@
        \fi
      \else
        \message{! Parent style "\p@rent" referenced for borders by "\m@rker"  does not exist!}%
      \fi
    \fi
  \fi
  \trace{ebi}{END.}%
}
\def\dep@ndlist{}
\def\inheritb@rders{%
  \ifx\empty\dep@ndlist
    \let\n@xt\relax
  \else
    \let\postp@nelist\empty
    \trace{ebi}{Dependency list:\dep@ndlist}%
    \let\d@\tryc@pyb@rders
    \x@\cstackdown\dep@ndlist,\E
    \trace{ebi}{Postponed:\postp@nelist}%
    \ifx\postp@nelist\empty
      \let\n@xt\relax
    \else
      \let\n@xt\inheritb@rders
    \fi
  \fi
}
  

