%:strip
% This optional plugin (see ptx-plugins) provides ptx2pdf users with 
% access to decorative ornaments distributed as part of latex package pgfornaments,
% It does not include those ornaments, which are available at https://www.ctan.org/pkg/pgfornament
%
% This code thus includes a  XeTeX-specific, partial reimplementation of some portions
% of the pgfornament package to use the pgf-vectorian (and other) ornaments,
% but without loading (any of) pgf. 
% It also adds auto-filling and scaling functionality not found in that original package.
%
% Copyright (c) 2021 by SIL International written by David Gardner
%
% Some portions of this code derive from the pgfornament.sty file  (v0.2)
%  (C) 2016 Alain Matthes,  which was released under LaTeX Project Public
%  License. 
% Other portions de-abstract the abstraction layers from the pdf drivers from
% the pgf package, released under the same licence.
% 
%
% THE SOFTWARE IS PROVIDED "AS IS", WITHOUT WARRANTY OF ANY KIND,  
% EXPRESS OR IMPLIED, INCLUDING BUT NOT LIMITED TO THE WARRANTIES OF  
% MERCHANTABILITY, FITNESS FOR A PARTICULAR PURPOSE AND  
% NONINFRINGEMENT. IN NO EVENT SHALL SIL INTERNATIONAL BE LIABLE FOR  
% ANY CLAIM, DAMAGES OR OTHER LIABILITY, WHETHER IN AN ACTION OF  
% CONTRACT, TORT OR OTHERWISE, ARISING FROM, OUT OF OR IN CONNECTION  
% WITH THE SOFTWARE OR THE USE OR OTHER DEALINGS IN THE SOFTWARE.
%
% Except as contained in this notice, the name of SIL International  
% shall not be used in advertising or otherwise to promote the sale,  
% use or other dealings in this Software without prior written  
% authorization from SIL International.
%%%%%%%%%%%%%%%%%%%%%%%%%%%%%%%%%%%%%%%%%%%%%%%%%%%%%%%%%%%%%%%%%%%%%%%
\def\makeatletter{\catcode`\@=11}% Used by external code
\makeatletter

\plugin@startif{ornaments}
%\plugins@needed{} % Just in case it becomes separate.

% 
\def\MSG#1{\immediate\write -1{#1}}
\def\makeatother{\catcode`\@=12}% Used by external code
\newdimen\pdf@xunit
\newdimen\pdf@yunit

\def\strip@pt#1{\x@\x@\x@\@strip@pt\x@\the\x@ #1\@pt\E}
{\catcode`P=12 \catcode`T=12 \lowercase{\gdef\@strip@pt#1PT#2\E{#1} \xdef\@pt{PT}}} 
\def\ornYalign{0}%
\let\x@\expandafter
\x@\let\csname orn@skip176\endcsname\relax
\let\ornamentmaxwidth\empty
\gdef\OrnamentsFamily{vectorian}
\gdef\pgfOrnamentsObject{pgflibrary\OrnamentsFamily.code.tex} 
\newtoks\tmptoks
\def\setorn@scale{%
  \tmptoks{}%
  \pdf@yunit=1bp
  \ifx\undefined\@pgfornamentY
    \trace{orn}{setorn@scale: Setting a scale for undefined object!}%
    \pdf@yunit=1bp
  \else
    \ifx\ornamentwidth\empty
      \tmptoks{\pdf@xunit=\pdf@yunit\relax
      \ifx\ornamentmaxwidth\empty\else
        \trace{orn}{Applying x-limit \ornamentmaxwidth}%
        \ifdim \@pgfornamentX\pdf@xunit > \ornamentmaxwidth
          \pdf@xunit=\dimexpr \ornamentmaxwidth / \@pgfornamentX\relax
          \pdf@yunit=\pdf@xunit
        \fi
      \fi
      }%
    \else
      \pdf@xunit=\dimexpr \ornamentwidth / \@pgfornamentX\relax
    \fi
    \ifx\ornamentheight\empty
      \x@\tmptoks\x@{\the\tmptoks\pdf@yunit=\pdf@xunit}%
    \else
        \pdf@yunit=\dimexpr \ornamentheight / \@pgfornamentY\relax
    \fi
    \the\tmptoks
    \trace{orn}{setorn@scale: xunit \the\pdf@xunit\space yunit \the\pdf@yunit. Object will be \the\dimexpr \@pgfornamentX\pdf@xunit\relax}%
  \fi
}

\def\check@orn@dot#1.#2.#3\E{%
  \edef\d@t@tmp{#2}\ifx\d@t@tmp\empty
    \pdf@xunit=\csname orn@scale@#1@x\endcsname
    \pdf@yunit=\csname orn@scale@#1@y\endcsname
  \else
    \setorn@scale
    \pdf@xunit= #1.#2\pdf@xunit
    \pdf@yunit= #1.#2\pdf@yunit
  \fi
}

\def\setornscale#1#2#3#4{\bgroup
  \get@ornamentDim{#2}%
  \ifx\dimen1=0pt
    \@notpgfornamentdim{#2}%
  \fi
  \edef\ornamentwidth{#3}%
  \edef\ornamentheight{#4}%
  \setorn@scale
  \x@\xdef\csname orn@scale@#1@x\endcsname{\the\pdf@xunit}%
  \x@\xdef\csname orn@scale@#1@y\endcsname{\the\pdf@yunit}%
\egroup}

\def\setornamenttransform#1{%
  \x@\let\x@\ornament@xform\csname ornament@xform@#1\endcsname
  \ifx\ornament@xform\relax
    \let\ornament@xform\ornament@xform@u
  \fi
  \x@\s@tornamenttransform@m\ornament@xform % Calculate pdf@aa, etc.
} 

\def\s@tornamenttransform@m#1#2#3#4{%
  \def\pdf@aa{#1}\def\pdf@ab{#2}\def\pdf@ba{#3}\def\pdf@bb{#4}%
}
\def\s@tornamenttransform#1#2#3#4{%
  \def\pdf@aa{#1}\def\pdf@ab{#2}\def\pdf@ba{#3}\def\pdf@bb{#4}%
  \def\pdf@xo{0}\def\pdf@yo{0}%
  \calc@ornamentdim
  \ifdim\dimen0<0pt
    \def\pdf@xo{\strip@pt{\dimexpr -\dimen0\relax}}%
  \fi
  \ifdim\dimen1<0pt
    \def\pdf@yo{\strip@pt{\dimexpr -\dimen1\relax}}%
  \fi
  \pdf@transformcm{\pdf@aa}{\pdf@ab}{\pdf@ba}{\pdf@bb}{\pdf@xo}{\pdf@yo}%
}
\def\orn@qpoint#1#2{%Apply x and y scaling to dimensinons
  {\strip@pt{\dimexpr #1\pdf@xunit\relax}} {\strip@pt{\dimexpr #2\pdf@yunit}} }


% Path construction:
\def\pdf@lineto#1#2{\addto@macro\pdfpath{#1 #2 }\sysprotocol@literal{l}}
\def\pdf@moveto#1#2{\addto@macro\pdfpath{#1 #2 }\sysprotocol@literal{m}}
\def\pdf@curveto#1#2#3#4#5#6{%
  \addto@macro\pdfpath{#1 #2 #3 #4 #5 #6 }\sysprotocol@literal{c}}
\def\pdf@rect#1#2#3#4{\addto@macro\pdfpath{#1 #2 #3 #4 }\sysprotocol@literal{re}}
\def\pdf@closepath{\sysprotocol@literal{h}}


% Path usage:
\newif\ifpgfsys@eorule % Using evenodd rule?
\pgfsys@eoruletrue
\def\pdf@stroke{\sysprotocol@literal{S}}\def\pdf@closestroke{\sysprotocol@literal{s}} \def\pdf@fill{\ifpgfsys@eorule\sysprotocol@literal{f*}\else\sysprotocol@literal{f}\fi}
\def\pdf@fillstroke{\ifpgfsys@eorule\sysprotocol@literal{B*}\else\sysprotocol@literal{B}\fi}
\def\pdf@clipnext{\ifpgfsys@eorule\sysprotocol@literal{W*}\else\sysprotocol@literal{W}\fi \pdf@discardpath\ifx\orn@clip@rect\empty\global\let\orn@clip@rect\orn@last@rect\fi}
%\def\pdf@clipnext{\pdf@stroke\pdf@discardpath}%
\def\pdf@discardpath{\sysprotocol@literal{n}}
\def\pdf@parse@clip@rect#1 #2 #3 #4{\trace{orn}{ParseClip: #1 #2 #3 #4}\dimen2=#1 pt \dimen3=#2 pt \dimen4=#3 pt \dimen 5=#4 pt}
%Pen
\def\pdf@setlinewidth#1{\addto@macro\pdfpath{#1 }\sysprotocol@literal{w}}
\def\pdf@color@rgb@stroke#1 #2 #3\E{\sysprotocol@literal{#1 #2 #3 RG}}
\def\pdf@color@rgb@fill#1 #2 #3\E{\sysprotocol@literal{#1 #2 #3 rg}}
\def\pdf@beginpic{\special{pdf:bcontent}}
\def\pdf@endpic{\special{pdf:econtent}}

% Transformation:
\def\pdf@transformcm#1#2#3#4#5#6{%
  \addto@macro\pdfpath{#1 #2 #3 #4 #5 #6 }\sysprotocol@literal{cm}}
\def\pdf@transformText{%
  \pdf@transformcm{\strip@pt{\pdf@xunit}}{0}{0}{\strip@pt{\pdf@yunit}}{0}{0}}

% Scopes
\def\pdf@beginscope{\sysprotocol@literal{q}}
\def\pdf@endscope{\sysprotocol@literal{Q}}

\long\def\addto@macro#1#2{%
  %\begingroup
    \x@\x@\x@\def\x@\x@\x@#1\x@\x@\x@{\x@#1#2}}%
 %\endgroup}
\def\pdf@special#1{\special{pdf:code #1}}
\def\pdfcommands{}
\def\pdfpath{}
\def\sysprotocol@literal#1{\let\tmp\pdfpath\xdef\pdfpath{}\x@\pdf@special{\tmp#1}\xdef\pdfcommands{}}
\edef\orn@clip{clip}
\edef\orn@stroke{stroke}
\edef\orn@bb{boundingbox}

\def\orn@usepath#1{%
  \edef\tmp{#1}%
  \ifx\tmp\orn@clip
    \pdf@clipnext
  \else
    \ifx\tmp\orn@stroke
      \pdf@stroke
    \else
      \ifx\tmp\orn@bb
        \pdf@clipnext
      \else
        \errmessage{orn@usepath: #1 unrecognised}%
      \fi
    \fi
  \fi
}

\def\orn@setup{%
  \def\i{\orn@usepath{clip}}%
  \def\k{\orn@usepath{stroke}}%
  \def\ubb{\orn@usepath{boundingbox}}%
  \let\o\pdf@closepath
  \def\p ##1##2{\orn@qpoint{##1}{##2}}%
  \def\m ##1 ##2 {\edef\tmp{\p{##1}{##2}}\x@\pdf@moveto\tmp}%
  \def\l ##1 ##2 {\edef\tmp{\p{##1}{##2}}\x@\pdf@lineto\tmp}%
  \def\r ##1 ##2 ##3 ##4 {\edef\tmp{\p{##1}{##2}\p{##3}{##4}}\xdef\orn@last@rect{\tmp}\x@\pdf@rect\tmp}%
  \def\c ##1 ##2 ##3 ##4 ##5 ##6 {%
    \edef\tmp{\p{##1}{##2}\p{##3}{##4}\p{##5}{##6}}\x@\pdf@curveto\tmp}%
}

\def\LocalOrnament#1#2#3#4{% Define a local ornament #1, natural size #2 x #3, having code #4
	\x@\gdef\csname localornament@#1@X\endcsname{#2}%
	\x@\gdef\csname localornament@#1@Y\endcsname{#3}%
	\x@\gdef\csname localornament@#1@code\endcsname{#4}%
}
\def\StringOrnament#1#2#3{\x@\global\x@\font\csname localornament@#1@font\endcsname "#2" at 32pt \setbox0\hbox{\csname localornament@#1@font\endcsname #3}%
    \trace{orn}{Defining ornament #1 as box (\the\ht0+\the\dp0) x \the\wd0}%
    \bgroup
      %\tracingassigns=1
      \dimen0=\wd0\dimen1=\dimexpr \dimexpr \ht0+\dp0\relax\relax
      \ifnum 1=\ifdim \the\dimen0=0pt 1 \else\ifdim \the\dimen1=0pt 1 \else 0\fi\fi
	\x@\edef\csname localornament@#1@code\endcsname{\hbox{\csname localornament@#1@font\endcsname #3}}%
	\x@\xdef\csname localornament@#1@X\endcsname{1}%
	\x@\xdef\csname localornament@#1@Y\endcsname{1}%
        \message{Cannot define zero-size ornament #1 properly}%
      \else
        \count255=\dimen0\advance\count255 by 32768 \divide\count255 by 65536
	\x@\xdef\csname localornament@#1@X\endcsname{\the\count255}%
        \count255=\dimen1\advance\count255 by 32768 \divide\count255 by 65536
	\x@\xdef\csname localornament@#1@Y\endcsname{\the\count255}%
        %\tracingassigns=0
	\x@\xdef\csname localornament@#1@code\endcsname{\noexpand\pdf@transformText\raise \the\dp0\hbox{\csname localornament@#1@font\endcsname #3}}%
      \fi
    \egroup
  }

\def\localornament@dims#1{\x@\let\x@\@pgfornamentX\csname localornament@#1@X\endcsname
  \x@\let\x@\@pgfornamentY\csname localornament@#1@Y\endcsname}%
\LocalOrnament{0}{10}{10}{ }% A space
\LocalOrnament{-1}{10}{10}{\m 0.0 5 \l 10 5 \k } % A centred line of equal lenght to a space
\LocalOrnament{-2}{20}{10}{\m 0.0 5 \l 20 5 \k } % A centred line twice as long as a space
\LocalOrnament{-3}{50}{10}{\m 0.0 5 \l 50 5 \k } % A centred line five times longer than as a space
	
\def\get@ornamentDim#1{%
  \ifcsname localornament@#1@X\endcsname
    \localornament@dims{#1}%
  \else
    \ifcsname @pgfornamentDim\endcsname
      \@pgfornamentDim{#1}%
    \else
      \errmessage{@pgfornamentDim not defined. Probably the path from the TEXINPUTS environment variable does not include \pgfOrnamentsObject}%
    \fi
  \fi
}
\def\get@ornament@Code{%
  \ifcsname localornament@\pdf@ornamentNum @code\endcsname
    \csname localornament@\pdf@ornamentNum @code\endcsname
  \else
    {\m@kedigitsother
      \catcode`\%=5
      \catcode`\{=1 
      \catcode`\}=2
      \input \OrnamentsFamily\pdf@ornamentNum.pgf%
    \m@kedigitsletters}%
  \fi
}

\def\OrnamentTweakY#1#2#3{%
  \x@\xdef\csname orn@adj@#2@#1@Y\endcsname{#3}%
}
\def\OrnamentTweakX#1#2#3{%
  \x@\xdef\csname orn@adj@#2@#1@X\endcsname{#3}%
}

\OrnamentTweakY{vectorian}{84}{2.5}
\OrnamentTweakX{pgfhan}{76}{-15.5}
\OrnamentTweakX{pgfhan}{77}{-15.5}
\OrnamentTweakX{pgfhan}{74}{-8}
\OrnamentTweakY{pgfhan}{74}{-8}
\OrnamentTweakX{pgfhan}{75}{-8}
\OrnamentTweakY{pgfhan}{75}{-8}
\OrnamentTweakY{pgfhan}{67}{-8}
\OrnamentTweakX{pgfhan}{67}{-8}
\OrnamentTweakY{pgfhan}{65}{-64}
\OrnamentTweakY{pgfhan}{64}{-24}
\OrnamentTweakX{pgfhan}{64}{-8}

\def\calc@ornamentdim{%
    \dimen0=\dimexpr \pdf@xunit*\numexpr \pdf@aa*\@pgfornamentX\relax + \pdf@yunit*\numexpr\pdf@ba*\@pgfornamentY\relax\relax
    \dimen1=\dimexpr \pdf@xunit*\numexpr \pdf@ab*\@pgfornamentX\relax + \pdf@yunit * \numexpr\pdf@bb*\@pgfornamentY\relax\relax
}
\def\ornamentstretch{1}
\newif\ifboxorn % Set to true to surroud all ornaments with boxes.
\boxornfalse
\def\boxornRGB{0.5 1.0 1.0}
\def\orn@do@ornament{\bgroup%
    \ifpdf@ornament@stretch
      \pdf@xunit=\ornamentstretch\pdf@xunit
    \fi
    \calc@ornamentdim
    \ifdim \dimen0<0pt
      \dimen0=-\dimen0
    \fi
    \ifdim\dimen1<0pt
      \dimen1=-\dimen1
    \fi
    \global\let\orn@clip@rect\empty
    \global\let\orn@last@rect\empty
    \trace{orn}{orn@do@ornament(\ifpdf@ornament@stretch *\ornamentstretch\fi) \the\dimen0 x \the\dimen1}%
    \setbox0\hbox to \dimen0{%
      \pdf@beginscope
      \pdf@beginpic
      \x@\s@tornamenttransform\ornament@xform
      \pdf@setlinewidth{\ornamentlinewidth}%
      \ifx\ornamentlinecol\empty\else
        \x@\pdf@color@rgb@stroke\ornamentlinecol\E
      \fi
      \ifx\ornamentfillcol\empty\else
        \ifx\ornamentfillcol\fill@none
          \ornamentfillfalse
        \else
          \ornamentfilltrue
          \x@\pdf@color@rgb@fill\ornamentfillcol\E%
        \fi
      \fi
      \ifornamentfill
        \let\s\pdf@fillstroke
      \else
        \let\s\k
      \fi
      \get@ornament@Code
      \pdf@endpic
      \pdf@endscope\hss
    }%
    \ifx\orn@clip@rect\empty
      \dimen2=0pt\dimen3=0pt\dimen4=0pt\dimen5=0pt
      \trace{orn}{No clip rect}%
    \else
      \x@\pdf@parse@clip@rect\orn@clip@rect
    \fi
    \ifdim\dimen2=0pt \else
      \setbox0=\hbox to \dimen0{\kern-\dimen2\unhbox0\kern\dimen2}%
    \fi
    \ht0=\dimen1%\@pgfornamentY\pdf@yunit
    \dimen7=\dimen3 % dimen7=vadjust. Start by adjusting to crop-box
    %Manual adjustments?
    \dimen8=0pt 
    \ifcsname orn@adj@\pdf@ornamentNum @\OrnamentsFamily @Y\endcsname
      \x@\let\x@\tmp\csname orn@adj@\pdf@ornamentNum @\OrnamentsFamily @Y\endcsname
      \dimen6=\dimexpr \tmp\pdf@xunit* \pdf@ab+ \tmp\pdf@yunit * \pdf@bb\relax
      \advance\dimen7 by -\dimen6
    \fi
    \ifcsname orn@adj@\pdf@ornamentNum @\OrnamentsFamily @X\endcsname
      \x@\let\x@\tmp\csname orn@adj@\pdf@ornamentNum @\OrnamentsFamily @X\endcsname
      \message{Adjusting by \tmp}%
      \dimen6=\dimexpr \tmp\pdf@xunit* \pdf@aa+ \tmp\pdf@yunit * \pdf@ba\relax
      \advance\dimen8 by -\dimen6
    \fi
    %Is cut box < reported size? Center cutbox. UNHELPFUL AND BUGGY 
    %\dimen6=\dimexpr \dimen5 - \dimen3\relax % Height of cut-box
    %\ifdim\dimen6<0pt \dimen6=-\dimen6\fi
    %\ifdim\dimen6>0pt
      %\ifdim\dimen1>\dimen6
	%\advance \dimen7 by 0.5\dimexpr \dimen6 -\dimen1\relax
      %\fi
    %\fi
    \ifboxorn
      \setbox0=\hbox{%
        \special{color push rgb \boxornRGB}%
        \raise\dimen1\hbox{\vrule height 0.02pt width\dimen0}%
        \kern-\dimen0
        \kern -0.02pt \vrule height \dimen1 width 0.02pt depth 0pt
        \vrule height 0.02pt width\dimen0  depth 0pt
        \vrule height \dimen1 width 0.02pt depth 0pt \kern -0.02pt 
        \kern-\dimen0
        \ifdim\dimen8=0pt \else\kern -\dimen8 \message{kerning by \the\dimen8}\fi%
        \ifdim\dimen7=0pt
	  \box0
        \else
          \lower \dimen7\box0
        \fi%
        \ifdim\dimen8=0pt \else\kern \dimen8\fi
        \special{color pop}%
      }%
      \ht0=\dimen1
      \dimen7=0pt
      \dimen8=0pt
      %\showbox0
    \fi
    \advance\dimen7  by \ornYalign\dimen1
    %\dimen7=-\ornYalign\dimen1
    \ifdim\dimen8=0pt \else\kern -\dimen8 \message{kerning by \the\dimen8}\fi
    \ifdim\dimen7=0pt
      \box0
    \else
      \ifhmode 
        \lower \dimen7\box0
      \else
        \box0
      \fi
    \fi%
    \ifdim\dimen8=0pt \else\kern \dimen8\fi
\egroup}
\def\doOrnament#1{%
  \begingroup
    \orn@setup
    \x@\ornament@parse#1||||\E
    \setorn@scale
    \orn@do@ornament
  \endgroup
}% 

  
\def\endstack\E{}
\def\E{}

\def\ornament@xform@u{{1}{0}{0}{1}} % Normal
\def\ornament@xform@d{{-1}{0}{0}{-1}} %Rotate 180 / Flip both
\def\ornament@xform@l{{0}{-1}{1}{0}} % Rotate +90
\def\ornament@xform@r{{0}{1}{-1}{0}} % Rotate -90
\def\ornament@xform@h{{-1}{0}{0}{1}} % Flip Horizontally 
\def\ornament@xform@v{{1}{0}{0}{-1}} % Flip vertically
\def\ornament@xform@L{{0}{-1}{-1}{0}} % Rotate +90 and Flip vertically
\def\ornament@xform@R{{0}{1}{1}{0}} % Rotate -90 and Flip vertically
\setornamenttransform{u}

\def\ptn@minus{-}
\def\ptn@ast{*}
\def\ptn@plus{+}
\def\ptn@quest{?}
\def\ptn@range{(}
\def\ptn@eq{=}
\def\ptn@sim{"} % Ditto mars
\def\ptn@ex{x}

\def\parse@repeat@class#1-#2)#3\E{\ifnum 0#2=0\set@repeat@class{#3}{#1}{}\else\set@repeat@class{#3}{#1}{#2}\fi}
\newif\ifpdf@ornament@stretch
\newif\ifpdf@ornament@repeat
\def\StringOrnamentFont{AGA Arabesque Desktop}
\newcount\orn@strings
\orn@strings=1000
\def\chk@cplx#1#2\E{\edef\tmpa{#1}\edef\tmpb{#2}\uppercase{\edef\tmpu{#2}}}
\def\string@check #1"#2"#3\E{%
  \edef\tmp{#1}\ifx\tmp\empty
    \edef\tmp{#2}\ifx\tmp\empty\else
      \stringornamenttrue
      \edef\string@rnament{#2}%
      \ifcsname orn@string\string@rnament\endcsname\else
        \global\advance\orn@strings by 1 
        \x@\xdef\csname orn@string\string@rnament\endcsname{\the\orn@strings}%
        \StringOrnament{\the\orn@strings}{\StringOrnamentFont}{\string@rnament}%
      \fi
      \x@\let\x@\pdf@ornamentNum\csname orn@string\string@rnament\endcsname
    \fi
  \fi
}
%#1 ornament
%#2 orientation
%#3 Scale/mult
%#4 Size {=|[a-z]|N.NN}  If a number (1.0 is valid, 2. is not) then the item's size will be this much times the normal calulated size.
% if =, the previous scaling will be applied, if it's set, but not '=' or a
% number in formal N.N, then a pre-set scale [which must have been set by setscale] will be applied 
\newif\iflocalornament
\newif\ifstringornament
\def\@@ornsetnum #1#2\E{%
 %\trace{eb}{Ornament number is "#1#2"}%
 \edef\pdf@ornamentNum{#1#2}%
}
\def\ornament@parse#1|#2|#3|#4|#5\E{%#5 is scrap
  \trace{orn}{ornament@parse:#1:#2:#3}%
  \localornamentfalse
  \stringornamentfalse
  \string@check#1""\E
  \ifstringornament
    \get@ornamentDim{\pdf@ornamentNum}%
  \else
    \x@\@@ornsetnum #1\empty\E
    \get@ornamentDim{\pdf@ornamentNum}%
  \fi
  %\ifx\dimen1=0pt
  %  \localornamenttrue
  %  \not@pgfornamentdim{#2}%
  %\fi
  \trace{orn}{Dimensions: \@pgfornamentX, \@pgfornamentY}%
  \setornamenttransform{#2}%
  \pdf@ornament@stretchfalse
  \pdf@ornament@repeatfalse
  \let\orn@repeat@class\empty
  \edef\tmp{#4}%
  \ifx\tmp\empty\ornscaletrue
  \else
    \ornscalefalse
    %\tracingassigns=1
    \ifx\tmp\ptn@eq\else %a scale of = means keep the same scaling.
      \x@\check@orn@dot\tmp..\E
    \fi
    \trace{orn}{orn@scale@#4: xunit \the\pdf@xunit\space yunit \the\pdf@yunit}%
    %\tracingassigns=0
  \fi
  \edef\tmp{#3}%
  \ifx\tmp\empty\else
    \ifx\tmp\ptn@minus
      \pdf@ornament@stretchtrue
      \trace{orn}{Stretchd #1}%
    \else
      \ifx\tmp\ptn@ex% No variable, xleader-fill
	\pdf@ornament@repeattrue
      \else
	\x@\chk@cplx#3\E %Split off first code
        %\uppercase{\edef\tmpu{\tmpb}}%
        \ifx\tmpb\tmpu\else % variable needs to be  upper case 
          \trace{orn}{\tmpb\space is lowercase! Treat as \tmpu}%
          \let\tmpb\tmpu %make it upper
          \pdf@ornament@stretchtrue
        \fi
        \ifx\tmpa\ptn@sim
	  \pdf@ornament@repeattrue
	  \edef\orn@repeat@class{\tmpb}% Varable, range set elsewhere
	\else 
	  \ifx\tmpa\ptn@ast % Any number
	    \pdf@ornament@repeattrue
	    \set@repeat@class{\tmpb}{0}{}%
	  \else
	    \ifx\tmpa\ptn@plus % 1 or more
	      \pdf@ornament@repeattrue
	      \set@repeat@class{\tmpb}{1}{}%
	    \else
	      \ifx\tmpa\ptn@quest % 0 or 1
		\pdf@ornament@repeattrue
		\set@repeat@class{\tmpb}{0}{1}%
	      \else
		\ifx\tmpa\ptn@range % Range
		  \pdf@ornament@repeattrue
		  \x@\parse@repeat@class\tmpb\E
		\else
		  \ifx\tmpa\ptn@eq % Preserve variable
		    \pdf@ornament@repeattrue
		    \ifcsname best@mult@class@num@\tmpb\endcsname
		      \x@\let\x@\tmp\csname best@mult@class@num@\tmpb\endcsname
		      \set@repeat@class{\tmpb}{\tmp}{\tmp}%
		    \else
		      \errmessage{Cannot re-use undefined variable \tmpb}%
		      \set@repeat@class{\tmpb}{1}{}%
		    \fi
		  \else
		    \message{Ornamental border: unrecognised control char '#3'}%
		  \fi
		\fi
	      \fi
	    \fi
	  \fi
	\fi
      \fi
    \fi
  \fi
}
\def\orn@rpt@c@missing{0}

\def\set@repeat@class#1#2#3{\edef\orn@repeat@class{#1}%
  \ifx\orn@repeat@class\empty
    \edef\orn@rpt@c@missing{\the\numexpr 1+\orn@rpt@c@missing\relax}%
    \trace{orn}{Missing repeat class}%
    \edef\orn@repeat@class{Z\orn@rpt@c@missing}%
  \fi
  \x@\edef\csname mult@repeat@min@\orn@repeat@class\endcsname{#2}%
  \x@\edef\csname mult@repeat@max@\orn@repeat@class\endcsname{#3}%
}

\newif\ifornscale
\def\orn@an@ornament#1,#2\E{%Without changing the scale, output several ornaments
  \x@\ornament@parse#1||||\E%
  \ifornscale
    \setorn@scale
  \fi
  \ifpdf@ornament@repeat
    \ifx\orn@repeat@class\empty
      \trace{orn}{Repeating ornament using xleaders for \orn@repeat@class}%
      \xleaders\hbox{\orn@do@ornament}\hskip 0pt plus 1 fil\relax
    \else
      \x@\let\x@\tmp \csname best@mult@class@num@\orn@repeat@class\endcsname
      \ifx\tmp\relax
	\trace{orn}{Repeating ornament using xleaders for \orn@repeat@class}%
	\xleaders\hbox{\orn@do@ornament}\hskip 0pt plus 1 fil\relax
      \else
	\trace{orn}{Repeating ornament #1 (\tmp times)}%
	\bgroup\setbox1=\hbox{\orn@do@ornament}\count255=\tmp\loop\ifnum\count255>0\copy1\advance\count255 by -1\repeat\egroup%
      \fi
    \fi
  \else
    \trace{orn}{Simple ornament #1}%
    \orn@do@ornament
  \fi
  \def\tmp{#2}%
  \ifx\tmp\empty
    \let\nxt=\endstack
  \else
    \let\nxt\orn@an@ornament
  \fi
  \x@\nxt\tmp\E
}

\def\Ornaments#1{%
  \begingroup
    \def\orn@rpt@c@missing{0}%
    \orn@setup
    \x@\orn@an@ornament#1,\E%
  \endgroup
}% 
\newdimen\fill@target % Target dimension
\newcount\mult@count
\newdimen\fill@fixed % length of rigid elements
\newdimen\fill@stretch % length of flexible elements
\newdimen\fill@mult % length of (non-stretchy) optional elements
\newdimen\fill@mult@min % Smallest optional element (0pt if none)
\newdimen\fill@mult@min@used % Smallest optional element currently in use. (0pt if none)

\newtoks\mult@classes
\mult@classes{}
\def\clear@mult@class#1{\x@\let\csname mult@class@dim@#1\endcsname\relax\x@\xdef\csname best@mult@class@num@#1\endcsname{0}}
\def\m@lti@inc#1{\x@\edef\csname mult@class@num@#1\endcsname{\the\numexpr \csname mult@class@num@#1\endcsname +1\relax}}
\def\m@lti@dec#1{\x@\edef\csname mult@class@num@#1\endcsname{\the\numexpr \csname mult@class@num@#1\endcsname -1\relax}}
\def\O@tp@t{}
\def\m@lti@print#1{\edef\O@tp@t{\O@tp@t #1: \csname mult@class@num@#1\endcsname\space}}
\def\best@m@lti@print#1{\edef\O@tp@t{\O@tp@t #1: \csname best@mult@class@num@#1\endcsname\space}}
\def\print@multi{\immediate\write-1{\O@tp@t}\let\O@tp@t\empty}
\def\foundb@st#1{\x@\global\x@\let\csname best@mult@class@num@#1\x@\endcsname\csname mult@class@num@#1\endcsname}
\newcount\best@score
\newdimen\best@adjust
\newdimen\best@stretch

\def\apply@checks#1{%
  \ifnum\csname mult@class@num@#1\endcsname < \csname mult@repeat@min@#1\endcsname \advance\count255 by \numexpr 1500 * (\csname mult@repeat@min@#1\endcsname - \csname mult@class@num@#1\endcsname) \relax\fi
  \x@\ifx\csname mult@repeat@max@#1\endcsname\empty\else
    \ifnum\csname mult@class@num@#1\endcsname > \csname mult@repeat@max@#1\endcsname \advance\count255 by \numexpr  1500 *(\csname mult@class@num@#1\endcsname - \csname mult@repeat@max@#1\endcsname)\relax \fi
  \fi
}


\def\prep@dims#1#2{%Prepare dimensions for calcsc@re. \t@mp is the natural length of the object \t@mpstr=\t@mp if the object is stretchy
  \ifnum #2 > 0 
    \ifdim\fill@mult@min@used=0pt
      \fill@mult@min@used=\t@mp
    \else\ifdim\fill@mult@min@used > \t@mp
        \fill@mult@min@used=\t@mp
    \fi\fi
    \dimen0=-\fill@mult@min@used % Sane limit for overfull under-fill
    \ifdim \t@mpstr>0pt
      \advance\fill@mult by #2\dimexpr \t@mp -\t@mpstr\relax  % mult-value also includes stretchy portion
      \advance\fill@stretch by #2\dimexpr \t@mpstr\relax %
    \else
      \advance\fill@mult by #2\dimexpr \t@mp \relax 
    \fi
  \fi
  \ifdim\fill@stretch=0pt
    \dimen0=-100sp% Have *some* overrun, in case of rounding errors
    \dimen2=1pt%
  \else
    \dimen2=\fill@mult@min % underrun of > smallest optional chunk is silly
    \dimen0=-0.5\fill@stretch % Shrink up to 50% is OK.
  \fi
  \dimen1=\dimexpr \fill@target - \fill@mult - \fill@stretch - \fill@fixed\relax
  \x@\edef\csname mult@class@num@#1\endcsname{\the\count255}%
  \ifdim \dimen0 =0pt % Sanity...
    \dimen0=-100sp
  \fi
  \trace{orn}{Try: remaining:\the\dimen1 (\the\fill@target - \the\fill@mult - \the\fill@stretch - \the\fill@fixed), flex:\the\dimen0, \the\dimen2}% 
}
  
\def\calcsc@re{\bgroup \dimen4=\ifdim\dimen1<0pt 0.1\dimen0\else 0.1\dimen2 \fi\count255=\dimen4 
  \trace{orn}{\the\dimen1 / \the\count255 }% 
  \dimen4=\dimexpr 10\dimen1 / \count255\relax \count255=\dimen4
  \trace{orn}{precheck score: \the\dimen4 (\the\count255)}%
  \let\D@\apply@checks\the\mult@classes
  \let\D@\m@lti@print\the\mult@classes
  \trace{orn}{postcheckdscore: \the\dimen4 (\the\count255) \the\fill@stretch}%
  \print@multi
  \ifnum\count255<\best@score 
    \global\best@score=\count255
    \global\best@adjust=\dimen1
    \global\best@stretch=\fill@stretch
    \let\D@\foundb@st\the\mult@classes
    \trace{orn}{New best score:\the\best@score, \the\dimen1, \the\best@stretch}%
 \else
    \trace{orn}{rejecting score:\the\count255, \the\dimen1}%
 \fi\egroup}

\def\setup@meas@ornament{%
  \fill@mult@min=0pt\fill@stretch=0pt\fill@fixed=0pt\best@score=\maxdimen
  \best@stretch=\fill@stretch
  \mult@count=0
}

\edef\mult@class@list{}
\def\clear@vars{%
  \let\D@\clear@mult@class
  \the\mult@classes
  \edef\mult@class@list{}%
  \mult@classes{}%
}

\def\orn@SetStretch#1#2#3#4{%
  \trace{orn}{SetStretch #1 #2 #3 #4}%
  \fill@target=#1%
  \ifornscale
    \setorn@scale
  \fi
  \begingroup
    \edef\orn@dir{#2}%
    \ifx\orn@dir\ptn@ex
      \def\@@orndim{0}\else\def\@@orndim{1}\fi
    \setup@meas@ornament
    \edef\tmp{#3}%
    \ifx\tmp\ptn@eq\else
      \clear@vars%
    \fi
    \x@\orn@meas@ornament#4,\E%Measure the (horizontal) dimensions
    \dimen1=\dimexpr \fill@target - \fill@fixed - \fill@stretch \relax
    %Set limits for stretch (dimen2) / shrink (dimen0)
    \ifdim\dimen2=0pt
      \dimen2=1em
    \fi
    \relax
    %\tracingifs=1
    \global\best@adjust=\dimen1
    \global\best@stretch=\fill@stretch
    %
    \ifdim\dimen1 < 0pt
      \trace{orn}{No space for optional parts}%
      \ifdim\dimen1 < -0.7\fill@stretch
	 \trace{orn}{Oversquashed fill}%
      \fi
      \clear@vars
    \else
      \trace{orn}{Calculating \the\mult@count\space optional parts (\mult@class@list)}%
      \ifnum\mult@count>0
	\x@\multi@try\mult@class@list,\E{-1}%
	\relax
	\let\D@\best@m@lti@print\the\mult@classes
        \trace{orn}{Final result: stretch:\the\best@stretch, score:\the\best@score}\print@multi
      \fi
    \fi
    \dimen4=\dimexpr \best@stretch+\best@adjust\relax
    \ifdim\best@stretch=0pt 
    \else
      \trace{orn}{Stretch = (a:\the\best@adjust,  s:\the\best@stretch) (a+s:\the\dimen4)/s)}%
      \dimen5=\dimexpr \best@stretch*4\relax%
      \multiply\dimen4 by 4\relax
      \dimen6=\dimen4
      %\tracingassigns=1
      \ifdim\dimen6<0pt \multiply \dimen6 by -1\relax\fi
      \ifdim \dimen5 > \dimen6 \dimen6=\dimen5 \fi
      \loop \ifdim\dimen6<0.4\maxdimen
	\multiply\dimen5 by 2\relax
	\multiply\dimen4 by 2\relax
	\multiply\dimen6 by 2\relax
	\repeat
      %\tracingassigns=0
      \count255=\dimen5 \divide\count255 by 65535 % Integer part only
      \trace{orn}{\the\dimen4 / \the\count255}%
      \dimen4=\dimexpr \dimen4 / \count255\relax%Scale by 32
      \xdef\ornamentstretch{\strip@pt{\dimen4}}%
    \fi
    \trace{orn}{Ornament stretch set to \ornamentstretch}%
  \endgroup
}

\def\E{}
\def\add@mult@class#1{\x@\mult@classes\x@{\the\mult@classes\D@{#1}}}
\newif\iftemp
\def\orn@meas@ornament#1,#2\E{%
  \x@\ornament@parse#1||||\E %Set relevant transform matrix and various bools
  \ifornscale
    \setorn@scale
  \fi
  \tmptoks{}%
  \calc@ornamentdim %dimen0=x, dimen1=y
  \ifdim\dimen\@@orndim<0pt
    \dimen\@@orndim=-\dimen\@@orndim
  \fi
  \trace{orn}{Ornament #1 is \the\dimen0 by \the\dimen1 \ifpdf@ornament@stretch Stretchy\fi\ifpdf@ornament@repeat Repeat\fi}%
  \ifpdf@ornament@repeat
    \advance\mult@count by 1
    \ifdim\fill@mult@min=0pt
      \fill@mult@min=\dimen\@@orndim
    \else 
      \ifdim\fill@mult@min>\dimen\@@orndim
        \fill@mult@min=\dimen\@@orndim
      \fi
    \fi
    \x@\let\x@\tmp\csname mult@class@dim@\orn@repeat@class\endcsname
    \ifx\relax\tmp
      \trace{orn}{New class \orn@repeat@class}%
      \ifx\orn@repeat@class\empty\else\x@\add@mult@class\x@{\orn@repeat@class}\fi
      \x@\edef\csname mult@class@dim@\orn@repeat@class\endcsname{\the\dimen\@@orndim}%
      \ifpdf@ornament@stretch
        \x@\edef\csname mult@class@stretch@dim@\orn@repeat@class\endcsname{\the\dimen\@@orndim}%
        \x@\edef\csname mult@class@dim@\orn@repeat@class\endcsname{\the\dimen\@@orndim}%
      \else
        \x@\edef\csname mult@class@dim@\orn@repeat@class\endcsname{\the\dimen\@@orndim}%
        \x@\edef\csname mult@class@stretch@dim@\orn@repeat@class\endcsname{0pt}%
      \fi
      \x@\edef\csname mult@class@num@\orn@repeat@class\endcsname{0}%
      \edef\mult@class@list{\ifx\mult@class@list\empty\else \mult@class@list,\fi \orn@repeat@class}%
    \else
      \x@\edef\csname mult@class@dim@\orn@repeat@class\endcsname{\the\dimexpr\tmp+\the\dimen\@@orndim\relax}%
      \ifpdf@ornament@stretch
        \x@\let\x@\tmpz\csname mult@class@stretch@dim@\orn@repeat@class\endcsname
        \x@\edef\csname mult@class@stretch@dim@\orn@repeat@class\endcsname{\the\dimexpr\tmpz+\the\dimen\@@orndim\relax}%
      \fi
    \fi
    \trace{orn}{\orn@repeat@class: \csname mult@class@dim@\orn@repeat@class\endcsname, \csname mult@class@stretch@dim@\orn@repeat@class\endcsname}%
  \else
    \ifpdf@ornament@stretch
      \advance\fill@stretch by \dimen\@@orndim %stretchy ones subtract roughly their own length
    \else
      \advance\fill@fixed by \dimen\@@orndim
    \fi
  \fi
  \def\tmp{#2}%
  \ifx\tmp\empty
    \let\nxt=\endstack
  \else
    \let\nxt\orn@meas@ornament
  \fi
  \x@\nxt\tmp\E
}

\def\multi@try#1,#2\E#3{\bgroup
    \trace{orn}{multitry '#1' {#3}}%
    \x@\let\x@\t@mp\csname mult@class@dim@#1\endcsname
    \x@\let\x@\t@mpstr\csname mult@class@stretch@dim@#1\endcsname
    \count255=#3
    \ifnum\count255=-1
      \ifdim 3\dimexpr \dimen1-\dimen0\relax > \t@mp
	\dimen3=\dimexpr \dimen1 \ifdim -\dimen0 >\t@mp  + \t@mp \else -\dimen0 \fi\relax
	\dimen4=\t@mp
	\count255=\dimen4
	\count255=\numexpr  \dimen3 / \count255 \relax
	\trace{orn}{ #1 (\t@mp) fits into \the\dimen3 \the\count255\space times}%
	\ifcsname mult@repeat@max@#1\endcsname
          \x@\ifx\csname mult@repeat@max@#1\endcsname\empty\else
            \ifnum \csname mult@repeat@max@#1\endcsname < \count255
              \count255=\csname mult@repeat@max@#1\endcsname
              \trace{orn}{That's above max for #1 (\the\count255); limiting range}%
              \advance \count255 by 1
            \fi
          \fi
        \fi
	\ifcsname mult@repeat@min@#1\endcsname
	  \ifnum \csname mult@repeat@min@#1\endcsname>\count255
	    \count255=\csname mult@repeat@min@#1\endcsname
	    \trace{orn}{ min for #1 (\the\count255) doesn't fit, but try it anyway (it might be shrinkable)}%
	  \fi
	\fi
      \else
	\count255=0
      \fi
    \fi
    \def\tmp{#2}%
    \global\tmptoks{}%
    \ifx\tmp\empty % No more optional items
      \bgroup
        \prep@dims{#1}{\count255}%
	\calcsc@re
      \egroup
      \ifnum\count255>0 %We included some over-run, Try one under...
	\bgroup
	  \advance\count255 by -1
          \prep@dims{#1}{\count255}%
	  \calcsc@re
	\egroup
      \fi
      \ifdim\t@mpstr >0pt % Item can shrink: also try 1 over. 
        \bgroup
          \advance\count255 by 1
          \prep@dims{#1}{\count255}%
          \calcsc@re
        \egroup
      \fi
    \else
      \ifdim\t@mpstr>0pt
          \advance\count255 by 1 % item can shrink; start at 1 over
      \fi
      \loop\unless\ifnum\count255<0
	\bgroup
          \prep@dims{#1}{\count255}%
	  \x@\multi@try#2\E{-1}%
	\egroup
	\advance\count255 by -1
      \repeat
    \fi
  \egroup
}  
% Parameters:
% 1 ornament number
% 2 output x dimension (or empty)
% 3 output y dimension (or empty)
% 4 mirror -> h = horiz, v=vert, c=both
% 5 rotation: up is -> u=up d=down l=left r=right

\def\UseOrnament#1#2#3#4#5{%
    \orn@setup
    \def\ornamentwidth{#2}%
    \def\ornamentheight{#3}%
    \def\ornamentmirror{#4}%
    \def\ornamentorientation{#5}%
    \ifx\ornamentorientation\empty
      \def\ornamentorientation{u}% Up is up
    \fi
    \x@\ornament@parse#1||||\E
    \setorn@scale
    \doOrnament{#1}%
}

\newif\ifornamentfill
\def\chkset#1#2{\edef\tmp{#2}\ifx\tmp\empty\else\x@\global\x@\let\csname #1\endcsname\tmp\fi}

% Parameters:
% 1 stroke width
% 2 stroke colour (or empty for no change)
% 3 fill colour (or 'none' for no fill, empty for no change)
\def\SetupOrnament#1#2#3{% Line width / Line colour / Fill colour
  \chkset{ornamentlinewidth}{#1}%
  \chkset{ornamentlinecol}{#2}%
  \chkset{ornamentfillcol}{#3}%
}

%\tracingassigns=1
%\tracingmacros=1

%Format for pattern:
% Ornament number | transformation | Fitting Strategy | scale
% Fit is what should happnen to adjust spacing:
% [empty]  - Nothing - Just produce one ornament
% -     - stretch ornament to fill space 
% x	- Fill space using \xleaders 
% *V	- Variable V should be any integer including 0
% +V	- Variable V should be any integer above 0
% ?V	- Variable V should be 0 or 1
% (L-H)V - Variable V should be any integer between L and H (inclusve)
% ~V	- Variable V (no change to rules)
% =V	- Variable V (unchanged from reuslt.)
% Items sharing the same  variable V use the same range limits
% The last range limit is the one that will take effect. 
\newbox\orn@bdr@bot
\newbox\orn@bdr@left
\newbox\orn@bdr@right
\newbox\orn@bdr@top
% Box forming:
% Given X and Y values,
\def\ornamentheight{10pt}
\def\trybox#1#2#3{%
  \trace{orn}{TRYBOX}%
  \def\ornamentheight{#3}%
  \def\ornamentwidth{}%
  \def\ornYalign{1}%
  \orn@SetStretch{#1}{x}{}{\patternT}\setbox\orn@bdr@top\hbox to #1{\Ornaments{\patternT}}%
  \def\ornYalign{0}%
  \orn@SetStretch{#1}{x}{=}{\patternB}\setbox\orn@bdr@bot\hbox to #1{\Ornaments{\patternB}}%
  \trace{orn}{#2 - \the\ht\orn@bdr@top -\the\dp\orn@bdr@top -\the\ht\orn@bdr@bot -\the\dp\orn@bdr@bot}%
  \dimen1=\dimexpr #2 - \ht\orn@bdr@top -\dp\orn@bdr@top -\ht\orn@bdr@bot -\dp\orn@bdr@bot\relax
  %\def\ornamentheight{}
  %\def\ornamentwidth{#3}
  \trace{orn}{Box side is \the\dimen1}%
  \orn@SetStretch{\the\dimen1}{y}{=}{\patternL}\setbox\orn@bdr@left\vbox to \dimen1{\lineskip=0pt \baselineskip=0pt \Ornaments{\patternL}}%
  %\showbox\orn@bdr@left
  \ifx\patternL\patternR
    \setbox\orn@bdr@right\copy\orn@bdr@left
  \else
    \orn@SetStretch{\the\dimen1}{y}{=}{\patternR}\setbox\orn@bdr@right\vbox to \dimen1{\lineskip=0pt \baselineskip=0pt \Ornaments{\patternR}}%
  \fi
  %\showboxdepth=9
  %\showboxdepth=10
  %\showbox\orn@bdr@right
  \vbox to #2{\baselineskip=0pt \lineskip=0pt \hsize=#1\box\orn@bdr@top\hbox to #1{\box\orn@bdr@left\hfill\box\orn@bdr@right}\box\orn@bdr@bot}%
}
\def\OrnamentBorder#1{%
  \trace{orn}{OrnamentBorder}%
  \def\ornamentheight{#1}%
  \def\ornamentwidth{}%
  \def\ornYalign{1}%
  \ifx\b@drtop\tr@e
    \orn@SetStretch{\b@rderbox@width}{x}{}{\patternT}\setbox\orn@bdr@top\hbox to \b@rderbox@width{\Ornaments{\patternT}}%
  \else
    \setbox\orn@bdr@top\hbox{}%
  \fi
  \def\ornYalign{0}%
  \ifx\b@drbottom\tr@e
    \orn@SetStretch{\b@rderbox@width}{x}{=}{\patternB}\setbox\orn@bdr@bot\hbox to \b@rderbox@width{\Ornaments{\patternB}}%
  \else
    \setbox\orn@bdr@bot\hbox{}%
  \fi
  \trace{orn}{#1 - \the\ht\orn@bdr@top -\the\dp\orn@bdr@top -\the\ht\orn@bdr@bot -\the\dp\orn@bdr@bot}%
  \dimen1=\dimexpr \b@rderbox@height - \ht\orn@bdr@top -\dp\orn@bdr@top -\ht\orn@bdr@bot -\dp\orn@bdr@bot\relax
  \dimen2=\dimexpr \dimen1 - \ht\b@rderbox\relax
  \divide\dimen2 by \b@drvsides
  %\def\ornamentheight{}
  %\def\ornamentwidth{#3}
  \trace{orn}{Box side portion is \the\dimen1. Orig: \the\ht\b@rderbox +\the\dp\b@rderbox x\the\wd\b@rderbox, final:\b@rderbox@height }%
  \tempfalse
  \def\border@l@pad{0pt}\def\border@r@pad{0pt}%
  \setbox\orn@bdr@right\box\voidb@x
  \ifx\b@drleft\tr@e
    \orn@SetStretch{\the\dimen1}{y}{=}{\patternL}\setbox\orn@bdr@left\vbox to \dimen1{\lineskip=0pt \lineskiplimit=0pt \baselineskip=0pt \Ornaments{\patternL}}%
    \def\border@l@pad{\border@hpadding}%
    \ifx\patternL\patternR
      \ifx\b@drright\tr@e
	\setbox\orn@bdr@right\copy\orn@bdr@left
        \def\border@r@pad{\border@hpadding}%
      \fi
    \fi
  \else
    \setbox\orn@bdr@left\vbox{}%
  \fi
  %\showbox\orn@bdr@left
  \ifx\b@drright\tr@e
    \ifvoid\orn@bdr@right
      \orn@SetStretch{\the\dimen1}{y}{=}{\patternR}\setbox\orn@bdr@right\vbox to \dimen1{\lineskip=0pt \lineskiplimit=0pt \baselineskip=0pt \Ornaments{\patternR}}%
      \def\border@r@pad{\border@hpadding}%
    \fi
  \fi
  \showboxdepth=3
  \showboxbreadth=9999
  %\showbox\orn@bdr@right
  \dimen7=\dimexpr\ht\orn@bdr@top+\dp\orn@bdr@top  \ifx\b@drtop\tr@e +\dimen2\fi \relax
  \dimen8=\dimexpr\wd\orn@bdr@left + \border@l@pad\relax
  \dimen9=\dimexpr\wd\orn@bdr@right + \border@r@pad\relax
  \setbox\b@rderbox\vbox to \b@rderbox@height{\baselineskip=0pt \lineskip=0pt \hsize=\b@rderbox@width
    \box\orn@bdr@top\hbox to \b@rderbox@width{\box\orn@bdr@left\hskip\border@l@pad\hss\hbox{\vbox to \dimen1{}}\hskip\border@r@pad\hss\box\orn@bdr@right}\baselineskip=\ht\orn@bdr@bot\box\orn@bdr@bot}%
  %\showbox#1
  %\x@\setbox#1\vbox{\vbox to 0pt{\vskip\dimen7\hbox to \b@rderbox@width{\hskip\dimen8\hss\box#1\hss\hskip\dimen9}\vss}\box\b@rderbox}%
}

\def\OrnamentLine#1#2#3#4{%#1 Line length #2 Ornament height #3 Direction {Capital to keep old variables)  #4 Pattern
  \def\ornamentheight{#2}%
  \def\tmp{#3}\lowercase{\def\tmpb{#3}}%
  \ifx\tmp\tmpb\def\tmp{}\else\def\tmp{=}\fi
  \orn@SetStretch{#1}{\tmpb}{\tmp}{#4}%
  \ifx\tmpb\ptn@ex\hbox \else \vbox\fi to #1{\Ornaments{#4}}%
}

%Stylesheet Options
\def\BorderPatternTop #1\relax{\initc@t\setsbp@ram{borderpatterntop}{#1}}
\def\BorderPatternBot #1\relax{\initc@t\setsbp@ram{borderpatternbot}{#1}}
\def\BorderPatternLeft #1\relax{\initc@t\setsbp@ram{borderpatternleft}{#1}}
\def\BorderPatternRight #1\relax{\initc@t\setsbp@ram{borderpatternright}{#1}}



\def\ornamentheight{20pt}\def\ornamentwidth{}
\def\SwitchOrnamentsFamily#1{%
  %\tracingassigns=1
  \edef\ornDimname{nopgfornamentDim-#1}%
  \ifcsname\ornDimname\endcsname\else 
    \bgroup\let\@pgfornamentDim\relax
      \def\OrnamentsFamily{#1}\includeifpresent{\pgfOrnamentsObject}%
      \ifx\@pgfornamentDim\relax\else\x@\global\x@\let\csname\ornDimname\endcsname\@pgfornamentDim\fi
    \egroup
  \fi
  \ifcsname\ornDimname\endcsname
    \x@\global\x@\let\x@\@pgfornamentDim\csname \ornDimname\endcsname
    \xdef\OrnamentsFamily{#1}%
  \else\trace{orn}{Could not load \pgfOrnamentsObject}%
  \fi
  %\tracingassigns=0
}

\addtoinithooks{\x@\SwitchOrnamentsFamily\x@{\OrnamentsFamily}}%
\catcode`\@=11
\def\m@rker{esb}

\def\DefaultOrnament{-1}
\def\mkb@rder@ornaments{% Plugin for sidebars
  \trace{eb}{mkb@rder@ornaments}%
  \def\ornamentheight{\b@drwidth}%
  \getsbp@ram{borderfillcolour}\ifx\cp@ram\relax\edef\ornamentfillcol{\fill@none}\else\let\ornamentfillcol\cp@ram\fi
  \trace{eb}{Colour:\meaning\b@drcol, Fill:\meaning\ornamentfillcol}
  \ifx\b@drcol\relax
    \def\ornamentlinecol{0 0 0}\else
    \let\ornamentlinecol\b@drcol
  \fi
  \getsbp@ram{borderstyleextra}\ifx\cp@ram\relax\else\x@\SwitchOrnamentsFamily\x@{\cp@ram}\fi
  \getsbp@ram{borderpatterntop}\ifx\cp@ram\relax\edef\patternT{\DefaultOrnament||-}\else\let\patternT\cp@ram\fi
  \getsbp@ram{borderpatternbot}\ifx\cp@ram\relax\edef\patternB{\DefaultOrnament|u|-}\else\let\patternB\cp@ram\fi
  \getsbp@ram{borderpatternleft}\ifx\cp@ram\relax\edef\patternL{\DefaultOrnament|l|-}\else\let\patternL\cp@ram\fi
  \getsbp@ram{borderpatternright}\ifx\cp@ram\relax\edef\patternR{\DefaultOrnament|r|-}\else\let\patternR\cp@ram\fi
  \edef\ornamentlinewidth{\strip@pt{\dimexpr\b@drlinewidth\FontSizeUnit\relax}}%
  \OrnamentBorder{\b@drwidth\FontSizeUnit}%
  %\setbox\b@rderbox\vbox{\trybox{\b@rderbox@width}{\b@rderbox@height}{\b@drwidth\FontSizeUnit}}%
  \trace{orn}{ORNAMENTS box \the\ht\b@rderbox+\the\dp\b@rderbox * \the\wd\b@rderbox}%
  %\showbox#1\copy #1 
}
\edef\b@rderparameters{\b@rderparameters,borderstyleextra,borderpatterntop,borderpatternleft,borderpatternright,borderpatternbot}

\x@\def\csname drawzrule-ornaments\endcsname{%
  \getp@ram{borderpatterntop}{zrule}{\styst@k}%
  \ifx\p@ram\relax
    \def\p@ttern{88||-}%
  \else
    \let\p@ttern\p@ram
  \fi
  \let\oldOrnamentsFamily\OrnamentsFamily
  \getp@ram{borderstyleextra}{zrule}{\styst@k}%
  \ifx\p@ram\relax\else 
    \x@\SwitchOrnamentsFamily\x@{\p@ram}%
  \fi
  \getp@ram{borderlinewidth}{zrule}{\styst@k}%
  \ifx\p@ram\relax
    \edef\ornamentlinewidth{0.5}%
  \else
    \edef\ornamentlinewidth{\p@ram}%
  \fi
  \getp@ram{bordercolour}{zrule}{\styst@k}%
  \ifx\p@ram\relax
    \def\ornamentlinecol{0 0 0}%
  \else 
    \let\ornamentlinecol\p@ram
  \fi
  \getp@ram{borderfillcolour}{zrule}{\styst@k}%
  \ifx\p@ram\relax
    \def\ornamentfillcol{none}%
  \else 
    \let\ornamentfillcol\p@ram
  \fi
  \edef\ornamentheight{\dimexpr\rule@thk \relax}%
  \edef\ornamentwidth{}%
  \getp@ram{verticalalign}{zrule}{\styst@k}%
  \ifx\p@ram\@lignTop
    \def\ornYalign{1}%
  \else\ifx\p@ram\@lignBot
      \def\ornYalign{0}%
    \else
      \def\ornYalign{.5}%
    \fi
  \fi
  \endgraf
  \edef\rule@adjust{\the\dimexpr \rule@adjust + \ornYalign \dimexpr\rule@thk\relax\relax}%
  \bgroup 
    \trace{m}{Rule will be \rule@wid (hsize=\the\hsize)}%
  \hbox{\raise\rule@adjust\hbox to \hsize{\hskip\leftskip\ifx\rule@pos\@lignLeft\else\hfil\fi
    \hbox to \rule@wid{\orn@SetStretch{\rule@wid}{x}{}{\p@ttern}\Ornaments{\p@ttern}%\hss\vrule height \rule@thk width 0.1pt
    }%
    \ifx\rule@pos\@lignRight\else\hfil\fi\hskip\rightskip}}%
  \egroup
  \ifx\oldOrnamentsFamily\OrnamentsFamily\else
    \trace{m}{Restoring \oldOrnamentsFamily (from \OrnamentsFamily)}%
    \SwitchOrnamentsFamily{\oldOrnamentsFamily}%
  \fi
}
\def\pgfsetmiterlimit#1{}
\edef\borderstylelist{\borderstylelist, ornaments}
\plugin@endif
\let\pgfsetroundjoin\relax
\let\typeout\message
\let\pgfsetrectcap\relax
