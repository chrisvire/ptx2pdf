% Test the 'hook' code.
\def\addmtlquteht{2pt}
\def\addmtlqutewd{1.5pt}
\def\addmtlqutelw{0.2pt}
\def\beginadd{\lower 1pt \hbox{\vrule width \addmtlqutelw height \addmtlquteht\vrule width \addmtlqutewd height \addmtlqutelw}\kern -1pt}
\def\endadd{\kern -1pt \lower 1pt \hbox{\vrule width \addmtlqutewd height \addmtlqutelw\vrule width \addmtlqutelw height \addmtlquteht}}
\sethook{before}{m}{\dimen1=\prevdepth\MSG{TEST: \the\dimen1}\kern-\dimen1\vbox to 0pt{\vskip 0.5pt\hrule width 10em height 0.0pt depth 0.5pt}\vskip 0pt}
\sethook{after}{m}{\kern .5pt \hrule width 10em height .2pt \kern 1pt \hrule width 10em height .2pt}
\sethook{start}{m}{\beginadd\beginadd}
\sethook{end}{m}{\endadd\endadd}
\sethook{before}{add}{|}
\sethook{after}{add}{|}
\sethook{start}{add}{\beginadd}
\sethook{end}{add}{\endadd}
\setbookhook{end}{JHN}{\global\def\doLines{\doGridLines}}
\csname ColorFontstrue\endcsname
%\tracing{o}
%\NoteAtEnd{f}
\NoteAtEnd{x}
\ParagraphedNotes{x}
\ptxfile{test.usfm}
%\sethook{final}{afterincludes}{\prepusfm\csname m@kecover\endcsname\unprepusfm}
\tracing{e}
\tracing{cov}
\tracing{m}
%\tracing{g}
\tracing{s}
%\tracing{orn}
\tracing{j}
\NoTransparencyfalse
\edef\timestamp{Year-Month-Day}
\endinput
